\section*{8 线性映射矩阵表示(II)}
\addcontentsline{toc}{section}{8 线性映射矩阵表示(II)}

\vspace{2ex}

\centerline{\heiti A组}
\begin{enumerate}
    \item 
\end{enumerate}

\centerline{\heiti B组}
\begin{enumerate}
    \item 为使得“每行元素之和”的条件有用,我们用 $\alpha=\begin{pmatrix}1 \\ 1 \\ \vdots \\ 1\end{pmatrix}$ 去乘以 $A$. 则 $A\alpha=\begin{pmatrix}k \\ k \\ \vdots \\ k\end{pmatrix}=k\alpha$.
    因为 $A$ 可逆所以 $k\neq 0$,同时由上面的式子有 $\alpha=kA^{-1}\alpha$,得 $A^{-1}\alpha=\dfrac{1}{k}\alpha(k\neq 0)$.
    故 $A^{-1}$ 每行和为 $\dfrac{1}{k}$ 成立.
    \item \begin{enumerate}
        \item 由题意 $r(A)=r(B)$. $A$ 可由初等变换为 $\begin{pmatrix}1 & 2 & a \\ 1 & 3 & 0 \\ 0 & 0 & 0\end{pmatrix}$,$B$ 可由初等变换为 $\begin{pmatrix}1 & a & 2 \\ 0 & 1 & 1 \\ 0 & 0 & 2-a\end{pmatrix}$.
        由于秩相同故 $2-a=0,a=2$.
        \item $(A,B)=\begin{pmatrix}1 & 2 & a & 1 & a & 2 \\ 1 & 3 & 0 & 0 & 1 & 1 \\ 2 & 7 & -a & -1 & 1 & 1\end{pmatrix}\rightarrow\begin{pmatrix}1 & 0 & 6 & 3 & 4 & 4 \\ 0 & 1 & -2 & -1 & -1 & -1 \\ 0 & 0 & 0 & 0 & 0 & 0\end{pmatrix}$
        
        $X = \begin{pmatrix}-6k_1+3 \\ 2k_1-1 \\ k_1\end{pmatrix},Y = \begin{pmatrix}-6k_2+4 \\ 2k_2-1 \\ k_2\end{pmatrix},Z = \begin{pmatrix}-6k_3+4 \\ 2k_3-1 \\ k_3\end{pmatrix}$
        
        $\begin{pmatrix}-6k_1+3 & -6k_2+4 & -6k_3+4 \\ 2k_1-1 & 2k_2-1 & 2k_3-1 \\ k_1 & k_2 & k_3\end{pmatrix}\rightarrow \begin{pmatrix}1 & 1 & 1 \\ 0 & 1 & 1 \\ 0 & 0 & k_3-k_2\end{pmatrix}$

        因为 $P$ 可逆,所以 $k_2\neq k_3$,故 $P=\begin{pmatrix}-6k_1+3 & -6k_2+4 & -6k_3+4 \\ 2k_1-1 & 2k_2-1 & 2k_3-1 \\ k_1 & k_2 & k_3\end{pmatrix},k_2\neq k_3, k_1,k_2,k_3 \in \mathbf{R}$
    \end{enumerate}
\end{enumerate}

\centerline{\heiti C组}
\begin{enumerate}
    \item 首先设
    \[A=\begin{pmatrix}a_1 & a_2 \\ a_3 & a_4\end{pmatrix},B=\begin{pmatrix}b_1 & b_2 \\ b_3 & b_4\end{pmatrix},C=\begin{pmatrix}c_1 & c_2 \\ c_3 & c_4\end{pmatrix}.\]
    由于 $A,B,C$ 在 $M_2(\mathbf{C})$ 中线性无关,所以将 $A,B,C$ 的元素排为一列,可知矩阵
    \[\begin{pmatrix}a_1 & b_1 & c_1 \\ a_2 & b_2 & c_2 \\ a_3 & b_3 & c_3 \\ a_4 & b_4 & c_4\end{pmatrix}\]
    的秩为 $3$,这里不妨设前三个行向量线性无关,即有
    \[D=\begin{pmatrix}a_1 & b_1 & c_1 \\ a_2 & b_2 & c_2 \\ a_3 & b_3 & c_3\end{pmatrix}\]
    为可逆矩阵.

    另外,注意到对任意的 $x_1,x_2,x_3$,有
    \[x_1A+x_2B+x_3C=\begin{pmatrix}a_1x_1+b_1x_2+c_1x_3 & a_2x_1+b_2x_2+c_2x_3 \\ a_3x_1+b_3x_2+c_3x_3 & a_4x_1+b_4x_2+c_4x_3\end{pmatrix}.\]
    现在考虑方程组
    \[\begin{cases}a_1x_1+b_1x_2+c_1x_3 = 0 \\ a_2x_1+b_2x_2+c_2x_3 = 1 \\ a_3x_1+b_3x_2+c_3x_3 = 1\end{cases}\]
    其系数矩阵为 $D$,这是一个可逆矩阵,所以上述方程存在唯一解,不妨记为 $(x_1',x_2',x_3')$,此时就有 $\lvert x_1'A+x_2'B+x_3'C \rvert = \begin{vmatrix}0 & 1 \\ 1 & \ast\end{vmatrix} = -1$.
    
    所以 $x_1'A+x_2'B+x_3'C$ 为可逆矩阵.
\end{enumerate}

\clearpage
