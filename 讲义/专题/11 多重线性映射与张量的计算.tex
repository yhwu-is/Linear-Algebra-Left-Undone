\chapter{多重线性映射与张量的计算}

\section{多重线性映射}

\vspace{2ex}
\centerline{\heiti \Large 内容总结}

\vspace{2ex}
\centerline{\heiti \Large 习题}

\vspace{2ex}
{\kaishu 在这里,我们形成了一致性的闭环:物理法则产生了复杂系统,复杂系统导致了意识的存在,而意识使得人能够理解数学:一种编码了物理法则的底层逻辑的语言.}
\begin{flushright}
    \kaishu
    —— R. 彭罗斯(Roger Penrose)
\end{flushright}

\centerline{\heiti A组}
\begin{enumerate}
    \item
\end{enumerate}

\centerline{\heiti B组}
\begin{enumerate}
    \item
\end{enumerate}

\centerline{\heiti C组}
\begin{enumerate}
    \item
\end{enumerate}
