\chapter{线性映射矩阵表示(II)}

\section{矩阵转置}
\subsection{基本概念}
实际上,矩阵的转置就是第$i$行变成了第$i$列,或者抽象表达为:
\[A=(a_{ij})_{m \times n},\enspace A^\mathrm{T}=(a'_{ji})_{n \times m},\enspace a_{ij}=a'_{ji}\]
写成矩阵形式就是:
\begin{definition}
    设$A=\begin{pmatrix}
        a_{11} & a_{12} & \cdots & a_{1n} \\
        a_{21} & a_{22} & \cdots & a_{2n} \\
        \vdots & \vdots & \ddots & \vdots \\
        a_{m1} & a_{m2} & \cdots & a_{mn}
    \end{pmatrix}$,称$\begin{pmatrix}
        a_{11} & a_{21} & \cdots & a_{m1} \\
        a_{12} & a_{22} & \cdots & a_{m2} \\
        \vdots & \vdots & \ddots & \vdots \\
        a_{1n} & a_{2n} & \cdots & a_{mn}
    \end{pmatrix}$为矩阵$A$的转置,记作$A^\mathrm{T}$.
\end{definition}

\subsection{基本性质}
\begin{enumerate}
    \item $(A^\mathrm{T})^\mathrm{T}=A$

    \item $(A+B)^\mathrm{T}=A^\mathrm{T}+B^\mathrm{T}$

    \item $(\lambda A)^\mathrm{T}=\lambda A^\mathrm{T},\enspace \lambda \in \mathbf{F}$

    \item $(AB)^\mathrm{T}=B^\mathrm{T}A^\mathrm{T}$,$(A_1A_2\cdots A_n)^\mathrm{T}=A_n^\mathrm{T}\cdots A_2^\mathrm{T}A_1^\mathrm{T}$

    \item $(A^\mathrm{T})^{-1}=(A^{-1})^\mathrm{T}$

    \item $(A^\mathrm{T})^m=(A^m)^\mathrm{T}$
\end{enumerate}

以上证明大都是平凡的,可以自己尝试完成.
\subsection{对阵矩阵与反对称矩阵}
\begin{definition}
    设$A=(a_{ij})_{n \times n}$,如果$\forall i,j=1,2,\ldots,n$均有$a_{ij}=a_{ji}$,
    则称$A$为对称矩阵. 若均有$a_{ij}=-a_{ji}$,则称$A$为反对称矩阵.
\end{definition}
易得$A$为对称矩阵的充要条件为$A=A^\mathrm{T}$,$A$为反对称矩阵的充要条件为$A=-A^\mathrm{T}$.
\begin{example}
    证明以下几点性质:
    \begin{enumerate}
        \item 反对称矩阵主对角元均为0;

        \item $AA^\mathrm{T}$和$A^\mathrm{T}A$均为对称矩阵;

        \item 设$A,B$为$n$阶对称和反对称矩阵,则$AB+BA$是反对称矩阵;

        \item 对称矩阵的乘积不一定对称;

        \item 可逆的对称(反对称)矩阵的逆矩阵也是对称(反对称)矩阵.
    \end{enumerate}
\end{example}

\section{初等矩阵}
\subsection{基本概念与性质}
\begin{definition}
    将单位矩阵$E$做一次初等变换得到的矩阵称为初等矩阵,与三种初等行、列变换对应的三类初等矩阵为:
    \begin{enumerate}
        \item 将单位矩阵第$i$行(或列)乘$c$,得到初等倍乘矩阵$E_i(c)$;

        \item 将单位矩阵第$i$行乘$c$加到第$j$行,或将第$j$列乘$c$加到第$i$列,得到初等倍加矩阵$E_{ij}(c)$;

        \item 将单位矩阵第$i,j$行(或列)对换,得到初等对换矩阵$E_{ij}$.
    \end{enumerate}
\end{definition}
请各位同学以矩阵形式写出以上三类矩阵.注意:
\begin{enumerate}
    \item 倍加变化请一定注意$i$和$j$在行列的情况下的不同;

    \item 三类矩阵不是三个矩阵,例如行列选择不唯一,常数选择不唯一;

    \item 注意三种初等矩阵都是可逆的,且$E_i^{-1}(c)=E_i\left(\dfrac{1}{c}\right)$,$E_{ij}^{-1}(c)=E_{ij}(-c)$,$E_{ij}^{-1}=E_{ij}$;

    \item 三种初等矩阵的转置:$E_i^\mathrm{T}(c)=E_i(c)$,$E_{ij}^\mathrm{T}(c)=E_{ji}(c)$,$E_{ij}^\mathrm{T}=E_{ij}$;
\end{enumerate}

初等矩阵大家非常关心为什么左乘代表行变换,右乘代表列变换.以右乘为例,我们来看矩阵$A$和$B$相乘的任一列结果.我们可以将矩阵$A$
按列做分块矩阵得到$\begin{pmatrix}\alpha_1,\ldots,\alpha_n\end{pmatrix}$,$\alpha_i$即表示$A$的第$i$列.然后矩阵$B$的第$j$列为列向量$(x_1,\ldots,x_n)^\mathrm{T}$,
由于矩阵$A$与$B$相乘结果第$j$列就是$A$与$B$的第$j$列相乘结果(回顾矩阵乘法的计算方式),则有$B$的第$i$列等于
$x_1\alpha_1+\cdots+x_n\alpha_n$即为$A$的全部列向量的线性组合,故右乘矩阵$A$得到矩阵的任一列都是$A$的全部列向量的线性组合,
所以右乘可以代表列变换.注意我这里并没有限制矩阵$B$为初等矩阵或可逆矩阵.

实际上左乘表示行变换可以用类似方法说明,只需按行对$B$分块即可.这一思想是特别重要的,在很多时候如果我们意识到左右乘是对被乘矩阵的行列
重新线性组合,思路会清晰很多.

关于初等矩阵还有一个相当重要的定理:

\begin{theorem}
    任意可逆矩阵都可以被表示为若干个初等矩阵的乘积.
\end{theorem}
定理证明只需要回忆高斯消元法可以将可逆矩阵化为单位矩阵即可.

利用矩阵初等变换我们可以获得本学期需要学习的三个矩阵标准形,因此这一内容虽然很基本但是非常重要:
\begin{enumerate}
    \item 相抵矩阵:本章已学习的内容,在之后会详细说明;
    \item 相似矩阵:若$P$为初等矩阵,对矩阵做$P^{-1}AP$变换即可得到与$A$相似的矩阵;
    \item 相合矩阵:两个矩阵,其中一个可以通过做相同的初等行列变换的到另一个矩阵(若$P$为初等矩阵,
    $P^{\mathrm{T}}AP$就是对$A$做了一次相同的初等行列变换).
\end{enumerate}
请同学们思考:如何从线性映射矩阵表示的角度理解初等变换与标准形的关系?在B组习题中将有练习进行体会
(实际上对矩阵表示的基做``初等变换''就是对表示矩阵做了初等变换,这两种变换行列方向不一致且矩阵互逆).

\section{矩阵的逆}
\subsection{基本概念}
\begin{definition}
    设$A \in \mathbf{M}_n(\mathbf{F})$. 若存在$B \in \mathbf{M}_n(\mathbf{F})$使得$AB=BA=E$,则称矩阵$A$可逆,
    并把$B$称为$A$的逆矩阵,记作 $ B = A^{-1} $.
\end{definition}
注意,逆矩阵定义基于方阵,非方阵没有上述逆矩阵.广义逆矩阵允许非方阵,但那是另一个定义,
我们不需要掌握.对于可逆矩阵,注意以下两个定理:
\begin{theorem}
    可逆矩阵$A$的逆矩阵唯一.
\end{theorem}
\begin{theorem}
    $AB=E \iff A$与$B$互为逆矩阵.
\end{theorem}
这两个定理的证明教材中有,特别注意唯一性的证明,反证法的思路一定要掌握,十分经典.
还需要强调的一点是,逆矩阵来源于逆映射.
\subsection{基本性质}
\begin{enumerate}
    \item 注意没有加法性质(请举出反例),对于数乘有$(\lambda A)^{-1}=\lambda^{-1}A^{-1}$;

    \item $(AB)^{-1}=B^{-1}A^{-1},\enspace (A_1A_2\cdots A_k)^{-1}=A_k^{-1}\cdots A_2^{-1}A_1^{-1}$;

    \item $(A^k)^{-1}=(A^{-1})^k,\enspace A^kA^m=A^{k+m},\enspace (A^k)^m=A^{km}$;

    \item 若$A$和$B$可逆,则$A\neq O$且$B\neq O$能推出$AB\neq O$,并且$A$可逆且$AB=O$可以推出$B=O$,除此之外还有消去律成立,即$A$则有$AB=AC \implies B=C$成立.
\end{enumerate}

还需要熟练掌握可逆矩阵的几个等价条件:
\begin{theorem}
    设$A \in \mathbf{M}_n{\mathbf{F}}$,则下列命题等价:
    \begin{enumerate}
        \item $A$可逆;

        \item $r(A)=n$;

        \item $A$的$n$个行(列)向量线性无关;

        \item 齐次线性方程组$AX=0$只有零解;

        \item $|A|\neq 0$.
    \end{enumerate}
\end{theorem}
\begin{example}
    已知矩阵 $A=\begin{pmatrix}a & b & c \\ d & e & f \\ h & x & y\end{pmatrix}$ 的逆是 $A^{-1}=\begin{pmatrix}-1 & -2 & -1 \\ 2 & 1 & 0 \\ 0 & -3 & -1\end{pmatrix}$,

$B=\begin{pmatrix}a-2b & b-3c & -c \\ d-2e & e-3f & -f \\ h-2x & x-3y & -y\end{pmatrix}$.求矩阵 $X$ 满足:

\[X+\left(B(A^TB^2)^{-1}A^T\right)^{-1}=X\left(A^2(B^TA)^{-1}B^T\right)^{-1}(A+B)\]
\end{example}

\subsection{逆矩阵的求解(基本方法)}
\begin{enumerate}
    \item 利用解线性方程组的方法:假设$AX=b$,使用高斯消元法求解;

    \item 利用初等矩阵的方法(初等行变换为常用方法).
\end{enumerate}

注意,基于初等变换的方法是非常重要的,我们很多时候不要被题目吓到去采用其他
偏门的方法,实际上很多时候拿到一个具体的矩阵求逆,使用的方法就是初等行变换.

\begin{example}
    用上述两种方法求矩阵$A=\begin{pmatrix}1 & -1 & 1 \\ 0 & 1 & 2 \\ 1 & 0 & 4\end{pmatrix}$的逆矩阵.
\end{example}

\subsection{矩阵方程}
\begin{enumerate}
    \item 考虑以下情形(其中出现的矩阵除$X$外均可逆,$X$不一定是列向量):
    \begin{enumerate}[label=(\arabic*)]
        \item $AX=B \implies X=A^{-1}B, \enspace XA=B \implies X=BA^{-1}$;
        \item $AXB=C \implies X=A^{-1}CB^{-1}$;
    \end{enumerate}
    \item 考虑以下情形:$AX=B$但$A$不可逆($X$不一定是列向量),直接高斯消元即可;
    \item 考虑以下求解方式的合理性:
    \begin{enumerate}[label=(\arabic*)]
        \item 若求$A^{-1}$,只需对$(A,E)$只做初等行变换,可以得到$(E,A^{-1})$;
        \item 若求$A^{-1}B$,只需对$(A,B)$只做初等行变换,可以得到$(E,A^{-1}B)$;
        \item 若求$BA^{-1}$,只需对$\begin{pmatrix}
            A \\ B
        \end{pmatrix}$只做初等列变换,可以得到$\begin{pmatrix}
            E \\ BA^{-1}
        \end{pmatrix}$;
        \item 对$\begin{pmatrix}
            A & E \\ E & O
        \end{pmatrix}$的前$n$行与$n$列做相同的行列变换,可以得到$\begin{pmatrix}
            P^\mathrm{T}AP & P^\mathrm{T} \\ P & O
        \end{pmatrix}$.
    \end{enumerate}
\end{enumerate}

\begin{example}
    设$A=\begin{pmatrix}1 & 0 & 0 \\ 1 & 1 & 0 \\ 1 & 1 & 1\end{pmatrix},\
    B=\begin{pmatrix}0 & 1 & 1 \\ 1 & 0 & 1 \\ 1 & 1 & 0\end{pmatrix}$,求矩阵$X$满足:
    \[AXA+BXB=AXB+BXA+A(A-B)\]
\end{example}

\subsection{一些相似的定理}
\begin{theorem} \label{thm:6:线性映射对向量坐标的影响}
    \textbf{\heiti 线性映射对向量坐标的影响}

    设$\sigma \in \mathcal{L}(V_1,V_2)$关于$V_1$和$V_2$的基$B_1$和基$B_2$的矩阵为$A=(a_{ij})_{m \times n}$,
    且$\alpha$与$\sigma(\alpha)$在基$B_1$和基$B_2$下的坐标分别为$X$和$Y$,则$Y=AX$.
\end{theorem}
上述即教材定理4.1,这一定理给出一个向量经过线性映射之后,其坐标的变化. 我们可以用下图表示:

\begin{figure}[h]
    \centering
    \begin{tikzpicture}[>=Stealth]
        \node (V) at (0,0) {$V$};
        \node (W) at (3,0) {$W$};
        \node (Fn) at (0,-3) {$\mathbf{F}^n$};
        \node (Fm) at (3,-3) {$\mathbf{F}^m$};
        \draw[->,thick] (V) -- node[below]{表示矩阵:$A$} (W);
        \draw[<->,thick] (V) -- node[right]{同构} (Fn);
        \draw[<->,thick] (W) -- node[left]{同构} (Fm);
        \draw[->,thick] (Fn) -- node[above]{$\sigma(\alpha)=A\alpha$} (Fm);
    \end{tikzpicture}
\end{figure}

图中我们可以看出通过坐标映射后得到的新映射即为\autoref{thm:6:线性映射对向量坐标的影响} 描述的映射.

在描述下一定理之前,我们首先介绍过渡矩阵(变换矩阵)的概念.
\begin{definition}
    设$B_1=\{\alpha_1,\alpha_2,\ldots,\alpha_n\}$与$B_2=\{\beta_1,\beta_2,\ldots,\beta_n\}$是线性空间
    $V(\mathbf{F})$的任意两组基,$B_2$中每个基向量被基$B_1$表示为
    \[ \left\{
    \begin{array}{rcl}
        \beta_1&=&a_{11}\alpha_1+a_{21}\alpha_2+\cdots+a_{n1}\alpha_n \\
        \beta_2&=&a_{12}\alpha_1+a_{22}\alpha_2+\cdots+a_{n2}\alpha_n \\
        &\vdots& \\
        \beta_n&=&a_{1n}\alpha_1+a_{2n}\alpha_2+\cdots+a_{nn}\alpha_n
    \end{array}
    \right. \]
    将上式用矩阵表示为
    \[(\beta_1,\beta_2,\cdots,\beta_n)=(\alpha_1,\alpha_2,\cdots,\alpha_n)\begin{pmatrix}
        a_{11} & a_{12} & \cdots & a_{1n} \\
        a_{21} & a_{22} & \cdots & a_{2n} \\
        \vdots & \vdots & \ddots & \vdots \\
        a_{n1} & a_{n2} & \cdots & a_{nn}
    \end{pmatrix}\]
    我们将这一矩阵称为即$B_1$变为基$B_2$的变换矩阵(或过渡矩阵).
\end{definition}
简单而言就是将$B_2$中的向量在$B_1$下的坐标按列排列.需要注意表述中是$B_1$变为基$B_2$还是反过来,
这两个矩阵互逆.注意过渡矩阵一定是基与基之间的表示矩阵,并且过渡矩阵一定可逆.
\begin{theorem}
    \textbf{\heiti 基的选择对向量坐标的影响}

    设线性空间$V$的两组基为$B_1$和$B_2$,且基$B_1$到$B_2$的变换矩阵(过渡矩阵)为$A$,如果
    $\xi \in V(\mathbf{F})$,且在$B_1$和$B_2$下的坐标分别为$X$和$Y$,则$Y=A^{-1}X$.
\end{theorem}
上述即教材定理4.10,描述同一个向量在不同基下坐标之间的关系.事实上,这与本节同构关系紧密,因为
同构意味着两个线性空间结构一致,故同构映射可以保持向量组的线性关系不变.在同构关系下,
线性组合对应线性组合,线性无关对应线性无关,线性相关对应线性相关.我们有如下定理:
\begin{theorem}
    设$(\alpha_1,\alpha_2,\ldots,\alpha_n)$是线性无关的向量组,且
    \[(\beta_1,\beta_2,\ldots,\beta_s)=(\alpha_1,\alpha_2,\ldots,\alpha_n)A\]
    则向量组$(\beta_1,\beta_2,\ldots,\beta_s)$的秩等于矩阵$A$的秩.
\end{theorem}
定理的证明需要用到坐标映射是同构映射这一事实,我们不难发现等式左侧向量组与$A$的列向量组是等价的.
事实上我们也可以由此发现,过渡矩阵一定是可逆矩阵.
\begin{theorem}
    已知$\beta_i=a_{1i}\alpha_1+a_{2i}\alpha_2+\cdots+a_{ni}\alpha_n\enspace(i=1,2,\ldots,n)$,
    且$A=(a_{ij})$可逆,则$\alpha_1,\alpha_2,\ldots,\alpha_n$与$\beta_1,\beta_2,\ldots,\beta_n$
    等价.
\end{theorem}
实际上这一定理与上一定理的思想都是类似的,我们可以看一个例题练习一下:
\begin{example}
    已知$\beta_1=\alpha_2+\alpha_3,\enspace\beta_2=\alpha_1+\alpha_3,\enspace\beta_3=\alpha_1+\alpha_2$,
    证明$\alpha_1,\alpha_2,\alpha_3$与$\beta_1,\beta_2,\beta_3$等价.
\end{example}
\begin{theorem}
    \textbf{\heiti 基的选择对映射矩阵的影响}

    设线性变换$\sigma \in \mathcal{L}(V,V)$,$B_1=\{\alpha_1,\ldots,\alpha_n\}$和$B_2=\{\beta_1,\ldots,\beta_n\}$
    是线性空间的$V(\mathbf{F})$的两组基,基$B_1$变为基$B_2$的变换矩阵为$C$,如果$\sigma$在基$B_1$下的矩阵为$A$,
    则$\sigma$关于基$B_2$所对应的矩阵为$C^{-1}AC$.
\end{theorem}
上述即教材定理7.4,研究同一个映射在不同基下表示矩阵之间的关系.实际上我们将在下一专题初等矩阵一节进一步讨论.
这一定理的证明需要用到我们之前描述的两种线性映射矩阵表示的统一性.

\vspace{2ex}
\centerline{\heiti \Large 内容总结}

\vspace{2ex}

\centerline{\heiti \Large 习题}
\vspace{2ex}
{\kaishu }
\begin{flushright}
    \kaishu

\end{flushright}
\centerline{\heiti A组}
\begin{enumerate}
    \item
\end{enumerate}
\centerline{\heiti B组}
\begin{enumerate}
    \item
\end{enumerate}
\centerline{\heiti C组}
\begin{enumerate}
    \item
\end{enumerate}
