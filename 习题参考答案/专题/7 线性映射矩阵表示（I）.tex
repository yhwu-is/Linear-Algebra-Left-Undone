\section*{7 线性映射矩阵表示(I)}
\addcontentsline{toc}{section}{7 线性映射矩阵表示(I)}

\vspace{2ex}

\centerline{\heiti A组}
\begin{enumerate}
    \item 教材 P154/9,此处略.
    \item 教材 P154/10,此处略.
\end{enumerate}

\centerline{\heiti B组}
\begin{enumerate}
    \item $T(\beta_1,\beta_2,\cdots,\beta_n)=(\beta_1,\beta_2,\cdots,\beta_n)\begin{pmatrix}0 & 0 & 0 & \cdots & 0 & a_1 \\ 1 & 0 & 0 & \cdots & 0 & a_2 \\ 0 & 1 & 0 & \cdots & 0 & a_3 \\ \vdots & \vdots & \vdots & & \vdots & \vdots \\ 0 & 0 & 0 & \cdots & 1 & a_n\end{pmatrix}$
    所以 $A=\begin{pmatrix}0 & 0 & 0 & \cdots & 0 & a_1 \\ 1 & 0 & 0 & \cdots & 0 & a_2 \\ 0 & 1 & 0 & \cdots & 0 & a_3 \\ \vdots & \vdots & \vdots & & \vdots & \vdots \\ 0 & 0 & 0 & \cdots & 1 & a_n\end{pmatrix}$ 是 $T$ 关于基 $B$ 的表示矩阵.

    $T$ 是同构 $\Leftrightarrow$ $T$ 是双射 $\Leftrightarrow$ $r(T) = n$,所以 $A$ 满秩即 $a_1\neq 0$ 时 $T$ 是同构映射.
    \item \begin{enumerate}
        \item 略.
        \item 设 $(\sigma(f_1),\sigma(f_2),\sigma(f_3))=(f_1,f_2,f_3)A$,可用待定系数法解得 $A=\begin{pmatrix}-1 & -2 & -2 \\ 3 & 2 & 3 \\ -1 & -1 & -1\end{pmatrix}$.
        \item 用待定系数求出 $f=-2f_1+3f_2$,故 $\sigma(f)=-2\sigma(f_1)+3\sigma(f_2)=-4+3x+2x^2$.
    \end{enumerate}
    \item \begin{enumerate}
        \item 略.
        \item $\lambda \neq -1$ 时,$A,B$ 均可逆,故 $\varphi^{-1}(X)=A^{-1}XB^{-1}$.
        \item 取 $V$ 的一组基 $\begin{pmatrix}1 & 0 \\ 0 & 0\end{pmatrix},\begin{pmatrix}0 & 1 \\ 0 & 0\end{pmatrix},\begin{pmatrix}0 & 0 \\ 1 & 0\end{pmatrix},\begin{pmatrix}0 & 0 \\ 0 & 1\end{pmatrix}$,有
        \[\mathrm{Im}\sigma = \mathrm{span}(\varphi(\alpha_1),\varphi(\alpha_2),\varphi(\alpha_3),\varphi(\alpha_4))=\mathrm{span}(\begin{pmatrix}1 & 2 \\ -1 & -2\end{pmatrix},\begin{pmatrix}-1 & -1 \\ 1 & 1\end{pmatrix})\]
        \[\mathrm{Ker}\sigma = \mathrm{span}(\begin{pmatrix}2 & -3 \\ 0 & 1\end{pmatrix},\begin{pmatrix}1 & 0 \\ 1 & 0\end{pmatrix})\]
        (省略步骤,答案不唯一)
        \item 取 $\begin{pmatrix}1 & 2 \\ -1 & -2\end{pmatrix},\begin{pmatrix}-1 & -1 \\ 1 & 1\end{pmatrix},\begin{pmatrix}0 & 0 \\ 1 & 0\end{pmatrix},\begin{pmatrix}0 & 0 \\ 0 & 1\end{pmatrix}$ 即可.
        
        此时矩阵为 $\begin{pmatrix}2 & -2 & -1 & 0 \\ 4 & -2 & 0 & -1 \\ 0 & 0 & 0 & 0 \\ 0 & 0 & 0 & 0\end{pmatrix}$(答案不唯一).
    \end{enumerate}
    \item 求基的过程与求 $V=\{X\in \mathbf{R}^4\ |\ x_1+x_2+x_3=0\}$ 类似,求 $V$ 的基只需求解方程组 $x_1+x_2+x_3=0$ 即可,得到基础解系 $\begin{pmatrix}-1 \\ 0 \\ 1 \\ 0\end{pmatrix},\begin{pmatrix}-1 \\ 1 \\ 0 \\ 0\end{pmatrix},\begin{pmatrix}0 \\ 0 \\ 0 \\ 1\end{pmatrix}$.
    
    换回本题,有基为 $A_1=\begin{pmatrix}-1 & 0 \\ 1 & 0\end{pmatrix},A_2=\begin{pmatrix}-1 & 1 \\ 0 & 0\end{pmatrix},A_3=\begin{pmatrix}0 & 0 \\ 0 & 1\end{pmatrix}$.
    而 $\sigma(A_1)=A_1+A_2,\sigma(A_2)=A_1+A_2,\sigma(A_3)=2A_3$,可得 $\sigma(A_1+A_2)=2(A_1+A_2),\sigma(A_1-A_2)=0,\sigma(A_3)=2A_3$.

    所以取基 $\{A_1-A_2,A_1+A_2,A_3\}$ 有
    \[(\sigma(A_1-A_2),\sigma(A_1+A_2),\sigma(A_3))=(A_1-A_2,A_1+A_2,A_3)\begin{pmatrix}0 & 0 & 0 \\ 0 & 2 & 0 \\ 0 & 0 & 2\end{pmatrix}\]
    为对角矩阵.
    \item \begin{enumerate}
        \item 求核空间:设 $f(x)=ax^3+bx^2+cx+d$,由 $f(-1)=f(0)=f(1)=0$ 解得 $f(x)=a(x^3-x)$. 故 $N(T)=\mathrm{span}(x^3-x)$.
        
        求像空间:取常用基 $\{1,x,x^2,x^3\}$,我们要求 $T(1),T(x),T(x^2),T(x^3)$ 的极大线性无关组,我们发现 $T(x)=T(x^3)$,故先舍弃 $T(x^3)$,然后令 $k_1T(1)+k_2T(x)+k_3T(x^2)=0$,可解得 $k_1=k_2=k_3=0$,故 $T(1),T(x),T(x^2)$ 线性无关,故 $R(T)=\mathrm{span}(\begin{pmatrix}1 & 1 \\ 1 & 1\end{pmatrix},\begin{pmatrix}0 & 1 \\ -1 & 0\end{pmatrix},\begin{pmatrix}0 & 1 \\ 1 & 0\end{pmatrix})$.
        \item 由上一问有 $\mathrm{dim}N(T)=1,\mathrm{dim}R(T)=3$,又有 $\mathrm{dim}\mathbf{R}[x]_4=4$. 则维数公式成立.
    \end{enumerate}
    \item
        \begin{enumerate}
            \item
                对非齐次线性方程组$AX=\xi_1$,\\
                $\bar{A}=\begin{pmatrix}
                    1 & -1 & -1 & -1 \\
                    -1 & 1 & 1 & 1 \\
                    0 & -4 & -2 & -2
                \end{pmatrix}
                \rightarrow\begin{pmatrix}
                    1 & -1 & -1 & -1 \\
                    0 & 1 & \frac 12 & \frac 12 \\
                    0 & 0 & 0 & 0
                \end{pmatrix}
                \rightarrow\begin{pmatrix}
                    1 & 0 & -\frac 12 & -\frac 12 \\
                    0 & 1 & \frac 12 & \frac 12 \\
                    0 & 0 & 0 & 0
                \end{pmatrix}$,则\\
                $\xi_2=C_1\begin{pmatrix}
                    \frac 12 \\
                    -\frac 12 \\
                    1
                \end{pmatrix} + \begin{pmatrix}
                    -\frac 12 \\
                    \frac 12 \\
                    0
                \end{pmatrix}=\frac 12\begin{pmatrix}
                    C_1 - 1 \\
                    -C_1 + 1 \\
                    2C_1
                \end{pmatrix}$(其中$C_1$为任意常数).\\
                $A^2=\begin{pmatrix}
                    2 & 2 & 0 \\
                    -2 & -2 & 0 \\
                    4 & 4 & 0
                \end{pmatrix}$,对齐次线性方程组$A^2X=\xi_1$,\\
                $\bar{B}=\begin{pmatrix}
                    A^2 & \xi_1
                \end{pmatrix}=\begin{pmatrix}
                    2 & 2 & 0 & 1 \\
                    -2 & -2 & 0 & 1 \\
                    4 & 4 & 0 & -2
                \end{pmatrix}
                \rightarrow\begin{pmatrix}
                    1 & 1 & 0 & -\frac 12 \\
                    0 & 0 & 0 & 0 \\
                    0 & 0 & 0 & 0
                \end{pmatrix}$,\\
                则$A^2X=\xi_1$的通解\\
                $\xi_3=C_2\begin{pmatrix}
                    -1 \\
                    1 \\
                    0
                \end{pmatrix}+C_3\begin{pmatrix}
                    0 \\
                    0 \\
                    1
                \end{pmatrix}+\begin{pmatrix}
                    -\frac 12 \\
                    0 \\
                    0
                \end{pmatrix}=\begin{pmatrix}
                    -C_2 - \frac 12 \\
                    C_2 \\
                    C_3
                \end{pmatrix}$(其中$C_2, C_3$为任意常数).
            \item
                因为$|\xi_1,\xi_2,\xi_3|=\frac 12\begin{vmatrix}
                    -1 & C_1 - 1 & -C_2 - \frac 12 \\
                    1 & -C_1 + 1 & C_2 \\
                    -2 & 2C_1 & C_3
                \end{vmatrix}=-\frac 12\neq 0$,\\
                所以$\xi_1,\xi_2,\xi_3$线性无关.
        \end{enumerate}
    \item \begin{enumerate}
        \item 对新的一组基,使用过渡矩阵进行表达如下:
        \[(\beta_{1}, \beta_{2}, \beta_{3})=(\alpha_{1}, \alpha_{2}, \alpha_{3})\begin{pmatrix}
        2 & 1 & -1 \\
        1 & 1 & 1 \\
        3 & 2 & 1    
        \end{pmatrix}=(\alpha_{1}, \alpha_{2}, \alpha_{3}) C\]
        其中 $C$ 是可逆矩阵,且
        \[(\alpha_{1},\ \alpha_{2},\ \alpha_{3})=(\beta_{1},\ \beta_{2},\ \beta_{3}) C^{-1}\]
        将上式代入已知条件得
        \[\sigma\left(\left(\beta_{1},\ \beta_{2},\ \beta_{3}\right) C^{-1}\right)=\left(\left(\beta_{1},\ \beta_{2},\ \beta_{3}\right) C^{-1}\right) A\]
        容易验证(只需利用线性变换和矩阵的等价性然后利用矩阵乘法结合律即可)上式左端等于 $(\sigma(\beta_{1}, \beta_{2}, \beta_{3})) C^{-1}$,所以
        \[(\sigma(\beta_{1},\ \beta_{2},\ \beta_{3})) C^{-1}=(\beta_{1},\ \beta_{2},\ \beta_{3})(C^{-1} A)\]
        从而得 $\sigma(\beta_{1},\ \beta_{2},\ \beta_{3})=(\beta_{1},\ \beta_{2},\ \beta_{3})(C^{-1} A C)$,故 $\sigma$ 关于基 $\{\beta_{1},\ \beta_{2},\ \beta_{3}\}$ 下对应的矩阵 $B=C^{-1} A C=\begin{pmatrix}2 & 0 & 1 \\ 0 & 2 & 1 \\ 3 & 1 & -1\end{pmatrix}$.
        \item $\sigma$ 的值域是 $A$ 列向量组的极大线性无关组,由于 $A $ 的第 $1$ 列可以由第 $2$ 列和第 $3$ 列线性表示,从而 $\sigma(V)=L(2 \alpha_{1}+\alpha_{2},\ -\alpha_{1}+\alpha_{3})$.$\operatorname{Ker} \sigma$ 是线性方程组 $AX=0$ 的解空间,从而 $\mathrm{Ker} \sigma=\mathrm{span}(\alpha_{1}-2 \alpha_{2}-3 \alpha_{3})$.
        \item 由于 $\alpha_{1}$ 不能由 $2 \alpha_{1}+\alpha_{2}$ 和 $-\alpha_{1}+\alpha_{3}$ 线性表示,可以把 $\sigma(V)$ 的基扩充为 $V$ 的基 $\{\alpha_{1},\ 2 \alpha_{1}+\alpha_{2},\ -\alpha_{1}+\alpha_{3}\}$,$\sigma$ 在这个基下对应的矩阵是 $\begin{pmatrix}0 & 0 & 0 \\ 2 & 5 & -2 \\ 3 & 6 & -2\end{pmatrix}$.
        \item 由于 $\alpha_{1},\ \alpha_{2}$ 不能由 $\alpha_{1}-2 \alpha_{2}-3 \alpha_{3}$ 线性表示,可以把 $\mathrm{Ker} \sigma$ 的基扩充为 $V$ 的基 $\{\alpha_{1},\ \alpha_{2},\ \alpha_{1}-2 \alpha_{2}-3 \alpha_{3}\}$,$\sigma$ 在这个基下对应的矩阵是 $\begin{pmatrix}2 & 2 & 0 \\ 0 & 1 & 0 \\ 1 & 0 & 0\end{pmatrix}$.
    \end{enumerate}
    \item \begin{enumerate}
        \item 因为 $A^k=\begin{pmatrix}\lambda_1^k & 0 \\ 0 & \lambda_2^k\end{pmatrix}$,所以 $f(A)=a_mA^m+a_{m-1}A^{m-1}+\cdots+a_1A+a_0E = \begin{pmatrix}f(\lambda_1) & 0 \\ 0 & f(\lambda_2)\end{pmatrix}$.
        \item $A=PBP^{-1}$,则 $A^2=(PBP^{-1})(PBP^{-1})=PB^2P^{-1}$. 由归纳法得 $A^k=PB^kP^{-1}$,于是
        \[f(A)=a_mPB^mP^{-1}+a_{m-1}PB^{m-1}P^{-1}+\cdots+a_0=P\begin{pmatrix}f(\lambda_1) & 0 \\ 0 & f(\lambda_2)\end{pmatrix}P^{-1}=Pf(B)P^{-1}\]
    \end{enumerate}
\end{enumerate}

\centerline{\heiti C组}
\begin{enumerate}
    \item
\end{enumerate}

\clearpage
