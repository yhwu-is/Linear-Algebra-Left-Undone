\chapter{朝花夕拾} \label{chap:朝花夕拾}

我为这一讲取了一个很有诗意的名字,用以说明这一节我们重在对往日所学知识的回忆. 我们一路走来为了能进入这一讲做了太多准备工作,包括一开始难以理解的抽象空间和映射,以及后面具象但充满技巧性的矩阵与行列式. 但``吹尽狂沙始到金'',这一讲希望读者跟随我们的脚步,回忆起这一路上学习的核心概念和定理,为我们线性方程组一般理论的讨论画下一个完美的句点.

在解的一般理论中,我们将首先讨论有无解以及有解时唯一解和无穷解对应的情况,然后介绍了方程系数矩阵为方阵时用行列式表达的一般理论,即著名的 Cramer 法则. 然后分别讨论齐次与非齐次线性方程组解的结构分别具有什么特征,并且从对偶空间的角度给出了线性方程组解的几何解释. 接下来,我们将利用一般理论讨论一些秩有关的等式和不等式,也将讨论线性方程组中一些特殊的题型. 愿读者在每一个定理的证明和每一个例题的解答中,都能进一步体会之前所学的知识,加深理解,有所感悟.

\section{线性方程组解的一般理论}

\subsection{线性方程组解的一般理论}

\begin{theorem}{线性方程组有解的充要条件}{有解条件}
    线性方程组有解的充分必要条件是其系数矩阵与增广矩阵有相同的秩.
\end{theorem}
定理的证明非常简单,这里简要介绍思路:将方程组视为$x_1\beta_1+x_2\beta_2+\cdots+x_n\beta_n=\vec{b}$($\beta_i$就是系数矩阵的第$i$列),则有解的条件为$\vec{b}$可以被$\beta_1,\ldots,\beta_n$线性表示,这等价于向量组$(\beta_1,\ldots,\beta_n)$与$(\beta_1,\ldots,\beta_n,\vec{b})$等价,故定理成立. 下面是一个应用的例子:

\begin{example}{}{}
    设$n$阶矩阵$A$的行列式$|A|\neq 0$,记$A$的前$n-1$列形成的矩阵为$A_1$,$A$的第$n$列为$\vec{b}$,问:线性方程组$A_1X=\vec{b}$是否有解?
\end{example}
\begin{solution}
    无解;$|A|\neq 0$可知$r(A)=n$,$r(A_1)=n-1$,因此系数矩阵$A_1$的秩小于增广矩阵$A$的秩$n$,故无解.
\end{solution}

下面这一定理讨论方程组有解的情况下,唯一解和无穷解的条件,根据高斯消元法这一定理是很显然的:
\begin{theorem}{}{方程组解}
    当方程组有解时(注意这个前提),以下定理成立:
    \begin{enumerate}
        \item 如果它的系数矩阵$A$的秩等于未知量的数目$n$,则方程组有唯一解;

        \item 如果$A$的秩小于$n$,则方程组有无穷多个解.
    \end{enumerate}
\end{theorem}

当系数矩阵为方阵时,这一定理就是著名的\nameref{thm:Cramer}结论的一部分. 从历史角度来开,引入行列式是用于求解线性方程组的. 瑞士数学家克莱姆(Cramer)于 1750 年在他的《线性代数分析导言》中发表了这一方法. 事实上莱布尼兹〔1693〕,以及麦克劳林〔1748〕亦研究了这一法则,但他们的记法不如克莱姆清晰. 接下来我们介绍这一充满历史底蕴的定理:
\begin{theorem}{Cramer法则}{Cramer} \index{Cramer@Cramer 法则 (Cramer's rule)}
    对线性方程组
    \begin{gather}
        \label{eq:13:线性方程组1}
        \begin{cases} \begin{aligned}
                a_{11}x_1+a_{12}x_2+\cdots+a_{1n}x_n & = 0             \\
                a_{21}x_1+a_{22}x_2+\cdots+a_{2n}x_n & = 0             \\
                                                     & \vdotswithin{=} \\
                a_{n1}x_1+a_{n2}x_2+\cdots+a_{nn}x_n & = 0
            \end{aligned} \end{cases}
        \\
        \label{eq:13:线性方程组2}
        \begin{cases} \begin{aligned}
                a_{11}x_1+a_{12}x_2+\cdots+a_{1n}x_n & = b_1           \\
                a_{21}x_1+a_{22}x_2+\cdots+a_{2n}x_n & = b_2           \\
                                                     & \vdotswithin{=} \\
                a_{n1}x_1+a_{n2}x_2+\cdots+a_{nn}x_n & = b_n
            \end{aligned} \end{cases}
    \end{gather}

    令$D=\begin{vmatrix}
            a_{11} & a_{12} & \cdots & a_{1n} \\
            a_{21} & a_{22} & \cdots & a_{2n} \\
            \vdots & \vdots & \ddots & \vdots \\
            a_{n1} & a_{n2} & \cdots & a_{nn}
        \end{vmatrix}$,称为系数行列式.

    令$D_1=\begin{vmatrix}
            b_1    & a_{12} & \cdots & a_{1n} \\
            b_2    & a_{22} & \cdots & a_{2n} \\
            \vdots & \vdots & \ddots & \vdots \\
            b_n    & a_{n2} & \cdots & a_{nn}
        \end{vmatrix},\ldots,D_n=\begin{vmatrix}
            a_{11} & a_{12} & \cdots & b_1    \\
            a_{21} & a_{22} & \cdots & b_2    \\
            \vdots & \vdots & \ddots & \vdots \\
            a_{n1} & a_{n2} & \cdots & b_n
        \end{vmatrix}$.

    \begin{enumerate}
        \item 方程组 \ref{eq:13:线性方程组1} 只有零解$\iff D \neq 0$,\\
              方程组 \ref{eq:13:线性方程组1} 有非零解(无穷多解)$\iff D=0$,即$r(A)<n$;

        \item 方程组 \ref{eq:13:线性方程组2} 有唯一解$\iff D \neq 0$,此时$x_i=\dfrac{D_i}{D}\enspace(i=1,2,\ldots,n)$,\\
              当$D=0$时,方程组 \ref{eq:13:线性方程组2} 要么无解,要么有无穷多解.
    \end{enumerate}
\end{theorem}

不难看出,Cramer 法则不仅是一般理论在系数矩阵下的特殊情况,也给出了有解时如何求解的方法.

\begin{proof}
    我们不区分齐次与非齐次方程组进行证明,实际上齐次的结论只是下面证明的特例. 事实上,对于任意线性方程组$AX=b$,其中$b$可以是零向量,若$D=|A|\neq 0$,则$A$可逆,因此方程组有解
    \[X=A^{-1}b=\frac{1}{|A|}A^*b=\frac{1}{D}A^*b,\]
    即
    \[\begin{pmatrix}
            x_1 \\ x_2 \\ \vdots \\ x_n
        \end{pmatrix}= \dfrac{1}{D}\begin{pmatrix}
            A_{11} & A_{21} & \cdots & A_{n1} \\
            A_{12} & A_{22} & \cdots & A_{n2} \\
            \vdots & \vdots & \ddots & \vdots \\
            A_{1n} & A_{2n} & \cdots & A_{nn}
        \end{pmatrix}\begin{pmatrix}
            b_1 \\ b_2 \\ \vdots \\ b_n
        \end{pmatrix},\]
    于是根据\autoref{def:递归式定义},
    \[x_i=\frac{1}{D}(b_1A_{1i}+b_2A_{2i}+\cdots+b_nA_{ni})=\frac{D_i}{D}\enspace(i=1,2,\ldots,n)\]
    事实上此时解唯一,因为可逆矩阵$A$应当是可逆线性映射$\sigma$关于某组基的表示矩阵. 对于可逆映射而言,首先必须是单射,因此$\sigma(a)=b$只能有唯一解,因此$AX=b$只能有唯一解.

    反之,若方程组$AX=b$只有唯一解,说明方阵$A$对应的线性映射$\sigma$是单射,并且因为$A$是方阵,因此$\sigma$出发空间与到达空间维数相同,由\autoref{thm:双射等价条件} 可知$\sigma$是双射,因此$A$可逆,从而$|A|\neq 0$. 综上可以证明方程组$AX=b$有唯一解$\iff D=|A|\neq 0$.

    剩下的无解、无穷解的结论就很显然了. 若$AX=b$无解或有无穷解,反证法,若$D\neq 0$,则$A$可逆,从而$AX=b$有唯一解,矛盾,故$D=0$. 反之,若$D=0$,则$A$不可逆,反证法,若$AX=b$有唯一解,$A$可逆,矛盾,故$AX=b$无解或有无穷解(完全就是前面$AX=b$有唯一解$\iff D=|A|\neq 0$的推论).
\end{proof}

我们可以用 Cramer 法则求解线性方程组,但要注意只有方程个数与未知数个数相等时才能使用,并且需要系数行列式不为0.
\begin{example}{}{}
    求解方程组$\begin{cases}
            x_1+x_2+x_3=1          \\
            a_1x_1+a_2x_2+a_3x_3=0 \\
            a_1^2x_1+a_2^2x_2+a_3^2x_3=0
        \end{cases}$,其中$a_1,a_2,a_3$两两不等.
\end{example}

\begin{solution}
    $a_1,a_2,a_3$两两不等时,我们有
    \[D=\begin{vmatrix}
            1     & 1     & 1     \\
            a_1   & a_2   & a_3   \\
            a_1^2 & a_2^2 & a_3^2
        \end{vmatrix}=(a_2-a_1)(a_3-a_1)(a_3-a_2)\neq 0,\]
    根据Cramer法则,方程组有唯一解,且
    \[D_1=\begin{vmatrix}
            1 & 1     & 1     \\
            0 & a_2   & a_3   \\
            0 & a_2^2 & a_3^2
        \end{vmatrix}=(a_3-a_2)a_2a_3,\]
    \[D_2=\begin{vmatrix}
            1     & 1 & 1     \\
            a_1   & 0 & a_3   \\
            a_1^2 & 0 & a_3^2
        \end{vmatrix}=(a_1-a_3)a_1a_3,\]
    \[D_3=\begin{vmatrix}
            1     & 1     & 1 \\
            a_1   & a_2   & 0 \\
            a_1^2 & a_2^2 & 0
        \end{vmatrix}=(a_2-a_1)a_1a_2,\]
    因此
    \[x_1=\dfrac{D_1}{D}=\dfrac{a_2a_3}{(a_2-a_1)(a_3-a_1)},\]\[x_2=\dfrac{D_2}{D}=\dfrac{a_1a_3}{(a_1-a_2)(a_3-a_2)},\]\[x_3=\dfrac{D_3}{D}=\dfrac{a_1a_2}{(a_1-a_3)(a_2-a_3)}.\]
\end{solution}

因此通过上述定理我们首先了解了线性方程组有无解的一般准则,然后讨论了有解前提下唯一解、无穷解对应于什么情况,并且给出了方阵情况下的情况,以及有解时的求解的方法. 事实上,有关线性方程组解的情况的讨论至此文意已尽. 无论是理论层面或是解决题目的方面,上述定理都为我们提供了足量的信息.

下面这一例子非常重要,一方面是 Cramer 法则的应用,更重要的是我们将引入``对角占优矩阵'',这是一个在现实应用中非常常见的矩阵类型.
\begin{example}{对角占优}{对角占优}
    设 $A = (a_{ij})_{n \times n}$ 是一个 $n$ 阶矩阵,若 $|a_{ii}| > \displaystyle\sum_{j \neq i}|a_{ij}|$(即满足对角占优条件),证明:$|A|\neq 0$.
\end{example}
故这一例子告诉我们,对角占优矩阵一定可逆.
\begin{proof}
    采用反证法证明. 设 $\lvert A \rvert = 0$,则线性方程组 $AX = 0$ 有非零解,设为 $X_0 = (x_1, x_2, \ldots, x_n)^{\mathrm{T}}$,记
    \[\lvert x_k \rvert = \max \{\lvert x_1 \rvert, \lvert x_2 \rvert, \ldots, \lvert x_n \rvert\}.\]
    由 $X_0 \neq 0$ 可知 $\lvert x_k \rvert > 0$,考虑 $AX = 0$ 的第 $k$ 个方程,有 $\displaystyle\sum_{j=1}^n a_{kj}x_j = 0$,于是
    \[\lvert a_{kk} \rvert \lvert x_k \rvert = \lvert -\displaystyle\sum_{j \neq k}a_{kj}x_j \rvert \leqslant \sum_{j \neq k}\lvert a_{kj} \rvert \lvert x_k \rvert.\]
    这里同时使用了三角不等式和 $x_k$ 的定义. 约去 $\lvert x_k \rvert$ 后可得 $\lvert a_{kk} \rvert \leqslant \displaystyle\sum_{j \neq k} \lvert a_{kj} \rvert$,这与条件矛盾. 所以 $\lvert A \rvert \neq 0$.
\end{proof}

\subsection{齐次线性方程组解的一般理论}

接下来我们将分别针对齐次和非齐次线性方程组的情况展开关于解的结构性质的讨论. 回顾\autoref{ex:常见子空间} 中的讨论,对于齐次线性方程组$AX=0$,我们有:
\begin{theorem}{}{}
    齐次线性方程组$AX=\vec{0}$的解空间为$\mathbf{R}^n$的子空间.
\end{theorem}
这一结论告诉我们,齐次线性方程组解构成线性空间,这是一个重要的结构性结论,这一线性空间我们通常记为 $N(A)$. 在确认其为线性空间后,我们来研究该线性空间的基本性质,即 $N(A)$ 的基和维数. 在高斯消元法求解线性方程组的最后一步,我们选取了自由未知量,然后将线性方程组的解表达为了如下形式:
\[X = k_1X_1 + k_2X_2 + \cdots + k_{n-r}X_{n-r},\]
其中 $X_1,\ldots,X_{n-r}$ 我们称为线性方程组的基础解系. 现在我们将证明基础解系实际上就是 $N(A)$ 的一组基,并且 $r$ 代表的就是系数矩阵的秩:

\begin{theorem}{}{齐次维数}
    矩阵 $A \in \mathbf{M}_{m \times n}(\mathbf{F})$,若 $r(A) = r$,则该齐次线性方程组 $AX = \vec{0}$ 的解空间维数为 $n - r$,且其一组基就是基础解系 $X_1,\ldots,X_{n-r}$.
\end{theorem}
\begin{proof}
    \begin{enumerate}
        \item 维数部分证明:事实上,本定理的维数部分可以改写为类似于维数公式的形式,即
        \begin{equation}\label{eq:15:齐次维数公式}
            r(A) + \dim N(A) = n.
        \end{equation}
        其中 $N(A)$ 表示 $AX=\vec{0}$ 的解空间,区别在于维数公式中 $A$ 应当替换为线性映射 $\sigma$.

        我们令$A$是线性映射$\sigma$在出发空间和到达空间基下的矩阵表示,根据矩阵的秩的定义,$r(A)=r(\sigma)$;又根据\autoref{eq:7:方程组与核空间2} 的讨论,$\ker\sigma$和$N(A)$之间是坐标的一一对应关系,因此$\dim N(A)=\dim\ker\sigma$. 因此我们有\autoref{eq:15:齐次维数公式} 也成立,证毕.

        \item 基部分证明:对 $A$ 作初等行变换可将其化为有 $r$ 个非零行的行简化阶梯形矩阵,再作一些列对换(实际上就是相应地改变未知元的次序)可使阶梯形矩阵的左上角为 $r$ 阶单位矩阵,为了表述简便,不妨假设
        \[A \xrightarrow{\text{初等变换}}\begin{pmatrix}
            1 & 0 & \cdots & 0 & c_{1,r+1} & \cdots & c_{1n} \\
            0 & 1 & \cdots & 0 & c_{2,r+1} & \cdots & c_{2n} \\
            \vdots & \vdots & \ddots & \vdots & \vdots & \ddots & \vdots \\
            0 & 0 & \cdots & 1 & c_{r,r+1} & \cdots & c_{rn} \\
            0 & 0 & \cdots & 0 & 0 & \cdots & 0 \\
            \vdots & \vdots & \ddots & \vdots & \vdots & \ddots & \vdots \\
            0 & 0 & \cdots & 0 & 0 & \cdots & 0
        \end{pmatrix},\]
    \end{enumerate}
    即我们把所有主元都调换到了前 $r$ 列,那么这一方程组的解和 $AX = \vec{0}$ 的差别只是元素的位置可能发生了改变,并不影响线性相关性. 取后 $n - r$ 个变量为自由未知量,我们可以直接写出变换后的矩阵对应的方程组的解,即
    \[X = k_1X_1 + k_2X_2 + \cdots + k_{n-r}X_{n-r},\]
    其中
    \begin{gather*}
        X_1 = (-c_{1,r+1},-c_{2,r+1},\ldots,-c_{r,r+1},1,0,\ldots,0)^\mathrm{T},\\
        X_2 = (-c_{1,r+2},-c_{2,r+2},\ldots,-c_{r,r+2},0,1,\ldots,0)^\mathrm{T},\\
        \vdots\\
        X_{n-r} = (-c_{1n},-c_{2n},\ldots,-c_{rn},0,\ldots,1)^\mathrm{T}.
    \end{gather*}
    显然 $X_1,\ldots,X_{n-r}$ 是线性无关的,并且向量组长度等于 $N(A)$ 的维数,因此是 $N(A)$ 的一组基,证毕.
\end{proof}

我们可以用\autoref{thm:齐次维数} 解决很多问题,下面是一个最简单的例子:
\begin{example}{}{}
    若$n$元齐次线性方程组$AX = \vec{0}$的解都是$BX = \vec{0}$的解. 证明:$r(B) \leqslant r(A)$.
\end{example}

\begin{proof}
    由\autoref{thm:齐次维数},$r(A)=n-\dim N(A)$,$r(B)=n-\dim N(B)$,由于$AX=\vec{0}$的解都是$BX=\vec{0}$的解,即$N(A) \subset N(B)$,因此$\dim N(A) \leqslant \dim N(B)$,故$r(B) \leqslant r(A)$.
\end{proof}

实际上在前面的讨论中,无论是\nameref{thm:有解条件}的结论,都与列向量组成的线性空间有关,仿佛从未出现过行向量有关的定理. 事实上,我们将在未来讨论了内积空间正交性后展开对行向量空间的讨论,现在囿于概念上的缺乏无法叙述相关定理.

\subsection{非齐次线性方程组解的一般理论}

回顾\autoref{ex:常见子空间} 中的讨论我们发现,非齐次线性方程组的解不构成线性空间,但我们可以尝试将其与齐次线性方程组解空间联系起来研究. 对于非齐次线性方程组
\begin{equation} \label{eq:15:非齐次}
    x_1\beta_1+x_2\beta_2+\cdots+x_n\beta_n=\vec{b}
\end{equation}
我们将$n$元齐次线性方程组
\begin{equation} \label{eq:15:齐次}
    x_1\beta_1+x_2\beta_2+\cdots+x_n\beta_n=\vec{0}
\end{equation}
称为其导出组,则我们有:
\begin{theorem}{}{通解加特解}
    如果$n$元非齐次线性方程组有解,则它的解集$U=\{\gamma_0+\eta \mid \eta \in W\}$.
\end{theorem}
其中$\gamma_0$为\autoref{eq:15:非齐次} 的一个解(称为特解),$W$为\autoref{eq:15:齐次} 的解空间(\autoref*{eq:15:齐次} 的解称为通解). 更具体地,设非齐次线性方程组$AX=b$有特解$X_0$,方程$AX=0$的基础解系为$X_1,\ldots,X_n$,则这一定理告诉我们$AX=b$的任何一个解都可以写为
\[X=X_0+k_1X_1+k_2X_2+\cdots+k_nX_n,\]
的形式. 事实上,对于这种通解+特解的结构,如果读者在此之前学习了商空间一节,那么我们就会发现$U$实际上就是$W$的一个仿射子集. 当然如果没有学习相关概念,我们可以想象一个三元非齐次线性方程$ax + by + cz = d$ 和齐次线性方程$ax + by + cz = 0$. 非齐次线性方程的解显然对应一个不过原点的平面,而齐次则过原点. 我们便可以认为是齐次线性方程解平面沿着特解对应的向量平移到非齐次线性方程的解平面,这便是这一结论的几何解释. 同时我们可以得到下述结论:
\begin{enumerate}
    \item $n$元非齐次线性方程组 \ref*{eq:15:非齐次} 的两个解的差是它的导出组 \ref*{eq:15:齐次} 的一个解;

    \item $n$元非齐次线性方程组 \ref*{eq:15:非齐次} 的一个解与它的导出组 \ref*{eq:15:齐次} 的一个解之和仍是非齐次线性方程组 \ref*{eq:15:非齐次} 的一个解.
\end{enumerate}

\begin{proof}
    事实上,\ref*{eq:15:非齐次} 可以写为$AX=\vec{0}$.
    \begin{enumerate}
        \item 设$\gamma_1,\gamma_2$分别是非齐次线性方程组 \ref*{eq:15:非齐次} 的两个解,则
              \[A(\gamma_1-\gamma_2)=\vec{b}-\vec{b}=\vec{0}.\]

        \item 设$\gamma_1$是非齐次线性方程组 \ref*{eq:15:非齐次} 的一个解,$\eta_1$是齐次线性方程组 \ref*{eq:15:齐次} 的一个解,则
              \[A(\gamma_1+\eta_1)=\vec{b}+\vec{0}=\vec{b}.\]
    \end{enumerate}
\end{proof}
实际上根据上述几何描述形象理解这两个结论也不困难. 下面我们将通过一些例子进一步探讨非齐次线性方程组解的结构问题:
\begin{example}{}{非齐次解的进一步结构}
    若$X_0$是非齐次线性方程组$AX=\vec{b}$的一个特解,$X_1,\ldots,X_p$是$AX=\vec{0}$的基础解系,证明:
    \begin{enumerate}
        \item $X_0,X_1,X_2,\ldots,X_p$线性无关;

        \item $X_0,X_0+X_1,X_0+X_2,\ldots,X_0+X_p$线性无关;

        \item $AX=\vec{b}$的任一个解$X$可表示为
              \[X=k_0X_0+k_1(X_0+X_1)+k_2(X_0+X_2)+\cdots+k_p(X_0+X_p),\]
              其中$k_0+k_1+k_2+\cdots+k_p=1$.
    \end{enumerate}
\end{example}

\begin{proof}
    \begin{enumerate}
        \item 设$k_0X_0+k_1X_1+k_2X_2+\cdots+k_pX_p=\vec{0}$,则
              \[A(k_0X_0+k_1X_1+k_2X_2+\cdots+k_pX_p)=\vec{0}.\]
              因为$X_1,\ldots,X_p$是$AX=\vec{0}$的基础解系,所以$AX_i=\vec{0}$,$i=1,2,\ldots,p$,故$k_0AX_0=k_0\vec{b}=\vec{0}$,由于$\vec{b}\neq \vec{0}$,因此$k_0=0$. 因此
              \[k_1X_1+k_2X_2+\cdots+k_pX_p=\vec{0},\]
              因为$X_1,\ldots,X_p$是$AX=\vec{0}$的基础解系,所以它们线性无关,故$k_1=k_2=\cdots=k_p=0$,故$k_0=k_1=k_2=\cdots=k_p=0$,故$X_0,X_1,X_2,\ldots,X_p$线性无关.

        \item 设$k_0X_0+k_1(X_0+X_1)+k_2(X_0+X_2)+\cdots+k_p(X_0+X_p)=\vec{0}$,则
              \[(k_0+k_1+k_2+\cdots+k_p)X_0+k_1X_1+k_2X_2+\cdots+k_pX_p=\vec{0}.\]
              由上一问知,$X_0,X_1,X_2,\ldots,X_p$线性无关,故$\begin{cases}
                      k_0+k_1+k_2+\cdots+k_p=0 \\
                      k_1=k_2=\cdots=k_p=0
                  \end{cases}$因此$k_0=k_1=k_2=\cdots=k_p=0$,故$X_0,X_0+X_1,X_0+X_2,\ldots,X_0+X_p$线性无关.

        \item 设$X$是$AX=\vec{b}$的任一个解,则由\autoref{thm:通解加特解} 可知,$X$可以被表示为
              \begin{align*}
                  X & =X_0+k_1X_1+k_2X_2+\cdots+k_pX_p                                         \\
                    & =(1-k_1-k_2-\cdots-k_p)X_0+k_1(X_0+X_1)+k_2(X_0+X_2)+\cdots+k_p(X_0+X_p)
              \end{align*}
              令$k_0=1-k_1-k_2-\cdots-k_p$,则$k_0+k_1+k_2+\cdots+k_p=1$,命题得证.
    \end{enumerate}
\end{proof}

本例说明了非齐次线性方程组的特解事实上与对应的齐次线性方程组的基础解系也是线性无关的. 事实上本例的结论和解决思路都是非常重要,请读者务必掌握.

\begin{example}{}{非齐次线性无关解}
    设$A$为$s \times n$矩阵,且$r(A)=r$,证明:非齐次线性方程组$AX=\vec{b}$至多存在$n-r+1$个线性无关的解向量.
\end{example}

\begin{proof}
    当$AX=\vec{b}$无解时有0个解向量,我们知道$r\leqslant n$,因此$n-r+1\geqslant 1$,故命题成立.

    当$AX=\vec{b}$有解时,设$X_0$是非齐次线性方程组$AX=\vec{b}$的一个特解,根据\autoref{thm:齐次维数} 知,$AX=\vec{0}$的解空间维数为$n-r$,故设$X_1,\ldots,X_{n-r}$是$AX=\vec{0}$的基础解系,则由\autoref{ex:非齐次解的进一步结构} 知,$X_0,X_0+X_1,X_0+X_2,\ldots,X_0+X_{n-r}$线性无关,并且它们都是$AX=\vec{b}$的解,因此非齐次线性方程组$AX=\vec{b}$存在$n-r+1$个线性无关的解向量.

    下面要说明$AX=\vec{b}$任意$n-r+2$个解向量必定线性相关. 利用反证法,设$\eta_1,\eta_2,\ldots,\eta_{n-r+2}$是$AX=\vec{b}$的$n-r+2$个线性无关解向量,则$\eta_2-\eta_1,\eta_3-\eta_1,\ldots,\eta_{n-r+2}-\eta_1$均为$AX=\vec{0}$的解向量. 设
    \[k_1(\eta_2-\eta_1)+k_2(\eta_3-\eta_1)+\cdots+k_{n-r+1}(\eta_{n-r+2}-\eta_1)=\vec{0},\]
    即
    \[k_1\eta_2+k_2\eta_3+\cdots+k_{n-r+1}\eta_{n-r+2}-(k_1+k_2+\cdots+k_{n-r+1})\eta_1=\vec{0},\]
    因为$\eta_1,\eta_2,\ldots,\eta_{n-r+2}$线性无关,所以
    \[k_1=k_2=\cdots=k_{n-r+1}=0,\]
    即$\eta_2-\eta_1,\eta_3-\eta_1,\ldots,\eta_{n-r+2}-\eta_1$线性无关,因此$AX=\vec{0}$的解向量至少有$n-r+1$个线性无关,这与$AX=\vec{0}$的解空间维数为$n-r$矛盾,故假设不成立,命题得证.
\end{proof}

\section{线性方程组解的几何解释}

接下来从对偶空间的角度探讨一下线性方程组的结构问题. 考虑方程组 $Ax = b$,我们将其写开:

\[
\begin{cases}
\begin{aligned}
    a_{11} x_1 + a_{12} x_2 + \cdots + a_{1n} x_n & = b_1 \\
    a_{21} x_1 + a_{22} x_2 + \cdots + a_{2n} x_n & = b_2 \\
    \vdotswithin{=} \\
    a_{m1} x_1 + a_{m2} x_2 + \cdots + a_{mn} x_n & = b_m
\end{aligned}
\end{cases}
\]

因此从几何直观来看,线性方程组无非是一族超平面的交. 现在,用线性泛函的语言重述一下这个性质,记:

\[
\varphi_k(x_1, x_2, \cdots, x_n) = a_{k1} x_1 + a_{k2} x_2 + \cdots + a_{kn} x_n
\]

则我们知道,上述线性方程组的可行解就是

\[
\bigcap_{i = 1}^k \varphi_i^{-1} (b_i)
\]

我们知道,这样的一组纤维构成一个仿射子集,而仿射子集无非就是一个超平面,所以,现在上面的直观已经被用代数的语言描述了. 接下来,让我们考察仿射子集的交,也就是以下引理:

\begin{lemma}{}{}
    仿射子集的交如果非空,则它依然是仿射子集.
\end{lemma}

从直观上看,这个结果就是说,两个线性的东西的交集依然是线性的. 不妨设想一下三维情形,两个平面交于一条直线,而直线依然是一个线性的对象. 在下面的证明中,我们会给出这条直线的构造,而在后面关于射影几何代数的未竟专题当中,这将称为我们讨论的一个核心.

\begin{proof}
    考虑两个仿射子集 $v_1 + W_1, v_2 + W_2$. 如果向量 $u \in v_1 + W_1 \cap v_2 + W_2$,那么也就是说它可以表示成

    \[
    u = v_1 + w_1 = v_2 + w_2
    \]

    为了证明它是一个仿射子集,显然的想法就是,它减掉某一个东西之后构成一个线性空间. 那么这个东西是什么呢?很简单,只要假定它存在就可以了. 令 $u_0 \in v_1 + W_1 \cap v_2 + W_2$,记其为 $v_1 + \hat w_1 = v_2 + \hat w_2$,则

    \[
    u - u_0 = w_1 - \hat w_1 = w_2 - \hat w_2 \in W_1 \cap W_2
    \]

    因此,下面我们直接考虑证明 $v_1 + W_1 \cap v_2 + W_2 = u_0 + W_1 \cap W_2$. 现在,我们已经证明了左边包含于右边. 考虑 $u_0 + w \in u_0 + W_1 \cap W_2$,则我们知道

    \[
    u_0 + w = v_1 + \hat w_1 + w = v_1 + (\hat w_1 + w) \in v_1 + W_1
    \]

    同理可证 $u_0 + w \in v_2 + W_2$. 因此右边包含于左边. 从此得证.
\end{proof}

参照这个构造,应用归纳法,我们知道:

\[
\bigcap_{i = 1}^n v_i + W_i = v_0 + \bigcap_{i = 1}^n W_i
\]

其中 $v_0$ 为一个已经选出的代表元. 因此,我们展开线性方程的可行解的表达方式,根据\autoref{thm:纤维的结构}有 $\varphi_i^{-1}(b)$ 可以表达为 $v_i + \ker \varphi$ 的形式,所以:
\[
\bigcap_{i = 1}^m \varphi_i^{-1} (b_i) = v_0 + \bigcap_{i = 0}^m \ker \varphi_i
\]

我们将前面部分称为特解,后面部分称为通解. 而通解如我们所见就是 $Ax = 0$ 的解. 于是,参照\autoref{thm:零化子维数}以及\autoref{lem:NU性质},$Ax = 0$ 的解空间的维数就是 $\dim Z(\spa\{\varphi_1,\ldots,\varphi_m\}) = \dim V - \dim\spa\{\varphi_1,\ldots,\varphi_m\}$. 因此,我们只需考察在 $m$ 个线性泛函当中,有多少个是线性无关的. 而实际上我们知道线性泛函 $a^\mathrm{T}x = 1$ 与向量 $a$ 一一对应,故这一问题就转化为求这些向量拼成的矩阵的秩的问题了. 而这些向量本身来源于线性方程组的系数,因此拼成的矩阵就是系数矩阵,所以我们从几何的角度也得到了\autoref{thm:齐次维数}的结论.

\section{理论应用}

本节我们将综合线性方程组解的一般理论和之前所学的知识讨论一些秩的等式/不等式,以及其它一些经典的相关问题. 我们首先来看四个最为经典的问题:
\begin{example}{}{线性方程组理论与秩不等式}
    利用线性方程组解的一般理论,证明以下命题:
    \begin{enumerate}
        \item  设$A,B$分别是$m \times n$和$n \times s$矩阵,则$r(AB)\leqslant\min\{r(A),r(B)\}$;

        \item 设$A,B$分别是$m \times n$和$n \times s$矩阵,且$AB=O$,证明:$r(A)+r(B)\leqslant n$;

        \item 设$A$是$m \times n$实矩阵,证明:$r(A^\mathrm{T}A)=r(A)$;

        \item $A^2=A \iff r(A)+r(E-A)=n$.
    \end{enumerate}
\end{example}

\begin{proof}
    \begin{enumerate}
        \item 因为$BX=\vec{0}$可以导出$ABX=\vec{0}$,因此$N(B)\subset N(AB)$,因此$\dim N(B)\leqslant\dim N(AB)$,因此$r(AB)=n-\dim N(AB)\leqslant n-\dim N(B)=r(B)$

              又由$r(AB)=r((AB)^{\mathrm{T}})=r(B^{\mathrm{T}}A^{\mathrm{T}})\leqslant r(A^{\mathrm{T}})=r(A)$(最后一个小于等于理由同上面证明$r(AB)\leqslant r(B)$),因此$r(AB)\leqslant\min\{r(A),r(B)\}$.

        \item 将$B$按列分块为$(B_1,B_2,\ldots,B_s)$,则$AB=(AB_1,AB_2,\ldots,AB_s)=O$,因此$AB_i=\vec{0}$,$i=1,2,\ldots,s$,因此每个$B_i$都是齐次线性方程组$AX=\vec{0}$的解,因此$B$的列向量张成的空间包含于$AX=\vec{0}$的解空间,因此$r(B)\leqslant n-r(A)$,即$r(A)+r(B)\leqslant n$.

        \item 由前证$r(AB)\leqslant\min\{r(A),r(B)\}$,可知$r(AA^\mathrm{T})\leqslant r(A)$,因此只需证明$r(AA^\mathrm{T})\geqslant r(A)$,即只需证$N(AA^\mathrm{T})\subset N(A)$.

              设$AA^\mathrm{T}X=\vec{0}$,因此$X^\mathrm{T}(AA^\mathrm{T})X=0$,即$(A^\mathrm{T}X)^\mathrm{T}(A^\mathrm{T}X)=0$. 由于$A^\mathrm{T}X\in\mathbf{R}^n$,我们设$A^\mathrm{T}X=(a_1,a_2,\ldots,a_n)^\mathrm{T}$,则$(A^\mathrm{T}X)^\mathrm{T}(A^\mathrm{T}X)=a_1^2+a_2^2+\cdots+a_n^2=0$可得$a_1=a_2=\cdots=a_n=0$,即$A^\mathrm{T}X=\vec{0}$,因此$X \in N(A)$,因此$N(AA^\mathrm{T})\subset N(A)$,得证.

        \item 对于$A^2=A$,考虑分块矩阵$\begin{pmatrix}
                      A & O \\ O & A-E
                  \end{pmatrix}$,对其进行如下分块矩阵初等变换:
              \begin{align*}
                  \begin{pmatrix}
                      A & O \\ O & A-E
                  \end{pmatrix} & \to\begin{pmatrix}
                                         A & E-A \\ O & A-E
                                     \end{pmatrix}
                  \to\begin{pmatrix}
                         A & E \\ O & A-E
                     \end{pmatrix}
                  \to\begin{pmatrix}
                         O & E \\ -A(A-E) & A-E
                     \end{pmatrix}                  \\
                                   & \to\begin{pmatrix}
                                            O & E \\ -A(A-E) & O
                                        \end{pmatrix}.
              \end{align*}
              事实上第一次变换是第二行乘以$-E$加到第一行,第二次变换是第一列加到第二列,第三次变换是第二列乘以$-A$加到第一列,第四次变换是第一行乘以$E-A$加到第二行,注意每一步都是分块矩阵初等变换,因此不改变矩阵的秩,因此有
              \[r\begin{pmatrix}
                      A & O \\ O & A-E
                  \end{pmatrix}=r\begin{pmatrix}
                      O & E \\ -A(A-E) & O
                  \end{pmatrix}.\]
              即$r(A)+r(A-E)=n+r(-A(A-E))$,故$r(A)+r(A-E)=n$等价于$r(-A(A-E))=0$,即$-A(A-E)=O$,即$A^2=A$,得证.

              事实上,如果我们只需要证明$A^2=A \implies r(A)+r(E-A)=n$,我们可以用如下方法:由$A^2=A$可知$A(A-E)=O$,由本例第二问知$r(A)+r(A-E)\leqslant n$,又根据秩不等式$r(A)+r(B)\geqslant r(A+B)$,因此$r(A)+r(E-A)\geqslant r(A+(E-A))=r(E)=n$. 综上可知,$r(A)+r(E-A)=n$.
    \end{enumerate}
\end{proof}

实际上,我们解决此类问题,很多时候等式都需要拆为小于等于和大于等于同时成立进行证明,经常利用维数公式变形的齐次线性方程组解的一般理论,将问题转化为对像与核空间的研究,然后利用包含关系(复杂的题目可能涉及子空间交与和的维数公式)以及已知的简单秩不等式进行证明. 可能部分题目较为困难,但至少请掌握上面例题中的情况. 接下来我们讨论伴随矩阵的秩的问题,这一例题的结论和证明都非常重要:
\begin{example}{}{伴随矩阵的秩}
    设$A^*$为矩阵$A$的伴随矩阵,证明:
    \[r(A^*)=\begin{cases}
            n & r(A)=n \\ 1 & r(A)=n-1 \\ 0 & r(A) < n-1
        \end{cases}.\]
\end{example}

\begin{proof}
    \begin{enumerate}
        \item 当$r(A)=n$时,$A$可逆,因此$A^*=|A|A^{-1}$,因此$r(A^*)=r(|A|A^{-1})=r(A^{-1})=n$.

        \item 当$r(A)=n-1$时,$|A|=0$,因此$AA^*=|A|E=O$,即$A^*$的列向量都是方程$AX=0$的解,故$A^*$列向量张成的空间包含于$AX=0$的解空间,因此$r(A^*)\leqslant n-r(A)=1$.

              而$r(A)=n-1$表明$A$中存在非零的$n-1$阶子式,因此存在$A$的代数余子式$A_{ij}$不为0,因此$A^*$不为0,因此$r(A^*)\geqslant 1$,因此$r(A^*)=1$.

        \item 当$r(A)<n-1$时,$A$的任意一个$n-1$阶子式都等于0,即任意一个代数余子式$A_{ij}$都等于0,因此$A^*=O$,因此$r(A^*)=0$.
    \end{enumerate}
\end{proof}

接下来是一个利用\nameref{thm:线性空间维数公式}以及\nameref{thm:线性映射基本定理}中``设小扩大''的思想解决的问题:
\begin{example}{}{维数公式技巧例题}
    已知$A,B$分别是数域$\mathbf{F}$上的$l \times k$和$k \times n$矩阵,$X$是$n \times 1$的列向量. 证明:所有满足$ABX=0$的$BX$构成一个线性空间$V$,且$\dim V = r(B) - r(AB)$.
\end{example}

\begin{proof}
    $V$是线性空间只需要说明其中元素关于加法数乘封闭即可,因为这样$V$就是$\mathbf{F}^k$的子空间. 这一证明非常基本,我们在此略过.

    记$V_1=\{X\mid BX=0\},\enspace V_2=\{X\mid ABX=0\}$,则$V_1\subseteq V_2$,因为$\forall X\in V_1$,有$BX=0$,因此$ABX=A0=0$,即$X\in V_2$,因此$V_1\subseteq V_2$. 利用``设小扩大''的思想,取$V_1$的一组基$\alpha_1,\ldots,\alpha_r$,则可以扩充为$V_2$的一组基,记为$\alpha_1,\ldots,\alpha_r,\alpha_{r+1},\ldots,\alpha_m$,则$r=n-r(B)$,$s=n-r(AB)$,于是
    \begin{align*}
        V & =\{BX\mid ABX=0\}                                                \\
          & =\spa(B\alpha_1,\ldots,B\alpha_r,B\alpha_{r+1},\ldots,B\alpha_m) \\
          & =\spa(B\alpha_{r+1},\ldots,B\alpha_m).
    \end{align*}
    下面证明$B\alpha_{r+1},\ldots,B\alpha_m$线性无关. 为此,设
    \[c_{r+1}B\alpha_{r+1}+\cdots+c_mB\alpha_m=0,\]
    则
    \[B(c_{r+1}\alpha_{r+1}+\cdots+c_m\alpha_m)=0,\]
    因此$c_{r+1}\alpha_{r+1}+\cdots+c_m\alpha_m\in V_1$,因此存在$c_1,\ldots,c_r$使得
    \[c_{r+1}\alpha_{r+1}+\cdots+c_m\alpha_n=c_1\alpha_1+\cdots+c_r\alpha_r,\]
    即
    \[c_{r+1}\alpha_{r+1}+\cdots+c_m\alpha_m-c_1\alpha_1-\cdots-c_r\alpha_r=0.\]
    由于$\alpha_1,\ldots,\alpha_m$线性无关,因此
    \[c_{r+1}=\cdots=c_m=c_1=c_2=\cdots=c_r=0,\]
    因此$B\alpha_{r+1},\ldots,B\alpha_m$线性无关,因此$V$的维数为$s-r=(n-r(AB))-(n-r(B))=r(B)-r(AB)$,得证.
\end{proof}

\section{线性方程组拓展题型}

本节我们将介绍与线性方程组有关的一些题型,可能与高中数学讨论``题型''的学习风格有些类似. 需要注意的是,除了含参问题外,其余问题我们都将分别从齐次和非齐次两个方面进行讨论,给出问题的一般解法. 但实际上这里给出的解法并非能直接套用到所有的题目中,在习题中我们会遇到更多特别的题目. 因此更重要的应当是理解解题思路,而不是死记硬背解题方法.

\subsection{含参数的线性方程组问题}

此类问题一般考察对于含参数的线性方程组,参数取值如何时有解/无解/有唯一解等. 本质而言,\nameref{thm:有解条件}完全可以解决这一问题.

事实上,利用\autoref{thm:方程组解} 在有解情况下只需计算行列式判断非常方便,但判断无解需要利用\nameref{thm:有解条件},其中线性相关性的判断通常仍然需要我们对系数矩阵进行高斯-若当消元法. 我们来看一个简单的例子:
\begin{example}{}{}
    当$k$取何值时,方程组:
    \[\begin{cases}
            x_1+x_2+kx_3=4 \\ -x_1+kx_2+x_3=k^2 \\ x_1-x_2+2x_3=-4
        \end{cases}\]
    有唯一解、无解、有无穷多解?在有解的情况下,求出方程组的全部解.
\end{example}
\begin{solution}
    对于有无解的区别,我们一般都考虑直接使用高斯-若当消元法. 由高斯-若当消元法有(省略中间步骤直接得到阶梯矩阵):
    \[\begin{pmatrix}
            1  & 1  & k & 4   \\
            -1 & k  & 1 & k^2 \\
            1  & -1 & 2 & -4
        \end{pmatrix}\to\begin{pmatrix}
            1 & 1 & k                     & 4      \\
            0 & 2 & k-2                   & 8      \\
            0 & 0 & \dfrac{(k+1)(4-k)}{2} & k(k-4)
        \end{pmatrix}.\]
    \begin{enumerate}
        \item 当$k=-1$时,增广矩阵秩为3,系数矩阵秩为2(或者说最后一行出现矛盾方程),无解;

        \item 当$k=4$时,增广矩阵和系数矩阵秩均为2,方程有无穷多解,解得同解为$k_1(-3,-1,1)^{\mathrm{T}}+(0,4,0)^{\mathrm{T}}(k_1\in\mathbf{R})$;

        \item 当$k\neq-1,4$时,增广矩阵和系数矩阵秩均为3,方程有唯一解,解为
              \[(\dfrac{k^2+2k}{k+1},\dfrac{k^2+2k+4}{k+1},-\dfrac{2k}{k+1})^{\mathrm{T}}.\]
    \end{enumerate}

    事实上,本题方程个数与未知数个数相等,因此可以运用 \nameref{thm:Cramer}解决. 首先求解系数矩阵$A$的行列式为
    \[|A|=\begin{vmatrix}
            1  & 1  & k \\
            -1 & k  & 1 \\
            1  & -1 & 2
        \end{vmatrix}=(k+1)(4-k),\]
    由Cramer法则,当$|A|\neq 0$时,方程组有唯一解,当$|A|=0$时,方程组无解或有无穷多解,其余关于有无解的讨论与上面一致.
\end{solution}

\subsection{线性方程组同解问题}

两个线性方程组同解实际上有两种情况:
\begin{enumerate}
    \item 两线性方程组都无解(注意齐次没有这种情况,因为一定有零解);

    \item 两线性方程组都有解且有相同的解集.
\end{enumerate}

下面的定理给出了两线性方程组同解的充要条件. 实际上,这两个定理的证明很值得作为练习综合运用所学知识:
\begin{theorem}{}{}
    $n$元齐次线性方程组 $A_{m_1 \times n}X=\vec{0}$与 $B_{m_2 \times n}X=\vec{0}$同解的充要条件是$r\begin{pmatrix}
            A \\ B
        \end{pmatrix}=r(A)=r(B)$.
\end{theorem}

\begin{proof}
    \begin{enumerate}
        \item 必要性:事实上$\begin{pmatrix}
                      A \\ B
                  \end{pmatrix}X=0\iff AX=0,BX=0$,因此由$AX=0,BX=0$同解可知,$\begin{pmatrix}
                      A \\ B
                  \end{pmatrix}X=0\iff AX=0$,因此$r\begin{pmatrix}
                      A \\ B
                  \end{pmatrix}=r(A)$,同理可证$r\begin{pmatrix}
                      A \\ B
                  \end{pmatrix}=r(B)$,因此$r\begin{pmatrix}
                      A \\ B
                  \end{pmatrix}=r(A)=r(B)$.

        \item 充分性:事实上$\begin{pmatrix}
                      A \\ B
                  \end{pmatrix}X=0$的解一定是$AX=0$的解,设这两个方程的解空间依次为$U_1,U_2$,因此$U_1$是$U_2$的子空间. 而$r\begin{pmatrix}
                      A \\ B
                  \end{pmatrix}=r(A)$表明$U_1=U_2$,即$\begin{pmatrix}
                      A \\ B
                  \end{pmatrix}X=0$和$AX=0$同解. 同理可知$\begin{pmatrix}
                      A \\ B
                  \end{pmatrix}X=0$和$BX=0$同解,因此$A,B$同解.
    \end{enumerate}
\end{proof}

\begin{theorem}{}{}
    $n$元非齐次线性方程组 $A_{m_1 \times n}X=\vec{b}$与 $B_{m_2 \times n}X=\vec{d}$同解的充要条件是
    \begin{enumerate}
        \item $r(A)\neq r(A,\vec{b})$且$r(B)\neq r(B,\vec{d})$;或

        \item $r\begin{pmatrix}
                      A & \vec{b} \\ B & \vec{d}
                  \end{pmatrix}=r\begin{pmatrix}
                      A \\ B
                  \end{pmatrix}=r(A)=r(A,\vec{b})=r(B)=r(B,\vec{d})$.
    \end{enumerate}
\end{theorem}

\begin{proof}
    事实上两个条件分别对应于两方程均有解和均无解的情况,无解情况显然正确,下面讨论有解情况:
    \begin{enumerate}
        \item 必要性:事实上$\begin{pmatrix}
                      A & \vec{b} \\ B & \vec{d}
                  \end{pmatrix}X=0\iff AX=\vec{b},BX=\vec{d}$,因此由$AX=\vec{b},BX=\vec{d}$同解可知,$\begin{pmatrix}
                      A & \vec{b} \\ B & \vec{d}
                  \end{pmatrix}X=0\iff AX=\vec{b}$,因此$r\begin{pmatrix}
                      A & \vec{b} \\ B & \vec{d}
                  \end{pmatrix}=r(A,\vec{b})$,同理可证$r\begin{pmatrix}
                      A & \vec{b} \\ B & \vec{d}
                  \end{pmatrix}=r(B,\vec{d})$,因此$r\begin{pmatrix}
                      A & \vec{b} \\ B & \vec{d}
                  \end{pmatrix}=r(A,\vec{b})=r(B,\vec{d})$,由于此时对应有解情况,故$r\begin{pmatrix}
                      A & \vec{b} \\ B & \vec{d}
                  \end{pmatrix}=r(A,\vec{b})=r(B,\vec{d})=r(A)=r(B)$.

        \item 充分性:事实上$\begin{pmatrix}
                      A & \vec{b} \\ B & \vec{d}
                  \end{pmatrix}X=0$的解一定是$AX=\vec{b}$的解,设这两个方程的解集合(此时非齐次线性方程组不是子空间)分别为$S_1,S_2$,因此$S_1$是$S_2$的子集. 且由\autoref{ex:非齐次线性无关解} 可知,$S_1$的秩为$n-r\begin{pmatrix}
                      A \\ B
                  \end{pmatrix}+1$,$S_2$的秩为$n-r(A)+1$,因此$S_1=S_2$,即$\begin{pmatrix}
                      A & \vec{b} \\ B & \vec{d}
                  \end{pmatrix}X=0$和$AX=\vec{b}$同解. 同理可知$\begin{pmatrix}
                      A & \vec{b} \\ B & \vec{d}
                  \end{pmatrix}X=0$和$BX=\vec{d}$同解,得证.
    \end{enumerate}
\end{proof}

事实上,在了解上述定理证明后我们会发现这些条件都是非常自然的. 我们来看一个例子来运用上述定理:
\begin{example}{}{}
    已知方程组\begin{gather*}
        \begin{cases}
            x_1+2x_2+3x_3=0  \\
            2x_1+3x_2+5x_3=0 \\
            x_1+x_2+ax_3=0
        \end{cases} \\
        \begin{cases}
            x_1+bx_2+cx_3=0 \\
            2x_1+b^2x_2+(c+1)x_3=0
        \end{cases}
    \end{gather*}
    同解,求$a,b,c$的值.
\end{example}
\begin{solution}
    设第一个方程系数矩阵为$A$,第二个方程系数矩阵为$B$,则
    \[\begin{pmatrix}
            A \\ B
        \end{pmatrix}=\begin{pmatrix}
            1 & 2 & 3 \\ 2 & 3 & 5 \\ 1 & 1 & a \\ 1 & b & c \\ 2 & b^2 & c+1
        \end{pmatrix}\to\begin{pmatrix}
            1 & 0 & 1 \\ 0 & 1 & 1 \\ 0 & 0 & a-2 \\ 0 & 0 & c-b-1 \\ 0 & 0 & c-b^2-1
        \end{pmatrix},\]
    因此由同解条件$r\begin{pmatrix}
            A \\ B
        \end{pmatrix}=r(A)=r(B)$可知必有$r\begin{pmatrix}
            A \\ B
        \end{pmatrix}=r(A)=r(B)=2$(因为$r(B)\leqslant 2$,$r\begin{pmatrix}
            A \\ B
        \end{pmatrix}\geqslant 2$).

    从而有$a-2=c-b-1=c-b^2-1=0$. 因此$a=2,b=0,c=1$或$a=2,b=1,c=2$. 当$a=2,b=0,c=1$时,$r(B)=1\neq r(A)=2$,舍去. 故$a=2,b=1,c=2$.
\end{solution}

\subsection{线性方程组公共解问题}

公共解即为两线性方程组解集的交集,我们从齐次和非齐次讨论有公共解的条件:
\begin{theorem}{}{}
    对于$n$元齐次线性方程组 (1) $A_{m_1 \times n}X=\vec{0}$与 (2) $B_{m_2 \times n}X=\vec{0}$有
    \begin{enumerate}
        \item (1) 与 (2) 有非零公共解的充要条件是$r\begin{pmatrix} A \\ B \end{pmatrix}<n$;

        \item 设$\eta_1,\eta_2,\ldots,\eta_s\enspace(s=n-r(B))$是 (2) 的基础解系,则(1) 与 (2) 有非零公共解的充要条件是$A\eta_1,A\eta_2,\ldots,A\eta_s$线性相关;

        \item 设$\gamma_1,\gamma_2,\ldots,\gamma_t\enspace(t=n-r(A))$是(1) 的基础解系,$\eta_1,\eta_2,\ldots,\eta_s\enspace(s=n-r(B))$是 (2) 的基础解系,则(1) 与 (2) 有非零公共解的充要条件是
              \[\gamma_1,\gamma_2,\ldots,\gamma_t,\eta_1,\eta_2,\ldots,\eta_s\]
              线性相关.
    \end{enumerate}
\end{theorem}

\begin{proof}
    \begin{enumerate}
        \item 必要性:设$X_0$是两方程组的非零公共解,即$AX_0=\vec{0}$且$BX_0=\vec{0}$,因此$X_0$是$\begin{pmatrix}
                      A \\ B
                  \end{pmatrix}X=\vec{0}$的解,即线性方程组$\begin{pmatrix}
                      A \\ B
                  \end{pmatrix}X=\vec{0}$有非零解,因此$r\begin{pmatrix}
                      A \\ B
                  \end{pmatrix}\leqslant n$.

              充分性:由$r\begin{pmatrix}
                      A \\ B
                  \end{pmatrix}\leqslant n$可知线性方程组$\begin{pmatrix}
                      A \\ B
                  \end{pmatrix}X=\vec{0}$有非零解,设为$X_0$,因此有$AX_0=\vec{0}$且$BX_0=\vec{0}$,即$X_0$是两方程组的非零公共解,得证.

        \item 必要性:设$X_0$是两方程组的非零公共解,则$X_0$可由$\eta_1,\eta_2,\ldots,\eta_s$线性表示,即
              \[X_0=k_1\eta_1+k_2\eta_2+\cdots+k_s\eta_s,\]
              其中$k_1,k_2,\ldots,k_s$不全为0,因此$AX_0=A(k_1\eta_1+k_2\eta_2+\cdots+k_s\eta_s)=k_1A\eta_1+k_2A\eta_2+\cdots+k_sA\eta_s=\vec{0}$,由于$k_1,k_2,\ldots,k_s$不全为0,因此$A\eta_1,A\eta_2,\ldots,A\eta_s$线性相关.

              充分性:由$A\eta_1,A\eta_2,\ldots,A\eta_s$线性相关知,存在不全为0的$k_1,k_2,\ldots,k_s$使得
              \[k_1A\eta_1+k_2A\eta_2+\cdots+k_sA\eta_s=\vec{0},\]
              因此$A(k_1\eta_1+k_2\eta_2+\cdots+k_s\eta_s)=\vec{0}$,因此$X_0=k_1\eta_1+k_2\eta_2+\cdots+k_s\eta_s$是$AX=\vec{0}$的非零解(非零的原因在于如果$X_0$为零向量,那么因为$\eta_1,\eta_2,\ldots,\eta_s$线性无关,则$k_1,k_2,\ldots,k_s$均为0,矛盾).

              又$X_0$是$BX=\vec{0}$的解(因为表示为了$BX=\vec{0}$基础解系的线性组合),因此$X_0$是两方程组的非零公共解,得证.

        \item 必要性:设$X_0$是两方程组的非零公共解,则$X_0$可由$\gamma_1,\gamma_2,\ldots,\gamma_t$和$\eta_1,\eta_2,\ldots,\eta_s$线性表示,即
              \begin{align*}
                  X_0 & =k_1\gamma_1+k_2\gamma_2+\cdots+k_t\gamma_t \\
                      & =l_1\eta_1+l_2\eta_2+\cdots+l_s\eta_s
              \end{align*}
              其中$k_1,k_2,\ldots,k_t,l_1,l_2,\ldots,l_s$不全为0,因此
              \[k_1\gamma_1+k_2\gamma_2+\cdots+k_t\gamma_t-l_1\eta_1-l_2\eta_2-\cdots-l_s\eta_s=\vec{0},\]
              因此$\gamma_1,\gamma_2,\ldots,\gamma_t,\eta_1,\eta_2,\ldots,\eta_s$线性相关.

              充分性:由$\gamma_1,\gamma_2,\ldots,\gamma_t,\eta_1,\eta_2,\ldots,\eta_s$线性相关知,存在不全为0的
              \[k_1,k_2,\ldots,k_t,l_1,l_2,\ldots,l_s\]
              使得
              \[k_1\gamma_1+k_2\gamma_2+\cdots+k_t\gamma_t+l_1\eta_1+l_2\eta_2+\cdots+l_s\eta_s=\vec{0},\]
              令$X_0=k_1\gamma_1+k_2\gamma_2+\cdots+k_t\gamma_t=-(l_1\eta_1+l_2\eta_2+\cdots+l_s\eta_s)$,因此$X_0$是$AX=\vec{0}$和$BX=\vec{0}$的非零公共解(非零的原因在于如果$X_0$为零向量,那么因为$\gamma_1,\gamma_2,\ldots,\gamma_t,\eta_1,\eta_2,\ldots,\eta_s$线性无关,则$k_1,k_2,\ldots,k_t,l_1,l_2,\ldots,l_s$均为0,矛盾).
    \end{enumerate}
\end{proof}

\begin{theorem}{}{非齐次线性方程组公共解}
    对于$n$元非齐次线性方程组(1) $A_{m_1 \times n}X=\vec{b}$与 (2) $B_{m_2 \times n}X=\vec{d}$,若(1) 与 (2) 都有解,则
    \begin{enumerate}
        \item (1) 与 (2) 有公共解的充要条件是$r\begin{pmatrix}
                      A \\ B
                  \end{pmatrix}=r\begin{pmatrix}
                      A & \vec{b} \\ B & \vec{d}
                  \end{pmatrix}$;

        \item 若$r(B)=s$,且$\eta_1,\eta_2,\ldots,\eta_{n-s+1}$是 (2) 的$n-s+1$个线性无关的解,则(1) 与 (2) 有公共解的充要条件是$b$是$A\eta_1,A\eta_2,\ldots,A\eta_{n-s+1}$的凸组合,即存在数$k_1,k_2,\ldots,k_{n-s+1}$使得
              \[\vec{b}=k_1A\eta_1+k_2A\eta_2+\cdots+k_{n-s+1}A\eta_{n-s+1},\]
              其中$k_1+k_2+\cdots+k_{n-s+1}=1$;

        \item 若$r(A)=t$,$r(B)=s$,$\gamma_1,\gamma_2,\ldots,\gamma_{n-t+1}$是(1) 的$n-t+1$个线性无关的解,$\eta_1,\eta_2,\ldots,\eta_{n-s+1}$是 (2) 的$n-s+1$个线性无关的解,则(1) 与 (2) 有公共解的充要条件是存在数$k_1,k_2,\ldots,k_{n-t+1}$和$l_1,l_2,\ldots,l_{n-s+1}$使得
              \[k_1\gamma_1+k_2\gamma_2+\cdots+k_{n-t+1}\gamma_{n-t+1}-l_1\eta_1-l_2\eta_2-\cdots-l_{n-s+1}\eta_{n-s+1}=\vec{0}\]
              其中$k_1+k_2+\cdots+k_{n-t+1}=1$,$l_1+l_2+\cdots+l_{n-s+1}=1$.
    \end{enumerate}
\end{theorem}

\begin{proof}
    \begin{enumerate}
        \item 必要性:设$X_0$是两方程组的公共解,即$AX_0=\vec{b}$且$BX_0=\vec{d}$,因此$X_0$是$\begin{pmatrix}
                      A \\ B
                  \end{pmatrix}X=\begin{pmatrix}
                      \vec{b} \\ \vec{d}
                  \end{pmatrix}$的解,因此这一方程系数矩阵和增广矩阵的秩相等,故$r\begin{pmatrix}
                      A \\ B
                  \end{pmatrix}=r\begin{pmatrix}
                      A & \vec{b} \\ B & \vec{d}
                  \end{pmatrix}$.

              充分性:由$r\begin{pmatrix}
                      A \\ B
                  \end{pmatrix}=r\begin{pmatrix}
                      A & \vec{b} \\ B & \vec{d}
                  \end{pmatrix}$可知,方程$\begin{pmatrix}
                      A \\ B
                  \end{pmatrix}X=\begin{pmatrix}
                      \vec{b} \\ \vec{d}
                  \end{pmatrix}$有解(根据\nameref{thm:有解条件}),记为$X_0$,则
              \[\begin{pmatrix}
                      A \\ B
                  \end{pmatrix}X_0=\begin{pmatrix}
                      \vec{b} \\ \vec{d}
                  \end{pmatrix},\]
              因此$AX_0=\vec{b}$且$BX_0=\vec{d}$,即$X_0$是两方程组的公共解,得证.

        \item 首先说明,$\eta_2-\eta_1,\eta_3-\eta_1,\ldots,\eta_{n-s+1}-\eta_1$是$BX=\vec{0}$的基础解系. 事实上,由于$\eta_1,\eta_2,\ldots,\eta_{n-s+1}$线性无关,因此$\eta_2-\eta_1,\eta_3-\eta_1,\ldots,\eta_{n-s+1}-\eta_1$线性无关,否则存在不全为0的$k_2,k_3,\ldots,k_{n-s+1}$使得
              \[k_2(\eta_2-\eta_1)+k_3(\eta_3-\eta_1)+\cdots+k_{n-s+1}(\eta_{n-s+1}-\eta_1)=\vec{0},\]
              即$k_2\eta_2+k_3\eta_3+\cdots+k_{n-s+1}\eta_{n-s+1}=(k_2+k_3+\cdots+k_{n-s+1})\eta_1$,因此$\eta_1,\eta_2,\ldots,\eta_{n-s+1}$线性相关,矛盾. 因此$\eta_2-\eta_1,\eta_3-\eta_1,\ldots,\eta_{n-s+1}-\eta_1$是$BX=\vec{0}$的基础解系.

              必要性:设$X_0$是两方程组的公共解,则$X_0$可表示为
              \[X_0=\eta_1+k_2(\eta_2-\eta_1)+k_3(\eta_3-\eta_1)+\cdots+k_{n-s+1}(\eta_{n-s+1}-\eta_1),\]
              因此
              \begin{align*}
                  AX_0 & =A\eta_1+k_2(\eta_2-\eta_1)+k_3(\eta_3-\eta_1)+\cdots+k_{n-s+1}(\eta_{n-s+1}-\eta_1))    \\
                       & =A\eta_1+k_2A(\eta_2-\eta_1)+k_3A(\eta_3-\eta_1)+\cdots+k_{n-s+1}A(\eta_{n-s+1}-\eta_1)  \\
                       & =(1-k_2-k_3-\cdots-k_{n-s+1})A\eta_1+k_2A\eta_2+k_3A\eta_3+\cdots+k_{n-s+1}A\eta_{n-s+1} \\
                       & =\vec{b}.
              \end{align*}
              令$k_1=1-k_2-k_3-\cdots-k_{n-s+1}$,则$b$是$A\eta_1,A\eta_2,\ldots,A\eta_{n-s+1}$的凸组合,即存在数$k_1,k_2,\ldots,k_{n-s+1}$使得
              \[\vec{b}=k_1A\eta_1+k_2A\eta_2+\cdots+k_{n-s+1}A\eta_{n-s+1},\]
              其中$k_1+k_2+\cdots+k_{n-s+1}=1$.

              充分性:由存在数$k_1,k_2,\ldots,k_{n-s+1}$使得
              \[\vec{b}=k_1A\eta_1+k_2A\eta_2+\cdots+k_{n-s+1}A\eta_{n-s+1},\]
              其中$k_1+k_2+\cdots+k_{n-s+1}=1$可知
              \[k_1=1-k_2-k_3-\cdots-k_{n-s+1},\]
              故有
              \begin{align*}
                  \vec{b} & =(1-k_2-k_3-\cdots-k_{n-s+1})A\eta_1+k_2A\eta_2+k_3A\eta_3+\cdots+k_{n-s+1}A\eta_{n-s+1} \\
                          & =A(\eta_1+k_2(\eta_2-\eta_1)+k_3(\eta_3-\eta_1)+\cdots+k_{n-s+1}(\eta_{n-s+1}-\eta_1))
              \end{align*}
              令$X_0=\eta_1+k_2(\eta_2-\eta_1)+k_3(\eta_3-\eta_1)+\cdots+k_{n-s+1}(\eta_{n-s+1}-\eta_1)$,则$AX_0=\vec{b}$,且
              \begin{align*}
                  BX_0 & =B(\eta_1+k_2(\eta_2-\eta_1)+k_3(\eta_3-\eta_1)+\cdots+k_{n-s+1}(\eta_{n-s+1}-\eta_1))  \\
                       & =B\eta_1+k_2B(\eta_2-\eta_1)+k_3B(\eta_3-\eta_1)+\cdots+k_{n-s+1}B(\eta_{n-s+1}-\eta_1) \\
                       & =\vec{d},
              \end{align*}
              因此$X_0$是两方程组的公共解,证毕.

        \item 同2的证明知,$\gamma_2-\gamma_1,\gamma_3-\gamma_1,\ldots,\gamma_{n-t+1}-\gamma_1$是$AX=\vec{0}$的基础解系,$\eta_2-\eta_1,\eta_3-\eta_1,\ldots,\eta_{n-s+1}-\eta_1$是$BX=\vec{0}$的基础解系.

              必要性:设$X_0$是两方程组的公共解,则$X_0$可表示为
              \begin{align*}
                  X_0 & =\gamma_1+k_2(\gamma_2-\gamma_1)+k_3(\gamma_3-\gamma_1)+\cdots+k_{n-t+1}(\gamma_{n-t+1}-\gamma_1) \\
                      & =\eta_1+l_2(\eta_2-\eta_1)+l_3(\eta_3-\eta_1)+\cdots+l_{n-s+1}(\eta_{n-s+1}-\eta_1),
              \end{align*}
              因此
              \begin{align*}
                  \gamma_1+k_2(\gamma_2-\gamma_1)+k_3(\gamma_3-\gamma_1)+\cdots+k_{n-t+1}(\gamma_{n-t+1}-\gamma_1)-\eta_1-l_2(\eta_2-\eta_1) \\-l_3(\eta_3-\eta_1)-\cdots-l_{n-s+1}(\eta_{n-s+1}-\eta_1)=\vec{0},
              \end{align*}
              即
              \begin{align*}
                  (1-k_2-k_3-\cdots-k_{n-t+1})\gamma_1+k_2\gamma_2+k_3\gamma_3+\cdots+k_{n-t+1}\gamma_{n-t+1} \\-(1-l_2-l_3-\cdots-l_{n-s+1})\eta_1-l_2\eta_2-l_3\eta_3-\cdots-l_{n-s+1}\eta_{n-s+1}=\vec{0},
              \end{align*}
              令$k_1=1-k_2-k_3-\cdots-k_{n-t+1}$,$l_1=1-l_2-l_3-\cdots-l_{n-s+1}$,则
              \[k_1\gamma_1+k_2\gamma_2+k_3\gamma_3+\cdots+k_{n-t+1}\gamma_{n-t+1}-l_1\eta_1-l_2\eta_2-l_3\eta_3-\cdots-l_{n-s+1}\eta_{n-s+1}=\vec{0},\]
              其中$k_1+k_2+\cdots+k_{n-t+1}=1$,$l_1+l_2+\cdots+l_{n-s+1}=1$,得证.

              充分性:由$k_1+k_2+\cdots+k_{n-t+1}=1$,$l_1+l_2+\cdots+l_{n-s+1}=1$可知
              \[k_1=1-k_2-k_3-\cdots-k_{n-t+1},\enspace l_1=1-l_2-l_3-\cdots-l_{n-s+1},\]
              因此
              \begin{align*}
                  \vec{0} & =k_1\gamma_1+k_2\gamma_2+k_3\gamma_3+\cdots+k_{n-t+1}\gamma_{n-t+1}                               \\&-l_1\eta_1-l_2\eta_2-l_3\eta_3-\cdots-l_{n-s+1}\eta_{n-s+1} \\
                          & =(1-k_2-k_3-\cdots-k_{n-t+1})\gamma_1+k_2\gamma_2+k_3\gamma_3+\cdots+k_{n-t+1}\gamma_{n-t+1}      \\&-(1-l_2-l_3-\cdots-l_{n-s+1})\eta_1-l_2\eta_2-l_3\eta_3-\cdots-l_{n-s+1}\eta_{n-s+1} \\
                          & =\gamma_1+k_2(\gamma_2-\gamma_1)+k_3(\gamma_3-\gamma_1)+\cdots+k_{n-t+1}(\gamma_{n-t+1}-\gamma_1) \\&-\eta_1-l_2(\eta_2-\eta_1)-l_3(\eta_3-\eta_1)-\cdots-l_{n-s+1}(\eta_{n-s+1}-\eta_1).
              \end{align*}
              令$X_0=\gamma_1+k_2(\gamma_2-\gamma_1)+k_3(\gamma_3-\gamma_1)+\cdots+k_{n-t+1}(\gamma_{n-t+1}-\gamma_1)=\eta_1+l_2(\eta_2-\eta_1)+l_3(\eta_3-\eta_1)+\cdots+l_{n-s+1}(\eta_{n-s+1}-\eta_1)$,则$AX_0=\vec{b}$且$BX_0=\vec{d}$,因此$X_0$是两方程组的公共解,证毕.
    \end{enumerate}
\end{proof}

这两个定理看起来非常长,实则无需特别去记忆,只需要通过证明理解其含义即可,有时候做题甚至可以直接暴力解方程然后比较两方程组的解也能完成求解. 下面我们看一个简单的例子:
\begin{example}{}{}
    设四元齐次线性方程组(1) 为\[\begin{cases}
            2x_1+3x_2-x_3=0 \\ x_1+2x_2+x_3-x_4=0
        \end{cases}\]已知另一个四元齐次线性方程组 (2) 的基础解系为
    \[\alpha_1=(2,-1,a+2,1)^\mathrm{T},\enspace\alpha_2=(-1,2,4,a+8)^\mathrm{T}\]
    \begin{enumerate}
        \item 求方程组 (1) 的一个基础解系;

        \item 当$a$为何值时,方程组 (1) 和 (2) 有非零公共解,并求出非零公共解.
    \end{enumerate}
\end{example}

\begin{solution}
    \begin{enumerate}
        \item 直接给出结论:方程组 (1) 的一个基础解系为
              \[\beta_1=(5,-3,1,0)^{\mathrm{T}},\enspace\beta_2=(-3,2,0,1)^{\mathrm{T}}.\]

        \item 根据前述定理,两方程由非零公共解当且仅当$\alpha_1,\alpha_2,\beta_1,\beta_2$线性相关,事实上我们可以将这$\beta_1,\beta_2,\alpha_1,\alpha_2$按列排成矩阵
              \[A=\begin{pmatrix}
                      5 & -3 & 2 & -1 \\ -3 & 2 & -1 & 2 \\ 1 & 0 & a+2 & 4 \\ 0 & 1 & 1 & a+8
                  \end{pmatrix},\]
              事实上,要求两方程组公共解事实上就是求两个子空间$W_1=\spa(\beta_1,\beta_2)$和$W_2=\spa(\alpha_1,\alpha_2)$的交集. 回顾我们在之前所学习的知识,我们可以利用求解极大线性无关组的思想,首先对$A$进行初等行变换得到阶梯矩阵
              \[\begin{pmatrix}
                      1 & 0 & a+2 & 4 \\ 0 & 1 & 1 & a+8 \\ 0 & 0 & 3a+3 & -2a-2 \\ 0 & 0 & -5a-5 & 3a+3
                  \end{pmatrix},\]
              使得$\alpha_1,\alpha_2,\beta_1,\beta_2$线性相关,则必有$r(A)<4$,即$|A|=0$,由此解得$a=-1$. 此时阶梯矩阵为
              \[\begin{pmatrix}
                      1 & 0 & 1 & 4 \\ 0 & 1 & 1 & 7 \\ 0 & 0 & 0 & 0 \\ 0 & 0 & 0 & 0
                  \end{pmatrix},\]
              不难看出$\beta_1,\beta_2$是向量组$\beta_1,\beta_2,\alpha_1,\alpha_2$的极大线性无关组,因此$\alpha_1,\alpha_2$可以完全由$\beta_1,\beta_2$线性表示,故$W_1\cap W_2=W\spa(\alpha_1,\alpha_2)$,即公共解为$\alpha_1,\alpha_2$的线性组合,即
              \[k_1\alpha_1+k_2\alpha_2=(2k_1-k_2,-k_1+2k_2,2k_1+4k_2,k_1+k_2)^\mathrm{T},\]
              其中$k_1,k_2$为任意常数.
    \end{enumerate}
\end{solution}

\subsection{线性方程组反问题}

此类问题即已知方程组的解,要给出原方程组. 我们仍按齐次与非齐次分开的思路讨论此类问题的一般解法. 这里我们之间通过例子来讲解方法:
\begin{example}{}{}
    已知$n$维列向量组$\alpha_1,\ldots,\alpha_s$线性无关,求一齐次线性方程组以$\alpha_1,\ldots,\alpha_s$为基础解系.
\end{example}

\begin{solution}
    设所求的齐次线性方程组为$AX=0$,令$B=(\alpha_1,\ldots,\alpha_s)$,则$B$为$n\times s$矩阵且$AB=O$,于是$B^\mathrm{T}A^\mathrm{T}=O$,解线性方程组$B^\mathrm{T}X=0$,得到其基础解系为
    \[\beta_1,\beta_2,\ldots,\beta_{n-s},\]
    令$A^\mathrm{T}=(\beta_1,\beta_2,\ldots,\beta_{n-s})$即可,因为此时$B^\mathrm{T}A^\mathrm{T}=O$,故$AB=O$.
\end{solution}

\begin{example}{}{}
    设向量组$\alpha_1,\ldots,\alpha_s$线性无关,求一非齐次线性方程组$AX=\vec{b}$,使其解集以$\alpha_1,\ldots,\alpha_s$为极大线性无关组.
\end{example}

\begin{solution}
    根据\autoref{thm:非齐次线性方程组公共解},$\alpha_2-\alpha_1,\ldots,\alpha_s-\alpha_1$是$AX=\vec{0}$的基础解系,因此根据齐次线性方程组反问题的解法可以得到$A$,然后令$b=A\alpha_1$即可符合题意.
\end{solution}

下面我们来看一个具体的例子来运用上面介绍的方法:
\begin{example}{}{}
    已知$\alpha_1=(1,2,-1,0,4)^\mathrm{T},\enspace\alpha_2=(-1,3,2,4,1)^\mathrm{T},\enspace\alpha_3=(2,9,-1,4,13)^\mathrm{T}$,且有$W=\spa(\alpha_1,\alpha_2,\alpha_3)$.
    \begin{enumerate}
        \item 求以$W$为解空间的一个齐次线性方程组;

        \item 求以$W'=\{\eta+\alpha \mid \alpha\in W\}$为解集的一个非齐次线性方程组,其中$\eta=(1,2,1,2,1)^\mathrm{T}$.
    \end{enumerate}
\end{example}

\begin{solution}
    首先我们通过求解极大线性无关组的方法可以得到,$\alpha_1,\alpha_2,\alpha_3$的极大线性无关组是$\alpha_1,\alpha_2$. 然后我们基于此根据前面两个例题介绍的方法,我们有如下求解过程:
    \begin{enumerate}
        \item 设所求的齐次线性方程组为$AX=0$,令$B=(\alpha_1,\alpha_2)$,解线性方程组$B^\mathrm{T}X=0$,得到其基础解系为
              \[\beta_1=(7,-1,5,0,0)^\mathrm{T},\enspace\beta_2=(8,-4,0,5,0)^\mathrm{T},\enspace\beta_3=(-2,-1,0,0,1)^\mathrm{T},\]
              令$A^\mathrm{T}=(\beta_1,\beta_2,\beta_3)$即可,即
              \[A=\begin{pmatrix}
                      7 & -1 & 5 & 0 & 0 \\ 8 & -4 & 0 & 5 & 0 \\ -2 & -1 & 0 & 0 & 1
                  \end{pmatrix}.\]
              故所求的线性方程组为$\begin{cases}
                      7x_1-x_2+5x_3=0 \\ 8x_1-4x_2+5x_4=0 \\ -2x_1-x_2+x_5=0
                  \end{cases}$.

        \item 这一问可以利用前面例子中给出的方法,具体步骤不在此赘述(注意因为$\eta$不在$W$中,因此$W'$中可以有三个线性无关向量,不可想当然认为只有$\alpha_2-\alpha_1$是$AX=0$(设$A$为本题要求的方程组的系数矩阵)的解). 我们这里给出更简单的方法,读者只需回顾根据\autoref{thm:通解加特解} 即可发现奥秘. 事实上,将$x_1=1,x_2=2,x_3=1,x_4=2,x_5=1$代入有$7x_1-x_2+5x_3=10$,$8x_1-4x_2+5x_4=10$,$-2x_1-x_2+x_5=-3$,因此$\eta$就是方程组$\begin{cases}
                      7x_1-x_2+5x_3=10 \\ 8x_1-4x_2+5x_4=10 \\ -2x_1-x_2+x_5=-3
                  \end{cases}$的一个特解,从而根据\autoref{thm:通解加特解} 可以知道上述方程组符合题意.
    \end{enumerate}
\end{solution}

事实上,这里求出的线性方程组不一定唯一,但我们会发现解出的不同线性方程组的系数矩阵之间都可以通过初等行变换相互转化,即这些线性方程组是等价的.

\begin{summary}

    在这一讲中我们综合了之前所学的知识,结束了关于线性方程组一般理论的讨论. 我们首先在讨论了线性方程组有解的充要条件,然后讨论了有解前提下唯一解和无穷解的条件,给出了系数矩阵为方阵的 Cramer 法则. 事实上我们几乎所有关于有解、无解、唯一解的讨论至此意已尽.

    接下来我们进一步讨论了齐次与非齐次线性方程组解的结构. 齐次线性方程组的解空间非常基本,因为它构成了线性空间,它的维数我们也利用线性映射基本定理得出. 而非齐次线性方程组的所有解则是齐次线性方程组解空间的平移(实际上就是所谓仿射子集),我们也在各条性质以及例子中看到了其与对应的齐次线性方程组的解的关联,以及解的线性相关性等.

    除此之外,我们利用上面讨论的定理和性质,结合之前所学的一些秩不等式讨论了线性方程组理论的一些应用,这一节中的内容非常重要,请务必掌握. 最后我们也讨论了四个与线性方程组有关的问题,在题型一般解法的讨论中给出了一些定理的证明,希望读者能在证明中体会本节需要各位熟练运用和理解的证明方法.

    至此,线性方程组的一般理论意已尽,事实上这代表着代数的一个分支的研究基本结束. 但还有很多任务等着我们,让我们先休息一下,走入下一讲史海拾遗对历史的介绍,在闪耀着无数数学家智慧光芒的故事里,开启我们对未竟工作的探索.

\end{summary}

\begin{exercise}
    \exquote[鲁迅,《朝花夕拾》]{即使我说二二得四,三三见九,也没有一字不错. 这些既然都错,则绅士口头的二二得七,三三见千等等,自然就不错了.}

    \begin{exgroup}
        \item 证明以下关于线性方程组解的理论的基本定理:

        第一组(齐次线性方程组解空间的一般理论)
        \begin{enumerate}
            \item 设矩阵 $A \in \mathbf{M}_{m\times n}(\mathbf{F})$,若 $r(A)=r$,则齐次线性方程组 $AX=\vec{0}$ 的解空间 $N(A)$ 是 $\mathbf{F}^n$ 的一个 $n-r$ 维子空间.

            \item 设 $A$ 为 $m \times n$ 矩阵,则
                  \begin{enumerate}
                      \item 齐次线性方程组 $AX=\vec{0}$ 只有零解等价于 $r(A)=n$;

                      \item 齐次线性方程组 $AX=\vec{0}$ 有非零解(无穷解)等价于 $r(A)<n$.
                  \end{enumerate}

            \item 设 $A$ 为 $n$ 阶矩阵,则
                  \begin{enumerate}
                      \item 齐次线性方程组 $AX=\vec{0}$ 只有零解等价于 $|A|\neq 0$;

                      \item 齐次线性方程组 $AX=\vec{0}$ 有非零解(无穷解)等价于 $|A|=0$.
                  \end{enumerate}
        \end{enumerate}

        第二组(非齐次线性方程组解空间的一般理论)
        \begin{enumerate}[resume*]
            \item 对于非齐次线性方程组 $AX=\vec{b}$,下列命题等价:
                  \begin{enumerate}
                      \item $AX=\vec{b}$ 有解;

                      \item $\vec{b} \in R(A)$,即 $\vec{b}$ 可被 $A$ 的列向量组线性表示;

                      \item $r(A,\vec{b})=r(A)$,即增广矩阵的秩等于系数矩阵的秩.
                  \end{enumerate}
        \end{enumerate}

        第三组(线性方程组解的结构的一般理论)
        \begin{enumerate}[resume*]
            \item 设 $X_1,X_2,\ldots,X_s$ 为齐次线性方程组 $AX=\vec{0}$ 的一组解,则 $k_1X_1+k_2X_2+\cdots+k_sX_s$ 也为齐次线性方程组 $AX=\vec{0}$ 的解,其中 $k_1,k_2,\ldots,k_s$ 为任意常数.

            \item 设 $\eta_0$ 为非齐次线性方程组 $AX=\vec{b}$ 的一个解,$X_1,X_2,\ldots,X_s$ 为齐次线性方程组 $AX=\vec{0}$ 的一组解,则 $k_1X_1+k_2X_2+\cdots+k_sX_s+\eta_0$ 也为非齐次线性方程组 $AX=\vec{b}$ 的解.

            \item 设 $\eta_1,\eta_2$ 为非齐次线性方程组 $AX=\vec{b}$ 的两个解,则 $\eta_2-\eta_1$ 为齐次线性方程组 $AX=\vec{0}$ 的解.

            \item 设 $X_1,X_2,\ldots,X_s$ 为非齐次线性方程组 $AX=\vec{b}$ 的一组解,则 $k_1X_1+k_2X_2+\cdots+k_sX_s$ 也为非齐次线性方程组 $AX=\vec{b}$ 的解的充分必要条件是 $k_1+k_2+\cdots+k_s=1$.

            \item 设 $X_1,X_2,\ldots,X_s$ 为非齐次线性方程组 $AX=\vec{b}$ 的一组解,则 $k_1X_1+k_2X_2+\cdots+k_sX_s$ 为齐次线性方程组 $AX=\vec{0}$ 的解的充分必要条件是 $k_1+k_2+\cdots+k_s=0$. 判断以下关于线性方程组解的理论的说法是否正确并说明理由:
        \end{enumerate}

        第四组(一些经典的判断题)
        \begin{enumerate}[resume*]
            \item 方程组 $AX=\vec{b}$ 有唯一解等价于方程组 $AX=\vec{0}$ 只有零解.

            \item 设 $A$ 是 $m \times n$ 矩阵,$B$ 是 $n \times s$ 矩阵,若 $AB=O$,则 $B$ 的列向量为方程组 $AX=\vec{0}$ 的解.

            \item 设 $A$ 是 $n$ 阶非零矩阵,则存在非零矩阵 $B$,使得 $AB=O$ 等价于 $r(A)<n$.

            \item 方程组 $AX=\vec{0}$ 的解为 $BX=\vec{0}$ 的解,则 $r(A) \geqslant r(B)$.

            \item 方程组 $AX=\vec{0}$ 与 $BX=\vec{0}$ 为同解方程组等价于 $r(A)=r(B)$.
        \end{enumerate}
        \begin{answer}
            \begin{enumerate}
                \item 参考教材定理6.1.

                \item 提示. 考虑$A=\begin{pmatrix}
                              a_{11} & a_{12} & \cdots & a_{1n} \\
                              a_{21} & a_{22} & \cdots & a_{2n} \\
                              \vdots & \vdots & \ddots & \vdots \\
                              a_{m1} & a_{m2} & \cdots & a_{mn}
                          \end{pmatrix}$. 令$\alpha_1=\begin{pmatrix}
                              a_{11} \\
                              a_{21} \\
                              \vdots \\
                              a_{m1}
                          \end{pmatrix},\ldots,\alpha_n=\begin{pmatrix}
                              a_{1n} \\
                              a_{2n} \\
                              \vdots \\
                              a_{mn}
                          \end{pmatrix}$. $AX=0\implies x_1\alpha_1+\cdots+x_n\alpha_n=0$\\
                      只有零解表明$\alpha_1,\ldots,\alpha_n$线性无关,故$A$列满秩,$r(A)=n$.\\
                      有非零解(无穷解)则相反.

                \item 提示同上.

                \item 参考教材定理6.2.

                \item $A(k_1X_1+\cdots+k_sX_s)=k_1AX_1+\cdots+k_sAX_s=0+\cdots+0=0$.

                \item $A(k_1X_1+\cdots+k_sX_s+\eta_0)=k_1AX_1+\cdots+k_sAX_s+A\eta_0=b$

                \item $A(\eta_1-\eta_2)=A\eta_1-A\eta_2=b-b=0$

                \item 令$\bar{X}=k_1X_1+\cdots+k_sX_s$,有
                      \begin{align*}
                          A\bar{X}=b & \iff A(k_1X_1+\cdots+k_sX_s)=b        \\
                                     & \iff k_1AX_1+\cdots+k_sAX_s=b         \\
                                     & \iff (k_1+\cdots+k_s)b=b              \\
                                     & \iff k_1+\cdots+k_2=1\quad (b\neq 0).
                      \end{align*}

                \item 类似上题.

                \item 错误.\\
                      必要性正确. 令$A$为$m\times n$的矩阵,则$AX=b$有唯一解表明$r(A)=n$. 而$r(A)=n\implies AX=0$只有零解成立.\\
                      充分性错误. 注意到$AX=0$只有零解$\implies r(A)=n$. 但$r(A)=m$不代表$AX=b$有唯一解,因为还有无解的可能.\\
                      (齐次线性方程组一定有解,但非齐次线性方程组不一定有解,一定要注意这个差别)\\
                      简单的反例如$A=\begin{pmatrix}
                              1 & 0 \\
                              0 & 1 \\
                              0 & 0
                          \end{pmatrix},b=\begin{pmatrix}
                              0 \\
                              0 \\
                              1
                          \end{pmatrix}$

                \item 正确. 令$B=(\beta_1,\ldots,\beta_s)$. $AB=0\implies A(\beta_1,\ldots,\beta_s)=0$,即$(A\beta_1, \ldots, A\beta_s)=0$,故$B$的列向量为方程组$AX=0$的解. (注:$A(\beta_1,\ldots,\beta_s)=(A\beta_1, \ldots, A\beta_s)$利用分块矩阵性质即可)

                \item 利用上一题的结论易知正确.

                \item 设$A$的解空间为$N(A)$,$B$的解空间为$N(B)$,则由题意有$N(A)\subseteq N(B)$,故$\dim{N(A)}\leqslant \dim{N(B)}$. 而$r(A)+\dim{N(A)}=r(B)+\dim{N(B)}$(参考第1题),故$r(A)\geqslant r(B)$正确.

                \item 错误.\\
                      必要性正确. $AX=0$与$BX=0$为同解方程组可得$N(A)=N(B)\implies\dim{N(A)}=\dim{N(B)}$. 同上一题,可得$r(A)=r(B)$.\\
                      充分性错误. 反例有$A=\begin{pmatrix}
                              1 & 0 & 0 \\
                              0 & 1 & 0
                          \end{pmatrix}, B=\begin{pmatrix}
                              1 & 0 & 0 \\
                              0 & 0 & 1
                          \end{pmatrix}$.\\
                      实际上,只需考虑解空间均为$\mathbf{R}^n$的子空间,$\mathbf{R}^n=\spa(\alpha_1,\ldots,\alpha_n)$. 令$N(A)=\spa(\alpha_i), N(B)=\spa(\alpha+j),i\neq j$即为反例.
            \end{enumerate}
        \end{answer}

        \item 设$A$为四阶矩阵,$r(A)<4$,且$A_{21}\neq 0$,求方程组$AX=\vec{0}$的通解.
        \begin{answer}
            $r(A^*)=
              \begin{cases}
                  1 & r(A)=3 \\
                  0 & r(A)<3
              \end{cases}$,而已知$A_{21}\neq 0$,故$r(A^*)=1$,故$r(A)=3$,故$AX = 0$解空间维数为$4-r(A)=1$. 而$|A|=0$,由行列式展开可知
          \[ a_{i1}A_{21}+a_{i2}A_{22}+a_{i3}A_{23}+a_{i4}A_{24}=0 \qquad i=1,2,3,4 \]
          故解为$k(A_{21},A_{22},A_{23},A_{24})^\mathrm{T},\enspace k\in \mathbf{R}$.
        \end{answer}

        \item 设$A=(\alpha_1,\alpha_2,\alpha_3,\alpha_4)$为四阶矩阵,方程组$AX=\vec{0}$的通解为$X=k(1,0,-4,0)^\mathrm{T}$,求$A^*X=0$的基础解系.
        \begin{answer}
            由题意得$r(A)=3<4$且$\alpha_1-4\alpha_3=0$,或$\alpha_1=4\alpha_3$. 由$r(A)=3$得$r(A^*)=1$,$A^*X=0$的基础解系含3个线性无关的解向量. 由$A*A=|A|E=O$得$\alpha_1,\alpha_2,\alpha_3,\alpha_4$为$A^*X=0$的解,从而$\alpha_1,\alpha_2,\alpha_4$或$\alpha_2,\alpha_3,\alpha_4$为$A^*X=0$的基础解系.
        \end{answer}

        \item 设$A$为$n$阶实矩阵,$W=\{\beta\in\mathbf{R}^n \mid \alpha^\mathrm{T}A\beta=0,\enspace \forall \alpha\in\mathbf{R}^n\}$,证明:
        \begin{enumerate}
            \item $\dim W+r(A)=n$;

            \item $W$为$\mathbf{R}^n$的子空间.
        \end{enumerate}
        \begin{answer}
            $\forall \alpha \in \mathbf{R}^n, \alpha^\mathrm{T}A\beta =0\implies A\beta =0$.
        \end{answer}

        \item 已知4级方阵$A=(\alpha_1,\alpha_2,\alpha_3,\alpha_4)$的列向量$\alpha_1,\alpha_2,\alpha_4$线性无关,且$\alpha_1=2\alpha_2-\alpha_3$,若$\beta=\alpha_1-\alpha_2+3\alpha_4$,求方程组$AX=\beta$的通解.
        \begin{answer}
            $r(A)=3\implies \dim{N(A)}=4-r(A)=1$.

          $\beta = x_1\alpha_1+x_2\alpha_2+x_3\alpha_3+x_4\alpha_4\implies$特解为$(1,-1,0,3)^\mathrm{T}$

          $\alpha_1-2\alpha_2+\alpha_3=0\implies$导出组基础解系为$(1,-2,1,0)^\mathrm{T}$.

          故通解为$k(1,-2,1,0)^\mathrm{T}+(1,-1,0,3)^\mathrm{T},\enspace k\in\mathbf{R}$
        \end{answer}

        \item 设四元非齐次线性方程组的系数矩阵的秩为3,已知$\eta_1,\eta_2,\eta_3$是它的三个解向量,且$\eta_1=(2,3,4,5)^\mathrm{T},\eta_2+\eta_3=(1,2,3,4)^\mathrm{T}$,求该方程组的通解.
        \begin{answer}
            设方程组$Ax=b$,由$r(A)=3$知其导出组$Ax=0$的基础解系只含一个解向量.

          而$A\eta_1=b,A(\eta_2+\eta_3)=2b$,故$2\eta_1-(\eta_2+\eta_3)=(3,4,5,6)^\mathrm{T}$为$Ax=0$的基础解系,从而所求通解为
          \[ x = \eta_1+c(3,4,5,6)^\mathrm{T} = (2,3,4,5)^\mathrm{T} + c(3,4,5,6)^\mathrm{T} \]
          其中$c$为任意常数.
        \end{answer}

        \item 设$\beta_1,\beta_2,\beta_3$是$n$元非齐次线性方程组$AX=\vec{b}$的三个线性无关的解,且$r(A)=n-2$,求:
        \begin{enumerate}
            \item 导出组$AX=\vec{0}$的一个基础解系;

            \item $AX=2\vec{b}$的一般解.
        \end{enumerate}
        \begin{answer}
            \begin{enumerate}
                \item 取$\beta_1-\beta_2,\beta_1-\beta_3$(取法不唯一),有$A(\beta_1-\beta_2)=A(\beta_1-\beta_3)=0$且$\beta_1-\beta_2,\beta_1-\beta_3$线性无关,否则$\beta_1,\beta_2,\beta_3$线性相关.

                \item 取$2\beta_1+k_1(\beta_1+\beta_2)+k_2(\beta_2+\beta_3)$可行(取法不唯一).
            \end{enumerate}
        \end{answer}

        \item 已知$A$是一个$s\times n$矩阵,证明:线性方程组$AX=\vec{b}$对任意列向量$\vec{b}_{s\times 1}$都有解的充要条件是$A$行满秩.
        \begin{answer}
            对于任意的$b_{s\times 1}$,$AX=b$都有解,说明$A$的列向量可以线性表出出任意$s$维向量$b$,从而$r(A)\geqslant s$,而$r(A)\leqslant s$,所以$r(A)=s$.

          反之,如果$r(A)=s$,则$A$有$s$个线性无关的列向量,它们就是$s$维空间$V$的一组基,对于$V$中任意向量$b$,都可以由$A$的列向量线性表出,从而$AX=b$有解.

          当然,也可以直接用秩不等式:$s=r(A)\leqslant r(A,b)\leqslant s$得到$r(A,b)=s$,所以$AX=b$有解.
        \end{answer}

        \item 设$A,B$分别是$m \times n$和$n \times s$矩阵,且$r(B)=n$,证明:若$AB=O$,则$A=O$.
        \begin{answer}
            $AB=0\implies r(A)+r(B)\leqslant n,r(B)=n\implies r(A)\leqslant 0\implies A=O$.
        \end{answer}

        \item 设$A \in \mathbf{F}^{m \times n},B \in \mathbf{F}^{(n-m) \times n}\enspace(m<n)$,$V_1,V_2$分别为齐次线性方程组$AX=\vec{0}$和$BX=\vec{0}$的解空间,证明:$\mathbf{F}^n=V_1\oplus V_2$的解的充要条件是$\begin{pmatrix} A \\ B \end{pmatrix}X=\vec{0}$只有零解.
        \begin{answer}
            必要性. 有条件可知$V_1\cap V_2=\{0\}$,$\begin{pmatrix}A\\B\end{pmatrix}\in\mathbf{F}^{n\times n}$,而$\begin{pmatrix}A\\B\end{pmatrix}x=0$只有零解,故$\begin{pmatrix}A\\B\end{pmatrix}$可逆,从而$r(A)=m,r(B)=n-m$,于是
          \begin{align*}
              \dim V_1 & =n-r(A)=n-m,       \\
              \dim V_2 & =n-r(B)=n-(n-m)=m.
          \end{align*}
          又$V_1+V_2$是$\mathbf{F}^n$的子空间,且$\dim(V_1+V_2)=\dim V_1+\dim V_2-\dim(V_1\cap V_2)=\dim\mathbf{F}^n$,故$\mathbf{F}^n=V_1\oplus V_2$.

          充分性. 若$\begin{pmatrix}A\\B\end{pmatrix}x=0$有非零解$x_1$,则$x_1\in V_1\cap\V_2$. 这与$\mathbf{F}^n=V_1\oplus V_2$矛盾.
        \end{answer}

        \item 齐次线性方程组\[\begin{cases}
                x_2+ax_3+bx_4=0  \\
                -x_1+cx_3+dx_4=0 \\
                ax_1+cx_2-ex_4=0 \\
                bx_1+dx_2+ex_3=0
            \end{cases}\]的一般解以$x_3,x_4$作为自由未知量.
        \begin{enumerate}
            \item 求$a,b,c,d,e$满足的的条件;

            \item 求齐次线性方程组的基础解系.
        \end{enumerate}
        \begin{answer}
            \begin{enumerate}
                \item 由条件知,方程组系数矩阵的为2,系数矩阵
                      \[ A=\begin{pmatrix}
                              0  & 1 & a & b  \\
                              -1 & 0 & c & d  \\
                              a  & c & 0 & -e \\
                              b  & d & e & 0
                          \end{pmatrix}\rightarrow
                          \begin{pmatrix}
                              0  & 1 & a          & b       \\
                              -1 & 0 & c          & d       \\
                              0  & 0 & 0          & ad-e-bc \\
                              0  & 0 & -(ad-e-bc) & 0
                          \end{pmatrix} \]
                      故$ad-e-bc=0$.

                \item 易求得基础解系为$(c,-a,1,0)^\mathrm{T},(d,-b,0,1)^\mathrm{T}$.
            \end{enumerate}
        \end{answer}
    \end{exgroup}

    \begin{exgroup}
        \item 设$A$为方阵,证明:$A^2=E \iff r(A+E)+r(A-E)=n$.
        \begin{answer}
        \begin{enumerate}
            \item 充分性:有 $(A+E)(A-E) = A^2 - E = 0$, 由定理 $BC=0 \implies r(B)+r(C) \leq n$,得 $r(A+E)+r(A-E) \leq n$;而 $r(A+E)+r(A-E) \geq r(A+E+A-E) = n$,故 $r(A+E)+r(A-E)=n$.
            \item 必要性:由$r(A+E)+r(A-E)=n$可得出 $A$ 的特征值仅可能为 1 或者 -1,并且可以对角化.

                因此,记 $A = PBP^{-1}$,其中 $B$ 为对角矩阵,则 $(A+E)(A-E)=P(B-E)(B+E)P^{-1}=0$,故命题得证.
        \end{enumerate}
        \end{answer}

        \item 设$a_1,a_2,\ldots,a_n$为互不相等的实数,$b_1,b_2,\ldots,b_n$为任意给定的实数. 证明:存在唯一的$n-1$次多项式,满足$f(a_i)=b_i,\enspace i=1,2,\ldots,n$.
        \begin{answer}
            设 $f(x) = c_0+c_1x+c_2x^2+\cdots+c_{n-1}x^{n-1}$,则有
            \[\begin{cases} \begin{aligned}
                        c_0+c_1a_1+c_2a_1^2+\cdots+c_{n-1}a_1^{n-1} & = b_1,            \\
                        c_0+c_1a_2+c_2a_2^2+\cdots+c_{n-1}a_2^{n-1} & = b_2,            \\
                                                                    & \vdotswithin{ = } \\
                        c_0+c_1a_n+c_2a_n^2+\cdots+c_{n-1}a_n^{n-1} & = b_n.            \\
                    \end{aligned} \end{cases}\]
            注意此处我们研究的对象是 $(c_0, c_1, \ldots, c_{n-1})^{\mathrm{T}}$. 因为系数行列式
            \[D = \begin{vmatrix}
                    1      & a_1    & \cdots & a_1^{n-1} \\
                    1      & a_2    & \cdots & a_2^{n-1} \\
                    \vdots & \vdots & \ddots & \vdots    \\
                    1      & a_n    & \cdots & a_n^{n-1} \\
                \end{vmatrix} = \prod_{1 \leqslant j < i \leqslant n} (a_i-a_j) \neq 0.\]
            所以由 Cramer 法则,上述关于 $(c_0, c_1, \ldots, c_{n-1})^{\mathrm{T}}$ 的方程组有唯一解,所以满足条件的多项式函数 $f$ 是唯一存在的.
        \end{answer}

        \item 设$A$为$m \times n$矩阵,$r(A)=m$,$B$是$m$阶可逆矩阵,已知$A$的行空间$R(A^\mathrm{T})$是方程组$CX=\vec{0}$的解空间,证明:$BA$的行向量也是$CX=\vec{0}$的一个基础解系.
        \begin{answer}
            由$r(A)=m$可知,$A$的$m$个行向量($A^\mathrm{T}$的$m$个列向量)线性无关,它们是方程组$CX=0$的一个基础解系. 由$CA^\mathrm{T}=O$和$n-r(C)=m$,得$r(C)=n-m$. 因此,由$CA^\mathrm{T}B^\mathrm{T}=C(BA^\mathrm{T})=O$,可知$(BA)^\mathrm{T}$的$m$个行向量线性无关,它们也是$CX=0$的一个基础解系.
        \end{answer}

        \item 设$A$是数域$\mathbf{F}$上的一个$n$阶可逆方阵,$A$的前$r$个行向量组成的矩阵为$B$,后$n-r$个行向量组成的矩阵为$C$,$n$元线性方程组$BX=0$与$CX=0$的解空间分别为$V_1,V_2$. 证明:$\mathbf{F}^n=V_1\oplus V_2$.
        \begin{answer}
            先记$W=V_1+V_2$,若$\alpha\in V_1\cap V_2$,则$B\alpha=C\alpha=0$,所以
            \[A\alpha=\begin{pmatrix}
                    B \\
                    C
                \end{pmatrix}\alpha=0,\]
            由于$A$可逆,因此$\alpha=0$,即$V_1\cap V_2=\{0\}$,因此$V_1+V_2$是直和,因此只需证$W=\mathbf{F}^n$即可. 事实上,我们知道$r(B)=r,r(C)=n-r$,因此$\dim V_1=n-r,\enspace \dim V_2=n-(n-r)=r$,所以
            \[\dim W=\dim V_1+\dim V_2=n-r+r=n=\dim \mathbf{F}^n,\]
            又$W=V_1\oplus V_2\subseteq \mathbf{F}^n$,因此$W=\mathbf{F}^n$,故得证.
        \end{answer}

        \item 设$A$是$n$阶矩阵,且$A_{11}\neq 0$,证明:方程组$AX=\vec{b}$($\vec{b}$为非零向量)有无穷多解的充要条件为$A^*\vec{b}=\vec{0}$.
        \begin{answer}
            必要性:设$AX=b$有无穷多解,则$r(A)<n$,从而$|A|=0$,于是$A^*b=A^*AX=|A|X=0$.

          充分性:设$A^*b=0$,即方程组$A^*b=0$有非零解,则$r(A^*)<n$,又$A_{11}\neq 0$,所以$r(A)=n-1$.

          令$A=(\alpha_1,\alpha_2,\ldots,\alpha_n)$,因为$A^*A=|A|E=O$,所以$\alpha_1,\alpha_2,\ldots,\alpha_n$为方程组$A^*X=0$的解,又因为$A_{11}\neq 0$,所以$\alpha_1,\alpha_2,\ldots,\alpha_n$线性无关.

          由$r(A^*)=1$,得方程组$A^*X=0$的基础解系含有$n-1$个线性无关的解向量,所以$\alpha_2,\alpha_3,\ldots,\alpha_n$为方程组$A^*X=0$的一个基础解系.

          因为$A^*b=0$,所以$b$为方程组$A^*X=0$的一个解,从而$b$可由$\alpha_2,\alpha_3,\ldots,\alpha_n$线性表示,$b$也可由$\alpha_1,\alpha_2,\ldots,\alpha_n$线性表示,于是$r(A) = r\begin{pmatrix}A & b\end{pmatrix}=n-1<n$,故方程组$AX=b$有无穷多解.
        \end{answer}

        \item 若$n$阶矩阵$A$的各行、各列元素之和都为0,证明:$|A|$的所有元素的代数余子式都相等.
        \begin{answer}
            由各列的元素之和等于0,得$|A|=0$. 利用教材第6章习题7的结论:

          如果$r(A)<n-1$,则$r(A^*)=0$,$A^*=O$,所以,$A_{ij}=0,\enspace i,j=1,2,\ldots,n$;

          如果$r(A)=n-1$,则$r(A^*)=1$,且$AA^*=O$,于是$A^*$的每一列$(A_{i1},A_{i2},\cdots A_{in})^\mathrm{T},\enspace i=1,2,\ldots,n$都是$AX=0$的解.

          由于$AX=0$的解空间的维数为
          \[ n-r(A)=n-(n-1)=1 \]
          所以,$AX=0$的任意两个解成比例. 又元素全部为1的$n$元向量$e=(1,1,\ldots,1)^\mathrm{T}$满足方程组$AX=0$,因此$A^*$任意一列都与$e$成比例,即
          \[ A_{i1}=A_{i2}=\cdots=A_{in} \qquad i=1,2,\ldots,n \]
          所以,$A^*$的每一列元素(即$|A|$的每一行元素的代数余子式)都相等.

          同理,$(A^\mathrm{T})^*$的每一列元素(即$|A^\mathrm{T}|$的每一行元素,也是$|A|$的每一列元素的代数余子式)都相等,即
          \[ A_{1j}=A_{2j}=\cdots=A_{nj} \qquad j=1,2,\ldots,n \]
        \end{answer}

        \item 已知 $\alpha_1,\alpha_2,\ldots,\alpha_s$ 是齐次线性方程组 $AX=\vec{0}$ 的一组基础解系,向量组
        \[\beta_1=t_1\alpha_1+t_2\alpha_2,\ \beta_2=t_1\alpha_2+t_2\alpha_3,\ \ldots,\ \beta_{s-1}=t_1\alpha_{s-1}+t_2\alpha_s\]
        试问当实数 $t_1,t_2$ 满足何条件时,$AX=\vec{0}$ 有基础解系包含向量 $\beta_1,\beta_2,\ldots,\beta_{s-1}$,并写出该基础解系中的其余向量.
        \begin{answer}
            见2019-2020学年线性代数I(H)期末第五题.
        \end{answer}

        \item 已知线性方程组
        \[\begin{cases} \begin{aligned}
                    a_{11}x_1+a_{12}x_2+\cdots+a_{1,2n}x_{2n} & =0              \\
                    a_{21}x_1+a_{22}x_2+\cdots+a_{2,2n}x_{2n} & =0              \\
                                                              & \vdotswithin{=} \\
                    a_{n1}x_1+a_{n2}x_2+\cdots+a_{n,2n}x_{2n} & =0              \\
                \end{aligned}\end{cases}\]
        的一个基础解系为$(b_{11},b_{12},\ldots,b_{1,2n})^\mathrm{T},(b_{21},b_{22},\ldots,b_{2,2n})^\mathrm{T},(b_{n1},b_{n2},\ldots,b_{n,2n})^\mathrm{T}$,求解线性方程组
        \[\begin{cases} \begin{aligned}
                    b_{11}x_1+b_{12}x_2+\cdots+b_{1,2n}x_{2n} & =0              \\
                    b_{21}x_1+b_{22}x_2+\cdots+b_{2,2n}x_{2n} & =0              \\
                                                              & \vdotswithin{=} \\
                    b_{n1}x_1+b_{n2}x_2+\cdots+b_{n,2n}x_{2n} & =0              \\
                \end{aligned} \end{cases}.\]
        (注:本题的一般形式:设$A=(a_{ij})_{m\times n},m<n,r(A)=m$;$B=(b_{ij})_{n\times(n-m)},m<n,r(B)=n-m$.已知齐次线性方程组$A\vec{X}=\vec{0}$的解空间$N(A)=R(B)$(矩阵$B$的列空间),求齐次线性方程组$\displaystyle\sum_{i=1}^{n}b_{ij}y_i=0,j=1,2,\ldots,n-m$的一个基础解系.)
        \begin{answer}
            令原方程组$AX=0$,解系组成矩阵$B$. 则新方程组为$B^\mathrm{T}X=0$. 而$B^\mathrm{T}A^\mathrm{T}=(AB)^\mathrm{T}=0$,推测$A^\mathrm{T}$的列向量为基础解系. 而由维数公式,
          \[ r(A)=\dim V-\dim\ker(A) = 2n-n=n. \]
          故$A^\mathrm{T}$列空间维数为$n$,即$r_c(A^\mathrm{T})=n$. 而由题意$r(B)=n$(否则不是基础解系),故
          \[ \dim{\ker(B)}=\dim{V^{\prime}}-r(B)=2n-n=n=r_c(A^\mathrm{T}). \]
          猜想成立.
        \end{answer}

        \item 设$A,B\in \mathbf{F}^{n\times n}$,且$r(A)=r,\enspace r(B)=s,\enspace r\begin{pmatrix} A \\ B \end{pmatrix}=k$.
        \begin{enumerate}
            \item 证明:满足$AX=O$的$n$阶方阵$X$全体构成$\mathbf{F}^{n\times n}$的子空间,并求其维数;

            \item 令满足$AX=O$的$n$阶方阵$X$全体构成的子空间为$V_1$,满足$BX=O$的$n$阶方阵$X$全体构成的子空间为$V_2$,求$V_1+V_2$的维数.
        \end{enumerate}
        \begin{answer}
            \begin{enumerate}
                \item 略.

                \item \begin{align*}
                          \dim(V_1+V_2) & =\dim V_1+\dim V_2-\dim(V_1\cap V_2) \\
                                        & =n(n-r)+n(n-s)-n(n-k).
                      \end{align*}
            \end{enumerate}
        \end{answer}

        \item 设$S(A)=\{B \in \mathbf{M}_n(\mathbf{F}) \mid AB=0\}$.
        \begin{enumerate}
            \item 证明:$S(A)$为$\mathbf{M}_n(\mathbf{F})$的子空间;

            \item 设$r(A)=r$,求$S(A)$的一组基和维数.
        \end{enumerate}

        \item 设$A$是元素全为1的$n$阶方阵.
        \begin{enumerate}
            \item 求行列式$|aE+bA|$,其中$a,b$为实常数;

            \item 已知$1<r(aE+bA)<n$,试确定$a,b$满足的条件,并求下列线性子空间的维数:
                  \[W=\{x \mid (aE+bA)x=0,\enspace x\in\mathbf{R}\}.\]
        \end{enumerate}
        \begin{answer}
            \begin{enumerate}
                \item 容易计算$|aE=bA|=(a+nb)a^{n-1}$.

                \item 由$1<r(aE+bA)<n$知$|aE+bA|=0$. 故$a\neq 0$,且$a+nb=0$,此时$aE+bA$左上角的$n-1$阶子式
                      \[ \begin{vmatrix}
                              a+b    & b      & \cdots & b      \\
                              b      & a+b    & \cdots & b      \\
                              \vdots & \vdots & \ddots & \vdots \\
                              b      & b      & \cdots & a+b
                          \end{vmatrix}=(a+(n-1)b)a^{n-2}=\frac{a^{n-1}}{n}\neq 0, \]
                      故$\dim{W}=n-r(aE+bA)=n-(n-1)=1$.
            \end{enumerate}
        \end{answer}

        \item 已知线性方程组
        \[\begin{cases} \begin{aligned}
                    a_{11}x_1+a_{12}x_2+\cdots+a_{1n}x_n & =b_1            \\
                    a_{21}x_1+a_{22}x_2+\cdots+a_{2n}x_n & =b_2            \\
                                                         & \vdotswithin{=} \\
                    a_{n1}x_1+a_{n2}x_2+\cdots+a_{nn}x_n & =b_n            \\
                \end{aligned} \end{cases}\]
        的系数矩阵与
        \[\begin{pmatrix}
                a_{11} & a_{12} & \cdots & a_{1n} & b_1    \\
                a_{21} & a_{22} & \cdots & a_{2n} & b_2    \\
                \vdots & \vdots & \ddots & \vdots & \vdots \\
                a_{n1} & a_{n2} & \cdots & a_{nn} & b_n    \\
                b_1    & b_2    & \cdots & b_n    & 0
            \end{pmatrix}\]
        秩相等,求证:上述线性方程组有解.

        \item 设$A=(a_{ij})_{m\times n}$,$\vec{b}$和$X$为$m$元列向量,$Y$为$n$元列向量,证明:
        \begin{enumerate}
            \item 若$AY=\vec{b}$有解,则$A^\mathrm{T}X=\vec{0}$的任一组解都满足$\vec{b}^\mathrm{T}X=\vec{0}$;

            \item 方程组$AY=\vec{b}$有解的充要条件是方程组$\begin{pmatrix}
                          A^\mathrm{T} \\ \vec{b}^\mathrm{T}
                      \end{pmatrix}X=\begin{pmatrix}
                          \vec{0} \\ 1
                      \end{pmatrix}$无解(其中$\vec{0}$是$n$元零向量).
        \end{enumerate}
        \begin{answer}
            见教材P213第6题.
        \end{answer}

        \item 判断:设 $A$ 是复数域上 $m \times n$ 阶矩阵,则矩阵秩 $r\left(A^T A\right)=r(A)$.
        \begin{answer}
            假. 取$A=\begin{pmatrix}
                1-i & 1+i  \\
                1+i & -1+i
            \end{pmatrix}$,有$A^\mathrm{T}A=0$.
        \end{answer}

        \item 证明:对于$m \times n$实矩阵$A$,方程$A^\mathrm{T}AX = A^\mathrm{T}\vec{b}$总是有解,且$A$为方阵时,$A^\mathrm{T}AX = \vec{0}$和$AX=\vec{0}$同解. 当$r(A)=n$时求其解,并证明$A(A^\mathrm{T}A)^{-1}A^\mathrm{T}$是幂等的对称矩阵.
        \begin{answer}
            见教材P214第8题.
        \end{answer}

        \item 设$A,B,C$为$n$阶实方阵,且$BAA^\mathrm{T}=CAA^\mathrm{T}$,证明:$BA=CA$.
        \begin{answer}
            由$XA=0$与$XAA^{\mathrm{T}}=0$同解. 又由条件知$(C-B)AA^\mathrm{T}=0$,故$(C-B)A=0$. 即$CA=BA$.
        \end{answer}

        \item 设$A$为数域$\mathbf{F}$上的$n$阶方阵,又设线性空间$\mathbf{F^n}$的两个子空间为$W_1=\{X\in\mathbf{F}^n \mid AX=\vec{0}\}$,$W_2=\{X\in\mathbf{F}^n \mid (A-E)X=\vec{0}\}$. 证明:$A^2=A \iff \mathbf{F}^n=W_1\oplus W_2$.
        \begin{answer}
            由$r(A)+r(E-A)=n \iff A^2=A$, 易证.
        \end{answer}

        \item $n$阶方阵$A,B$满足$AB=BA$,证明:$r(AB)+r(A+B) \leqslant r(A)+r(B)$.
        \begin{answer}
            方法一. 用分块矩阵的方法,我们知道
          \[ \begin{pmatrix}
                  A & O \\
                  O & B
              \end{pmatrix}
              \rightarrow
              \begin{pmatrix}
                  A & O \\
                  A & B
              \end{pmatrix}
              \rightarrow
              \begin{pmatrix}
                  A & A   \\
                  A & A+B
              \end{pmatrix}. \]
          结合$AB=BA$,我们知道
          \[ \begin{pmatrix}
                  A & A   \\
                  A & A+B
              \end{pmatrix}
              \underbrace{
                  \begin{pmatrix}
                      A+B & O \\
                      -A  & E
                  \end{pmatrix}
              }_{\text{非广义初等变换,难以想到}}
              =
              \begin{pmatrix}
                  AB & A   \\
                  O  & A+B
              \end{pmatrix}. \]
          于是
          \[ r(A)+r(B)=r
              \begin{pmatrix}
                  A & O \\
                  O & B
              \end{pmatrix}=
              \begin{pmatrix}
                  A & A   \\
                  A & A+B
              \end{pmatrix}\geqslant r
              \begin{pmatrix}
                  AB & A   \\
                  O  & A+B
              \end{pmatrix}\geqslant
              r(AB)+r(A+B). \]
          方法二. 设方程组$AX=0$与$BX=0$的解空间分别是$V_1, V_2$,方程组$ABX=BAX=0$与$(A+B)X=0$的解空间分别为$W_1, W_2$,则$V_1\subseteq W_1, V_2\subseteq W_1$,从而$V_1+V_2\subseteq W_1$,同时$V_1\cap V_2\subseteq W_1$,同时$V_1\cap V_2\subseteq W_2$,利用维数公式就有
          \[ \dim V_1+\dim V_2=\dim(V_1+V_2)+\dim(V_1\cap V_2)\leqslant \dim W_1+\dim W_2. \]
          即
          \[ (n-r(A))+(n-r(B))\leqslant (n-r(AB))+(n-r(A+B)). \]
          化简便知$r(A)+r(B)\geqslant r(AB)+r(A+B)$.
        \end{answer}

        \item 请按序证明以下结论:
        \begin{enumerate}
            \item $A,B$分别是$s \times m,m \times n$矩阵,则$ABX=\vec{0}$与$BX=\vec{0}$同解的充要条件是$r(AB)=r(B)$;

            \item $A,B$分别是$s \times m,m \times n$矩阵,且$r(AB)=r(B)$,则对任意的$n \times t$矩阵都有$r(ABC)=r(BC)$;

            \item 设$A$是$n$阶方阵,则存在正整数$k$使得$r(A^k)=r(A^{k+1})=r(A^{k+2})=\cdots$,且对任意正整数$m$,有$r(A^n)=r(A^{n+m})$.
        \end{enumerate}
        \begin{answer}
            \begin{enumerate}
                \item 必要性:$ABX=0$与$BX=0$同解可知它们基础解系所含向量个数相同,即
                      \[ n-r(AB)=n-r(B)\implies r(AB)=r(B). \]
                      充分性:由必要性,当$r(AB)=r(B)$时,$ABX=0$与$BX=0$的基础解系所含向量个数相同,而$BX=0$的解都是$ABX=0$的解,所以$ABX=0$与$BX=0$同解.

                \item $r(AB)=r(B)$说明$ABX=0$与$BX=0$同解,用$CX$代替$X$就得到$ABCX=0$与$BCX=0$同解,从而有$r(ABC)=r(BC)$.\\
                      \textbf{推论.}设$A$是一个方阵,且存在正整数$k$使得$r(A^{k+1})=r(A^k)$,递推就有
                      \[ r(A^k)=r(A^{k+1})=r(A^{k+2})=\cdots. \]

                \item 当$A$可逆时,结论是显然的. 当$A$不可逆时,有$r(A)\leqslant n-1$,现在考虑$n+1$个矩阵$A,A^2,\ldots,A^{n+1}$,有
                      \[ n-1\geqslant r(A)\geqslant r(A^2)\geqslant\cdots\geqslant r(A^n)\geqslant r(A^{n+1})\geqslant 0. \]
                      这$n+1$个矩阵的秩只能从$0,1,\ldots,n-1$这$n$个数中取,所以必有两个矩阵的秩相同,即存在$m(1\leqslant m\leqslant n)$使得$r(A^m)=r(A^{m+1})$,由上面的推论可得:
                      \[ r(A^m)=r(A^{m+1})=\cdots=r(A^{n})=r(A^{n+1})=\cdots. \]
                      特别的,对于任意的正整数$k$有$r(A^n)=r(A^{n+k})$.
            \end{enumerate}
        \end{answer}

        \item 如果齐次线性方程组\[\begin{cases}
                x_1+x_2+bx_3-x_4+x_5=0   \\
                2x_1+3x_2+x_3+x_4-2x_5=0 \\
                x_2+ax_3+3x_4-4x_5=0     \\
                -3x_1-3x_2-3bx_3+bx_4+(a+2)x_5=0
            \end{cases}\]的解空间是3维的,试求$a,b$的值与解空间的基. 解空间可能为2维吗?
        \begin{answer}
            增广矩阵$A=\begin{pmatrix}
                1  & 1  & b   & -1 & 1   \\
                2  & 3  & 1   & 1  & -2  \\
                0  & 1  & a   & 3  & -4  \\
                -3 & -3 & -3b & b  & a+2
            \end{pmatrix}$. 由解空间维数为3可知,$r(A)=2$. 由于$\beta_1=\begin{pmatrix}
                1 \\
                2 \\
                0 \\
                3
            \end{pmatrix}
            \beta_2=\begin{pmatrix}
                1 \\
                3 \\
                1 \\
                -3
            \end{pmatrix}$线性无关,则后三列必能被其表示. 对于$\alpha_2=\begin{pmatrix}
                -1 \\
                1  \\
                3  \\
                b
            \end{pmatrix}=k_1\beta_1+k_2\beta_2\implies k_1=-4,k_2=3\implies b=3$\\
        $\alpha_3=\begin{pmatrix}
                1  \\
                -2 \\
                -4 \\
                a+2
            \end{pmatrix}=k_1\beta_1+k_2\beta_2\implies k_1=5,k_2=1\implies a=-25$. \\
        而$\alpha_1=\begin{pmatrix}
                b \\
                1 \\
                a \\
                -3b
            \end{pmatrix}=\begin{pmatrix}
                3  \\
                1  \\
                -5 \\
                -9
            \end{pmatrix}=k_1\beta_1+k_2\beta_2\implies k_1=8,k_2=-5$成立. 故$a=-5,b=3$. 解空间的基略.\\
        若解空间为2维,则$r(A)=3$,令$A=\begin{pmatrix}
                \alpha_1 \\
                \alpha_2 \\
                \alpha_3 \\
                \alpha_4
            \end{pmatrix}$,由于$\alpha_1,\alpha_2$必线性无关,故$\alpha_3,\alpha_4$中有且仅有一个可被$\alpha_1,\alpha_2$表示. 若为$\alpha_3$,$\alpha_2-2\alpha_1=(0,1,1-2b,3,-4)=(0,1,4,3,-4)$,则$a=1-2b$. 故$a+2b=1$,且$\alpha_4$不能被表示,故$\alpha_4\neq -3\alpha_1\implies b\neq 3, a\neq -5$. 故$\begin{cases}
                a+2b=1 \\
                b\neq 3, a\neq -5
            \end{cases}$即可.
        \end{answer}

        \item 设$W_1,W_2$分别为$n$元齐次线性方程组$AX=\vec{0}$和$BX=\vec{0}$的解空间,试构造两个$n$元齐次线性方程组,使它们的解空间分别为$W_1 \cap W_2$和$W_1+W_2$.
        \begin{answer}
            $W_1\cap W_2:\begin{pmatrix}
                  A \\ B
              \end{pmatrix}X=0$. \qquad
          $W_1+ W_2:\begin{cases}
                  AX = 0 \\
                  BX = 0
              \end{cases}$.
        \end{answer}

        \item 已知方程组$\begin{cases}
                x_1+x_2+ax_3+x_4=1 \\ -x_1+x_2-x_3+bx_4=2 \\ 2x_1+x_2+x_3+x_4=c
            \end{cases}$与$\begin{cases}
                x_1+x_4=-1 \\ x_2-2x_4=d \\ x_3+x_4=e
            \end{cases}$同解,求$a,b,c,d,e$.
        \begin{answer}
            方法一. 令$x_4=t$,则方程组$\begin{cases}
                x_1+x_4=-1 \\
                x_2-2x_4=d \\
                x_3+x_4=e
            \end{cases}$的一般解为$\begin{cases}
                x_1=-1-t \\
                x_2=d+2t \\
                x_3=e-t  \\
                x_4=t
            \end{cases}$.

        代入方程组$\begin{cases}
                x_1+x_2+ax_3+x_4=1  \\
                -x_1+x_2-x_3+bx_4=2 \\
                2x_1+x_2+x_3+x_4=c
            \end{cases}$可得$\begin{cases}
                (2-a)t=2-d-ae \\
                (b+4)t=1-d+e  \\
                0=c-d-e+2
            \end{cases}$. 由$t$的任意性,可得$a=2,b=-4$. 从而$\begin{cases}
                0=2-d-2e \\
                0=1-d+e  \\
                0=c-d-e+2
            \end{cases}$,解得$d=\dfrac{4}{3},e=\dfrac{1}{3},c=-\dfrac{1}{3}$.

        方法二. 由于
        \[\begin{pmatrix}
                A & b \\
                B & d
            \end{pmatrix}=
            \begin{pmatrix}
                1  & 1 & a  & 1  & 1  \\
                -1 & 1 & -1 & b  & 2  \\
                2  & 1 & 1  & 1  & c  \\
                1  & 0 & 0  & 1  & -1 \\
                0  & 1 & 0  & -2 & d  \\
                0  & 0 & 1  & 1  & e
            \end{pmatrix}\rightarrow
            \begin{pmatrix}
                1 & 0 & 0 & 0     & 0            \\
                0 & 0 & 0 & b-a+6 & 3-2d+(1-a)e  \\
                0 & 0 & 0 & 2a-4  & c-2+d(2a-1)e \\
                0 & 0 & 0 & a-2   & d-2+ae       \\
                0 & 1 & 0 & 0     & 0            \\
                0 & 0 & 1 & 0     & 0
            \end{pmatrix}.\]
        易知$r(B)=3$,由$r\begin{pmatrix}
                A & b \\
                B & d
            \end{pmatrix}=r(B)$可得$\begin{cases}
                a-2=0          \\
                2a-4=0         \\
                b-a+6=0        \\
                3-2d+(1-a)e=0  \\
                c-2+d(2a-1)e=0 \\
                d-2+ae=0
            \end{cases}$,解得$\begin{cases}
                a=2          \\
                b=-4         \\
                c=-\dfrac 13 \\[1ex]
                d=\dfrac 43  \\[1ex]
                e=\dfrac 13
            \end{cases}$.
        \end{answer}

        \item 设有两个非齐次线性方程组 (1) 和 (2),它们的通解分别是$X=\gamma+t_1\eta_1+t_2\eta_2=\delta+k_1\xi_1+k_2\xi_2$. 其中$\gamma=(5,-3,0,0)^\mathrm{T},\eta_1=(-6,5,1,0)^\mathrm{T},\eta_2=(-5,4,0,1)^\mathrm{T},\delta=(-11,3,0,0)^\mathrm{T},\xi_1=(8,-1,1,0)^\mathrm{T},\xi_2=(10,-2,0,1)^\mathrm{T}$,求这两个方程组的公共解.
        \begin{answer}
            注意如果$X$是两个方程组的公共解,这等价于存在$t_1,t_2,k_1,k_2$使得
          \[ X=\gamma+t_1\eta_1+t_2\eta_2=\delta+k_1\xi_1+k_2\xi_2. \]
          从而对应有
          \[ t_1\eta_1+t_2\eta_2-k_1\xi_1-k_2\xi_2=\gamma-\delta. \]
          将$t_1,t_2,k_1,k_2$设为未知量,对上述方程组的增广矩阵进行初等行变换,可得
          \[ (\eta_1,\eta_2,-\xi_1,-\xi_2,\delta-\gamma)=
              \begin{pmatrix}
                  -6 & -5 & -8 & -10 & -16 \\
                  5  & 4  & 1  & 2   & 6   \\
                  1  & 0  & -1 & 0   & 0   \\
                  0  & 1  & 0  & -1  & 0
              \end{pmatrix}\rightarrow
              \begin{pmatrix}
                  1 & 0 & -1 & 0  & 0  \\
                  0 & 1 & 0  & -1 & 0  \\
                  0 & 0 & 1  & 1  & 1  \\
                  0 & 0 & 0  & 1  & 2.
              \end{pmatrix} \]
          故$(t_1,t_2,k_1,k_2)'$有唯一解$(-1,2,-1,2)'$. 因此公共解为
          \[ X=\gamma-\eta_1+2\eta_2=\delta-\xi_1+2\xi_2=(1,0,-1,2)'. \]
        \end{answer}

        \item 若方程组$\begin{cases}
                x_1+x_2+x_3=0   \\
                x_1+2x_2+ax_3=0 \\
                x_1+4x_2+a^2x_3=0
            \end{cases}$与$x_1+2x_2+x_3=a-1$有公共解,求$a$的值及所有公共解.
        \begin{answer}
            方程组 (1)、(2) 有公共解,即方程组$\begin{cases}
                x_1+x_2+x_3=0,     \\
                x_1+2x_2+ax_3=0,   \\
                x_1+4x_2+a^2x_3=0, \\
                x_1+2x_2+x_3=a-1
            \end{cases}$有解,对其增广矩阵进行初等行变换,
        \begin{align*}
            \bar{A}= & \begin{pmatrix}
                           1 & 1 & 1   & 0   \\
                           1 & 2 & a   & 0   \\
                           1 & 4 & a^2 & 0   \\
                           1 & 2 & 1   & a-1
                       \end{pmatrix}\rightarrow
            \begin{pmatrix}
                1 & 1 & 1     & 0   \\
                0 & 1 & a-1   & 0   \\
                0 & 3 & a^2-1 & 0   \\
                0 & 1 & 0     & a-1
            \end{pmatrix}\rightarrow            \\
                     & \begin{pmatrix}
                           1 & 1 & 1        & 0   \\
                           0 & 1 & a-1      & 0   \\
                           0 & 0 & a^2-3a+2 & 0   \\
                           0 & 0 & 1-a      & a-1
                       \end{pmatrix}\rightarrow
            \begin{pmatrix}
                1 & 1 & 1   & 0          \\
                0 & 1 & a-1 & 0          \\
                0 & 0 & 1-a & a-1        \\
                0 & 0 & 0   & (a-1)(a-2)
            \end{pmatrix}.
        \end{align*}
        当$a\neq 1$且$a\neq 2$时,方程组 (1)、(2) 没有公共解.

        当$a=1$时,$\bar{A}\rightarrow\begin{pmatrix}
                1 & 1 & 1 & 0 \\
                0 & 1 & 0 & 0 \\
                0 & 0 & 0 & 0 \\
                0 & 0 & 0 & 0
            \end{pmatrix}\rightarrow\begin{pmatrix}
                1 & 0 & 1 & 0 \\
                0 & 1 & 0 & 0 \\
                0 & 0 & 0 & 0 \\
                0 & 0 & 0 & 0
            \end{pmatrix}$,因为$r(A)=r(\bar{A})=2$,所以方程组 (1)、(2) 有公共解,公共解为$X=C\begin{pmatrix} -1 \\ 0  \\ 1 \end{pmatrix}$(其中$C$为任意常数).

        当$a=2$时,$\bar{A}\rightarrow\begin{pmatrix}
                1 & 1 & 1  & 0 \\
                0 & 1 & 1  & 0 \\
                0 & 0 & -1 & 1 \\
                0 & 0 & 0  & 0
            \end{pmatrix}\rightarrow\begin{pmatrix}
                1 & 0 & 0 & 0  \\
                0 & 1 & 0 & 1  \\
                0 & 0 & 1 & -1 \\
                0 & 0 & 0 & 0
            \end{pmatrix}$,方程组 (1)、(2) 有唯一的公共解为$X=\begin{pmatrix}
                0 \\
                1 \\
                -1
            \end{pmatrix}$.
        \end{answer}
    \end{exgroup}

    \begin{exgroup}
        \item 用方程组的理论证明:一个$n$次多项式不可能有多于$n$个不同的根.
        \begin{answer}
            令$f(x)=a_nx^n+\cdots+a_1x+a_0$为一个$n$次多项式,设$\lambda_1,\ldots,\lambda_{n+1}$为$f(x)$的$n+1$个不同的根,则有齐次线性方程组$\begin{cases} \begin{aligned}
                \lambda_1^n a_n+\cdots+\lambda_1a_1+a_0         & =0,               \\
                \lambda_2^n a_n+\cdots+\lambda_2a_1+a_0         & =0,               \\
                                                                & \vdotswithin{ = } \\
                \lambda_{n+1}^n a_n+\cdots+\lambda_{n+1}a_1+a_0 & =0.
            \end{aligned} \end{cases}$. 其系数矩阵为$A=\begin{pmatrix}
            \lambda_1^n     & \lambda_1^{n-1}     & \cdots & 1      \\
            \lambda_2^n     & \lambda_2^{n-1}     & \cdots & 1      \\
            \vdots          & \vdots              & \ddots & \vdots \\
            \lambda_{n+1}^n & \lambda_{n+1}^{n-1} & \cdots & 1
        \end{pmatrix}$. 显然$|A|\neq 0$,从而$a_n=a_{n-1}=\cdots=a_1=a_0=0$,这与$f(x)$是$n$次多项式矛盾.
        \end{answer}

        \item 相容(即有解)的线性方程组$AX=\vec{b}$在怎样的条件下,其解中第$k$个未知量$x_k$都是同一个值?你给的条件是否是充分必要的?
        \begin{answer}
            见教材P214第7题.
        \end{answer}

        \item 已知$A$是$n$阶对称矩阵,$\beta$为$n$元非零列向量,$B=\begin{pmatrix}
                A & \beta \\ \beta^\mathrm{T} & 0
            \end{pmatrix}$,证明:
        \begin{enumerate}
            \item 若$r(A)=n$,则$B$可逆的充要条件是$\beta^\mathrm{T}A^{-1}\beta \neq \vec{0}$;

            \item 若$r(A)=r$,则$r(B)=r$的充要条件是方程组$\begin{cases}
                          AX=\beta \\ \beta^\mathrm{T}X=\vec{0}
                      \end{cases}$有解;

            \item 若$r(A)=n-1$,则$B$可逆的充要条件是$AX=\beta$无解.
        \end{enumerate}
        \begin{answer}
            \begin{enumerate}
                \item 由于$r(A)=n$,所以$A$可逆,由打洞原理可知$|B|=|A|(0-\beta' A^{-1}\beta)=-|A|\beta' A^{-1}\beta$,从而$B$可逆的充要条件为$\beta' A^{-1}\beta\neq 0$.

                \item 必要性. 由于
                      \[ r=r(A)\leqslant r(A,\beta)\leqslant r\begin{pmatrix} A & \beta \\ \beta' & 0 \end{pmatrix}=r(B)=r \]
                      再结合$A'=A$,可知
                      \[ r\begin{pmatrix} A & \beta \\ \beta' & 0 \end{pmatrix} =r(A,\beta)=r\begin{pmatrix} A \\ \beta' \end{pmatrix} \]
                      于是由定理15.1可知方程组$\begin{pmatrix} A \\ \beta' \end{pmatrix}X=\begin{pmatrix} \beta \\ 0 \end{pmatrix}$有解,即$\begin{cases} AX=\beta, \\ \beta' X=0. \end{cases}$有解.

                      充分性. 由于$\begin{cases} AX=\beta \\ \beta' X=0 \end{cases}$有解,从而$AX=\beta$也有解,即有$r(A)=r(A,\beta)$. 另外$\begin{cases} AX=\beta \\ \beta' X=0 \end{cases}$有解也说明$\begin{pmatrix} A \\ \beta' \end{pmatrix}X=\begin{pmatrix} \beta \\ 0 \end{pmatrix}$有解,于是结合定理15.1可知$r\begin{pmatrix} A & \beta \\ \beta' & 0 \end{pmatrix}=r\begin{pmatrix} A \\ \beta' \end{pmatrix}$. 而显然$r\begin{pmatrix} A \\ \beta' \end{pmatrix}=r(A,\beta)=r(A)$,于是$r(B)=r\begin{pmatrix} A & \beta \\ \beta' & 0 \end{pmatrix}=r$.

                \item 必要性. 若$B$可逆,则$B$的行向量组线性无关,从而$r(A,\beta)=n$,又由于$r(A)=n-1$,所以$r(A,\beta)=r(A)+1$,从而由定理15.1知$AX=\beta$无解.

                      充分性. 由于$r(A)=n-1$,若$AX=\beta$无解,则由定理15.1可知$r(A,\beta)=r(A)+1=n$,于是
                      \[ r(B)=r\begin{pmatrix}
                              A      & \beta \\
                              \beta' & 0
                          \end{pmatrix}\geqslant r(A,\beta)=n \]
                      若$r(B)=n$,则$r(B)=r\begin{pmatrix}
                              A      & \beta \\
                              \beta' & 0
                          \end{pmatrix}=r(A,\beta)$,取转置结合$A'=A$有$r\begin{pmatrix}
                              A      & \beta \\
                              \beta' & 0
                          \end{pmatrix}=r\begin{pmatrix}
                              A \\
                              \beta'
                          \end{pmatrix}$,再次结合定理15.1可知方程组$\begin{cases}
                              AX=\beta, \\
                              \beta' X=0.
                          \end{cases}$有解,特别地,$AX=\beta$也有解,这就与已知产生了矛盾. 所以$r(B)=n+1$,即$B$可逆.
            \end{enumerate}
        \end{answer}

        \item 讨论$b_1,b_2,\ldots,b_n\enspace(n \geqslant 2)$满足什么条件时,下列方程组
        \[\begin{cases} \begin{aligned}
                    x_1+x_2     & =b_1            \\
                    x_2+x_3     & =b_2            \\
                                & \vdotswithin{=} \\
                    x_{n-1}+x_n & =b_{n-1}        \\
                    x_n+x_1     & =b_n
                \end{aligned} \end{cases}\]有解,并求解.
        \begin{answer}
            \begin{enumerate}
                \item 当$n$为偶数时,将增广矩阵$\bar{A}$的第$i\enspace(i=1,2,\ldots,n-1)$行乘以$(-1)^i$加到最后一行,得
                      \[ \bar{A}\rightarrow\begin{pmatrix}
                              1      & 1      & \cdots & 0      & 0      & b_1                                  \\
                              0      & 1      & \cdots & 0      & 0      & b_2                                  \\
                              0      & 0      & \cdots & 0      & 0      & b_3                                  \\
                              \vdots & \vdots & \ddots & \vdots & \vdots & \vdots                               \\
                              0      & 0      & \cdots & 1      & 1      & b_{n-1}                              \\
                              0      & 0      & \cdots & 0      & 0      & \displaystyle\sum_{i=1}^{n}(-1)^ib_i
                          \end{pmatrix}. \]
                      故当$\displaystyle\sum_{i=1}^{n}(-1)^ib_i=0$时,方程组有无穷多解,一般解为
                      \[\begin{cases} \begin{aligned}
                                  x_1     & =\displaystyle\sum_{i=1}^{n-1}(-1)^{i-1}b_i+(-1)^1x_n, \\
                                  x_2     & =\displaystyle\sum_{i=2}^{n-1}(-1)^{i-2}b_i+(-1)^2x_n, \\
                                          & \vdotswithin{ = }                                      \\
                                  x_{n-2} & =b_{n-2}-b_{n-1}+(-1)^{n-2}x_n,                        \\
                                  x_{n-1} & =b_{n-1}+(-1)^{n-1}x_n,                                \\
                              \end{aligned} \end{cases}\]
                      其中$x_n$为自由未知量.

                \item 当$n$为奇数时,有
                      \[ \bar{A}\rightarrow\begin{pmatrix}
                              1      & 1      & \cdots & 0      & 0      & b_1                                        \\
                              0      & 1      & \cdots & 0      & 0      & b_2                                        \\
                              0      & 0      & \cdots & 0      & 0      & b_3                                        \\
                              \vdots & \vdots & \ddots & \vdots & \vdots & \vdots                                     \\
                              0      & 0      & \cdots & 1      & 1      & b_{n-1}                                    \\
                              0      & 0      & \cdots & 0      & 2      & b_n+\displaystyle\sum_{i=1}^{n-1}(-1)^ib_i
                          \end{pmatrix}. \]
                      此时无论$b_1,b_2,\ldots,b_n\enspace(n\geqslant 2)$取何值,方程组都有唯一解为
                      \[\begin{cases} \begin{aligned}
                                  x_1     & =\displaystyle\sum_{i=1}^{n-1}(-1)^{i-1}b_i+(-1)^{n-1}x_n, \\
                                  x_2     & =\displaystyle\sum_{i=2}^{n-1}(-1)^{i-2}b_i+(-1)^{n-2}x_n, \\
                                          & \vdotswithin{ = }                                          \\
                                  x_{n-2} & =b_{n-2}-b_{n-1}+(-1)^2x_n,                                \\
                                  x_{n-1} & =b_{n-1}+(-1)^1x_n,                                        \\
                                  x_n     & =\frac 12b_n+\displaystyle\sum_{i=1}^{n-1}(-1)^ib_i.
                              \end{aligned} \end{cases}\]
            \end{enumerate}
        \end{answer}

        \item 已知$A$是$s \times n$矩阵,$B$是$m \times n$矩阵,$X,\vec{a},\vec{b}$分别是$n,s,m$元列向量,证明:
        \begin{enumerate}
            \item 齐次线性方程组$AX=\vec{0}$和$BX=\vec{0}$同解的充要条件是$A$与$B$行向量等价(列向量不一定);

            \item 齐次线性方程组$AX=\vec{0}$和$BX=\vec{0}$解空间分别为$V_1,V_2$,证明:$V_1 \subseteq V_2$的充要条件是存在$m \times s$矩阵$C$使得$B=CA$;

            \item 线性方程组$AX=\vec{a}$的解都是$BX=\vec{b}$的解的充要条件是增广矩阵$(B,\vec{b})$的每个行向量都可以被$(A,\vec{a})$的行向量线性表示;

            \item 线性方程组$AX=\vec{a}$与$BX=\vec{b}$同解的充要条件是$(A,\vec{a})$与$(B,\vec{b})$行向量等价.
        \end{enumerate}
        \begin{answer}
            \begin{enumerate}
                \item 必要性. 设$A$的行向量为$\alpha_1,\alpha_2,\ldots,\alpha_s$,$B$的行向量为$\beta_1,\beta_2,\ldots,\beta_m$,由$AX=0$与$BX=0$同解得$AX=0$与$\begin{pmatrix}A \\ B\end{pmatrix}=0$同解,从而系数矩阵的秩相同,即行向量$\alpha_1,\alpha_2,\ldots,\alpha_s$与$\alpha_1,\alpha_2,\ldots,\alpha_s,\beta_1,\beta_2,\ldots,\beta_m$秩相同. %“由命题3.2.5知道”?因此$\alpha_1,\alpha_2,\ldots,\alpha_s$与$\alpha_1,\alpha_2,\ldots,\alpha_s,\beta_1,\beta_2,\ldots,\beta_m$等价,即有$\beta_1,\beta_2,\ldots,\beta_m$可由$\alpha_1,\alpha_2,\ldots,\alpha_s$线性表出. 同理,$\alpha_1,\alpha_2,\ldots,\alpha_s$可由$\beta_1,\beta_2,\ldots,\beta_m$线性表出,即$A$与$B$的行向量等价.

                      充分性可由必要性的证明得到. 但是由于$A,B$列向量的维数都可能不同,所以不存在等价关系.

                \item 提示. $V_1\subseteq V_2$等价于$AX=0$与$\begin{pmatrix}
                              A \\
                              B
                          \end{pmatrix}X=0$同解.

                \item 必要性. 由题意可知$AX=a$与$\begin{pmatrix} A \\ B \end{pmatrix}X=\begin{pmatrix} a \\ b \end{pmatrix}$同解,从而它们的增广矩阵秩相同,即$r(A,a)=r\begin{pmatrix}
                              A & a \\
                              B & b
                          \end{pmatrix}$,可知$(A,a)$与$\begin{pmatrix}
                              A & a \\
                              B & b
                          \end{pmatrix}$的行向量组等价,从而$(B,b)$的每一个行向量都可以由$(A,a)$的行向量线性表出.\\
                      充分性可由必要性得到.

                \item 证明留给读者.
            \end{enumerate}
        \end{answer}

        \item 设$A=(a_{ij})_{n \times n}$是一个$n$级矩阵,证明:
        \begin{enumerate}
            \item 若$A$为复矩阵,且$|a_{ii}|>\displaystyle\sum_{j \neq i}|a_{ij}|$,那么$|A|\neq 0$;

            \item 若$A$为实矩阵,且$a_{ii}>\displaystyle\sum_{j \neq i}|a_{ij}|$,那么$|A|>0$;

            \item (推论)存在充分大的实数$M$,使得$t>M$时,$tE+A$可逆.
        \end{enumerate}
        \begin{answer}
            \begin{enumerate}
                \item \label{item:15:B:5:1}
                      采用反证法. 设 $\lvert A \rvert = 0$,则线性方程组 $AX = 0$ 有非零解,设为 $X_0 = (x_1, x_2, \ldots, x_n)^{\mathrm{T}}$,记
                      \[\lvert x_k \rvert = \max \{\lvert x_1 \rvert, \lvert x_2 \rvert, \ldots, \lvert x_n \rvert\}.\]
                      由 $X_0 \neq 0$ 可知 $\lvert x_k \rvert > 0$,考虑 $AX = 0$ 的第 $k$ 个方程,有 $\displaystyle\sum_{j=1}^n a_{kj}x_j = 0$,于是
                      \[\lvert a_{kk} \rvert \lvert x_k \rvert = \lvert -\displaystyle\sum_{j \neq k}a_{kj}x_j \rvert \leqslant \sum_{j \neq k}\lvert a_{kj} \rvert \lvert x_k \rvert.\]
                      约去 $\lvert x_k \rvert$ 后可得 $\lvert a_{kk} \rvert \leqslant \displaystyle\sum_{j \neq k} \lvert a_{kj} \rvert$,这与条件矛盾. 所以 $\lvert A \rvert \neq 0$.

                \item 构造实函数
                      \[f(t) = \begin{vmatrix}
                              a_{11}  & ta_{12} & ta_{13} & \cdots & ta_{1n} \\
                              ta_{21} & a_{22}  & ta_{23} & \cdots & ta_{2n} \\
                              ta_{31} & ta_{32} & a_{33}  & \cdots & ta_{3n} \\
                              \vdots  & \vdots  & \vdots  & \ddots & \vdots  \\
                              ta_{n1} & ta_{n2} & ta_{n3} & \cdots & a_{nn}  \\
                          \end{vmatrix}\]
                      由于 $A$ 是实矩阵,所以 $f(t)$ 是关于 $t$ 的一个实系数多项式(连续)函数,同时
                      \[f(0) = a_{11}a_{22}\cdots a_{nn} > 0.\]
                      当$t \in [0, 1]$ 时,还有
                      \[a_{ii} > \sum_{j \neq i} \lvert a_{ij} \rvert \geqslant \sum_{j \neq i} \lvert ta_{ij} \rvert.\]
                      由 \ref*{item:15:B:5:1} 可知 $f(t)$ 在 $[0, 1]$ 上非零,由连续函数的介值定理可知 $f(1) > 0$,即 $\lvert A \rvert > 0$.

                \item 此为直接推论不再赘述.
            \end{enumerate}
        \end{answer}
    \end{exgroup}
\end{exercise}
