\phantomsection
\section*{12 行列式}
\addcontentsline{toc}{section}{12 行列式}

\vspace{2ex}

\centerline{\heiti A组}
\begin{enumerate}
    \item 考虑滚动消去法,每行减下一行,得
          \[ \begin{vmatrix}
                  1-x    & 1      & 1      & 1      & \ldots & 1      & 1      \\
                  0      & 1-x    & 1      & 1      & \ldots & 1      & 1      \\
                  0      & 0      & 1-x    & 1      & \ldots & 1      & 1      \\
                  0      & 0      & 0      & 1-x    & \ldots & 1      & 1      \\
                  \vdots & \vdots & \vdots & \vdots & \ddots & \vdots & \vdots \\
                  0      & 0      & 0      & 0      & \ldots & 1-x    & 1      \\
                  x      & x      & x      & x      & \ldots & x      & 1      \\
              \end{vmatrix} \]
          第一行乘$(-x)$到第$n$行,得
          \[\begin{vmatrix}
                  1-x    & 1      & 1      & 1      & \ldots & 1      & 1      \\
                  0      & 1-x    & 1      & 1      & \ldots & 1      & 1      \\
                  0      & 0      & 1-x    & 1      & \ldots & 1      & 1      \\
                  0      & 0      & 0      & 1-x    & \ldots & 1      & 1      \\
                  \vdots & \vdots & \vdots & \vdots & \ddots & \vdots & \vdots \\
                  0      & 0      & 0      & 0      & \ldots & 1-x    & 1      \\
                  x^2    & 0      & 0      & 0      & \ldots & 0      & 1-x    \\
              \end{vmatrix}\]
          按第$n$行展开,得原式等于$(1-x)^n+(-1)^{n+1}x^2T_{n-1}$,其中
          \[T_{n-1}=\begin{vmatrix}
                  1      & 1      & 1      & \ldots & 1      & 1      \\
                  1-x    & 1      & 1      & \ldots & 1      & 1      \\
                  0      & 1-x    & 1      & \ldots & 1      & 1      \\
                  \vdots & \vdots & \vdots & \ddots & \vdots & \vdots \\
                  0      & 0      & 0      & \ldots & 1      & 1      \\
                  0      & 0      & 0      & \ldots & 1-x    & 1      \\
              \end{vmatrix}\]
          按第$n-1$行展开$T_{n-1}$,有$T_{n-1}=T_{n-2}+(-1)(1-x)T_{n-2}=xT_{n-2}$. 由$T_2=\begin{vmatrix}
                  1&1\\1-x&1\end{vmatrix}=x$,递推得$T_{n-1}=x^{n-2}$. 代入可解得原式$=(1-x)^n-(-x)^n$

    \item 按第一列展开,得$D_n=\displaystyle\prod_{i=1}^na_i+(-1)^{n+1}\prod_{i=1}^nb_i$.

    \item 按第一列展开,得
          \begin{align*}
              D_{n} & =(a+b) D_{n-1}-\begin{vmatrix}
                                         a b    & 0      & 0      & 0   & \cdots & 0      & 0      \\
                                         1      & a+b    & a b    & 0   & \cdots & 0      & 0      \\
                                         0      & 1      & a+b    & a b & \cdots & 0      & 0      \\
                                         \vdots & \vdots & \vdots & a+b & \ddots & \vdots & \vdots \\
                                         0      & 0      & 0      & 0   & \cdots & a+b    & a b    \\
                                         0      & 0      & 0      & 0   & \cdots & 1      & a+b
                                     \end{vmatrix} \\
                    & =(a+b) D_{n-1}-a b D_{n-2}
          \end{align*}
          \begin{enumerate}
              \item 方法一:特征方程法. 有特征方程$r^2-(a+b)r+ab=(r-a)(r-b)=0$,先考虑$a\neq b$,则$D_n=C_1a^n+C_2b^n$,$D_{1}=a+b$,$D_{2}=a^{2}+a b+b^{2}$,则有
                    \[\begin{cases}
                            C_1a+C_2b=a+b \\
                            C_1a^2+C_2b^2=a^2+ab+b^2
                        \end{cases},\]
                    考虑Cramer法则,首先判断系数行列式$\begin{vmatrix}
                            a   & b   \\
                            a^2 & b^2
                        \end{vmatrix}=ab(b-a)$,若$a=b=0$,原行列式为0;若$a=0,b\neq 0$,则有$C_2=1$,$D_n=b^n$;同理若$a=0,b\neq 0$,$D_n=a^n$. 考虑$a\neq 0,b\neq 0$,则系数行列式非0,根据Cramer法则可以解得
                    \[C_1=\frac{\begin{vmatrix}
                                a+b        & b   \\
                                a^2+ab+b^2 & b^2
                            \end{vmatrix}}{ab(b-a)}=\frac{a}{a-b},\quad
                        C_2=\frac{\begin{vmatrix}
                                a   & a+b        \\
                                a^2 & a^2+ab+b^2
                            \end{vmatrix}}{ab(b-a)}=-\frac{b}{a-b}\]
                    故$D_n=\displaystyle\frac{a^{n+1}-b^{n+1}}{a-b}$. 考虑到$a$或$b$为0时也符合该公式,故合并为同一公式. 再考虑$a=b$,则$D_n=(C_1+C_2n)a^n$,有方程
                    \[\begin{cases}
                            (C_1+C_2)a=2a \\
                            (C_1+2C_2)a^2=3a^2
                        \end{cases},\]
                    $a=0$,则$D_n=0$;$a\neq 0$,可以解得$C_1=C_2=1$,有$D_n=(n+1)a^n$. 考虑到$a=0$时$D_n$也符合该式,故合并. 综上有
                    \[D_n=\begin{cases}
                            \displaystyle\frac{a^{n+1}-b^{n+1}}{a-b}, & a\neq b \\
                            (n+1)a^n,                                 & a=b
                        \end{cases}.\]

              \item 方法二:递推式变形法. 递推式变形, 得
                    \[\mathrm{D}_{n}-a D_{n-1}=b\left(D_{n-1}-a D_{n-2}\right),\]
                    由于 $D_{1}=a+b, D_{2}=a^{2}+a b+b^{2}$,从而利用上述递推公式得
                    \begin{align*}
                        \mathrm{D}_{n}-a D_{n-1} & =b\left(D_{n-1}-a D_{n-2}\right)                \\
                                                 & =b^{2}\left(D_{n-2}-a D_{n-3}\right)            \\
                                                 & =\cdots=b^{n-2}\left(D_{2}-a D_{1}\right)=b^{n}
                    \end{align*}
                    故\begin{align*}
                        D_{n} & =a D_{n-1}+b^{n}=a\left(a D_{n-2}+b^{n-1}\right)+b^{n}     \\
                              & =\cdots=a^{n-1} D_{1}+a^{n-2} b^{2}+\cdots+a b^{n-1}+b^{n} \\
                              & =a^{n}+a^{n-1} b+\cdots+a b^{n-1}+b^{n}                    \\
                              & =\begin{cases}
                                     \displaystyle\frac{a^{n+1}-b^{n+1}}{a-b}, & a\neq b \\
                                     (n+1)a^n,                                 & a=b
                                 \end{cases}
                    \end{align*}
          \end{enumerate}

    \item 略.

    \item \begin{enumerate}
              \item 方法一:慢慢消去.
                    \begin{align*}
                        \begin{vmatrix}
                            a^{2} & (a+1)^{2} & (a+2)^{2} & (a+3)^{2} \\
                            b^{2} & (b+1)^{2} & (b+2)^{2} & (b+3)^{2} \\
                            c^{2} & (c+1)^{2} & (c+2)^{2} & (c+3)^{2} \\
                            d^{2} & (d+1)^{2} & (d+2)^{2} & (d+3)^{2}
                        \end{vmatrix}
                         & = \begin{vmatrix}
                                 a^{2} & 2a+1 & 4a+4 & 6a+9 \\
                                 b^{2} & 2b+1 & 4b+4 & 6b+9 \\
                                 c^{2} & 2c+1 & 4c+4 & 6c+9 \\
                                 d^{2} & 2d+1 & 4d+4 & 6d+9
                             \end{vmatrix} \\
                         & = \begin{vmatrix}
                                 a^{2} & 2a+1 & 4a+4 & 4 \\
                                 b^{2} & 2b+1 & 4b+4 & 4 \\
                                 c^{2} & 2c+1 & 4c+4 & 4 \\
                                 d^{2} & 2d+1 & 4d+4 & 4
                             \end{vmatrix}    \\
                         & = 4\begin{vmatrix}
                                  a^{2} & 2a & 4a & 1 \\
                                  b^{2} & 2b & 4b & 1 \\
                                  c^{2} & 2c & 4c & 1 \\
                                  d^{2} & 2d & 4d & 1
                              \end{vmatrix}=0
                    \end{align*}

              \item 关注到行列式中只有$(a^2, b^2, c^2, d^2)^\mathrm{T}, (a, b, c, d)^\mathrm{T}$和$(1, 1, 1, 1)^\mathrm{T}$三个线性无关列向量,则一眼看出线性相关. 因此设原行列式为$|\alpha_1, \alpha_2, \alpha_3, \alpha_4|$,有$\alpha_4=\alpha_1-3\alpha_2+3\alpha_3$,因此原行列式值为0.
          \end{enumerate}

    \item \begin{enumerate}
              \item \begin{align*}
                        D_n & =\begin{vmatrix}
                                   D_{n-1} & a   \\
                                   b       & a_n
                               \end{vmatrix},a=(0, \ldots, 0, -a_{n-1})^\mathrm{T}, b=(1, \ldots, 1) \\
                            & =a_nD_{n-1}+a_{n-1}\begin{vmatrix}
                                                     D_{n-2} & 1_{n-2} \\
                                                     1_{n-2} & 1
                                                 \end{vmatrix}
                        \quad \text{($1_{n-2}$表示由$(n-2)$个1构成的行/列向量)}                      \\
                            & =a_nD_{n-1}+a_{n-1}
                        \begin{vmatrix}
                            a_1    & 0      & 0      & \ldots & 0       & 1      \\
                            0      & a_2    & 0      & \ldots & 0       & 1      \\
                            0      & 0      & a_3    & \ldots & 0       & 1      \\
                            \vdots & \vdots & \vdots & \ddots & \vdots  & \vdots \\
                            0      & 0      & 0      & \ldots & a_{n-2} & 1      \\
                            0      & 0      & 0      & \ldots & 0       & 1      \\
                        \end{vmatrix} \quad \text{($1-n$列用第$n$列减)}                         \\
                            & =a_nD_{n-1}+\prod_{i=1}^{n-1}a_i
                    \end{align*}
                    递推可得$\displaystyle D_n=\left(\prod_{k=1}^na_k\right)
                        \left(1+\sum_{i=1}^n\frac{1}{a_i}\right)$.

              \item 将 1 写成 $1+0$, 将 $D$ 硬拆成 $2^{n}$ 个行列式, 只有如下的 $n+1$ 个行列式非 0:
                    \begin{align*}
                        D_n = & \begin{vmatrix}
                                    1      &       &        &       \\
                                    1      & a_{2} &        &       \\
                                    \vdots &       & \ddots &       \\
                                    1      &       &        & a_{n}
                                \end{vmatrix}
                        +\begin{vmatrix}
                             a_{1} & 1      &                \\
                                   & 1      &                \\
                                   & \vdots & \ddots &       \\
                                   & 1      &        & a_{n}
                         \end{vmatrix}+\cdots+                    \\
                              & \begin{vmatrix}
                                    a_{1} &       &        & 1      \\
                                          & a_{2} &        & 1      \\
                                          &       & \ddots & \vdots \\
                                          &       &        & 1
                                \end{vmatrix}
                        +\begin{vmatrix}
                             a_{1} &                      \\
                                   & a_{2} &              \\
                                   &       & \ddots &     \\
                                   &       &        & a_n
                         \end{vmatrix}                       \\
                        =     & \left(\prod_{k=1}^{n} a_{k}+\frac{1}{a_{2}}
                        \prod_{k=1}^{n} a_{k}\right)\left(1+\sum_{i=1}^{n} \frac{1}{a_{i}}\right)
                    \end{align*}
          \end{enumerate}

    \item 暴力拆成$2^n$个行列式,其中只有2个每列各不相同,其他都因为有相同列而为0. 因此有
          \begin{align*}
              \text{原式} & =\begin{vmatrix}
                                 a_{1}     & a_{2}     & \cdots & a_{n-1}     & a_{n}     \\
                                 a_{1}^{2} & a_{2}^{2} & \cdots & a_{n-1}^{2} & a_{n}^{2} \\
                                 \vdots    & \vdots    & \ddots & \vdots      & \vdots    \\
                                 a_{1}^{n} & a_{2}^{n} & \cdots & a_{n-1}^{n} & a_{n}^{n}
                             \end{vmatrix}+
              \begin{vmatrix}
                  a_{2}     & a_{3}     & \cdots & a_{n}     & a_{1}     \\
                  a_{2}^{2} & a_{3}^{2} & \cdots & a_{n}^{2} & a_{1}^{2} \\
                  \vdots    & \vdots    & \ddots & \vdots    & \vdots    \\
                  a_{2}^{n} & a_{3}^{n} & \cdots & a_{n}^{n} & a_{1}^{n}
              \end{vmatrix}                                               \\
                          & =(1+(-1)^{n-1})\begin{vmatrix}
                                               a_{1}     & a_{2}     & \cdots & a_{n-1}     & a_{n}     \\
                                               a_{1}^{2} & a_{2}^{2} & \cdots & a_{n-1}^{2} & a_{n}^{2} \\
                                               \vdots    & \vdots    & \ddots & \vdots      & \vdots    \\
                                               a_{1}^{n} & a_{2}^{n} & \cdots & a_{n-1}^{n} & a_{n}^{n}
                                           \end{vmatrix}                \\
                          & =(1+(-1)^{n-1})\prod_{i=1}^na_i
              \begin{vmatrix}
                  1           & 1           & \cdots & 1             & 1           \\
                  a_{1}       & a_{2}       & \cdots & a_{n-1}       & a_{n}       \\
                  \vdots      & \vdots      & \ddots & \vdots        & \vdots      \\
                  a_{1}^{n-1} & a_{2}^{n-1} & \cdots & a_{n-1}^{n-1} & a_{n}^{n-1}
              \end{vmatrix}                                     \\
                          & =(1+(-1)^{n-1})\left(\prod_{i=1}^na_i\right)\prod_{1\leqslant i<j\leqslant n}(a_j-a_i) \\
          \end{align*}

    \item \begin{enumerate}
              \item 连加法得结果为160;

              \item 连加法得结果为$(\lambda-1)(\lambda+3)^3$;

              \item 如果对数据敏感能看出线性相关,可以直接由$\alpha_4-\alpha_1=3(\alpha_3-\alpha_2)$证得其线性相关,则为0. 也可以使用滚动消去法,如\[
                        \begin{vmatrix}
                            1 & 4 & 9  & 16 \\
                            3 & 5 & 7  & 9  \\
                            5 & 7 & 9  & 11 \\
                            7 & 9 & 11 & 13
                        \end{vmatrix}=\begin{vmatrix}
                            1 & 4 & 9 & 16 \\
                            3 & 5 & 7 & 9  \\
                            2 & 2 & 2 & 2  \\
                            2 & 2 & 2 & 2
                        \end{vmatrix}=0.\]

              \item 当然是递推法的常见形式,但是只有4阶,所以可以暴算,转化为上三角行列式,如\[
                        \begin{vmatrix}
                            3 & 2           & 0            & 0             \\
                            0 & \frac{7}{3} & 2            & 0             \\
                            0 & 0           & \frac{15}{7} & 2             \\
                            0 & 0           & 0            & \frac{31}{15}
                        \end{vmatrix}=31.\]
          \end{enumerate}

    \item \begin{enumerate}
              \item 行列式中2比较多,用全是2的第二行去消,然后按第2行展开.
                    \[D=\begin{vmatrix}
                            -1     & 0      & \cdots & 0      & 0      \\
                            2      & 2      & \cdots & 2      & 2      \\
                            \vdots & \vdots & \ddots & \vdots & \vdots \\
                            0      & 0      & \cdots & n-3    & 0      \\
                            0      & 0      & \cdots & 0      & n-2
                        \end{vmatrix}=2\cdot(-1)\cdot(n-2)!=-2(n-2)!.\]

              \item \begin{enumerate}
                        \item 方法一:考虑滚动消去法,有
                              \begin{align*}
                                  D & =\begin{vmatrix}
                                           1      & 2      & 3      & \cdots & n-1    & n      \\
                                           1      & 1      & 1      & \cdots & 1      & 1-n    \\
                                           1      & 1      & 1      & \cdots & 1-n    & 1      \\
                                           \vdots & \vdots & \vdots & \ddots & \vdots & \vdots \\
                                           1      & 1      & 1-n    & \cdots & 1      & 1      \\
                                           1      & 1-n    & 1      & \cdots & 1      & 1
                                       \end{vmatrix}                                    \\&=\begin{vmatrix}
                                      0      & n+1    & 2      & \cdots & n-2    & n-1    \\
                                      0      & n      & 0      & \cdots & 0      & -n     \\
                                      0      & n      & 0      & \cdots & -n     & 0      \\
                                      \vdots & \vdots & \vdots & \ddots & \vdots & \vdots \\
                                      0      & n      & -n     & \cdots & 0      & 0      \\
                                      1      & 1-n    & 1      & \cdots & 1      & 1
                                  \end{vmatrix}\\
                                    & =(-1)^{n+1}\begin{vmatrix}
                                                     n+1    & 2      & \cdots & n-2    & n-1    \\
                                                     n      & 0      & \cdots & 0      & -n     \\
                                                     n      & 0      & \cdots & -n     & 0      \\
                                                     \vdots & \vdots & \ddots & \vdots & \vdots \\
                                                     n      & -n     & \cdots & 0      & 0
                                                 \end{vmatrix}                                   \\
                                    & =(-1)^{n+1}(-1)^{\frac{(n-2)(n-3)}{2}}
                                  \begin{vmatrix}
                                      n+1    & n-1    & \cdots & 3      & 2      \\
                                      n      & -n     & \cdots & 0      & 0      \\
                                      n      & 0      & \cdots & 0      & 0      \\
                                      \vdots & \vdots & \ddots & \vdots & \vdots \\
                                      n      & 0      & \cdots & 0      & -n
                                  \end{vmatrix}                                                  \\
                                    & =(-1)^{\frac{n^2-3n+8}{2}}\begin{vmatrix}
                                                                    \frac{n(n+1)}{2} & n-1    & \cdots & 3      & 2      \\
                                                                    0                & -n     & \cdots & 0      & 0      \\
                                                                    0                & 0      & \cdots & 0      & 0      \\
                                                                    \vdots           & \vdots & \ddots & \vdots & \vdots \\
                                                                    0                & 0      & \cdots & 0      & -n
                                                                \end{vmatrix}=(-1)^{\frac{n^2-n+12}{2}}\frac{n^{n-1}(n+1)}{2}
                              \end{align*}

                        \item 方法二:考虑连加法. 将第 $2,3, \ldots, n$ 列都加到第 1 列, 提出公因子 $\dfrac{1}{2} n(n+1)$, 再依次将第 $n-1$ 行乘 $(-1)$ 加到第 $n$ 行, $\cdots$, 第 2 行乘 $(-1)$ 加到第 3 行, 第 1 行乘 $(-1)$ 加到第 2 行, 然后对第 1 列展开, 得到一个 $n-1$ 阶行列式, 它的副对角元为 $1-$ $n$, 其余元素均为 1 . 再把它的各列加到第 1 列, 并把它的第 1 行乘 $(-1)$ 加到其余各行, 得
                              \begin{align*}
                                  D & =\frac{n(n+1)}{2}\begin{vmatrix}
                                                           1      & 2      & \cdots & n-1    & n      \\
                                                           1      & 3      & \cdots & n      & 1      \\
                                                           1      & 4      & \cdots & 1      & 2      \\
                                                           \vdots & \vdots & \ddots & \vdots & \vdots \\
                                                           1      & 1      & \cdots & n-2    & n-1
                                                       \end{vmatrix} \\
                                    & =\frac{n(n+1)}{2}\begin{vmatrix}
                                                           1      & 2      & 3      & \cdots & n      \\
                                                           0      & 1      & 1      & \cdots & 1-n    \\
                                                           \vdots & \vdots & \vdots & \ddots & \vdots \\
                                                           0      & 1      & 1-n    & \cdots & 1      \\
                                                           0      & 1-n    & 1      & \cdots & 1
                                                       \end{vmatrix}_{n} \\
                                    & =\frac{n(n+1)}{2}
                                  \begin{vmatrix}
                                      1      & 1      & \cdots & 1      & 1-n \\
                                      \vdots & \vdots & \ddots & \vdots & 1   \\
                                      1      & 1-n    & \cdots & 1      & 1   \\
                                      1-n    & 1      & \cdots & 1      & 1
                                  \end{vmatrix}                         \\
                                    & =\frac{n(n+1)}{2}\begin{vmatrix}
                                                           -1     & 1      & \cdots & 1      & 1-n    \\
                                                           0      & 0      & \cdots & -n     & n      \\
                                                           \vdots & \vdots & \ddots & \vdots & \vdots \\
                                                           0      & -n     & \cdots & 0      & n      \\
                                                           0      & 0      & \cdots & 0      & n
                                                       \end{vmatrix}_{n-1}
                              \end{align*}
                              将上式先对第 1 列展开, 得到一个 $n-2$ 阶行列式, 再将它对最后一行展开, 得
                              \begin{align*}
                                  D & =\frac{-n(n+1)}{2} n
                                  \begin{vmatrix}
                                      0      & 0      & \cdots & 0      & -n     \\
                                      0      & 0      & \cdots & -n     & 0      \\
                                      \vdots & \vdots & \ddots & \vdots & \vdots \\
                                      0      & -n     & \cdots & 0      & 0      \\
                                      -n     & 0      & \cdots & 0      & 0
                                  \end{vmatrix}_{n-3}                      \\
                                    & =-\frac{n^{2}(n+1)}{2}(-1)^{\frac{(n-3)(n-4)}{2}}(-n)^{n-3} \\
                                    & =(-1)^{\frac{n(n-1)}{2} }\frac{(n+1) n^{n-1}}{2}
                              \end{align*}
                              事实上,两种方法得到的答案是等价的.
                    \end{enumerate}
          \end{enumerate}

    \item 若 $b=c$, 则每行行和都相等, 考虑连加法, 把第 $2,3, \ldots, n$ 列都加到第 1 列, 提出公因子 $a+(n-1) c$, 再将第 1 行乘 $(-1)$ 加到其余各行, 得到上三角行列式. 于是
          \begin{align*}
              D_{n} & =(a+(n-1) c)\begin{vmatrix}
                                      1      & c      & c      & \cdots & c      \\
                                      1      & a      & c      & \cdots & c      \\
                                      1      & c      & a      & \cdots & c      \\
                                      \vdots & \vdots & \vdots & \ddots & \vdots \\
                                      1      & c      & c      & \cdots & a
                                  \end{vmatrix} \\
                    & =(a+(n-1) c)\begin{vmatrix}
                                      1      & c      & c      & \cdots & c      \\
                                      0      & a-c    & 0      & \cdots & 0      \\
                                      0      & 0      & a-c    & \cdots & 0      \\
                                      \vdots & \vdots & \vdots & \ddots & \vdots \\
                                      0      & 0      & 0      & \cdots & a-c
                                  \end{vmatrix} \\
                    & =(a+(n-1) c)(a-c)^{n-1} .
          \end{align*}
          若 $b \neq c$, 第 $i$ 行乘 $(-1)$ 加到第 $i-1$ 行($i$ 依次取 $n, n-1, \ldots, 2$),再对第 1 列展开,
          \[D_{n}=\begin{vmatrix}
                  a-b    & c-a    & 0      & \cdots & 0      \\
                  0      & a-b    & c-a    & \cdots & 0      \\
                  \vdots & \vdots & \vdots & \ddots & \vdots \\
                  0      & 0      & 0      & \cdots & c-a    \\
                  b      & b      & b      & \cdots & a
              \end{vmatrix}=(a-b)D_{n-1}+(-1)^{n+1}b(c-a)^{n-1},\]
          有了递推式,但是递推式中有$(-1)^{n+1}$,递推比较麻烦可能还需要处理. 如果用数归证明的话应该可以直接下手了,但是有更聪明的办法:取转置,有$D_n^\mathrm{T}=D_n$,同理有$D_n=(a-c)D_{n-1}+(-1)^{n+1}c(b-a)^{n-1}$.

    \item 证明:\begin{enumerate}
        \item (线性性)直接将公理化定义用递归式对第$i$列展开:
              \begin{align*}
                      & D(\alpha_1,\ldots,\lambda\alpha_{i}+\mu\beta_i,\ldots,\alpha_n)                                      \\
                  ={} & \sum_{k=1}^{n}(\lambda a_{ki}+\mu b_{ki})A_{ki}                                                      \\
                  ={} & \lambda \cdot \sum_{k=1}^{n}a_{ki}A_{ki}+\mu \cdot \sum_{k=1}^{n}b_{ki}A_{ki}                        \\
                  ={} & \lambda D(\alpha_1,\ldots,\alpha_{i},\ldots,\alpha_n)+\mu D(\alpha_1,\ldots,\beta_i,\ldots,\alpha_n)
              \end{align*}
              则线性性得证.

        \item (反对称性)使用数学归纳法证明. 显然,$D(\alpha_1,\alpha_2)=-D(\alpha_2,\alpha_1)$,然后做出归纳假设:对于任意正整数$i,j$,$1 \leqslant i, j \leqslant n - 1$且$i \neq j$,有:
              \[ D(\alpha_1,\ldots,\alpha_i,\ldots,\alpha_j,\ldots,\alpha_{n-1})=-D(\alpha_1,\ldots,\alpha_j,\ldots,\alpha_i,\ldots,\alpha_{n-1}) \]
              由此做出递推,对交换前后的行列式的首行做展开:
              \begin{align*}
                  D(\alpha_1,\ldots,\alpha_{i},\ldots,\alpha_{j},\ldots,\alpha_n)
                   & =\sum_{k=1}^{n}a_{1k}A_{1k}   \\
                  D(\alpha_1,\ldots,\alpha_{j},\ldots,\alpha_{i},\ldots,\alpha_n)
                   & =\sum_{k=1}^{n}a'_{1k}A'_{1k}
              \end{align*}
              其中,除第$i,j$项外,由归纳假设,其余项都满足$a_{1k}=a'_{1k},A_{1k}=-A'_{1k}$,则有$a_{1k}A_{1k}=-a'_{1k}A'_{1k},k\neq i,j$. 因此主要考察$a_{1i}A_{1i}+a_{1j}A_{1j}$与$a'_{1i}A'_{1i}+a'_{1j}A'_{1j}$这两项. 首先有$a'_{1i}=a_{1j},a'_{1j}=a_{1i}$. 然后将$A_{1i}$与$A'_{1j}$两项展开对比:
              \begin{align*}
                  A_{1i}  & =(-1)^{1+i}(\alpha'_{1},\ldots,\alpha'_{i-1},\alpha'_{i+1},\ldots,\alpha'_{j},\ldots,\alpha'_{n})                             \\
                  A'_{1j} & =(-1)^{1+j}(\alpha'_{1},\ldots,\alpha'_{i-1},\alpha'_{j},\alpha'_{i+1},\ldots,\alpha'_{j-1},\alpha'_{j+1},\ldots,\alpha'_{n})
              \end{align*}
              式中的$\alpha'_k$表示原列向量去掉首行元素后剩余$n-1$个元素组成的新列向量. 可以发现,$A'_{1j}$向左交换$j-(i+1)$次后与$A_{1i}$是绝对值一致的. 则根据归纳假设,有$(-1)^{j-(i+1)}A'_{1j}=(-1)^{1+j-(1+i)}A_{1i}$,即有$A'_{1j}=-A_{1i}$,所以$a_{1i}A_{1i}+a_{1j}A_{1j}=-(a'_{1j}A'_{1j}+a'_{1i}A'_{1i})$. 综上可证:
              \[ D(\alpha_1,\ldots,\alpha_{i},\ldots,\alpha_{j},\ldots,\alpha_n)=D(\alpha_1,\ldots,\alpha_{j},\ldots,\alpha_{i},\ldots,\alpha_n) \]

        \item (规范性)只需使用递归式定义,逐次展开即可.
    \end{enumerate}

\item 使用倍加列变换的性质,可得$|\alpha_1+\alpha_2+\alpha_3,\alpha_1+3\alpha_2+9\alpha_3,\alpha_1+4\alpha_2+16\alpha_3|=6|\alpha_1,\alpha_2,\alpha_3|=12$

\item \begin{enumerate}
        \item 首先由反对称矩阵的性质,有$A^T=-A$,其次由矩阵与行列式的性质,可得$|A^T|=|A|$,则可推出$|A|=|-A|=(-1)^n|A|$,又n为奇数,故$|A|=0$,矩阵$A$不可逆.

        \item 只需证明$AB$的秩小于$n$,即$|AB|=0$即可. $B$是奇数阶反对称矩阵,$|B|=0$,则$|AB|=|A||B|=0$得证.
    \end{enumerate}

\item 根据伴随矩阵的性质,$\begin{pmatrix}
            A & O \\ O & B
        \end{pmatrix}^* = \begin{vmatrix}
            A & O \\ O & B
        \end{vmatrix} \cdot \begin{pmatrix}
            A & O \\ O & B
        \end{pmatrix}^{-1}$. 其中,对$\begin{vmatrix}
            A & O \\ O & B
        \end{vmatrix}$做递归式展开,有$\begin{vmatrix}
            A & O \\ O & B
        \end{vmatrix}=\displaystyle\sum_{k=1}^{m}(-1)^{1+k}a_{1k}\begin{vmatrix}
            M_{1k} & O \\ O & B
        \end{vmatrix}$,依此逐次展开可得$\begin{vmatrix}
            A & O \\ O & B
        \end{vmatrix}=|A||B|=ab$. 又$ab\begin{pmatrix}
            A & O \\ O & B
        \end{pmatrix}^{-1}=ab\begin{pmatrix}
            A^{-1} & O \\ O & B^{-1}
        \end{pmatrix}=\begin{pmatrix}
            b\cdot aA^{-1} & O \\ O & a\cdot bB^{-1}
        \end{pmatrix}$,最终可得$\begin{pmatrix}
            A & O \\ O & B
        \end{pmatrix}^* = \begin{pmatrix}
            bA^* & O \\ O & aB^*
        \end{pmatrix}$.

    与前一个式子的展开类似,$\begin{pmatrix}
            O & A \\ B & O
        \end{pmatrix}^*=\begin{vmatrix}
            O & A \\ B & O
        \end{vmatrix}\cdot \begin{pmatrix}
            O & A \\ B & O
        \end{pmatrix}^{-1}$,其中对前一步推导做少量修正后可得$\begin{vmatrix}
            O & A \\ B & O
        \end{vmatrix}=(-1)^{mn}|A||B|=(-1)^{mn}ab$,则$(-1)^{mn}ab\begin{pmatrix}
            O & A \\ B & O
        \end{pmatrix}^{-1}=(-1)^{mn}\begin{pmatrix}
            O & a\cdot bB^{-1} \\ b\cdot aA^{-1} & O
        \end{pmatrix}$,最终可得$\begin{pmatrix}
            O & A \\ B & O
        \end{pmatrix}^*=(-1)^{mn}\begin{pmatrix}
            O & aB^* \\ bA^* & O
        \end{pmatrix}$.

\item 证明;\begin{enumerate}
        \item 对正整数$k$,有$(A^k)^*=(A^*)^k$. 从而若$A^m=A$,则$(A^*)^m=(A^m)^*=A^*$. 即$A$是幂等矩阵,则$A^*$也是幂等矩阵. 同样,若$A^m=0$,则$(A^*)^m=(A^m)^*=0$. 即$A$是幂零矩阵,则$A^*$也是幂零矩阵.

        \item $(A^*)^T=(A^T)^*=A^*$. 从而若$A$是对称矩阵,则$A^*$也是对称矩阵$. (A^*)^T=(A^T)^*=(-A)^*=(-1)^{n-1}A^*$. 从而若$A^*$为偶数阶时也为反对称矩阵,奇数阶时为对称矩阵.
    \end{enumerate}

\item 对于上三角矩阵,只需看其伴随矩阵的下半部分元素,即$A_{ij},j>i$. 对这些代数余子式,要么有一整行或整列为零,要么对角线存在零元素,则这些余子式均为零,伴随矩阵是上三角矩阵.\par 或者使用另法,当$A$是可逆矩阵,即$A$对角线元素不为零,则$A^*=|A|\cdot A^{-1}$,其中$A^{-1}$也是上三角矩阵,因此伴随矩阵$A^*$是上三角矩阵.

\item $|A|=0$,则$r(A)<n$,故$r(A^*)=\begin{cases}
            1, & r(A)=n-1 \\
            0, & r(A)<n-1
        \end{cases}$. 当$r(A^*)=0$时,答案显然成立;当$r(A^*)=1$时,由矩阵秩的定义,任意两行(列)必成比例,否则秩大于1,矛盾. 综上,原题得证.

\item $(\alpha_1-\alpha_2-2\alpha_3,2\alpha_1+\alpha_2-\alpha_3,3\alpha_1+\alpha_2+2\alpha_3)=(\alpha_1,\alpha_2,\alpha_3)\begin{pmatrix}
            1 & 2 & 3 \\ -1 & 1 & 1 \\ -2 & -1 & 2
        \end{pmatrix}$,又$\begin{vmatrix}
            1 & 2 & 3 \\ -1 & 1 & 1 \\ -2 & -1 & 2
        \end{vmatrix}=12\neq 0$,因此该矩阵可逆,则$(\alpha_1-\alpha_2-2\alpha_3,2\alpha_1+\alpha_2-\alpha_3,3\alpha_1+\alpha_2+2\alpha_3)$与$(\alpha_1,\alpha_2,\alpha_3)$秩相等,因此这三个向量线性无关.

\item 只要取$\alpha_3,\alpha_4 \in \mathbf{R}^4$,使得$D(\alpha_1,\alpha_2,\alpha_3,\alpha_4) \neq 0$即可. 易得$\begin{vmatrix}
            1  & 1  & 0 & 0 \\
            2  & 4  & 0 & 0 \\
            1  & -1 & 1 & 0 \\
            -1 & 1  & 0 & 1
        \end{vmatrix}=\begin{vmatrix}
            1 & 1 \\
            2 & 4
        \end{vmatrix} \cdot \begin{vmatrix}
            1 & 0 \\
            0 & 1
        \end{vmatrix}=2 \neq 0$. 于是可取$\alpha_3=(0,0,1,0)^T,\alpha_4=(0,0,0,1)^T$,能使得${\alpha_1,\alpha_2,\alpha_3,\alpha_4}$为$\mathbf{R}^4$的一组基.

    \item \begin{enumerate}
              \item 若 $ AB = kE \enspace(k \neq 0) $,则 $ A^{-1} = \dfrac{1}{k} B $. 由
                    \[ A^2 - A - 2E = A(A - E) - 2E = O \]
                    可得
                    \[ A(A - E) = 2E \text{~或~} A(E - A) = -2E \]
                    所以
                    \[ A^{-1} = \frac{1}{2}(A-E),\enspace (E - A)^{-1} = -\frac{1}{2}A \]

              \item 由
                    \[  A^2 - A - 2E = (A - 2E)(A + E) = O \]
                    可知,$ A + E $ 和 $ A - 2E $ 不能同时可逆,否则 $ A^2 - A - 2E $ 为零矩阵可逆,矛盾.
          \end{enumerate}

    \item \begin{enumerate}
              \item \begin{align*}
                        (A - E)(B - E) & = AB - AE - EB + E^2 \\
                                       & = AB - A - B + E     \\
                                       & = AB - (A + B) + E   \\
                                       & = AB - AB + E        \\
                                       & = E
                    \end{align*}
                    所以 $ A - E $ 与 $ B - E $ 互为逆矩阵.

              \item 由于 $ A - E $ 与 $ B - E $ 互为逆矩阵,所以
                    \begin{align*}
                        (B - E)(A - E) & = E              \\
                                       & = BA - B - A + E \\
                                       & = BA - AB + E
                    \end{align*}
                    所以 $ BA - AB = O $,即 $ AB = BA $.

              \item 由 $ A = AB - B = B(A - E) $ 可得 $ r(A) \leqslant r(B) $,同理可得 $ r(B) \leqslant r(A) $,所以 $ r(A) = r(B) $.
          \end{enumerate}
\end{enumerate}

\centerline{\heiti B组}
\begin{enumerate}
    \item 这三小问实际上都是对原行列式中的某一行进行了替换进行计算.
          \begin{enumerate}
              \item \begin{align*}
                        A_{21}+A_{22}+A_{23}+A_{24}
                         & = \begin{vmatrix}
                                 3 & 0  & 4  & 1 \\
                                 1 & 1  & 1  & 1 \\
                                 0 & -7 & 8  & 3 \\
                                 5 & 3  & -2 & 2
                             \end{vmatrix}
                        = \begin{vmatrix}
                              3 & -3 & 1  & -2 \\
                              1 & 0  & 0  & 0  \\
                              0 & -7 & 8  & 3  \\
                              5 & -2 & -7 & -3
                          \end{vmatrix}             \\
                         & = (-1)^{2+1} \begin{vmatrix}
                                            -3 & 1  & -1 \\
                                            -7 & 8  & 3  \\
                                            -2 & -7 & -3
                                        \end{vmatrix} \\
                         & = 148
                    \end{align*}

              \item \begin{align*}
                        A_{31}+A_{33}
                         & = 1A_{31}+0A_{32}+1A_{33}+0A_{34}
                        = \begin{vmatrix}
                              3 & 0 & 4  & 1 \\
                              2 & 3 & 1  & 4 \\
                              1 & 0 & 1  & 0 \\
                              5 & 3 & -2 & 2
                          \end{vmatrix}
                        = 3 \begin{vmatrix}
                                3 & 0 & 4  & 1 \\
                                2 & 1 & 1  & 4 \\
                                1 & 0 & 1  & 0 \\
                                5 & 1 & -2 & 2
                            \end{vmatrix}                   \\
                         & = 3 \begin{vmatrix}
                                   3 & 0 & 1  & 1 \\
                                   2 & 1 & -1 & 4 \\
                                   1 & 0 & 0  & 0 \\
                                   5 & 3 & -7 & 2
                               \end{vmatrix}
                        = 3 \cdot (-1)^{3+1} \begin{vmatrix}
                                                 0 & 1  & 1 \\
                                                 1 & -1 & 4 \\
                                                 3 & -7 & 2
                                             \end{vmatrix}  \\
                         & = -12
                    \end{align*}

              \item \begin{align*}
                        M_{41}+M_{42}+M_{43}+M_{44}
                         & = -A_{41}+A_{42}-A_{43}+A_{44}
                        = \begin{vmatrix}
                              3  & 0  & 4  & 1 \\
                              2  & 3  & 1  & 4 \\
                              0  & -7 & 8  & 3 \\
                              -1 & 1  & -1 & 1
                          \end{vmatrix}                     \\
                         & = \begin{vmatrix}
                                 3  & 3  & 1  & 4 \\
                                 2  & 5  & -1 & 6 \\
                                 0  & -7 & 8  & 3 \\
                                 -1 & 0  & 0  & 0
                             \end{vmatrix}
                        = {-1}^{4+1} \cdot (-1) \begin{vmatrix}
                                                    3  & 1  & 4 \\
                                                    5  & -1 & 6 \\
                                                    -7 & 8  & 3
                                                \end{vmatrix} \\
                         & = -78
                    \end{align*}
          \end{enumerate}

    \item 有问题,之后可能会考虑修改题目,此处略.

    \item \begin{align*}
              \lvert A+B^{-1} \rvert ={} & \lvert B^{-1}BA+B^{-1}E \rvert = \lvert B^{-1} \rvert \cdot \lvert BA+E \rvert                  \\
              ={}                        & \frac{1}{2} \lvert BA+A^{-1}A \rvert = \frac{1}{2} \lvert B+A^{-1} \rvert \cdot \lvert A \rvert \\
              ={}                        & \frac{3}{2} \lvert A^{-1}+B \rvert = 3
          \end{align*}

    \item 正交矩阵满足 $AA^{\mathrm{T}} = A^{\mathrm{T}}A = E$,所以 $\lvert AA^{\mathrm{T}} \rvert = \lvert A \rvert^2 = \lvert E \rvert = 1$. 而 $\lvert A \rvert < 0$,所以 $\lvert A \rvert = -1$.
          \[\lvert E+A \rvert = \lvert AA^{\mathrm{T}}+A \rvert = \lvert A \rvert \cdot \lvert A^{\mathrm{T}}+E \rvert = -\lvert (A+E)^{\mathrm{T}} \rvert = -\lvert E+A \rvert.\]
          故 $\lvert E+A \rvert = 0$.

          以上两道题都多次运用了 $E = AA^{-1} = A^{-1}A$ 的技巧,请大家留意.

    \item \begin{enumerate}
              \item 由于行列式
                    \[ D = \begin{vmatrix}
                            a_{11}     & a_{12}     & a_{13}     & \cdots & a_{1n}     \\
                            a_{11}     & a_{12}     & a_{13}     & \cdots & a_{1n}     \\
                            a_{21}     & a_{22}     & a_{23}     & \cdots & a_{2n}     \\
                            \vdots     & \vdots     & \vdots     & \ddots & \vdots     \\
                            a_{n-1, 1} & a_{n-1, 2} & a_{n-1, 3} & \cdots & a_{n-1, n}
                        \end{vmatrix} = 0\]
                    而 $M_1, -M_2, \ldots, (-1)^{n-1}M_n$ 恰是 $D$ 的第一行元素的代数余子式,所以将 $D$ 按第一行展开,可知\[a_{11}M_1+a_{12}(-M_2)+\cdots+a_{1n}(-1)^{n-1}M_n=0.\]
                    而 $D$ 的其他行元素与第一行元素的代数余子式乘积之和为 0,于是结论成立.

              \item 因为 $r(A) = n-1$,所以 $M_1, -M_2, \ldots, (-1)^{n-1}M_n$,不全为 0. 且该方程组解空间维数为 1,$M_1, -M_2, \ldots, (-1)^{n-1}M_n$ 正是该方程组的非零解,结论成立.
          \end{enumerate}

    \item $A^* = \lvert A \rvert A^{-1} = 2A^{-1}, B^* = \lvert B \rvert B^{-1} = B^{-1}$,故 \[\lvert 2A^*B^*-A^{-1}B^{-1} \rvert = \lvert 4A^{-1}B^{-1}-A^{-1}B^{-1} \rvert = \lvert 3A^{-1}B^{-1} \rvert = \dfrac{3^n}{\lvert A \rvert \lvert B \rvert} = \dfrac{3^n}{2}.\]

    \item \begin{enumerate}
              \item 设 $A = (a_{ij})$,则 $A^* = (A_{ji})$($A^*$ 的表达式). 而 $A^T = A^*$,有 $a_{ij} = A_{ij}, \forall i, j = 1, 2, \ldots, n$. 而 $A$ 非零,故 $\exists a_{kl} \neq 0$,从而有 \[ \lvert A \rvert = a_{k1}A_{k1}+\cdots+a_{kn}A_{kn} = a_{k1}^2+\cdots+a_{kn}^2 > 0.\]

              \item $\lvert A \rvert > 0$ 有 $A$ 是可逆的,故 $A^* = \lvert A \rvert A^{-1} = A^{\mathrm{T}}$,从而 $A^{\mathrm{T}}A = \lvert A \rvert E$. 两侧取行列式有 $\lvert A \rvert^2 = \lvert A \rvert^n $,结合 $\lvert A \rvert > 0$ 有 $\lvert A \rvert = 1$.

              \item $\lvert A \rvert = 1$,故 $A^{-1} = A^{\mathrm{T}}$,$A$ 是正交矩阵.

              \item \begin{align*}
                        \lvert E-A \rvert & = \lvert AA^{\mathrm{T}}-AE \rvert = \lvert A \rvert \lvert A^{\mathrm{T}}-E\rvert = \lvert A^{\mathrm{T}}-E\rvert = \lvert (A-E)^{\mathrm{T}} \rvert \\
                                          & = \lvert A-E \rvert = (-1)^n\lvert E-A \rvert.
                    \end{align*}
                    $n$ 为奇数,则 $\lvert E-A \rvert = -\lvert E-A \rvert$,即 $\lvert E-A \rvert = 0$.
          \end{enumerate}

    \item $r(A) = n-1$,则 $\lvert A \rvert = 0$ 且 $AX = 0$ 的解空间的维数为 1. 而考虑 $AA^* = \lvert A \rvert E = 0$,且 $\exists A_{ij} \neq 0$,所以 $(A_{i1}, A_{i2}, \ldots , A_{in})^{\mathrm{T}}$ 是所求的基础解系.

    \item 设 \[A = \begin{pmatrix}
                  a_{11} & a_{12} & \cdots & a_{1n} \\
                  a_{21} & a_{22} & \cdots & a_{2n} \\
                  \vdots & \vdots & \ddots & \vdots \\
                  a_{n1} & a_{n2} & \cdots & a_{nn} \\
              \end{pmatrix}\]
          则 \[A^* = \begin{pmatrix}
                  A_{11}     & A_{21}     & \cdots & A_{n-1, 1}   & A_{n1}     \\
                  A_{12}     & A_{22}     & \cdots & A_{n-1, 2}   & A_{n2}     \\
                  \vdots     & \vdots     & \ddots & \vdots       & \vdots     \\
                  A_{1, n-1} & A_{2, n-1} & \cdots & A_{n-1, n-1} & A_{n, n-1} \\
                  A_{1n}     & A_{2n}     & \cdots & A_{n-1, n}   & A_{nn}     \\
              \end{pmatrix}\]
          注意到目标行列式是 $A^*$ 中元素 $A_{nn}$ 的代数余子式,也就是 $(A^*)^*$ 中 $(n, n)$ 位置的元素. 由 {例13.9(4)} $(A^*)^* = \lvert A \rvert^{n-2}A$ 可知结论成立. % FIXME: xref

    \item 设 $f(x) = c_0+c_1x+c_2x^2+\cdots+c_{n-1}x^{n-1}$,则有
          \[\begin{cases} \begin{aligned}
                      c_0+c_1a_1+c_2a_1^2+\cdots+c_{n-1}a_1^{n-1} & = b_1,            \\
                      c_0+c_1a_2+c_2a_2^2+\cdots+c_{n-1}a_2^{n-1} & = b_2,            \\
                                                                  & \vdotswithin{ = } \\
                      c_0+c_1a_n+c_2a_n^2+\cdots+c_{n-1}a_n^{n-1} & = b_n.            \\
                  \end{aligned} \end{cases}\]
          注意此处我们研究的对象是 $(c_0, c_1, \ldots, c_{n-1})^{\mathrm{T}}$. 因为系数行列式
          \[D = \begin{vmatrix}
                  1      & a_1    & \cdots & a_1^{n-1} \\
                  1      & a_2    & \cdots & a_2^{n-1} \\
                  \vdots & \vdots & \ddots & \vdots    \\
                  1      & a_n    & \cdots & a_n^{n-1} \\
              \end{vmatrix} = \prod_{1 \leqslant j < i \leqslant n} (a_i-a_j) \neq 0.\]
          所以由 Cramer 法则,上述关于 $(c_0, c_1, \ldots, c_{n-1})^{\mathrm{T}}$ 的方程组有唯一解,所以满足条件的多项式函数 $f$ 是唯一存在的.

    \item 令 $A = (\alpha_1, \alpha_2, \ldots, \alpha_n), B = \begin{pmatrix}
                  \alpha_1^{\mathrm{T}}\alpha_1 & \alpha_1^{\mathrm{T}}\alpha_2 & \cdots & \alpha_1^{\mathrm{T}}\alpha_n \\
                  \alpha_2^{\mathrm{T}}\alpha_1 & \alpha_2^{\mathrm{T}}\alpha_2 & \cdots & \alpha_2^{\mathrm{T}}\alpha_n \\
                  \vdots                        & \vdots                        & \ddots & \vdots                        \\
                  \alpha_n^{\mathrm{T}}\alpha_1 & \alpha_n^{\mathrm{T}}\alpha_2 & \cdots & \alpha_n^{\mathrm{T}}\alpha_n \\
              \end{pmatrix}$,显然 $B = A^{\mathrm{T}}A$,由 {11.4 秩不等式第 4 个} 有 $r(B) = r(A)$. 进而 $n$ 维向量组 $(\alpha_1, \alpha_2, \ldots, \alpha_n)$ 线性无关等价于 $r(A) = n$,也就等价于 $r(B) = n$,进而等价于 $\lvert B \rvert \neq 0$,命题得证. % FIXME: xref

    \item $(\implies)$ 记 $A = (\alpha_1, \alpha_2, \ldots, \alpha_n), \varepsilon_i = (0, \ldots, 1, 0, \ldots, 0)^{\mathrm{T}}$(第 $i$ 个为 1),$E = (\varepsilon_1, \varepsilon_2, \ldots, \varepsilon_n)$. 因为 $\alpha_1, \alpha_2, \ldots, \alpha_n$ 是 $n$ 维线性无关的向量组,故 $\lvert A \rvert \neq 0$,即 $A$ 可逆,$A = EA, AA^{-1} = E$,\[(\alpha_1, \alpha_2, \ldots, \alpha_n) = (\varepsilon_1, \varepsilon_2, \ldots, \varepsilon_n)A, (\varepsilon_1, \varepsilon_2, \ldots, \varepsilon_n) = (\alpha_1, \alpha_2, \ldots, \alpha_n)A^{-1}.\] 即 $\alpha_1, \alpha_2, \ldots, \alpha_n$ 与 $\varepsilon_1, \varepsilon_2, \ldots, \varepsilon_n$ 等价. $\varepsilon_1, \varepsilon_2, \ldots, \varepsilon_n$ 为 $\mathbf{R}^n$ 的一个基,能表出任一 $n$ 维向量,故 $\alpha_1, \alpha_2, \ldots, \alpha_n$ 能表出 $\mathbf{R}^n$ 中任一 $n$ 维向量.

          $(\impliedby)$ 若 $\alpha_1, \alpha_2, \ldots, \alpha_n$ 能表出任一 $n$ 维向量,则 $\varepsilon_i = \lambda_{i1}\alpha_1+\cdots+\lambda_{in}\alpha_n, i = 1, 2, \ldots, n$. 即
          \[E = (\varepsilon_1, \varepsilon_2, \ldots, \varepsilon_n) = (\alpha_1, \alpha_2, \ldots, \alpha_n)\begin{pmatrix}
                  \lambda_{11} & \lambda_{21} & \cdots & \lambda_{n1} \\
                  \lambda_{11} & \lambda_{21} & \cdots & \lambda_{n1} \\
                  \vdots       & \vdots       & \ddots & \vdots       \\
                  \lambda_{11} & \lambda_{21} & \cdots & \lambda_{n1} \\
              \end{pmatrix}.\]
          进而 $\lvert \alpha_1, \alpha_2, \ldots, \alpha_n \rvert \begin{vmatrix}
                  \lambda_{11} & \lambda_{21} & \cdots & \lambda_{n1} \\
                  \lambda_{11} & \lambda_{21} & \cdots & \lambda_{n1} \\
                  \vdots       & \vdots       & \ddots & \vdots       \\
                  \lambda_{11} & \lambda_{21} & \cdots & \lambda_{n1} \\
              \end{vmatrix} = 1$,所以 $\lvert \alpha_1, \alpha_2, \ldots, \alpha_n \rvert \neq 0$,$\alpha_1, \alpha_2, \ldots, \alpha_n$ 线性无关.

    \item $A$ 的前 $s$ 列组成的 $s$ 阶子式为Vandermonde行列式
          \[D = \begin{vmatrix}
                  1      & a      & a^2    & \cdots & a^{n-1}    \\
                  1      & a^2    & a^4    & \cdots & a^{2(n-1)} \\
                  \vdots & \vdots & \vdots & \ddots & \vdots     \\
                  1      & a^s    & a^{2s} & \cdots & a^{s(n-1)} \\
              \end{vmatrix}.\]
          由于当 $0 < r < n$ 时,$a^r \neq 1$,因此 $a, a^2, \ldots, a^s$ 两两不同,进而 $D \neq 0$,于是 $r(A) \geqslant s$. 又因为 $A$ 的行数是 $s$,所以 $r(A) \leqslant s$. 从而 $r(A) = s$.

          \item \begin{enumerate}
            \item 可以硬拆成8个行列式,其中只有2个非零,则得到
                  \[D=\begin{vmatrix}
                          ax & ay & az \\
                          ay & az & ax \\
                          az & ax & ay
                      \end{vmatrix}+\begin{vmatrix}
                          by & bz & bx \\
                          bz & bx & by \\
                          bx & by & bz \\
                      \end{vmatrix}=(a+b)\begin{vmatrix}
                          x & y & z \\
                          y & z & x \\
                          z & x & y \\
                      \end{vmatrix}=(a^3+b^3)(3xyz-\sum x^2),\]
                  另解:分解成两个矩阵相乘再各自求行列式. 这其实是看出这是一种线性变换的本质之后的做法:
                  \[D=\begin{vmatrix}
                          \begin{pmatrix}
                              a & b & 0 \\
                              0 & a & b \\
                              b & 0 & a \\
                          \end{pmatrix}
                          \begin{pmatrix}
                              x & y & z \\
                              y & z & x \\
                              z & x & y \\
                          \end{pmatrix}
                      \end{vmatrix}.\]

            \item 其实直接用三阶行列式的公式也不错,这里介绍基于硬拆的方法
                  \begin{align*}
                      D & =x\begin{vmatrix}
                                x & xy    & xz    \\
                                y & y^2+1 & yz    \\
                                z & yz    & z^2+1
                            \end{vmatrix}+\begin{vmatrix}
                                              1 & xy    & xz    \\
                                              0 & y^2+1 & yz    \\
                                              0 & yz    & z^2+1
                                          \end{vmatrix} \\
                        & =x\begin{vmatrix}
                                x & 0 & 0 \\
                                y & 1 & 0 \\
                                z & 0 & 1
                            \end{vmatrix}+\begin{vmatrix}
                                              y^2+1 & yz    \\
                                              yz    & z^2+1
                                          \end{vmatrix}
                      =\sum x^2+1
                  \end{align*}
        \end{enumerate}

  \item 利用$|\lambda E_m-AB|=\lambda^{m-n}|\lambda E_n-BA|$
        \begin{align*}
            |2E-\alpha_1^\mathrm{T}\beta_1-\alpha_2^\mathrm{T}\beta_2|
             & =\begin{vmatrix}
                    2E-\begin{pmatrix}
                       \alpha_1^\mathrm{T} & \alpha_2^\mathrm{T}
                   \end{pmatrix}\begin{pmatrix}
                                    \beta_1 \\\beta_2
                                \end{pmatrix}
                \end{vmatrix}
            =2^{n-2}\begin{vmatrix}
                        2E-\begin{pmatrix}
                       \beta_1 \\\beta_2
                   \end{pmatrix}\begin{pmatrix}
                                    \alpha_1^\mathrm{T} & \alpha_2^\mathrm{T}
                                \end{pmatrix}
                    \end{vmatrix}         \\
             & =2^{n-2}\begin{vmatrix}
                           2-\beta_1\alpha_1^\mathrm{T} & -\beta_1\alpha_2^\mathrm{T}  \\
                           -\beta_2\alpha_1^\mathrm{T}  & 2-\beta_2\alpha_2^\mathrm{T}
                       \end{vmatrix} \\
             & =2^{n-2}
            \left(\left(2-\sum_{i=1}^na_ib_i\right)
            \left(2-\sum_{i=1}^nc_id_i\right)
            -\left(\sum_{i=1}^na_id_i\right)
            \left(\sum_{i=1}^nb_ic_i\right)\right)
        \end{align*}

  \item 由条件有$|E+A|=|AA^\mathrm{T}+A|=|A(A^\mathrm{T}+E)|=|A||A^\mathrm{T}+E|=-|A^\mathrm{T}+E|$,然而,由$(E+A)^\mathrm{T}=E+A^\mathrm{T}$,有$|E+A|=|A^\mathrm{T}+E|$,综上两式就有$|E+A|=0$.

    \item \begin{enumerate}
              \item 线性变换的验证省略.

              \item $(\implies)$ 若 $\lvert AB \rvert = 0$,则 $\lvert A \rvert = 0$ 或 $\lvert B \rvert = 0$,故 $A$ 或 $B$ 不可逆. 不妨假设 $A$ 不可逆,则存在 $X_0 \neq 0$ 使得 $AX_0 = 0$,$T(X_0) = AX_0B = 0$. 但 $T$ 是可逆的,所以 $T$ 是单射,$T(X) = 0 \Leftrightarrow X = 0$,矛盾.

                    $(\impliedby)$ $\lvert AB \rvert \neq 0$ 则 $A, B$ 可逆,故 $T(X) = AXB$ 的逆映射为 $T^{-1}(X) = A^{-1}XB^{-1}$.
          \end{enumerate}

    \item 因为 $r(A) < n, A_{11} \neq 0$,所以 $r(A) = n-1$,进而由 {例13.9 (6)} 可知 $r(A^*) = 1$,所以有 % FIXME: xref
          \[A^* = \begin{pmatrix} a_1 \\ a_2 \\ \vdots \\ a_n \end{pmatrix} (\lambda_1, \lambda_2, \ldots, \lambda_n).\]
          设 $(\lambda_1, \lambda_2, \ldots, \lambda_n) \begin{pmatrix} a_1 \\ a_2 \\ \vdots \\ a_n \end{pmatrix} = k$,则
          \begin{align*}
              (A^*)^2 & = \begin{pmatrix}a_1 \\ a_2 \\ \vdots \\ a_n \end{pmatrix}
              (\lambda_1, \lambda_2, \ldots, \lambda_n)
              \begin{pmatrix} a_1 \\ a_2 \\ \vdots \\ a_n \end{pmatrix}
              (\lambda_1, \lambda_2, \ldots, \lambda_n)                                     \\
                      & = \begin{pmatrix} a_1 \\ a_2 \\ \vdots \\ a_n \end{pmatrix} \cdot k
              \cdot (\lambda_1, \lambda_2, \ldots, \lambda_n) = k
              \begin{pmatrix} a_1 \\ a_2 \\ \vdots \\ a_n \end{pmatrix}
              (\lambda_1, \lambda_2, \ldots, \lambda_n) = kA^*
          \end{align*}

    \item $\forall a_i \in \mathbf{R}, i \in \mathbf{Z}_{+}$ 满足:若 $i \neq j$,则 $a_i \neq a_j$,考虑以Vandermonde行列式的形式进行排布. 设
          \[ A = \begin{pmatrix}
                  1         & 1         & \cdots & 1         & \cdots \\
                  a_1       & a_2       & \cdots & a_k       & \cdots \\
                  a_1^2     & a_2^2     & \cdots & a_k^2     & \cdots \\
                  \vdots    & \vdots    & \ddots & \vdots    &        \\
                  a_1^{n-1} & a_2^{n-1} & \cdots & a_k^{n-1} & \cdots
              \end{pmatrix}\]
          考虑 $A$ 的任意 $n$ 阶主子式 $D_n$,其均构成Vandermonde行列式,又 $i \neq j$ 有 $a_i \neq a_j$,所以值均不为 0,也就是说任意 $n$ 个向量都线性无关,其个数也恰好为 $n$,构成 $V$ 的一组基,命题得证.

    \item \begin{align*}
              f(A) g(A) & = (E + A + A^2 + \cdots + A^{m - 1})(E - A) \\
                        & = E - A^m = \begin{pmatrix}
                                          1 - a^m & -mba^{m - 1} \\
                                          0       & 1 - a^m
                                      \end{pmatrix}
          \end{align*}

    \item 求与 $ A $ 可交换的矩阵等价于求与 $ A - E $ 可交换的矩阵 $ X $. 可解得
          \begin{align*}
              X & = \begin{pmatrix}
                        x_{11} & -2x_{11} - 2x_{32} + 2x_{33} & 4x_{32} \\
                        0      & -x_{32} + x_{33}             & 2x_{32} \\
                        0      & x_{32}                       & x_{33}
                    \end{pmatrix} \\
                & = x_{11} \begin{pmatrix}
                               1 & -2 & 0 \\
                               0 & 0  & 0 \\
                               0 & 0  & 0
                           \end{pmatrix}
              + x_{32} \begin{pmatrix}
                           0 & -2 & 4 \\
                           0 & -1 & 2 \\
                           0 & 1  & 0
                       \end{pmatrix}
              + x_{33} \begin{pmatrix}
                           0 & 2 & 0 \\
                           0 & 1 & 0 \\
                           0 & 0 & 1
                       \end{pmatrix}
          \end{align*}
          其中 $ x_{11}, x_{32}, x_{33} $ 为任意实数. 由此我们也得到了 $ A $ 可交换的矩阵构成的子空间的一组基.

    \item \begin{enumerate}
              \item 不妨设 $ B = (b_{ij})_{3 \times 3} $ 与 $ A $ 可交换,即 $ AB = BA $,这等价于 $ (A - E)B = B(A - E) $,即
                    \[ \begin{pmatrix}
                              &   & 1 \\
                              & 1 &   \\
                            1 &   &
                        \end{pmatrix}
                        \begin{pmatrix}
                            b_{11} & b_{12} & b_{13} \\
                            b_{21} & b_{22} & b_{23} \\
                            b_{31} & b_{32} & b_{33}
                        \end{pmatrix} =
                        \begin{pmatrix}
                            b_{11} & b_{12} & b_{13} \\
                            b_{21} & b_{22} & b_{23} \\
                            b_{31} & b_{32} & b_{33}
                        \end{pmatrix}
                        \begin{pmatrix}
                              &   & 1 \\
                              & 1 &   \\
                            1 &   &
                        \end{pmatrix} \]
                    于是
                    \[ \begin{pmatrix}
                            b_{31} & b_{32} & b_{33} \\
                            b_{21} & b_{22} & b_{23} \\
                            b_{11} & b_{12} & b_{13}
                        \end{pmatrix} = \begin{pmatrix}
                            b_{11} & b_{12} & b_{13} \\
                            b_{21} & b_{22} & b_{23} \\
                            b_{31} & b_{32} & b_{33}
                        \end{pmatrix} \]
                    对应元素相等,可得 $ b_{31} = b_{13}, b_{32} = b_{12}, b_{33} = b_{11}, b_{21} = b_{23} $,即所有与 $ A $ 可交换的矩阵为
                    \[ \begin{pmatrix}
                            b_{11} & b_{12} & b_{13} \\
                            b_{21} & b_{22} & b_{21} \\
                            b_{13} & b_{12} & b_{11}
                        \end{pmatrix} \]
                    其中 $ b_{11}, b_{12}, b_{13}, b_{21}, b_{22} $ 为任意实数.

              \item 由于 $ AB + E = A^2 + B $,所以
                    \[ (A - E)B = A^2 - E = (A - E)(A + E) \]
                    又由于 $ |A - E| = -1 \neq 0 $,所以 $ A - E $ 可逆,进而
                    \[ B = A + E = \begin{pmatrix}
                            2 & 0 & 1 \\
                            0 & 3 & 0 \\
                            1 & 0 & 2
                        \end{pmatrix} \]
          \end{enumerate}

    \item \begin{enumerate}
              \item 首先有 $ E \in C(A) $,所以 $ C(A) $ 非空. $ \forall B_1, B_2 \in C(A), \lambda \in \mathbf{F} $,有
                    \begin{gather*}
                        A(B_1 + B_2) = AB_1 + AB_2 = B_1A + B_2A = (B_1 + B_2)A \\
                        A(\lambda B_1) = \lambda AB_1 = \lambda B_1A = (\lambda B_1)A
                    \end{gather*}
                    所以 $ C(A) $ 是 $ \mathbf{F}^{n \times n} $ 的子空间.

              \item $ C(E) = \mathbf{F}^{n \times n} $.

              \item 我们有结论:$ C(A) $ 为全体对角矩阵构成的集合. 故 $ \dim C(A) = n $. 基矩阵 $ B_k = (b_{ij})_{n \times n} $,其中 $ b_{ij} = \delta_{ij} \delta_{jk} $. $ B_1, B_2, \ldots, B_n $ 为一组基.
          \end{enumerate}

    \item \begin{enumerate}
              \item 由 $ A^k = O $ 有
                    \[ E = E - A^k = (E - A)(E + A + \cdots + A^{k - 1}) \]
                    故 $ E - A $ 可逆,且
                    \[ (E - A)^{-1} = E + A + \cdots + A^{k - 1} \]

              \item 若 $ k $ 为偶数,设 $ k = 2m $,由 $ A^k = O $ 有
                    \[ E = E - A^{2m} = (E + A)(E - A)(E + A^2 + A^4 + \cdots + A^{2m - 2}) \]
                    故 $ E + A $ 可逆,且
                    \[ (E + A)^{-1} = (E - A)(E + A^2 + A^4 + \cdots + A^{2m - 2}) \]

                    若 $ k $ 为奇数,设 $ k = 2m + 1 $,由 $ A^k = O $ 有
                    \[ E = E - A^{2m + 1} = (E - A)(E + A + A^2 + \cdots + A^{2m - 1} + A^{2m}) \]
                    故 $ E - A $ 可逆,且
                    \[ (E - A)^{-1} = E + A + A^2 + \cdots + A^{2m - 1} + A^{2m} \]

              \item 由于 $ A^k = O $,于是
                    \[ e^A = E + A + \frac{1}{2!} A^2 + \cdots + \frac{1}{(k - 1)!} A^{k - 1} \]
                    由 $ e^A e^{-A} = E $ 知 $ E + A + \dfrac{1}{2!} A^2 + \cdots + \dfrac{1}{(k - 1)!} A^{k - 1} $ 可逆,且
                    \[ (E + A + \frac{1}{2!} A^2 + \cdots + \frac{1}{(k - 1)!} A^{k - 1})^{-1} = e^{-A} \]
          \end{enumerate}
\end{enumerate}

\centerline{\heiti C组}
\begin{enumerate}
    \item \begin{align*}
              \begin{pmatrix} a_n \\ b_n \\ 2^n \end{pmatrix}
               & = \begin{pmatrix}
                       3 & 1 & 1 \\
                       2 & 4 & 2 \\
                       0 & 0 & 2
                   \end{pmatrix} \begin{pmatrix} a_{n - 1} \\ b_{n - 1} \\ 2^{n - 1} \end{pmatrix} \\
               & = \cdots                                                                          \\
               & = \begin{pmatrix}
                       3 & 1 & 1 \\
                       2 & 4 & 2 \\
                       0 & 0 & 2
                   \end{pmatrix}^n \begin{pmatrix} a_0 \\ b_0 \\ 1 \end{pmatrix}
              = \left(2E + \begin{pmatrix}
                               1 & 1 & 1 \\
                               2 & 2 & 2 \\
                               0 & 0 & 0
                           \end{pmatrix}\right)^n \begin{pmatrix} -1 \\ 3 \\ 1 \end{pmatrix}
          \end{align*}

    \item 略. 感兴趣的同学可以通过设未知数,利用初等变换计算.

    \item 对原矩阵进行分块初等变换化为上三角块矩阵后进行计算. 因为 $\lvert A \rvert \neq 0$,所以 $A$ 可逆,进而有如下变换:
          \[\begin{pmatrix}
                  E        & O \\
                  -CA^{-1} & E
              \end{pmatrix} \begin{pmatrix}
                  A & B \\
                  C & D \\
              \end{pmatrix} = \begin{pmatrix}
                  A & B          \\
                  O & D-CA^{-1}B
              \end{pmatrix},\]
          所以
          \begin{align*}
                  & \begin{vmatrix}
                        A & B \\
                        C & D \\
                    \end{vmatrix}
              = \begin{vmatrix}
                    E        & O \\
                    -CA^{-1} & E \\
                \end{vmatrix}
              \begin{vmatrix}
                  A & B \\
                  C & D \\
              \end{vmatrix}       \\
              ={} & \begin{vmatrix}
                        A & B          \\
                        O & D-CA^{-1}B
                    \end{vmatrix}
              = \lvert A \rvert \lvert D-CA^{-1}B \rvert = \lvert AD-ACA^{-1}B \rvert.
          \end{align*} 由于 $AC = CA$,所以有 $ACA^{-1} = CAA^{-1} = C$,所以
          \[\begin{vmatrix}
                  A & B \\
                  C & D \\
              \end{vmatrix} = \lvert AD-CB \rvert.\]

    \item 这道题目我们利用了一个分块矩阵作为中间``桥梁''使得其通过分块初等变换之后能分别得到两个方向上的结果. 考虑矩阵 $\begin{pmatrix}
                  A                   & \alpha \\
                  -\beta^{\mathrm{T}} & 1
              \end{pmatrix}$,有
          \[\begin{pmatrix}
                  A                   & \alpha \\
                  -\beta^{\mathrm{T}} & 1
              \end{pmatrix} \begin{pmatrix}
                  E                  & O \\
                  \beta^{\mathrm{T}} & E
              \end{pmatrix} = \begin{pmatrix}
                  A+\alpha \beta^{\mathrm{T}} & \alpha \\
                  O                           & 1
              \end{pmatrix},\]
          所以
          \[\begin{vmatrix}
                  A                   & \alpha \\
                  -\beta^{\mathrm{T}} & 1
              \end{vmatrix} = \begin{vmatrix}
                  A                   & \alpha \\
                  -\beta^{\mathrm{T}} & 1
              \end{vmatrix} \begin{vmatrix}
                  E                  & O \\
                  \beta^{\mathrm{T}} & E
              \end{vmatrix} = \begin{vmatrix}
                  A+\alpha \beta^{\mathrm{T}} & \alpha \\
                  O                           & 1
              \end{vmatrix} = \lvert A+\alpha \beta^{\mathrm{T}} \rvert.\]
          另一方面,
          \[\begin{pmatrix}
                  E                        & O \\
                  \beta^{\mathrm{T}}A^{-1} & 1
              \end{pmatrix} \begin{pmatrix}
                  A                   & \alpha \\
                  -\beta^{\mathrm{T}} & 1
              \end{pmatrix} = \begin{pmatrix}
                  A & \alpha                           \\
                  O & 1+\beta^{\mathrm{T}}A^{-1}\alpha
              \end{pmatrix}.\]
          注意 $\beta^{\mathrm{T}}A^{-1}\alpha$ 的最终结果是一个数. 进而
          \[\begin{vmatrix}
                  A                   & \alpha \\
                  -\beta^{\mathrm{T}} & 1
              \end{vmatrix} = \begin{vmatrix}
                  E                        & O \\
                  \beta^{\mathrm{T}}A^{-1} & 1
              \end{vmatrix} \begin{vmatrix}
                  A                   & \alpha \\
                  -\beta^{\mathrm{T}} & 1
              \end{vmatrix} = \begin{vmatrix}
                  A & \alpha                           \\
                  O & 1+\beta^{\mathrm{T}}A^{-1}\alpha
              \end{vmatrix} = \lvert A \rvert(1+\beta^{\mathrm{T}}A^{-1}\alpha).\]
          所以有
          \[\lvert A+\alpha \beta^{\mathrm{T}} \rvert = \lvert A \rvert(1+\beta^{\mathrm{T}}A^{-1}\alpha)\]

    \item 依旧是对分块矩阵做初等分块变换. 考虑到
          \begin{align*}
              \left(\begin{pmatrix}
                        E & E \\
                        O & E
                    \end{pmatrix}
              \begin{pmatrix}
                  A & B \\
                  B & A
              \end{pmatrix} \right)
              \begin{pmatrix}
                  E & -E \\
                  O & E
              \end{pmatrix}
               & = \begin{pmatrix}
                       A+B & A+B \\
                       B   & A
                   \end{pmatrix}
              \begin{pmatrix}
                  E & -E \\
                  O & E
              \end{pmatrix}      \\
               & = \begin{pmatrix}
                       A+B & O   \\
                       B   & A-B
                   \end{pmatrix},
          \end{align*}
          所以有 \begin{align*}
              \begin{vmatrix}
                  A & B \\
                  B & A
              \end{vmatrix}
               & = \begin{vmatrix}
                       E & E \\
                       O & E
                   \end{vmatrix}
              \begin{vmatrix}
                  A & B \\
                  B & A
              \end{vmatrix}
              \begin{vmatrix}
                  E & -E \\
                  O & E
              \end{vmatrix}
              = \begin{vmatrix}
                    A+B & O   \\
                    B   & A-B
                \end{vmatrix}    \\
               & =
              \lvert (A+B)(A-B) \rvert = \lvert A+B \rvert \lvert A-B \rvert
          \end{align*}

    \item 首先有 $\begin{vmatrix}
                  \lvert A \rvert & \lvert B \rvert \\
                  \lvert C \rvert & \lvert D \rvert
              \end{vmatrix} = \lvert A \rvert \lvert D \rvert-\lvert B \rvert \lvert C \rvert$.
          \begin{enumerate}
              \item 若 $\lvert A \rvert \neq 0$,即 $A$ 可逆. 因为矩阵的初等变换不改变矩阵的秩,所以由
                    \[\begin{pmatrix}
                            E        & O \\
                            -CA^{-1} & E
                        \end{pmatrix} \begin{pmatrix}
                            A & B \\
                            C & D
                        \end{pmatrix} \begin{pmatrix}
                            E & -A^{-1}B \\
                            O & E        \\
                        \end{pmatrix} = \begin{pmatrix}
                            A & O          \\
                            O & D-CA^{-1}B
                        \end{pmatrix},\]
                    条件 $r\left(\begin{pmatrix}
                                A & B \\
                                C & D
                            \end{pmatrix}\right) = n$ 以及 $A$ 可逆,可以得到
                    \[D-CA^{-1}B = O.\]
                    即若 $A$ 可逆,则 $D = CA^{-1}B$,并且
                    \[\begin{vmatrix}
                            \lvert A \rvert & \lvert B \rvert \\
                            \lvert C \rvert & \lvert D \rvert
                        \end{vmatrix} = \lvert A \rvert \lvert CA^{-1}B \rvert-\lvert B \rvert \lvert C \rvert = \lvert A \rvert \lvert C \rvert \lvert A^{-1} \rvert \lvert B \rvert-\lvert B \rvert \lvert C \rvert = 0.\]

              \item 若 $\lvert A \rvert = 0$,只需证 $\lvert B \rvert \lvert C \rvert = 0$. 若 $\lvert B \rvert \neq 0$,则由
                    \[\begin{pmatrix}
                            E        & O \\
                            -DB^{-1} & E
                        \end{pmatrix} \begin{pmatrix}
                            A & B \\
                            C & D
                        \end{pmatrix} \begin{pmatrix}
                            E        & O \\
                            -B^{-1}A & E
                        \end{pmatrix} = \begin{pmatrix}
                            O          & B \\
                            C-DB^{-1}A & O
                        \end{pmatrix},\]
                    有 $C-DB^{-1}A = O$. 注意到 $\lvert A \rvert = 0$,故 \[\lvert C \rvert = \lvert DB^{-1}A \rvert = \lvert D \rvert \lvert B^{-1} \rvert \lvert A \rvert = 0.\] 同理可证若 $\lvert C \rvert \neq 0$,则 $\lvert B \rvert = 0$.
          \end{enumerate}
          综上,结论成立.

    \item \begin{enumerate}
              \item \label{item:13:B:5:1}
                    采用反证法. 设 $\lvert A \rvert = 0$,则线性方程组 $AX = 0$ 有非零解,设为 $X_0 = (x_1, x_2, \ldots, x_n)^{\mathrm{T}}$,记
                    \[\lvert x_k \rvert = \max \{\lvert x_1 \rvert, \lvert x_2 \rvert, \ldots, \lvert x_n \rvert\}.\]
                    由 $X_0 \neq 0$ 可知 $\lvert x_k \rvert > 0$,考虑 $AX = 0$ 的第 $k$ 个方程,有 $\displaystyle\sum_{j=1}^n a_{kj}x_j = 0$,于是
                    \[\lvert a_{kk} \rvert \lvert x_k \rvert = \lvert -\displaystyle\sum_{j \neq k}a_{kj}x_j \rvert \leqslant \sum_{j \neq k}\lvert a_{kj} \rvert \lvert x_k \rvert.\]
                    约去 $\lvert x_k \rvert$ 后可得 $\lvert a_{kk} \rvert \leqslant \displaystyle\sum_{j \neq k} \lvert a_{kj} \rvert$,这与条件矛盾. 所以 $\lvert A \rvert \neq 0$.

              \item 构造实函数
                    \[f(t) = \begin{vmatrix}
                            a_{11}  & ta_{12} & ta_{13} & \cdots & ta_{1n} \\
                            ta_{21} & a_{22}  & ta_{23} & \cdots & ta_{2n} \\
                            ta_{31} & ta_{32} & a_{33}  & \cdots & ta_{3n} \\
                            \vdots  & \vdots  & \vdots  & \ddots & \vdots  \\
                            ta_{n1} & ta_{n2} & ta_{n3} & \cdots & a_{nn}  \\
                        \end{vmatrix}\]
                    由于 $A$ 是实矩阵,所以 $f(t)$ 是关于 $t$ 的一个实系数多项式(连续)函数,同时
                    \[f(0) = a_{11}a_{22}\cdots a_{nn} > 0.\] 当$t \in [0, 1]$ 时,还有
                    \[a_{ii} > \sum_{j \neq i} \lvert a_{ij} \rvert \leqslant \sum_{j \neq i} \lvert ta_{ij} \rvert.\]
                    由 \ref*{item:13:B:5:1} 可知 $f(t)$ 在 $[0, 1]$ 上非零,由连续函数的介值定理可知 $f(1) > 0$,即 $\lvert A \rvert > 0$.

              \item 此为直接推论不再赘述.
          \end{enumerate}

    \item \begin{enumerate}
              \item 设 $\begin{pmatrix}
                            A & C \\
                            O & B
                        \end{pmatrix}$ 的伴随矩阵为 $\begin{pmatrix}
                            X & Y \\
                            Z & W
                        \end{pmatrix}$. 而 $\begin{vmatrix}
                            A & C \\
                            O & B
                        \end{vmatrix} = \lvert A\rvert \lvert B \rvert$,所以有
                    \[\begin{pmatrix}
                            A & C \\
                            O & B
                        \end{pmatrix} \begin{pmatrix}
                            X & Y \\
                            Z & W
                        \end{pmatrix} = \lvert A\rvert \lvert B \rvert \begin{pmatrix}
                            E & O \\
                            O & E
                        \end{pmatrix}.\]
                    得到方程组
                    \[\begin{cases}
                            AX+CZ = \lvert A \rvert \lvert B \rvert E \\
                            AY+CW = O                                 \\
                            BZ    = O                                 \\
                            BW    = \lvert A \rvert \lvert B \rvert E
                        \end{cases}\]
                    考虑一般情况,我们不再单独讨论 $B$ 是否等于 $O$. 所以 $Z = O$, $X = \lvert B \rvert A^*$, $W = \lvert A \rvert B^*$, $Y = -A^*CB^*$. 即 \[\begin{pmatrix}
                            A & C \\
                            O & B
                        \end{pmatrix}^* = \begin{pmatrix}
                            \lvert B \rvert A^* & -A^*CB^*            \\
                            O                   & \lvert A \rvert B^*
                        \end{pmatrix}\]

              \item 若 $A$ 可逆,则可以通过以下的初等分块变换将其化为上三角块矩阵.
                    \[\begin{pmatrix}
                            E        & O \\
                            -CA^{-1} & E
                        \end{pmatrix} \begin{pmatrix}
                            A & B \\
                            C & D
                        \end{pmatrix} = \begin{pmatrix}
                            A & B          \\
                            O & D-CA^{-1}B
                        \end{pmatrix}.\]
                    两侧取伴随有
                    \begin{align*}
                            & \left(\begin{pmatrix}
                                            E        & O \\
                                            -CA^{-1} & E
                                        \end{pmatrix}
                        \begin{pmatrix}
                                A & B \\
                                C & D
                            \end{pmatrix}\right)^*
                        = \begin{pmatrix}
                              A & B \\
                              C & D
                          \end{pmatrix}^*
                        \begin{pmatrix}
                            E        & O \\
                            -CA^{-1} & E
                        \end{pmatrix}^*            \\
                        ={} & \begin{pmatrix}
                                  A & B          \\
                                  O & D-CA^{-1}B
                              \end{pmatrix}^*
                        = \begin{pmatrix}
                              \lvert D-CA^{-1}B \rvert A^* & -A^*B(D-CA^{-1}B)^*            \\
                              O                            & \lvert A \rvert (D-CA^{-1}B)^*
                          \end{pmatrix}
                    \end{align*}
                    而
                    \[\begin{pmatrix}
                            E        & O \\
                            -CA^{-1} & E
                        \end{pmatrix}^* \begin{pmatrix}
                            E        & O \\
                            -CA^{-1} & E
                        \end{pmatrix} = \begin{vmatrix}
                            E        & O \\
                            -CA^{-1} & E
                        \end{vmatrix} \begin{pmatrix}
                            E & O \\
                            O & E
                        \end{pmatrix} = \begin{pmatrix}
                            E & O \\
                            O & E
                        \end{pmatrix}\]
                    所以
                    \begin{align*}
                            & \begin{pmatrix}
                                  A & B \\
                                  C & D
                              \end{pmatrix}^*                                                                         \\
                        ={} & \begin{pmatrix}
                                  \lvert D-CA^{-1}B \rvert A^* & -A^*B(D-CA^{-1}B)^*            \\
                                  O                            & \lvert A \rvert (D-CA^{-1}B)^*
                              \end{pmatrix}
                        \begin{pmatrix}
                            E        & O \\
                            -CA^{-1} & E
                        \end{pmatrix}                                                                                \\
                        ={} & \begin{pmatrix}
                                  \lvert D-CA^{-1}B \rvert A^*+A^*B(D-CA^{-1}B)^*CA^{-1} & -A^*B(D-CA^{-1}B)^*            \\
                                  -\lvert A \rvert (D-CA^{-1}B)^*CA^{-1}                 & \lvert A \rvert (D-CA^{-1}B)^*
                              \end{pmatrix} &
                    \end{align*}
          \end{enumerate}

    \item \begin{enumerate}
              \item 因为 $n=2$ 时 $(A^*)^* = A$,所以 $B = (B^*)^* = A^*$,而 $B$ 的伴随矩阵是唯一的,所以存在唯一的2阶方阵 $A = B^*$ 使得 $A^* = B$.

              \item $(\impliedby)$ 由 {例13.9(6)} 可得. % FIXME: xref

                    $(\implies)$ \begin{enumerate}
                        \item $r(B) = n$ 时,若存在 $A$ 使得 $A^* = B$,则由 $(A^*)^* = \lvert A \rvert^{n-2}A$,有
                              \[A = \dfrac{1}{\lvert A \rvert^{n-2}}(A^*)^* = \dfrac{1}{\lvert A \rvert^{n-2}}B^* = \dfrac{1}{\lvert A \rvert^{n-2}}\lvert B \rvert B^{-1},\]
                              而
                              \[\lvert B \rvert = \lvert A^* \rvert = \lvert A \rvert^{n-1}, d\]
                              代入上式可得
                              \[A = \lvert A \rvert B^{-1} = \sqrt[n-1]{\lvert B \rvert} B^{-1}\]
                              从而满足 $A^* = B$ 的矩阵 $A$ 存在,且有 $n-1$ 个.

                        \item $r(B) = 1$ 时,存在可逆矩阵 $P, Q$ 使得
                              \[B = P\begin{pmatrix}
                                      1 & O \\
                                      O & O \\
                                  \end{pmatrix}Q.\]
                              若存在 $A$ 满足 $A^{*} = B$,则 $r(A) = n-1$,从而存在可逆矩阵 $G, H$ 使得
                              \[A = G\begin{pmatrix}
                                      0 & O       \\
                                      O & E_{n-1}
                                  \end{pmatrix}H,\]
                              则
                              \[A^* = H^*\begin{pmatrix}
                                      0 & O       \\
                                      O & E_{n-1}
                                  \end{pmatrix}^*G^* = H^*\begin{pmatrix}
                                      1 & O \\
                                      O & O \\
                                  \end{pmatrix}G^* = \lvert HG \rvert H^{-1}\begin{pmatrix}
                                      1 & O \\
                                      O & O \\
                                  \end{pmatrix}G^{-1},\]
                              由 $A^* = B$ 可得
                              \[\lvert HG \rvert H^{-1}\begin{pmatrix}
                                      1 & O \\
                                      O & O \\
                                  \end{pmatrix}G^{-1} = P\begin{pmatrix}
                                      1 & O \\
                                      O & O \\
                                  \end{pmatrix}Q,\]
                              即
                              \[\lvert HG \rvert\begin{pmatrix}
                                      1 & O \\
                                      O & O \\
                                  \end{pmatrix} = HP\begin{pmatrix}
                                      1 & O \\
                                      O & O \\
                                  \end{pmatrix}QG,\]
                              记 $C = HP$, $D = QG$,且分块为 $C = \begin{pmatrix}
                                      C_{11} & C_{12} \\
                                      C_{21} & C_{22}
                                  \end{pmatrix}$, $D = \begin{pmatrix}
                                      D_{11} & D_{12} \\
                                      D_{21} & D_{22}
                                  \end{pmatrix}$,其中 $C_{22}, D_{22}$ 是 $n-1$ 阶矩阵,则
                              \[\lvert HG \rvert\begin{pmatrix}
                                      1 & O \\
                                      O & O \\
                                  \end{pmatrix} = \begin{pmatrix}
                                      C_{11} & C_{12} \\
                                      C_{21} & C_{22}
                                  \end{pmatrix} \begin{pmatrix}
                                      1 & O \\
                                      O & O \\
                                  \end{pmatrix} \begin{pmatrix}
                                      D_{11} & D_{12} \\
                                      D_{21} & D_{22}
                                  \end{pmatrix} = \begin{pmatrix}
                                      C_{11}D_{11} & C_{11}D_{12} \\
                                      C_{21}D_{11} & C_{21}D_{12}
                                  \end{pmatrix},\]
                              于是
                              \[\lvert HG \rvert = C_{11}D_{11}, C_{11}D_{12} = O, C_{21}D_{11} = O, C_{21}D_{12} = O.\]
                              因为 $H, G$ 可逆,所以 $C_{11} \neq 0, D_{11} \neq 0$,于是 $C_{21} = O = D_{12}$. 从而
                              \begin{align*}
                                  A & = G\begin{pmatrix}
                                             0 & O       \\
                                             O & E_{n-1}
                                         \end{pmatrix}H
                                  = Q^{-1}D\begin{pmatrix}
                                               0 & O       \\
                                               O & E_{n-1}
                                           \end{pmatrix}CP^{-1}      \\
                                    & = Q^{-1}\begin{pmatrix}
                                                  D_{11} & O      \\
                                                  D_{21} & D_{22}
                                              \end{pmatrix}
                                  \begin{pmatrix}
                                      0 & O       \\
                                      O & E_{n-1}
                                  \end{pmatrix}
                                  \begin{pmatrix}
                                      C_{11} & C_{12} \\
                                      O      & C_{22}
                                  \end{pmatrix} P^{-1}               \\
                                    & = Q^{-1} \begin{pmatrix}
                                                   0 & O            \\
                                                   O & D_{22}C_{22}
                                               \end{pmatrix} P^{-1}.
                              \end{align*}
                              又
                              \[C_{11}D_{11} = \lvert HG \rvert = \lvert CP^{-1}Q^{-1}D \rvert = \dfrac{1}{\lvert PQ \rvert}\lvert DC \rvert.\]
                              接下来转为求 $\lvert DC \rvert$. 而
                              \[DC = \begin{pmatrix}
                                      D_{11} & O      \\
                                      D_{21} & D_{22}
                                  \end{pmatrix} \begin{pmatrix}
                                      C_{11} & C_{12} \\
                                      O      & C_{22}
                                  \end{pmatrix} = \begin{pmatrix}
                                      D_{11}C_{11} & D_{11}C_{12}              \\
                                      D_{21}C_{11} & D_{21}C_{12}+D_{22}C_{22}
                                  \end{pmatrix}.\]
                              考虑初等分块变换
                              \begin{align*}
                                  \begin{pmatrix}
                                      1                  & O       \\
                                      -D_{21}D_{11}^{-1} & E_{n-1}
                                  \end{pmatrix}DC
                                   & = \begin{pmatrix}
                                           1                  & O       \\
                                           -D_{21}D_{11}^{-1} & E_{n-1}
                                       \end{pmatrix}
                                  \begin{pmatrix}
                                      D_{11}C_{11} & D_{11}C_{12}              \\
                                      D_{21}C_{11} & D_{21}C_{12}+D_{22}C_{22}
                                  \end{pmatrix} \\
                                   & = \begin{pmatrix}
                                           D_{11}C_{11} & D_{11}C_{12} \\
                                           O            & D_{22}C_{22}
                                       \end{pmatrix},
                              \end{align*}
                              故
                              \[C_{11}D_{11} = \dfrac{1}{\lvert PQ \rvert}\lvert DC \rvert = \dfrac{1}{\lvert PQ \rvert}D_{11}C_{11} \lvert D_{22}C_{22} \rvert,\]
                              从而
                              \[\dfrac{1}{\lvert PQ \rvert} \lvert D_{22}C_{22} \rvert = 1,\]
                              即
                              \[\lvert D_{22}C_{22} \rvert = \lvert PQ \rvert.\]
                              命题得证.

                        \item $r(B) = 0$ 则是平凡情况,其是所有 $r \leqslant n-2$ 矩阵的伴随矩阵.
                    \end{enumerate}

              \item 由 $r(B^*) = r(A) = 1$ 可知 $r(B) = 2$. $B^*B = \lvert B \rvert E = 0$,由此可知 $B$ 的列向量为方程组 $B^*X = 0$ 的解,其基础解系为
                    \[\alpha_1 = (-1, 1, 0)^{\mathrm{T}}, \alpha_2 = (-1, 0, 1)^{\mathrm{T}}.\]
                    令 $B = (\alpha_1, \alpha_2, \alpha_3)$,其中 $\alpha_3 = k_1\alpha_1+k_2\alpha_2 = (k_1+k_2, -k_1, -k_2)^{\mathrm{T}}$. 由 $BB^* = 0$ 解得 $k_1 = k_2 = 1$,从而
                    \[B = \begin{pmatrix}
                            -1 & -1 & 2  \\
                            1  & 0  & -1 \\
                            0  & 1  & -1
                        \end{pmatrix}.\]
          \end{enumerate}
\end{enumerate}

\clearpage
