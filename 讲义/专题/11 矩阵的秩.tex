\chapter{矩阵的秩}

事实上,在前述介绍中我们已经有了充足的关于线性空间、线性映射以及矩阵的相关背景知识. 回顾之前讨论的\autoref{ex:2:线性空间引入} 中的例子以及线性映射核空间与齐次线性方程组的解的关联,我们是时候彻底揭开矩阵与抽象的线性空间和线性映射之间那层若隐若现的薄膜,而我们的工具正是本节讨论的核心——矩阵的秩. 相信在理解了本节内容后,我们可以说是一只手已经触碰到线性方程组解的一般理论了.

\vspace{2ex}
\centerline{\heiti \Large 内容总结}

本讲我们结合了之前所学的有关线性相关性、线性映射以及矩阵的基本知识和运算技巧,以矩阵的秩为中心展开讨论. 我们首先基于线性映射的秩给出矩阵的秩的定义,然后用多种方法证明了矩阵的秩等于行秩等于列秩,然后介绍了矩阵可逆的几个等价条件,这是非常常用的.

然后我们讨论了三个重要的定理,分别是基的选择对向量坐标的影响(引入了过渡矩阵)、线性映射对向量坐标的影响以及基的选择对映射矩阵的影响,并通过定理的证明以及一些例子综合运用了所学知识.

然后我们通过两种方法讨论了相抵标准形(其一是找到一组基使得线性映射矩阵表示为相抵标准形的形式,其二是基于初等变换不改变矩阵的秩(注意我们在正文中提到了初等变换的两个最重要的定理)),然后讨论了基于相抵标准形的矩阵分解. 最后我们讨论了大量秩不等式,其中我们综合运用线性相关性、线性映射像空间核空间性质以及分块矩阵初等变换等进行证明,希望读者能从中总结出一些证明的基本思路与方法.

\vspace{2ex}
\centerline{\heiti \Large 习题}

\vspace{2ex}
{\kaishu 也许我可以并非不适当地要求获得数学上亚当这一称号,因为我相信数学理性创造物由我命名比起同时代其它数学家加在一起还要多.}
\begin{flushright}
    \kaishu
    ——西尔维斯特
\end{flushright}

\centerline{\heiti A组}
\begin{enumerate}
    \item 给定$\mathbf{R}^4$的两组基
          \begin{gather*}
              \alpha_1=(1,1,1,1),\ \alpha_2=(1,1,-1,-1),\ \alpha_3=(1,-1,1,-1),\ \alpha_4=(1,-1,-1,1) \\
              \beta_1=(1,1,0,1),\ \beta_2=(2,1,3,1),\ \beta_3=(1,1,0,0),\ \beta_4=(0,1,-1,-1)
          \end{gather*}
          求由基$\alpha_1,\alpha_2,\alpha_3,\alpha_4$到基$\beta_1,\beta_2,\beta_3,\beta_4$的过渡矩阵,并求向量$\xi=(1,0,0,-1)$在基$\alpha_1,\alpha_2,\alpha_3,\alpha_4$下的坐标.

    \item 证明:矩阵添加一列(或一行),其秩或不变,或增加1.

    \item 设$A$是$s \times n$矩阵,$B$是$A$前$m$行构成的$m \times n$矩阵,证明:$r(B) \geqslant r(A) + m - s$.
\end{enumerate}

\centerline{\heiti B组}
\begin{enumerate}
    \item 利用列向量线性相关性,证明矩阵秩不等式:\[|r(A)-r(B)|\leqslant r(A\pm B) \leqslant r(A)+r(B)\]

    \item 设$W$是$n$维线性空间$V$的一个非平凡子空间,$W$中取一组基$\delta_1,\ldots,\delta_m$,按如下两种方式将其扩充为$V$的一组基:
          \begin{align*}
              B_1 & =\{\delta_1,\ldots,\delta_m,\alpha_{m+1},\ldots,\alpha_n\} \\
              B_2 & =\{\delta_1,\ldots,\delta_m,\beta_{m+1},\ldots,\beta_n\}
          \end{align*}
          设基$B_1$到$B_2$的过渡矩阵为$P$,求商空间$V/W$的基$\alpha_{m+1}+W,\ldots,\alpha_n+W$到$\beta_{m+1}+W,\ldots,\beta_n+W$的过渡矩阵.

    \item 证明:当$n$为奇数时,$\alpha_1,\alpha_2,\ldots,\alpha_n$线性无关的充要条件是$\alpha_1+\alpha_2,\alpha_2+\alpha_3,\ldots,\alpha_n+\alpha_1$线性无关.

    \item 设
          \[B_1=\left\{\begin{pmatrix}
                  1 & 0 \\ 0 & 0
              \end{pmatrix},\begin{pmatrix}
                  0 & 1 \\ 0 & 0
              \end{pmatrix},\begin{pmatrix}
                  0 & 0 \\ 1 & 0
              \end{pmatrix}\begin{pmatrix}
                  0 & 0 \\ 0 & 1
              \end{pmatrix}\right\},\]
          \[B_2=\left\{\begin{pmatrix}
                  1 & 0 \\ 0 & 0
              \end{pmatrix},\begin{pmatrix}
                  1 & 1 \\ 0 & 0
              \end{pmatrix},\begin{pmatrix}
                  1 & 1 \\ 1 & 0
              \end{pmatrix}\begin{pmatrix}
                  1 & 1 \\ 1 & 1
              \end{pmatrix}\right\}.\]
          \begin{enumerate}
              \item 证明:$B_2$也是线性空间$\mathbf{M}_2(\mathbf{R})$的基;

              \item 求基$B_2$变为基$B_1$的变换矩阵;

              \item 求$\mathbf{M}_2(\mathbf{R})$的一组基$B_3=\{A_1,A_2,A_3,A_4\}$,使得$A_i^2=A_i,\enspace i=1,2,3,4$;

              \item 已知矩阵$A$关于基$B_2$的坐标为$(1,1,1,1)^\mathrm{T}$,求$A$关于基$B_3$的坐标.
          \end{enumerate}

    \item (利用相抵标准形)证明以下结论:
          \begin{enumerate}
              \item 设$B_1,B_2$为$s \times n$列满秩矩阵,证明:存在$s$阶可逆矩阵$C$使得$B_2=CB_1$;

              \item 设$B_1,B_2$为$s \times n$行满秩矩阵,证明:存在$n$阶可逆矩阵$C$使得$B_2=B_1C$;

              \item 任意秩为$r$的矩阵都可以被分解为$r$个秩为1的矩阵之和;

              \item 已知$A$是$n$阶方阵,证明:存在$n$阶方阵$B$使得$A=ABA,\enspace B=BAB$.
          \end{enumerate}

    \item 设$B$是$3 \times 1$矩阵,$C$是$1 \times 3$矩阵,证明:$r(BC) \leqslant 1$. 反之,若$A$是秩为1的$3 \times 3$矩阵,证明:存在$3 \times 1$矩阵$B$和$1 \times 3$矩阵$C$,使得$A = BC$.

    \item 设$\alpha,\beta$为$n$维列向量,且$A=\alpha\alpha^\mathrm{T}+\beta\beta^\mathrm{T}$.
          \begin{enumerate}
              \item 证明:$r(A) \leqslant 2$;

              \item 若$\alpha,\beta$线性相关,证明:$r(A) \leqslant 1$.
          \end{enumerate}

    \item 设 $A \in \mathbf{M}_{m \times n}(\mathbf{F})$,$r(A)=r$,$k$ 是满足条件 $r \leqslant k \leqslant n$ 的任意整数,证明存在 $n$ 阶方阵 $B$,使得 $AB=O$,且 $r(A)+r(B)=k$.

    \item 设$A$是$m \times n$矩阵($m \leqslant n$),$r(A)=m$,证明:存在$n \times m$矩阵$B$使得$AB=E$.

    \item 设$A,B \in \mathbf{M}_n(\mathbf{F})$,$r(A)+r(B) \leqslant n$,证明:存在可逆矩阵$M$,使得$AMB=O$.

    \item 设$S(A)=\{B \in \mathbf{M}_n(\mathbf{F}) \mid AB=0\}$.
          \begin{enumerate}
              \item 证明:$S(A)$为$\mathbf{M}_n(\mathbf{F})$的子空间;

              \item 设$r(A)=r$,求$S(A)$的一组基和维数.
          \end{enumerate}

    \item 设$A$为$n$阶实方阵且$r(A)=r>0$,证明存在秩为$r$的实方阵$B$和$C$使得$AB=CA$. % 新题,需要答案

    \item 设$r(A)=r$,证明:存在$n$阶矩阵$B$满足$r(B)=n-r$,使得$AB=O$. % 新题,需要答案

    \item 设$A$和$B$是两个$n$阶实方阵,证明:$r(A)=r(AB)$当且仅当存在$n$阶实方阵$C$使得$A=ABC$. % 新题,需要答案
\end{enumerate}

\centerline{\heiti C组}
\begin{enumerate}
    \item (打洞法)已知$A$是一个$s \times n$矩阵,证明:$r(E_n-A^\mathrm{T}A)-r(E_s-AA^\mathrm{T})=n-s$.

    \item 利用打洞法完成以下两个问题(\ref*{item:11:C:1} 也可以不使用打洞法,可以思考其他方式解决):
          \begin{enumerate}
              \item 设$n$阶方阵$A,B,C,D$满足$AC+BD=E$,证明:$r(AB) = r(A)+r(B)-n$;

              \item \label{item:11:C:1}
                    $n$阶方阵$A,B$满足$AB=BA$,证明:$r(AB)+r(A+B)\leqslant r(A)+r(B)$.
          \end{enumerate}

    \item $f(x)=f_1(x)f_2(x)$是多项式,且$f_1(x)$与$f_2(x)$互素,则$f(A)=O$的充要条件是$r(f_1(A))+r(f_2(A))=n$. (注:此题的推论非常多,如$A^2=A$,$A^n=E$等形式的结论都可以利用这个例子推导出)

    \item 设$A,B$分别为$3 \times 2$和$2 \times 3$实矩阵. 若$AB=\begin{pmatrix}
                  8             & 0 & -4 \\[1ex]
                  -\dfrac{3}{2} & 9 & -6 \\[1ex]
                  -2            & 0 & 1
              \end{pmatrix}$,求$BA$.

    \item 设矩阵$A \in \mathbf{F}^{m \times n}$,$A$的秩$r(A)=r$,定义$\mathbf{F}^{n \times p}$到$\mathbf{F}^{m \times p}$的线性映射$\sigma$,使得$\forall X \in \mathbf{F}^{n \times p}$,$\sigma(X)=AX$. 求$\sigma$核空间的维数.
\end{enumerate}
