\chapter{有限维线性空间}

\section{线性相关性}
\subsection{线性相关性的定义}
研究线性相关性来源于我们希望知道有限维线性空间至少需要多少个向量张成,以下是其定义:
\begin{definition}
	设$V(\mathbf{F})$是一个线性空间,$\alpha_1,\alpha_2,\ldots,\alpha_m\in V$,如果存在
	不全为$0$的$\lambda_1,\lambda_2,\ldots,\lambda_m\in\mathbf{F}$,使得
	$$\lambda_1\alpha_1+\alpha_2\lambda_2+\cdots+\lambda_m\alpha_m=0$$
	成立,则称$\alpha_1,\alpha_2,\ldots,\alpha_m$线性相关,否则称线性无关(即系数只能为$0$).
\end{definition}
线性相关和线性无关的证明就是基于这个定义,请务必牢牢掌握.

直接由定义我们可以导出以下结论:

1. 线性空间中单个向量$\alpha$线性相关的充要条件是$\alpha$为零向量;

2. 任何含零向量的向量组都线性相关.

我们来看几个基本的例子:
\begin{example}
	\textup{(1)}判断$(1,1,0),(0,1,1),(1,0,-1)$的线性相关性;

	\textup{(2)}判断$(1,-3,1),(-1,2,-2),(1,1,3)$的线性相关性;

	\textup{(3)}判断$p_1(x)=1+x,p_2(x)=1-x,p_3(x)=x+x^2$的线性相关性;

	\textup{(4)}判断$1,\sin^2x,\cos^2x$的线性相关性;

	\textup{(5)}判断$1,2^x,2^{-x}$的线性相关性.
\end{example}
注意上述(3)到(5)题为不能代入特殊的$x$值来说明,例如(3)令$x=0$得到线性相关的做法是错误的,因为
(3)中线性空间就是多项式构成的线性空间,其中的元素就是多项式,不能代入值.注意(5)是特殊题型,
需要构造更多的方程来求解这一问题.
\subsection{线性相关性的定理}
本节内容十分重要,是理解线性空间结构的基础,希望同学们对以下定理及其证明十分熟练并且要有深刻的理解.
我们的主线思路是从不同方面理解线性相关性:
\begin{enumerate}
	\item 从线性组合看(定义)

	      向量组线性相关$\iff$它们有系数不全为0的线性组合等于零向量;

	      向量组线性无关$\iff$它们只有系数全为0的线性组合才会等于零向量.
	\item 从线性表示看(教材定理2.3)
	      \begin{theorem}
		      线性空间$V(\mathbf{F})$中的向量组$\alpha_1,\alpha_2,\ldots,\alpha_m(m \ge 2)$线性相关的充分必要条件是
		      $\alpha_1,\alpha_2,\ldots,\alpha_m$中有一个向量可由其余向量在域$\mathbf{F}$上线性表示.
	      \end{theorem}
	      这一定理等价描述为,向量组线性无关的充分必要条件是其中的向量无法互相表示.这是显然的,因为向量组能互相表示
	      利用定义可以轻松写出非零系数的线性表示.总结一下即为:

	      向量组线性相关$\iff$其中至少有一个向量可以由其余向量线性表示;

	      向量组线性无关$\iff$其中每一个向量都不能由其余向量线性表出.
	\item 从齐次线性方程组看(教材P66例3,实际上这一点与定义十分类似)

	      列向量组$\alpha_1,\alpha_2,\ldots,\alpha_m$线性相关$\iff$齐次线性方程组$x_1\alpha_1+x_2\alpha_2+\cdots+x_m\alpha_m=0$有非零解;

	      列向量组$\alpha_1,\alpha_2,\ldots,\alpha_m$线性无关$\iff$齐次线性方程组$x_1\alpha_1+x_2\alpha_2+\cdots+x_m\alpha_m=0$只有零解.
	\item 从向量组线性表示一个向量的方式看(教材定理2.4)
	      \begin{theorem}
		      若向量组$\alpha_1,\alpha_2,\ldots,\alpha_m$线性无关,而向量组$\beta,\alpha_1,\alpha_2,\ldots,\alpha_m$线性相关,
		      则$\beta$可由$\alpha_1,\alpha_2,\ldots,\alpha_m$线性表示,且表示法唯一.
	      \end{theorem}
	      总结如下:若向量组外另一向量可由这一组向量线性表示,则

	      向量组线性无关$\iff$表出方式唯一;

	      向量组线性相关$\iff$表示方式有无穷多种.

	      表示方式唯一的证明是经典的,即设有另一种表示方式,然后利用线性无关的定义说明这两种表示方式必定相同即可.
	\item 从向量组与它的部分组的关系看(教材P67例6)

	      如果向量组的一个部分组线性相关,那么整个向量组也线性相关;

	      如果向量组线性无关,那么它的任何一个部分组也线性无关.
\end{enumerate}

最后我们还要介绍一个定理,这一定理更加重要,证明可能较为复杂,但结论一定要熟练掌握:
\begin{theorem}\label{线性表示}
	设$V(\mathbf{F})$中向量组$\{\beta_1,\beta_2,\ldots,\beta_s\}$的每个向量可由另一向量组$\alpha_1,\alpha_2,\ldots,\alpha_r$
	线性表示.若$s>r$,则$\{\beta_1,\beta_2,\ldots,\beta_s\}$线性相关.
\end{theorem}
这一定理的等价命题为,$\{\beta_1,\beta_2,\ldots,\beta_s\}$线性无关则必有$s\le r$.

这一定理可通俗概括为:多的向量组可以被少的向量组线性表示,多的一定线性相关.反过来说,线性无关的向量只能被等长或更长的向量组线性表示.

\section{基与维数}
\subsection{秩与维数的概念}
\begin{definition}
	若线性空间$V(\mathbf{F})$的非空子集$S$中存在线性无关的向量组$B=\{\alpha_1,\alpha_2,\cdots,\alpha_m\}$,
	且$S$中每个向量都可以由$B$线性表示,则$B$中向量的个数$r$叫做$S$的秩,记作$r(S)= r$(实际上即为极大线性无关组的长度).
\end{definition}
\begin{definition}
	若线性空间$V(\mathbf{F})$的有限子集$B=\{\alpha_1,\alpha_2,\cdots,\alpha_n\}$线性无关,且$L(B) = V$,则称$B$为$V$的一组基,
	并称$n$为$V$的维数,记作$\dim V = n$.
\end{definition}
实际上,综合上述两个定义我们可以看到,如果$S$为有限维线性空间$V(\mathbf{F})$的子空间,那么$S$的秩就是$S$的维数.

从维数的定义中我们可以理解到,一个线性空间的基的长度必定是唯一的,否则同一个线性空间将会出现多个不同的维数,这是不合理的.
证明可以直接利用上一小节中的定理1.3.3,反证法即可.当然我们也可以不利用定理1.3.3,直接通过线性无关的定义以及线性方程组的解的情况讨论
得到线性空间维数唯一的结论(大家可以自行尝试,实际上我的期中复习2021版中也有讲解).

我们需要提及一个概念,即自然基.例如三维空间的自然基为$(1,0,0),(0,1,0),(0,0,1)$,$n$维空间也有类似的推广(即$n$个只有一位为0其余全为1的向量).
对于多项式我们则将$1,x,x^2,\cdots$称为自然基,矩阵、函数等也有相关的常用的基.
\subsection{相关定理与性质}
根据秩与维数的概念与定理1.3.3(或教材定理2.5),我们可以得到以下直接的结论:

1. $r(S)=n$,则$S$中$n+1$个向量必线性相关,$S$中任何线性无关向量组至多含$n$个向量,且将含$n$个线性无关向量的
向量组称为$S$的极大线性无关组.同理,$n$维线性空间中$n+1$个向量必线性相关,其中含$n$个线性无关向量的向量组称为
线性空间的一组基;

注意:极大线性无关组有两个关键词:线性无关与张成空间,我们还有两个结论:
\begin{itemize}
	\item 设向量组的秩为$r$,则它的任意$r$个线性无关的向量都构成它的一个极大线性无关组;
	\item 设向量组的秩为$r$,则若向量组可以由其中的$r$个向量线性表出,那么这$r$个向量就是原向量组的一个极大线性无关组.
\end{itemize}

2. 线性空间的基的个数(维数)是唯一的,但基中向量选取不唯一;

3. 若$r(S)=r$,$B=\{\alpha_1,\alpha_2,\cdots,\alpha_r\}$是$S$的极大线性无关组,则$L(S)=L(B)$,
即$\dim L(S)=r(S)$(秩与维数统一).

接下来我们介绍等价向量组的概念.我们称可以互相线性表示的两个向量组为等价向量组.可以回忆教材1.3节等价关系,可知等价向量组必须满足自反性、对称性以及传递性.
这一点是容易证明的.注意,等价向量组必定秩相等,这是由定理1.3.3(教材定理2.5)可以直接导出的.
关于等价向量组,在后续专题过渡矩阵中还有更多的讨论.

实际上上述结论将向量组的线性扩张替换为线性空间,向量组的秩替换为线性空间的维数,向量组的极大线性无关组替换为线性空间的基,
我们能得到大量等价的结论.除此之外还有和线性相关性以及秩有关的很多结论,我们在这两节习题中展示.
\begin{example}
	证明以下两个结论:

	\textup{(1)}设$U$和$W$都是$V$的非零子空间,如果$U\subseteq W$,那么$\dim U \le \dim W$;
	
	\textup{(2)}设$U$和$W$都是$V$的非零子空间,$U\subseteq W$,且$\dim U = \dim W$,则$U = W$.
\end{example}
最后是一个很重要的定理,这一定理在之后大量的定理证明中都有运用:
\begin{theorem}
	如果$W$是$n$维线性空间$V$的一个子空间,则$W$的基可以扩充为$V$的基.
\end{theorem}
这一定理表明我们可以往一个向量组添加向量扩充为基,之前我们通过求极大线性无关组丢掉一些向量
使得我们获得一组基.定理证明是简单的,实际上这就是教材定理2.6.
\subsection{求解极大线性无关组}
我们首先考察一个阶梯矩阵$A$.我们不难发现,阶梯矩阵的主元(非零行的第一个非零元素)所在的列
构成$A$的列向量的一个极大线性无关组.证明可以利用行列式的秩的相关概念,取出矩阵的子式考虑其行列式.

但我们一般的题目不会直接给出能够成阶梯矩阵的向量,但我们知道,矩阵经过初等行变换不改变列之间的
线性关系(思考其证明),所以我们求极大线性无关组的方法就是,将向量按列排列成矩阵,然后对矩阵做
初等行变换,化为阶梯形,找到主元所在的列,对应的原向量即组成一个极大线性无关组.

注意极大线性无关组是不唯一的,但上面给出了一个程式化的方法.实际上如果能一眼看出结果的也不必如此麻烦
(当然题目直接要求极大线性无关组还是应当写具体过程的).
\begin{example}
	求向量组$\{\alpha_1=(1,-1,2,4),\alpha_2=(0,3,1,2),\alpha_3=(3,0,7,14),\alpha_4=(1,-1,2,0),\alpha_5=(2,1,5,6)\}$
	的极大线性无关组和秩.
\end{example}

\subsection{向量组的坐标}
坐标的概念实际上我们已经熟悉,例如高中学过的平面向量,平面向量的坐标表示就是二维平面的基$(0,1),(1,0)$
下的坐标表示.我们现在将这个概念拓展到更一般的线性空间:
\begin{definition}
	设$B=\{\beta_1,\beta_2,\cdots,\beta_n\}$是$n$维线性空间$V(\mathbf{F})$的一组基,如果$V$中元素$\alpha$
	表示为$\alpha=a_1\beta_1+a_2\beta_2+\cdots+a_n\beta_n$,则其系数组$a_1,a_2,\cdots,a_n$称为$\alpha$在
	基$B$下的坐标,记为$\alpha_B=(a_1,a_2,\cdots,a_n)$.
\end{definition}
我们很容易证明,$(\alpha+\beta)_B=\alpha_B+\beta_B$和$(\lambda\alpha)_B=\lambda\alpha_B$成立,
这表明坐标也保持了原空间的线性运算关系不变.并且我们不难证明坐标与向量是一一对应的,并且坐标处于$\mathbf{F}^n$空间中,
所以我们可以将对一般线性空间的研究转为研究简单的$\mathbf{F}^n$空间,这也是下一章中同构会提及的.

需要注意的是,$\mathbf{R}^n$中的向量在自然基下的坐标实际上就是向量本身,需要牢记.
\begin{example}
	分别求$p(x)=a_0+a_1x+a_2x^2$在基$B_1=\{1,x,x^2\}$和$B_2=\{1,x-1,(x-1)^2\}$下的坐标.
\end{example}

\vspace{2ex}
\centerline{\heiti \Large 内容总结}

\vspace{2ex}

\centerline{\heiti \Large 习题}
\vspace{2ex}
{\kaishu }
\begin{flushright}
	\kaishu

\end{flushright}
\centerline{\heiti A组}
\begin{enumerate}
	\item
\end{enumerate}
\centerline{\heiti B组}
\begin{enumerate}
	\item
\end{enumerate}
\centerline{\heiti C组}
\begin{enumerate}
	\item
\end{enumerate}