\documentclass{ctexbook}

\usepackage{geometry}
\geometry{a4paper}
%\usepackage[UTF8, heading = false, scheme = plain]{ctex}%格式
\usepackage{ctex}
\usepackage[utf8]{inputenc}
\usepackage{bm}
\usepackage{graphicx} %添加图片
% \usepackage{amsthm}
\usepackage{amsmath}
\renewcommand{\vec}[1]{\boldsymbol{#1}} % 生产粗体向量,而不是带箭头的向量
\usepackage{amssymb}
\usepackage{booktabs} % excel导出的大表格
\usepackage{rotating}
\usepackage{extarrows}

\usepackage{indentfirst}
\setlength{\parindent}{2em}

\usepackage{ntheorem}
\theoremheaderfont{\bf\heiti}
\theorembodyfont{\fangsong}

\usepackage{zhnumber}

% chapter 标题修改为第 * 讲
\ctexset{
    chapter={format={\centering\Huge\bfseries},name={第,讲},number=\arabic{chapter}},
    section={format={\raggedright\Large\bfseries},name={,},number={\arabic{chapter}.\arabic{section}}},
    subsection={format={\raggedright\large\bfseries},name={,},number={\arabic{chapter}.\arabic{section}.\arabic{subsection}}},
}

% hyperref 与 cleveref 需要最后引入
\usepackage{hyperref}
\usepackage{cleveref}
\hypersetup{
    colorlinks,
    pdfborder={0 0 0},
    bookmarksnumbered,
}

\newtheorem{definition}{定义}[chapter] % 中文
\newtheorem{example}{例}[chapter]
\newtheorem{lemma}{引理}[chapter]
\newtheorem{theorem}{定理}[chapter]

%\newenvironment{proof}{{\noindent\it 证明}\quad}{\hfill □ \square□\par}
\DeclareMathOperator{\Ima}{Im}% 定义新符号
\DeclareMathOperator{\Rank}{rank}% 定义求秩算子

\title{\heiti 浙江大学 2023-2024 学年 \\ 线性代数荣誉课辅学讲义}
\author{2023-2024 学年线性代数 I/II(H)辅学授课 \\ 吴一航\ \ yhwu\_is@zju.edu.cn}

\begin{document}
\frontmatter
\maketitle

\songti

% 插入空页
{\null
\thispagestyle{empty}
\newpage}
\setcounter{page}{1}

\pdfbookmark[0]{目录}{contents}
\tableofcontents

\mainmatter
\setcounter{page}{1} % 将页码计数设置为 1
\chapter{预备知识}

\section{等价类}

\section{高斯消元法}

\section{基本代数结构}

\vspace{2ex} 
\centerline{\heiti \Large 内容总结}

\vspace{2ex} 

\centerline{\heiti \Large 习题}
\vspace{2ex} 
{\kaishu }
\begin{flushright}
    \kaishu

\end{flushright}
\centerline{\heiti A组}
\begin{enumerate}
	\item 
\end{enumerate}
\centerline{\heiti B组}
\begin{enumerate}
	\item 
\end{enumerate}
\centerline{\heiti C组}
\begin{enumerate}
	\item 
\end{enumerate}
\chapter{线性空间}

\section{线性空间的定义}

\section{线性子空间}
我们首先看线性子空间的定义:
\begin{definition}
	设$W$是线性空间$V(\mathbf{F})$的非空子集,如果$W$对$V$中的运算也构成域$\mathbf{F}$
	上的线性空间,则称$W$是$V$的线性子空间(简称子空间).
\end{definition}
请一定注意定义中的非空子集,建议验证子空间时先验证非空。接下来是验证子空间的一般方法:
\begin{theorem}
	线性空间$V(\mathbf{F})$的非空子集$W$为$V$的子空间的充分必要条件是$W$对于$V(\mathbf{F})$的线性运算封闭.
\end{theorem}
这表明只要子空间中的元素满足对原空间的加法和数乘运算封闭即可。

注意线性空间有两个子空间称为平凡子空间,即仅含零元的子集$\{0\}$和其自身$V$。其它
子空间称为非平凡子空间。
\begin{example}
	\textup{(1)}说明$\mathbf{R}[x]_2$是$\mathbf{R}[x]_3$的子空间\textup{;}

	\textup{(2)}判断$W_1=\{(x,y,z)\ |\ \cfrac{x}{3}=\cfrac{y}{2}=z\},
	W_2=\{(x,,y,z)\ |\ x+y+z=1,x-y+z=1\}$是否为$\mathbf{R}^3$的子空间.
\end{example}
第二小问表明过原点的直线/平面构成三维空间的子空间,不过原点的无法保持线性性。

\section{线性扩张}
接下来我们讨论线性扩张及其性质,我们首先来看线性组合和线性表示的概念:
\begin{definition}
	设$V(\mathbf{F})$是一个线性空间,$\alpha_i\in V,\lambda_i\in \mathbf{F}(i=1,2,\cdots,m)$,
	则向量$\alpha=\lambda_1\alpha_1+\lambda_2\alpha_2+\cdots+\lambda_m\alpha_m$
	称为向量组$\{\alpha_1,\alpha_2,\cdots,\alpha_m\}$在域$\mathbf{F}$的线性组合,或说$\alpha$
	在域$\mathbf{F}$上可用向量组$\{\alpha_1,\alpha_2,\cdots,\alpha_m\}$线性表示.
\end{definition}
基于此,我们给出线性扩张的定义:
\begin{definition}
	设$S$是线性空间$V(\mathbf{F})$的非空子集,我们称
	$$L(S)=\{\lambda_1\alpha_1+\cdots+\lambda_k\alpha_k\ |\ \lambda_1,\cdots,\lambda_k\in\mathbf{F},\alpha_1,\cdots,\alpha_k\in S,k\in\mathbf{N^*}\}$$
	为$S$的线性扩张,即$S$中所有有限子集在域$\mathbf{F}$上的一切线性组合组成的$V(\mathbf{F})$的子集.
\end{definition}
下面的定理告诉我们可以通过线性扩张构造子空间:
\begin{theorem}
	线性空间$V(\mathbf{F})$的非空子集$S$的线性扩张$L(S)$是$V$中包含$S$的最小子空间.
\end{theorem}
这一定理的证明首先证明线性扩张是子空间,这是容易的,然后说明最小只需要说明$L(S)$是$V$中包含$S$的任意子空间的子集即可。

最后我们再说明有限维线性空间和无限维线性空间的定义,本课程研究的内容都在有限维线性空间:
\begin{definition}
	$V(\mathbf{F})$称为有限维线性空间,如果$V$中存在一个有限子集$S$使得$L(S)=V$,反之称为无限维线性空间.
\end{definition}
\begin{example}
	证明:$\mathbf{R}[x]_3$是有限维线性空间,$\mathbf{R}[x]$是无限维线性空间.
\end{example}

\vspace{2ex} 
\centerline{\heiti \Large 内容总结}

\vspace{2ex} 

\centerline{\heiti \Large 习题}
\vspace{2ex} 
{\kaishu 1520年以来,全世界只有85个机构存活至今,其中50家是大学。大学依靠梦想、希望生存下去——这就是大学的历史。}
\begin{flushright}
    \kaishu
    ——美国哥伦比亚大学校长L·C·柏林格
\end{flushright}
\centerline{\heiti A组}
\begin{enumerate}
	\item 检验下列集合对指定的加法和数乘运算是否构成实数域上的线性空间.
	
	(1)有理数集$\mathbf{Q}$对普通的数的加法和乘法;

	(2)集合$\mathbf{R}^2$对通常的向量加法和如下定义的数量乘法:$\lambda\cdot(x,y)=(\lambda x,y)$;

	(3)$\mathbf{R}_+^n$(即$n$元正实数向量)对如下定义的加法和数乘运算:
	$$(a_1,\cdots,a_n)+(b_1,\cdots,b_n)=(a_1b_1,\cdots,a_nb_n),$$
	$$\lambda\cdot(a_1,\cdots,a_n)=(a_1^\lambda,\cdots,a_n^\lambda).$$

	(4)请继续完成教材P86第二章习题第一题9-11小问关于函数的加法数乘定义线性空间的问题.
	\item 请完成教材P86-87第二章习题第三题的全部小问.第5小问平常问题较多,实际上就是要判断满足一定条件的
	多项式是否构成子空间.
\end{enumerate}
\centerline{\heiti B组}
\begin{enumerate}
	\item 设$V$是一个线性空间,$W$是$V$的子集,证明:$W$是$V$的子空间$\iff L(W)=W$.
	\item 回答以下两个问题:

	(1)设$\mathbf{R}^+$是所有正实数组成的集合,加法和数乘定义如下:$\forall a,b \in \mathbf{R}^+,k\in \mathbf{R},a\oplus b = ab, k\odot a = a^k$,$\mathbf{R}^+$关于这一加法和数乘构成一个实线性空间,求$\mathbf{R}^+$的一组基;

	(2)设$V$是一个$n$维实线性空间,证明:存在$V$中的一个由可列无穷多个向量组成的向量组$\{\alpha_i\ |\ i\in\mathbf{Z}^+\}$,使得其中任意$n$个向量组成的向量组都是$V$的一组基.
\end{enumerate}
\centerline{\heiti C组}
\begin{enumerate}
	\item 设$E$是域$F$的一个子域.
	
	(1)证明:$F$关于自身的加法和乘法构成一个$E$上的向量空间,并举一例;

	(2)举例说明:$E(E\neq F)$不是$F$上的线性空间;

	(3)证明:若$V$是$F$上的一个线性空间,则$V$也是$E$上的一个线性空间.
\end{enumerate}
\chapter{有限维线性空间}

\section{线性相关性}
\subsection{线性相关性的定义}
研究线性相关性来源于我们希望知道有限维线性空间至少需要多少个向量张成,以下是其定义:
\begin{definition}
    设$V(\mathbf{F})$是一个线性空间,$\alpha_1,\alpha_2,\ldots,\alpha_m\in V$,如果存在
    不全为0的$\lambda_1,\lambda_2,\ldots,\lambda_m\in\mathbf{F}$,使得
    \[\lambda_1\alpha_1+\alpha_2\lambda_2+\cdots+\lambda_m\alpha_m=0\]
    成立,则称$\alpha_1,\alpha_2,\ldots,\alpha_m$\keyterm{线性相关}[linearly dependent],否则称\keyterm{线性无关}[linearly independent](即系数只能为0).
\end{definition}
线性相关和线性无关的证明就是基于这个定义,请务必牢牢掌握.

直接由定义我们可以导出以下结论:
\begin{enumerate}
    \item 线性空间中单个向量$\alpha$线性相关的充要条件是$\alpha$为零向量;

    \item 任何含零向量的向量组都线性相关.
\end{enumerate}

我们来看几个基本的例子:
\begin{example}
    \begin{enumerate}[label=(\arabic*)]
        \item 判断$(1,1,0),(0,1,1),(1,0,-1)$的线性相关性;

        \item 判断$(1,-3,1),(-1,2,-2),(1,1,3)$的线性相关性;

        \item 判断$p_1(x)=1+x,\enspace p_2(x)=1-x,\enspace p_3(x)=x+x^2$的线性相关性;

        \item 判断$1,\enspace \sin^2x,\enspace \cos^2x$的线性相关性;

        \item 判断$1,\enspace 2^x,\enspace 2^{-x}$的线性相关性.
    \end{enumerate}
\end{example}
注意上述 (3) 到 (5) 题为不能代入特殊的$x$值来说明,例如 (3) 令$x=0$得到线性相关的做法是错误的,因为
(3) 中线性空间就是多项式构成的线性空间,其中的元素就是多项式,不能代入值.注意 (5) 是特殊题型,
需要构造更多的方程来求解这一问题.

\subsection{线性相关性的定理}
本节内容十分重要,是理解线性空间结构的基础,希望同学们对以下定理及其证明十分熟练并且要有深刻的理解.
我们的主线思路是从不同方面理解线性相关性:
\begin{enumerate}
    \item 从线性组合看(定义)

          向量组线性相关$\iff$它们有系数不全为0的线性组合等于零向量;

          向量组线性无关$\iff$它们只有系数全为0的线性组合才会等于零向量.
    \item 从线性表示看(教材定理2.3)
          \begin{theorem}
              线性空间$V(\mathbf{F})$中的向量组$\alpha_1,\alpha_2,\ldots,\alpha_m\enspace(m \geqslant 2)$线性相关的充分必要条件是
              $\alpha_1,\alpha_2,\ldots,\alpha_m$中有一个向量可由其余向量在域$\mathbf{F}$上线性表示.
          \end{theorem}
          这一定理等价描述为,向量组线性无关的充分必要条件是其中的向量无法互相表示.这是显然的,因为向量组能互相表示
          利用定义可以轻松写出非零系数的线性表示.总结一下即为:

          向量组线性相关$\iff$其中至少有一个向量可以由其余向量线性表示;

          向量组线性无关$\iff$其中每一个向量都不能由其余向量线性表出.
    \item 从齐次线性方程组看(教材P66例3,实际上这一点与定义十分类似)

          列向量组$\alpha_1,\alpha_2,\ldots,\alpha_m$线性相关$\iff$齐次线性方程组$x_1\alpha_1+x_2\alpha_2+\cdots+x_m\alpha_m=0$有非零解;

          列向量组$\alpha_1,\alpha_2,\ldots,\alpha_m$线性无关$\iff$齐次线性方程组$x_1\alpha_1+x_2\alpha_2+\cdots+x_m\alpha_m=0$只有零解.
    \item 从向量组线性表示一个向量的方式看(教材定理2.4)
          \begin{theorem}
              若向量组$\alpha_1,\alpha_2,\ldots,\alpha_m$线性无关,而向量组$\beta,\alpha_1,\alpha_2,\ldots,\alpha_m$线性相关,
              则$\beta$可由$\alpha_1,\alpha_2,\ldots,\alpha_m$线性表示,且表示法唯一.
          \end{theorem}
          总结如下:若向量组外另一向量可由这一组向量线性表示,则

          向量组线性无关$\iff$表示方式唯一;

          向量组线性相关$\iff$表示方式有无穷多种.

          表示方式唯一的证明是经典的,即设有另一种表示方式,然后利用线性无关的定义说明这两种表示方式必定相同即可.
    \item 从向量组与它的部分组的关系看(教材P67例6)

          如果向量组的一个部分组线性相关,那么整个向量组也线性相关;

          如果向量组线性无关,那么它的任何一个部分组也线性无关.
\end{enumerate}

最后我们还要介绍一个定理,这一定理更加重要,证明可能较为复杂,但结论一定要熟练掌握:
\begin{theorem}\label{thm:3:线性表示}
    设$V(\mathbf{F})$中向量组$ \beta_1,\beta_2,\ldots,\beta_s $的每个向量可由另一向量组$\alpha_1,\alpha_2,\ldots,\alpha_r$
    线性表示.若$s>r$,则$ \beta_1,\beta_2,\ldots,\beta_s $线性相关.
\end{theorem}
这一定理的等价命题为,$ \beta_1,\beta_2,\ldots,\beta_s $线性无关则必有$s\leqslant r$.

这一定理可通俗概括为:多的向量组可以被少的向量组线性表示,多的一定线性相关.反过来说,线性无关的向量只能被等长或更长的向量组线性表示.

\section{基与维数}
\subsection{秩与维数的概念}
\begin{definition}
	若线性空间$V(\mathbf{F})$的非空子集$S$中存在线性无关的向量组$B=\{\alpha_1,\alpha_2,\cdots,\alpha_m\}$,
	且$S$中每个向量都可以由$B$线性表示,则$B$中向量的个数$r$叫做$S$的\keyterm{秩}[rank],记作$r(S)= r$(实际上即为极大线性无关组的长度).
\end{definition}
\begin{definition}
    若线性空间$V(\mathbf{F})$的有限子集$B=\{\alpha_1,\alpha_2,\ldots,\alpha_n\}$线性无关,且$\spa(B) = V$,则称$B$为$V$的一组基,
    并称$n$为$V$的维数,记作$\dim V = n$.
\end{definition}
实际上,综合上述两个定义我们可以看到,如果$S$为有限维线性空间$V(\mathbf{F})$的子空间,那么$S$的秩就是$S$的维数.

从维数的定义中我们可以理解到,一个线性空间的基的长度必定是唯一的,否则同一个线性空间将会出现多个不同的维数,这是不合理的.
证明可以直接利用\autoref{thm:3:线性表示},反证法即可.当然我们也可以不利用\autoref*{thm:3:线性表示},直接通过线性无关的定义以及线性方程组的解的情况讨论
得到线性空间维数唯一的结论(大家可以自行尝试,实际上我的期中复习2021版中也有讲解).

我们需要提及一个概念,即自然基.例如三维空间的自然基为$(1,0,0),(0,1,0),(0,0,1)$. $n$维空间也有类似的推广(即$n$个只有一位为 1 其余全为 0 的向量).
对于多项式我们则将$1,x,x^2,\ldots$称为自然基,矩阵、函数等也有相关的常用的基.

\begin{example}\label{exp:3:不同数域的维数}
    线性空间$\mathbf{C(C)}$维数为1,不同于线性空间$\mathbf{C(R)}$维数为2.
\end{example}
\subsection{相关定理与性质}
根据秩与维数的概念与\autoref{thm:3:线性表示}(或教材定理2.5),我们可以得到以下直接的结论:
\begin{enumerate}
    \item $r(S)=n$,则$S$中$n+1$个向量必线性相关,$S$中任何线性无关向量组至多含$n$个向量,且将含$n$个线性无关向量的
    向量组称为$S$的极大线性无关组.同理,$n$维线性空间中$n+1$个向量必线性相关,其中含$n$个线性无关向量的向量组称为
    线性空间的一组基;

    注意:极大线性无关组有两个关键词:线性无关与张成空间,我们还有两个结论:
    \begin{itemize}
        \item 设向量组的秩为$r$,则它的任意$r$个线性无关的向量都构成它的一个极大线性无关组;
        \item 设向量组的秩为$r$,则若向量组可以由其中的$r$个向量线性表出,那么这$r$个向量就是原向量组的一个极大线性无关组.
    \end{itemize}

    \item 线性空间的基的个数(维数)是唯一的,但基中向量选取不唯一;

    \item 若$r(S)=r$,$B=\{\alpha_1,\alpha_2,\ldots,\alpha_r\}$是$S$的极大线性无关组,则$\spa(S)=\spa(B)$,
          即$\dim \spa(S)=r(S)$(秩与维数统一).
\end{enumerate}
接下来我们介绍等价向量组的概念.我们称可以互相线性表示的两个向量组为等价向量组.可以回忆教材1.3节等价关系,可知等价向量组必须满足自反性、对称性以及传递性.
这一点是容易证明的.注意,等价向量组必定秩相等,这是由\autoref{thm:3:线性表示}(教材定理2.5)可以直接导出的.
关于等价向量组,在后续专题过渡矩阵中还有更多的讨论.

实际上上述结论将向量组的线性扩张替换为线性空间,向量组的秩替换为线性空间的维数,向量组的极大线性无关组替换为线性空间的基,
我们能得到大量等价的结论.除此之外还有和线性相关性以及秩有关的很多结论,我们在这两节习题中展示.
\begin{example}
	证明以下两个结论:

	\textup{(1)}设$U$和$W$都是$V$的非零子空间,如果$U\subseteq W$,那么$\dim U \leqslant \dim W$;

	\textup{(2)}设$U$和$W$都是$V$的非零子空间,$U\subseteq W$,且$\dim U = \dim W$,则$U = W$.
\end{example}
最后是一个很重要的定理,这一定理在之后大量的定理证明中都有运用:
\begin{theorem}
	如果$W$是$n$维线性空间$V$的一个子空间,则$W$的基可以扩充为$V$的基.
\end{theorem}

假设一个向量$\alpha$有以下表示方式:\[\alpha=\lambda_1\beta_1+\lambda_2\beta_2+\cdots+\lambda_n\beta_n.\]
若$\alpha$在基下有第二种表示方式,即:\[\alpha=\mu_1\beta_1+\mu_2\beta_2+\cdots+\mu_n\beta_n.\]
两式相减可得:\[0=(\lambda_1-\mu_1)\beta_1+(\lambda_2-\mu_2)\beta_2+\cdots+(\lambda_n-\mu_n)\beta_n.\]
利用基线性无关的性质,即可得到每个$\lambda_j-\mu_j=0$,证明了定理的一方面.

另一方面,设每个$\alpha\in V$在$B$下表示方式唯一,说明$\alpha_1,\alpha_2,\ldots,\alpha_n$张成$V$.
要证明线性无关性,设$\lambda_1,\ldots,\lambda_n\in\mathbf{F}$使得\[0=\lambda_1\beta_1+\cdots+\lambda_n\beta_n.\]
由表示的唯一性,可得$\lambda_1=\cdots=\lambda_n=0$,因此$\alpha_1,\ldots,\alpha_n$线性无关,从而是$V$的一组基.

\vspace{2ex}
\centerline{\heiti \Large 内容总结}

\vspace{2ex}

\centerline{\heiti \Large 习题}
\vspace{2ex}
{\kaishu }
\begin{flushright}
    \kaishu

\end{flushright}
\centerline{\heiti A组}
\begin{enumerate}
    \item
\end{enumerate}
\centerline{\heiti B组}
\begin{enumerate}
    \item
\end{enumerate}
\centerline{\heiti C组}
\begin{enumerate}
    \item
\end{enumerate}

\chapter{线性空间的运算}

\section{线性空间的交、并、和}
\subsection{线性空间的交与和的概念}
\begin{definition}
    设$W_1,W_2$是线性空间$V(\mathbf{F})$的两个子空间,则
    \begin{align*}
    W_1 \cap W_2&=\{\alpha \mid \alpha\in W_1 \text{ 且 } \alpha\in W_2\} \\
    W_1 \cup W_2&=\{\alpha \mid \alpha\in W_1 \text{ 或 } \alpha\in W_2\} \\
    W_1 + W_2&=\{\alpha_1+\alpha_2 \mid \alpha_1\in W_1,\enspace\alpha_2\in W_2\}
    \end{align*}
    分别称为$W_1$和$W_2$的交、并、和.
\end{definition}
我们要注意,线性空间的交与和仍然是$V$的子空间,请各位同学自行证明.并且$V$的有限个子空间的交与和
仍然是$V$的子空间.

关于线性空间的并,我们必须注意线性空间的并不一定是线性空间,这很容易理解,
因为两个线性空间元素组合在一起,两个线性空间各取一个元素求和显然不一定在并集中,大家
可以自行举反例. 我们给出以下结论:
\[ W_1 \cup W_2 \text{ 为线性空间 } \iff W_1 \subseteq W_2 \text{ 或 } W_2 \subseteq W_1 \iff W_1 \cup W_2=W_1+W_2 \]

这一结论证明并不复杂,希望各位同学掌握. $V$的有限个子空间的并仍为$V$的子空间的充要条件是其中有一个
子空间能包含其他所有子空间.

我们可以从几何直观上理解这些概念,例如三维空间中两个不同的过原点的平面构成的线性空间的交是其交线(交线也过原点)构成的线性空间,
其和为整个三维空间.三维空间中一个平面与不在该平面上的直线的交只有零元,和为整个三维空间.
考试时我们遇到反例问题可以首先考虑这些简单的几何图形,当然无法解决时可以考虑$(1,0),(0,1),(1,1)$此类
简单的向量为基构成的空间.

关于线性空间的并我们还有一个重要的覆盖定理:
\begin{theorem}
    设$V_1,V_2,\ldots,V_s$是线性空间$V$的$s$个非平凡子空间,证明:$V$中至少存在一个向量
    不属于$V_1,V_2,\ldots,V_s$中的任何一个,即$V_1 \cup V_2 \cup \cdots \cup V_s\subsetneq V.$
\end{theorem}
这一定理表明,任何一个线性空间都不能被自身有限个非平凡子空间通过并得到.例如,有限条直线的并不可能是一个平面.
定理的证明可以使用数学归纳法,下面是一个应用的例子:
\begin{example}
    设$V_1,V_2,\ldots,V_s$是线性空间$V$的$s$个非平凡子空间,证明:存在$V$的一组基$\alpha_1,\alpha_2,\ldots,\alpha_n$
    都不在$V_1,V_2,\ldots,V_s$中.
\end{example}

\section{维数公式}
\begin{theorem}
    设$W_1,W_2$是线性空间$V(\mathbf{F})$的两个子空间,则
    \[\dim W_1+\dim W_2=\dim(W_1+W_2)+\dim(W_1\cap W_2).\]
\end{theorem}
上式称为子空间的维数公式,区别于下一专题中的线性映射基本定理的维数公式.这一定理的证明思想
是重要的,利用基的扩张等技巧,需要同学们熟练掌握,下面是一个证明思想类似的例子:
\begin{example}
    已知$A,B$分别是数域$\mathbf{F}$上的$s \times k$和$k \times n$矩阵,$X$是$n \times 1$
    的列向量. 证明:所有满足$ABX=0$的$BX$构成一个线性空间$V$,且$\dim V = r(B) - r(AB)$.
\end{example}

\section{线性空间的直和}
我们证明或者和空间很多时候都是利用和空间定义进行向量分解,这种分解唯一时即为直和.我们有如下定义:
\begin{definition}
    设$W_1,W_2$是线性空间$V(\mathbf{F})$的两个子空间. 若$W_1 \cap W_2=\{0\}$,则$W_1+W_2$叫做
    $W_1$与$W_2$的\keyterm{直和}[direct sum],记作$W_1\oplus W_2$.此时称$W_1,W_2$为\keyterm{互补子空间}[complementary subspaces],或$W_1$是$W_2$的补空间,
    或$W_2$是$W_1$的补空间.
\end{definition}
我们需要注意,一个线性子空间的补空间并不唯一,请同学们给出相应的例子.

直和有以下等价的命题,我们证明或者利用直和都可以任意选择:
\begin{theorem}
    对于子空间$W_1,W_2$,下列命题等价:
    \begin{enumerate}[label=(\arabic*)]
        \item $W_1+W_2$是直和,即$W_1 \cap W_2=\{0\}$;

        \item $W_1+W_2$中的每个向量$\alpha$的分解式$\alpha=\alpha_1+\alpha_2\enspace(\alpha_1\in W_1,\enspace\alpha_2\in W_2)$唯一;

        \item 零向量的分解式$0=\alpha_1+\alpha_2 \enspace(\alpha_1\in W_1,\enspace\alpha_2\in W_2)$仅当$\alpha_1=\alpha_2=0$时成立;

        \item $\dim (W_1+W_2)=\dim W_1+\dim W_2$.
    \end{enumerate}
\end{theorem}
我们也可以定义有限个子空间的直和,即$V=W_1\oplus W_2\oplus\cdots\oplus W_n \iff W_i \cap \sum\limits_{j \neq i}W_j=\{0\}$.
等价命题也是上述定理的推广,例如唯一分解、0的分解以及维数公式推广. 我们有一个与多空间直和相关的定理:
\begin{theorem}
    若$V=V_1\oplus V_2$,$V_1=V_{11}\oplus\cdots\oplus V_{1s}$,$V_2=V_{21}\oplus\cdots\oplus V_{2t}$,则
    \[V=V_{11}\oplus\cdots\oplus V_{1s}\oplus V_{21}\oplus\cdots\oplus V_{2t}\]
\end{theorem}
我们证明直和一般有两种思路,一种是先证和,再证直和,我们来看一个例子:
\begin{example}
    数域$\mathbf{F}$上所有$n$阶方阵组成的线性空间$V=\mathbf{M}_n(\mathbf{F})$,$V_1$表示所有对称矩阵
    组成的集合,$V_2$表示所有反对称矩阵组成的集合. 证明:$V_1,V_2$都是$V$的子空间,且$V=V_1\oplus V_2$.
\end{example}
还有一种证明$V=V_1\oplus V_2$的方式是先令$W=V_1+V_2$,先证明和为直和(即交为$\{0\}$)再证$W=V$即可,
下面是一个例子:
\begin{example}
    设$A$是数域$\mathbf{F}$上的一个$n$阶可逆方阵,$A$的前$r$个行向量组成的矩阵为$B$,后$n-r$个
    行向量组成的矩阵为$C$,$n$元线性方程组$BX=0$与$CX=0$的解空间分别为$V_1,V_2$. 证明:$\mathbf{F}^n=V_1\oplus V_2$.
\end{example}

\vspace{2ex}
\centerline{\heiti \Large 内容总结}

\vspace{2ex}

\centerline{\heiti \Large 习题}
\vspace{2ex}
{\kaishu }
\begin{flushright}
    \kaishu

\end{flushright}
\centerline{\heiti A组}
\begin{enumerate}
    \item
\end{enumerate}
\centerline{\heiti B组}
\begin{enumerate}
    \item
\end{enumerate}
\centerline{\heiti C组}
\begin{enumerate}
    \item
\end{enumerate}

\chapter{线性映射}

\section{线性映射的定义}

\section{线性映射的确定}

\section{线性映射的像与核}

\section{线性映射的矩阵表示}

\vspace{2ex} 
\centerline{\heiti \Large 内容总结}

\vspace{2ex} 

\centerline{\heiti \Large 习题}
\vspace{2ex} 
{\kaishu }
\begin{flushright}
    \kaishu

\end{flushright}
\centerline{\heiti A组}
\begin{enumerate}
	\item 
\end{enumerate}
\centerline{\heiti B组}
\begin{enumerate}
	\item 
\end{enumerate}
\centerline{\heiti C组}
\begin{enumerate}
	\item 
\end{enumerate}
\chapter{线性映射基本定理}

\section{线性映射的秩}
我们已知$\textup{Im }\sigma=\sigma(V_1)=L(\sigma(\alpha_1),\sigma(\alpha_2),\cdots,\sigma(\alpha_n))$,
我们基于此定义线性映射的秩:
\begin{definition}
	设$\sigma\in L(V_1,V_2)$,如果$\sigma(V_1)$是$V_2$的有限维子空间,则
	$\sigma(V_1)$的维数称为$\sigma$的秩,记作$r(\sigma)$,即$r(\sigma)=\dim \sigma(V_1)$.
\end{definition}
简单理解即线性映射的秩即为线性映射像空间的维数.

\section{线性映射基本定理}
这一定理是本学期最重要的定理之一,因其重要性也被冠以线性映射基本定理(有线维线性空间)的名号:
\begin{theorem}
	设$\sigma \in L(V_1,V_2)$,若$\dim V_1=n$,则
	$$r(\sigma)+\dim\ker\sigma=n.$$
\end{theorem}
这一定理的证明方式希望大家熟练掌握,下面是一个思想上类似的例子:
\begin{example}
	设$\sigma$为有限维线性空间$V$上的线性变换,$W$是$V$的子空间,证明:
	$$\dim\sigma(W)+\dim(\sigma^{-1}(0) \cap W)=\dim W.$$
\end{example}
基于线性映射基本定理,我们可以得到如下定理:
\begin{theorem}
	对$\sigma \in L(V_1,V_2)$且$\dim V_1=\dim V_2=n$,我们有
	$\ker\sigma=\{0\}\iff \sigma$为单射$\iff \sigma$为满射$\iff \sigma$为双射(可逆)$\iff r(\sigma)=n$(满秩).
\end{theorem}
显然这一定理前提适用于一切有限维空间上的线性变换.我们需要注意的是,上述第一个等价式不是基于线性映射基本定理得到的,
是教材定理3.1的内容,证明较为容易,建议先自己尝试证明.

线性映射基本定理还隐藏着一个结论,即不可能存在从低维空间到高维空间的满射(反证法代入维数公式即可,当然也可以利用线性相关性证明).

\section{像与核的进一步讨论}
关于线性变换的像和核有很多的包含关系或等式等结论,实际上很多问题都来源于线性映射基本定理及其推论,本节我们主要探讨这一话题.

我们首先说明几个重要的原则:

1. 解决此类问题大多需要综合利用维数公式及其推论,需要讲题给条件转化为合适的等价表述然后解决;

2. 注意集合相等的证明方式,实际上就是两个集合互相包含.实际上很多时候一边的包含是显然的,只需证明另一边;

3. 时刻注意线性映射的像和核的定义,线性空间的交、和与直和的概念,例如看到像需要想到其存在原像,看到和与直和要想到将向量分拆等.

接下来我们看一些经典的结论(已知$V$为有限维线性空间,$\sigma\in L(V,V)$),有余力的同学可以思考其证明,其中结论1最为常见:

1. 若$\sigma$为幂等变换(即$\sigma^2=\sigma$)有$V=\ker\sigma\oplus\textup{Im }\sigma$;

2. $r(\sigma^2)=r(\sigma) \iff V=\ker\sigma\oplus\textup{Im }\sigma$;

3. $\ker\sigma=\ker\sigma^2 \iff \ker\sigma \cap \textup{Im }\sigma=\{0\} \iff \textup{Im }\sigma=\textup{Im }\sigma^2 \iff V=\ker\sigma\oplus \textup{Im }\sigma$;

4. $\ker\sigma \subseteq \ker\sigma^2 \subseteq \ker\sigma^3 \subseteq \cdots$;

5. $\textup{Im }\sigma \supseteq \textup{Im }\sigma^2 \supseteq \textup{Im }\sigma^3 \supseteq \cdots$;

6. 存在正整数$m$使得对任意的$n>m$都有$\ker\sigma^n=\ker\sigma^m$,$\textup{Im }\sigma^n=\textup{Im }\sigma^m$;

7. 存在正整数$m$使得$V=\textup{Im }\sigma^m+\ker\sigma^m$;

8. $\dim(\ker\sigma+\textup{Im }\sigma) \ge \cfrac{n}{2}$,等号成立充要条件为$\ker\sigma=\textup{Im }\sigma$.

\section{可逆与同构}
\subsection{线性空间同构的概念}
\begin{definition}
	如果由线性空间$V_(\mathbf{F})$到$V_2(\mathbf{F})$存在一个线性双射$\sigma$,则称
	$V_(\mathbf{F})$和$V_2(\mathbf{F})$是同构的,记作$V_1(\mathbf{F}) \cong V_2(\mathbf{F})$,
	$\sigma$称为$V_(\mathbf{F})$到$V_2(\mathbf{F})$的一个同构映射.
\end{definition}
容易验证同构为等价关系,且对上述同构映射$\sigma$,$V_1$中向量组$\{\alpha_1,\alpha_2,\cdots,\alpha_m\}$与$V_2$中对应的
$\{\sigma(\alpha_1),\sigma(\alpha_2),\cdots,\sigma(\alpha_m)\}$有相同的线性相关性,这不难证明.

下面是同构的等价条件:
\begin{theorem}
	两个线性空间$V_(\mathbf{F})$和$V_2(\mathbf{F})$同构的充要条件是它们的维数相等.
\end{theorem}
上述即教材定理3.8,定理的证明是简单的,利用维数公式以及同构是等价关系即可.

我们需要指出,同构是本教材中最重要的概念之一,它统一了教材2-3章所学的内容,
将线性空间可以按维数划分为不同的等价类,并且表明线性空间最本质的结构就在于
基及其维数,之前第二章研究的线性相关性与向量组的秩等就是研究线性空间的内部结构,
而线性映射则将相同或不同结构的线性空间联系在一起,同构则表明只要线性空间维数相同,
则可以将两个空间中的所有元素一一对应.
\subsection{线性空间同构举例}
在上一节最后我们提到,同构则表明只要线性空间维数相同,则可以将两个空间中的所有元素
一一对应.本节则研究几个经典的一一对应的例子.

1. 坐标映射:请回顾上一专题中向量的坐标,证明坐标映射是同构映射(实际上是显然的,因为一个
向量在一组基下坐标唯一,而一个坐标对应唯一一个向量);

2. 若$\dim V_1(\mathbf{F})=m$,$\dim V_2(\mathbf{F})=n$,则$L(V_1,V_2) \cong F^{m \times n}$.
证明有两种方式,一种来源于教材定理3.7,较为复杂,我们实际上只需要通过线性映射矩阵表示即可说明.

下面我们通过几个例题进一步了解几个常见的例子(简单题基本只需要判断维数是否相等即可):
\begin{example}
	指出下面各组内的两个线性空间是否同构,若同构可以进一步思考同构映射的构造:

	\textup{(1)}最高次不超过$n-1$的多项式构成的线性空间$\mathbf{R}[x]_n$与$\mathbf{R}^n$;

	\textup{(2)}全体复数在实数域上的线性空间$\mathbf{C}(\mathbf{R})$与$\mathbf{R}^2$;

	\textup{(3)}全体二元复向量$\mathbf{C}^2$在实数域上构成的线性空间$\mathbf{C}^2(\mathbf{R})$与$\mathbf{R}[x]_4$;

	\textup{(4)}全体二元复向量$\mathbf{C}^2$在复数域上构成的线性空间$\mathbf{C}^2(\mathbf{C})$与$L(\mathbf{R}^4,\mathbf{R})$.
\end{example}

\subsection{一些相似的定理}
\begin{theorem}
	\textbf{线性映射对向量坐标的影响}
	
	设$\sigma \in L(V_1,V_2)$关于$V_1$和$V_2$的基$B_1$和基$B_2$的矩阵为$A=(a_{ij})_{m \times n}$,
	且$\alpha$与$\sigma(\alpha)$在基$B_1$和基$B_2$下的坐标分别为$X$和$Y$,则$Y=AX$.
\end{theorem}
上述即教材定理4.1,这一定理给出一个向量经过线性映射之后,其坐标的变化.我们可以用下图表示:
\begin{figure}[h]
	\centering
	\includegraphics[scale=0.75]{./figs/6/6-1.png}
\end{figure}

图中我们可以看出通过坐标映射后得到的新映射即为定理4.1描述的映射.

在描述下一定理之前,我们首先介绍过渡矩阵(变换矩阵)的概念.
\begin{definition}
	设$B_1=\{\alpha_1,\alpha_2,\cdots,\alpha_n\}$与$B_2=\{\beta_1,\beta_2,\cdots,\beta_n\}$是线性空间
	$V(\mathbf{F})$的任意两组基,$B_2$中每个基向量被基$B_1$表示为
	$$\begin{cases}
		\beta_1=a_{11}\alpha_1+a_{21}\alpha_2+\cdots+a_{n1}\alpha_n \\
		\beta_2=a_{12}\alpha_1+a_{22}\alpha_2+\cdots+a_{n2}\alpha_n \\
		\cdots \\
		\beta_n=a_{1n}\alpha_1+a_{2n}\alpha_2+\cdots+a_{nn}\alpha_n
	\end{cases}.$$
	将上式用矩阵表示为
	$$(\beta_1,\beta_2,\cdots,\beta_n)=(\alpha_1,\alpha_2,\cdots,\alpha_n)\begin{pmatrix}
		a_{11} & a_{12} & \cdots & a_{1n} \\
		a_{21} & a_{22} & \cdots & a_{2n} \\
		\cdots & \cdots &        & \cdots \\
		a_{n1} & a_{n2} & \cdots & a_{nn}
	\end{pmatrix}.$$
	我们将这一矩阵称为即$B_1$变为基$B_2$的变换矩阵(或过渡矩阵).
\end{definition}
简单而言就是将$B_2$中的向量在$B_1$下的坐标按列排列.需要注意表述中是$B_1$变为基$B_2$还是反过来,
这两个矩阵互逆.注意过渡矩阵一定是基与基之间的表示矩阵,并且过渡矩阵一定可逆.
\begin{theorem}
	\textbf{基的选择对向量坐标的影响}
	
	设线性空间$V$的两组基为$B_1$和$B_2$,且基$B_1$到$B_2$的变换矩阵(过渡矩阵)为$A$,如果
	$\xi \in V(F)$,且在$B_1$和$B_2$下的坐标分别为$X$和$Y$,则$Y=A^{-1}X$.
\end{theorem}
上述即教材定理4.10,描述同一个向量在不同基下坐标之间的关系.事实上,这与本节同构关系紧密,因为
同构意味着两个线性空间结构一致,故同构映射可以保持向量组的线性关系不变.在同构关系下,
线性组合对应线性组合,线性无关对应线性无关,线性相关对应线性相关.我们有如下定理:
\begin{theorem}
	设$(\alpha_1,\alpha_2,\cdots,\alpha_n)$是线性无关的向量组,且
	$$(\beta_1,\beta_2,\cdots,\beta_s)=(\alpha_1,\alpha_2,\cdots,\alpha_n)A,$$
	则向量组$(\beta_1,\beta_2,\cdots,\beta_s)$的秩等于矩阵$A$的秩.
\end{theorem}
定理的证明需要用到坐标映射是同构映射这一事实,我们不难发现等式左侧向量组与$A$的列向量组是等价的.
事实上我们也可以由此发现,过渡矩阵一定是可逆矩阵.
\begin{theorem}
	已知$\beta_i=a_{1i}\alpha_1+a_{2i}\alpha_2+\cdots+a_{ni}\alpha_n(i=1,2,\cdots,n)$,
	且$A=(a_{ij})$可逆,则$\alpha_1,\alpha_2,\cdots,\alpha_n$与$\beta_1,\beta_2,\cdots,\beta_n$
	是等价的.
\end{theorem}
实际上这一定理与上一定理的思想都是类似的,我们可以看一个例题练习一下:
\begin{example}
	已知$\beta_1=\alpha_2+\alpha_3$,$\beta_2=\alpha_1+\alpha_3$,$\beta_3=\alpha_1+\alpha_2$,
	证明$\alpha_1,\alpha_2,\alpha_3$与$\beta_1,\beta_2,\beta_3$等价.
\end{example}
\begin{theorem}
	\textbf{基的选择对映射矩阵的影响}
	
	设线性变换$\sigma \in L(V,V)$,$B_1=\{\alpha_1,\dots,\alpha_n\}$和$B_2=\{\beta_1,\dots,\beta_n\}$
	是线性空间的$V(F)$的两组基,基$B_1$变为基$B_2$的变换矩阵为$C$,如果$\sigma$在基$B_1$下的矩阵为$A$,
	则$\sigma$关于基$B_2$所对应的矩阵为$C^{-1}AC$.
\end{theorem}
上述即教材定理7.4,研究同一个映射在不同基下表示矩阵之间的关系.实际上我们将在下一专题初等矩阵一节进一步讨论.
这一定理的证明需要用到我们之前描述的两种线性映射矩阵表示的统一性.

\vspace{2ex} 
\centerline{\heiti \Large 内容总结}

\vspace{2ex} 

\centerline{\heiti \Large 习题}
\vspace{2ex} 
{\kaishu }
\begin{flushright}
    \kaishu

\end{flushright}
\centerline{\heiti A组}
\begin{enumerate}
	\item 
\end{enumerate}
\centerline{\heiti B组}
\begin{enumerate}
	\item 
\end{enumerate}
\centerline{\heiti C组}
\begin{enumerate}
	\item 
\end{enumerate}
\chapter{商空间与对偶}
\chapter{矩阵基本运算}

\section{矩阵基本运算}

\section{矩阵转置}

\section{初等矩阵}

\section{矩阵的逆}

\section{矩阵的逆的求解}

\chapter{矩阵运算进阶}

\section{特殊矩阵}

\section{分块矩阵}

\section{矩阵的幂}

\chapter{矩阵的秩}
本节内容理解难度较大,事实上这里利用了很多线性空间与线性映射的思想,
也有很多技巧性的内容,因此希望各位同学根据自己实际情况理解掌握.虽然很推荐
这部分内容采用与教材不同的思路去理解,更多利用线性空间与线性映射的抽象知识思考,但是如果
理解起来有一定困难也记住一些结论去解决一些问题.

还有一部分应当属于本节的内容将在专题五线性方程组的部分提及,因此本节不再专门
讲解利用线性方程组的思想解决矩阵的秩相关问题的部分.

\section{矩阵的秩}
我们首先给出矩阵的三个秩的定义:
\begin{definition}
    设$A$是线性映射$\sigma$对应的矩阵,我们把$\sigma$的秩也称为矩阵$A$的秩,
    记为$r(A)$.我们将矩阵$A$的所有行向量组成的秩称为$A$的\keyterm*{行秩}[row rank],
    所有列向量组成的向量组的秩称为$A$的\keyterm*{列秩}[column rank].
\end{definition}
对于以上三个秩我们有重要的定理如下:
\begin{theorem}
    任意矩阵的秩 = 行秩 = 列秩.
\end{theorem}
这一定理的证明,矩阵的秩 = 列秩的部分根据线性映射的相关概念是显然的,行秩的部分
教材中有较为繁琐的证明,在本讲义下面的内容会有更加形象的解释.

实际上,这一定理有两个重要的直接推论,一是将求矩阵的秩的问题转化为求矩阵行/列极大线性无关向量组的问题,
第二是矩阵的秩等于其转置的秩.

可能很多同学对于行秩、列秩相等以及转置的几何意义很感兴趣.实际上我们有两种获得转置矩阵的
方式,第一种来源于我们之前讨论的对偶空间上的线性映射对应的矩阵,这种方式可能不够直观.
另一种获得的方法基于内积,感兴趣的同学可以了解矩阵的伴随(不是行列式中的伴随矩阵).

我们可以研究矩阵及其转置的关系,我们可以用一个图形来表示:

\begin{figure}[h]
    \centering
    \small
    \begin{tikzpicture}
        \tikzset{->-/.style={decoration={
            markings,
            mark=at position .6 with {\arrow{stealth}}},postaction={decorate}}}

        \draw[rotate=45] (0,6) rectangle (-3,3) rectangle (-5,0)
            (-3,3) rectangle(-3.35,3.35)
            coordinate (xr) at (-2,4)
            coordinate (xn) at (-4,2)
            coordinate (x) at (-2,2)
            coordinate (0n) at (-3,3)
            node at (-1,5) {行空间}
            node at (-4,1) {$A$的核空间}
            node at (-1.5,6.5) {$\dim r$}
            node at (-4,4) {$\mathbf{R}^n$}
            node at (-6,3) {$\dim n-r$};

        \draw[rotate=30] (6,2) rectangle (3.5,-2) rectangle (0,-4)
            (3.5,-2) rectangle (3.85,-2.35)
            coordinate (b) at (4.5,0.5)
            coordinate (0m) at (3.5,-2)
            node at (5,1.5) {列空间}
            node at (2,-3) {$A^{\mathrm{T}}$的核空间}
            node at (7,0) {$\dim r$}
            node at (5,-3) {$\mathbf{R}^m$}
            node at (4,-4.5) {$\dim m-r$};

        \foreach \point in {xr, x, xn, 0n, b, 0m}
            \fill[black] (\point) circle (1pt);

        \node [left] at (xr) {$x_r$};
        \node [below right] at (x) {$x=x_r+x_n$};
        \node [left] at (xn) {$x_n$};
        \node [right] at (0n) {0};
        \node [right] at (b) {$b$};

        \draw[->-,very thick] (xr) -- node[above,sloped] {$Ax_r = b$} (b);
        \draw[->-,very thick] (x) -- node[below,sloped] {$Ax = b$} (b);
        \draw[->-,very thick] (xn) -- node[below,sloped] {$Ax_n = 0$} (0m);

        \draw[dashed,thick] (xr) -- (x) -- (xn);

    \end{tikzpicture}
\end{figure}

我们观察到以下几点:
\begin{enumerate}
    \item 矩阵的行空间与解空间(零空间)互为正交补(直观理解两个空间就是互相垂直且互为补空间),这一点应当是在正交的内容中有所提及的;
    \item 矩阵的列空间与其转置矩阵的零空间互为正交补,这一点实际与上一条等价.
\end{enumerate}

接下来我们来看行秩(列秩比较显然,此处不再详细展开).我们首先得到解空间($N(A)$)的维数,这可以直接
根据维数公式得到:$\dim N(A) =n-r(A)$,根据正交补的性质,我们的可以得到行秩即为
$n-(n-r(A))=r(A)$.于是我们得到了一个基于正交补的行秩解释.

\section{相抵标准形}
此处我们需要首先回顾一个基本定理:
\begin{theorem}
    初等变换不改变矩阵的秩(包括行变换和列变换).
\end{theorem}
由这一定理我们可以推导出相抵标准形:
\begin{theorem}
    若$r(A_{m \times n})=r$,则存在可逆矩阵$P$和$Q$,使得
    \[PAQ=\begin{pmatrix}
        E_r & 0 \\ 0 & 0
    \end{pmatrix}=U_r\]
    其中$E_r$表示$r$阶单位矩阵.
\end{theorem}
这一定理证明直接使用定理4以及可逆矩阵可以拆分为初等矩阵的乘积即可.
其中$U_r$称为相抵标准形.我们称两个矩阵相抵即两个矩阵可以通过一系列
初等变换可以互相转化.由此我们得到关于矩阵相抵的两个等价命题:

1. 矩阵$A$与$B$相抵$\iff$存在可逆矩阵$P$和$Q$使得$PAQ=B$;

2. 矩阵$A$与$B$相抵$\iff r(A)=r(B)$.

\begin{example}
    设$A=\begin{pmatrix}
        1 & 0 & 2 & -4 \\ 2 & 1 & 3 & -6 \\ -1 & -1 & -1 & 2
    \end{pmatrix}$. 求
    \begin{enumerate}
        \item $A$的秩$r$和相抵标准形;

        \item 3 阶可逆矩阵$P$和 4 阶可逆矩阵$Q$使得$PAQ=\begin{pmatrix}
            E_r & 0 \\ 0 & 0
        \end{pmatrix}$.
    \end{enumerate}
\end{example}

关于相抵标准形,我们需要在此补充一个常用的技术,即相抵标准形的分解:

我们对$s \times n$矩阵$\begin{pmatrix}
    E_r & O \\ O & O
\end{pmatrix}$有一种很重要的分解:
\[\begin{pmatrix}
    E_r & O \\ O & O
\end{pmatrix}=\begin{pmatrix}
    E_r \\ O
\end{pmatrix}\begin{pmatrix}
    E_r & O
\end{pmatrix}\]
由此我们可以知道任意一个非零矩阵都可以被分解成一个列满秩矩阵和一个
行满秩矩阵的乘积:

\[A=P\begin{pmatrix}
    E_r & O \\ O & O
\end{pmatrix}Q=P\begin{pmatrix}
    E_r \\ O
\end{pmatrix}\begin{pmatrix}
    E_r & O
\end{pmatrix}Q\]
记$P_1=P\begin{pmatrix}
    E_r \\ O
\end{pmatrix}$,$Q_1=\begin{pmatrix}
    E_r & O
\end{pmatrix}Q$,则$A=P_1Q_1$,且$P_1$和$Q_1$分别为列满秩、行满秩矩阵.

我们可以利用相抵标准形解决很多问题,例如下一节中部分秩不等式的证明:
\begin{example}
    \begin{enumerate}
        \item $r\begin{pmatrix}
            A & O \\ O & B
        \end{pmatrix}=r(A)+r(B)$.

        \item $r\begin{pmatrix}
            A & D \\ O & B
        \end{pmatrix}\geqslant r(A)+r(B),\enspace r\begin{pmatrix}
            A & O \\ C & B
        \end{pmatrix}\geqslant r(A)+r(B)$.
    \end{enumerate}
\end{example}

\section{秩不等式}
本节的内容实际上部分内容有一定的技巧性,对于荣誉课程来说还是以理解为主(所以
其实本节中提到的很多内容都只是介绍性的,而非要求大家熟练掌握,但是遇见了要有
一些基本的思路而不能完全不理解),可能下面列出定理的时候显得比较繁冗,但是实
际上我们更重视其中的理解而非硬套结论.

我们首先给出一些常见的秩相关的不等式或等式,这些式子希望各位同学能够理解其含义,
而非机械记忆套用.下面这些等式/不等式的证明方式非常多,实际上可以利用之前所说化为
相抵标准形的方法,也可以利用线性相关性的方法,也可以回到线性映射进行考量.总之
解决的方法非常多,希望各位同学能熟练推导理解.
\begin{enumerate}
    \item $r(A)=r(PA)=r(AQ)=r(PAQ)$,其中$P$、$Q$可逆
    \item $|r(A)-r(B)|\leqslant r(A\pm B) \leqslant r(A)+r(B)$
    \item $r(AB) \leqslant \min\{r(A),\ r(B)\}$
    \item $r(A)=r(A^\mathrm{T})=r(AA^\mathrm{T})=r(A^\mathrm{T}A)$(注意第二个等号需要实矩阵作为前提条件)
    \item $A \in \mathbf{F}^{s \times n}$,$B \in \mathbf{F}^{n \times m}$,
    则$r(AB) \geqslant r(A)+r(B)-n$.(可以视为结论6的推论,特例$AB=O$时有$r(A)+r(B)\leqslant n$)
    \item $r(ABC) \geqslant r(AB)+r(BC)-r(B)$.(还可以考虑$A,B,C$相等的特殊情况的结果)
\end{enumerate}

分块矩阵的相关公式在上一小节的例题中已经书写过,此处不再重复.

一般而言,解决较为复杂的秩的问题时,我们可以采用如下方法:
\begin{enumerate}
    \item 利用(分块)矩阵初等变换;

    \item 利用线性方程组解的一般理论(将在专题五讲解);

    \item 利用向量组线性相关性;

    \item 利用已知的矩阵秩的等式和不等式.实际上等式很多时候基于可逆矩阵变换或者两个不等号夹逼.
\end{enumerate}

相关方法的应用都在本节最后的习题中有所体现,当然首要的任务是掌握上述基本的秩不等式的证明,
很多也利用了上面的思想,并且解法不唯一.

\vspace{2ex}
\centerline{\heiti \Large 内容总结}

\vspace{2ex}

\centerline{\heiti \Large 习题}
\vspace{2ex}
{\kaishu }
\begin{flushright}
    \kaishu

\end{flushright}
\centerline{\heiti A组}
\begin{enumerate}
    \item
\end{enumerate}
\centerline{\heiti B组}
\begin{enumerate}
    \item
\end{enumerate}
\centerline{\heiti C组}
\begin{enumerate}
    \item
\end{enumerate}

\chapter{行列式(I)}

\section{行列式的几种定义}

\section{行列式的基本运算}

\section{伴随矩阵}

\section{Cramer法则}

\section{行列式的秩}
\chapter{行列式计算进阶}
\chapter{朝花夕拾}

\section{线性方程组解的一般理论}

\section{理论应用}

\section{线性方程组拓展题型}

\vspace{2ex} 
\centerline{\heiti \Large 内容总结}

\vspace{2ex} 

\centerline{\heiti \Large 习题}
\vspace{2ex} 
{\kaishu }
\begin{flushright}
    \kaishu

\end{flushright}
\centerline{\heiti A组}
\begin{enumerate}
	\item 
\end{enumerate}
\centerline{\heiti B组}
\begin{enumerate}
	\item 
\end{enumerate}
\centerline{\heiti C组}
\begin{enumerate}
	\item 
\end{enumerate}
\chapter{多项式}

\section{多项式的定义}

\section{带余除法}

\section{代数学基本定理}

\vspace{2ex} 
\centerline{\heiti \Large 内容总结}

\vspace{2ex} 

\centerline{\heiti \Large 习题}
\vspace{2ex} 
{\kaishu }
\begin{flushright}
    \kaishu

\end{flushright}
\centerline{\heiti A组}
\begin{enumerate}
	\item 
\end{enumerate}
\centerline{\heiti B组}
\begin{enumerate}
	\item 
\end{enumerate}
\centerline{\heiti C组}
\begin{enumerate}
	\item 
\end{enumerate}
\chapter{不变子空间}

\section{不变子空间的定义}
在介绍本节内容前,我们需要首先对算子(operator)这一名词进行解释.
\begin{definition}
	向量空间到其自身的线性映射成为算子.
\end{definition}
以上是《线性代数应该这样学》对于算子的定义,本章中出现算子一词也默认为此含义.
实际上,狭义的算子指从一个函数空间到另一个函数空间(或其自身)的映射,例如
微积分中学习的梯度,散度以及拉普拉斯算子等.本书中采用的是广义的定义,将算子
这一定义延伸至向量空间.

需要注意的是,done right喜欢用抽象的算子作为研究对象,一般的高等代数则更喜欢具象的矩阵,
实际上二者是完全统一的,很多定理虽然用算子描述,实际上也有相应的矩阵版本.因为实际上矩阵$A$可以视为
算子$T\alpha=A\alpha$在自然基下的矩阵,由此可以做到统一.

在我们接下来的叙述中,我们希望将算子在基下的矩阵表示尽可能简单.在下面的内容中,有三个关键概念是
要经常提及的:算子,矩阵以及多项式,接下来的主干内容就是围绕这三者之间的关系展开.我可以在此画一个
三角形,各边连线上的内容大家可以在阅读过程中自己补全.
\begin{figure}[h]
	\centering
	\includegraphics[scale=0.4]{./figs/15/15-1.png}
\end{figure}

我们在上一章中讨论了算子可对角化的充要条件.但事实上存在大量不可对角化
的算子,我们也希望得到其最简单的矩阵表示形式.在定理\ref{可对角化充要条件}
中我们可以看到,$V$上算子$T$可对角化当且仅当$V$可以分解为$T$的特征子空间的
直和:$$V=V_{\lambda_1}\oplus V_{\lambda_2}\oplus\cdots\oplus V_{\lambda_s},$$
其中$\lambda_1,\lambda_2,\cdots,\lambda_s$为$T$的所有不同本征值.我们注意到
$\alpha\in V_{\lambda_i},T\alpha=\lambda_i\alpha\in V_{\lambda_i}$,这启示
我们可以将$V$分解为具有这一性质的子空间的直和来研究矩阵的简化形式.满足这一性质的
空间对于我们的研究非常重要,我们需要给予它一个定义:
\begin{definition}
	设$T\in L(V)$,若$V$的子空间$U$满足$\forall \alpha\in U,T\alpha\in U$,
	则称$U$是$T$的不变子空间,简称为$T$-子空间.
\end{definition}
即不变子空间中的每一个向量在算子作用后仍在这一空间中.为了研究在某一子空间下映射的性质,
我们还需要引入映射的限制的概念:
\begin{definition}
	设$g:A\to B$是一个映射,在$A$的子集$A_0$上定义$f:A_0\to B$满足$f(a)=g(a),\forall a\in A_0$,
	则称$f$为$g$关于集合$A_0$的限制映射,记为$f=g|_{A_0}$.
\end{definition}
即限制映射就是将原映射的定义域进行收缩,但原定义域上的函数值保持不变.
事实上,若映射为线性映射,限制的集合为线性映射的不变子空间,则这一映射限制在这一空间上成为算子,
我们称其为\textbf{限制算子}.

如果$U$是$T$的不变子空间,那么$T$还可以诱导出商空间$V/U$上的一个线性变换$T/U$,满足
$(T/U)(v+U)=Tv+U$,其中$v\in V$,称之为\textbf{商算子}.

这一定义的线性性容易验证,这里需要提及的是合理性(即是否是良定义,或well-defined的).
事实上,对于一个映射,其合理性在于原像集合中的一个元素只能映射到像集中的唯一一个值
(否则不符合映射的定义).商算子的出发空间元素是等价类,因此如果出现$v+U=w+U$但$Tv+U\neq Tw+U$
的情况,这一定义描述的就不是映射,因此不是良定义.但我们可以验证这一映射是良定义的,详见教材106页.
\begin{example}
	设$T\in L(V,W)$,定义$\tilde{T}:(V/(\textup{null }T))\to W$如下:
	$$\tilde{T}(v+\textup{null }T)=Tv.$$

	\textup{(1)}$\tilde{T}$是良定义的,且是$(V/(\textup{null }T))$到$W$上的线性映射;

	\textup{(2)}$\tilde{T}$是单射;

	\textup{(3)}$\textup{range }\tilde{T}=\textup{range }T$;

	\textup{(4)}$V/(\textup{null }T)$同构于$\textup{range }T$.
\end{example}
\begin{example}
	设$T\in L(V)$,证明:

	\textup{(1)}$T/(\textup{range }T)=0$;

	\textup{(2)}$T/(\textup{null }T)$是单的当且仅当$\textup{null }T\cap\textup{range }T=\{0\}$.
\end{example}
教材例5.3给出了四个常见的不变子空间的例子,分别是两个平凡子空间和映射的像与核.教材8.20还给出了$p$
为多项式时,$\textup{null }p(T)$和$\textup{range }p(T)$也为$T$的不变子空间.但有时我们可能会遇到
更为复杂的情形,如下面的例子:
\begin{example}
	设$V$是$n$维复向量空间,$T\in L(V)$,若$T$有$n$个互异的本征值,求$T$的所有不变子空间的个数.
\end{example}
\begin{example}
	设$\mathbf{F}$为一数域,算子$T$定义为
	$$T(a,b)=(a,b)\begin{pmatrix}
		1 & -1 \\ 2 & 2
	\end{pmatrix},$$证明:

	\textup{(1)}当$\mathbf{F}=\mathbf{R}$时,$\mathbf{R}^2$无$T$的非零真不变子空间;

	\textup{(2)}当$\mathbf{F}=\mathbf{C}$时,$\mathbf{C}^2$有$T$的非零真不变子空间.
\end{example}
除此之外,还有一些问题我们将在讨论若当标准形时进行讨论.

\section{特征值与特征向量}
从本章起,我们主要研究线性变换(而非一般线性映射)和方阵的性质.
\subsection{特征值与特征向量的定义与求解}
首先介绍线性变换和矩阵的特征值与特征向量的概念:
\begin{definition}
	设$\sigma$是线性空间$V(\mathbf{F})$上的一个线性变换,如果存在数$\lambda\in\mathbf{F}$
	和非零向量$\xi\in V$使得$\sigma(\xi)=\lambda\xi$,则称数$\lambda$为$\sigma$的一个特征值,
	并称非零向量$\xi$为$\sigma$属于其特征值$\lambda$的特征向量.
\end{definition}
必须注意特征向量为非零向量,否则零向量对任意$\lambda$都满足上面定义,从而失去“特征”的含义.
但是特征值可以为0,此时实际上就是线性变换的零空间.

特征值与特征向量的几何意义在于,某一线性变换的特征向量在经过变换后得到的向量与原先向量共线.

我们称关于同一个特征值$\lambda$的所有特征向量构成的集合记为$V_\lambda=\{\xi\ |\ \sigma(\xi)=\lambda\xi,\xi\in V\}$,
称为$\sigma$关于其特征值$\lambda$的特征子空间.
\begin{example}
	证明:$V_\lambda$是$V$的子空间.
\end{example}
实际上,$\sigma(\xi)=\lambda\xi$等价于$(\lambda I-\sigma)(\xi)=0$,故特征子空间就是线性变换
$\lambda I-\sigma$的核,核中必有非零向量(特征向量非零),故$\lambda I-\sigma$在$\lambda$为特征值时
必不满秩,故我们可以通过$|\lambda E-A|=0$求解特征值,其中$A$为$\sigma$在某组基下的矩阵.
对于特征向量的求解,求出$(\lambda E-A)X=0$的非零解就是特征向量在基下的坐标.

上面是线性变换的特征值与特征向量的定义,我们在考试中更一般地会遇到下述矩阵的特征值与特征向量的定义:
\begin{definition}
	设矩阵$A\in M_n(\mathbf{F})$,如果存在数$\lambda\in\mathbf{F}$和非零向量$X\in\mathbf{F}^n$使得
	$AX=\lambda X$,则称数$\lambda$为$A$的一个特征值,称非零向量$X$为$A$属于其特征值$\lambda$的特征向量.
\end{definition}
\begin{example}
	设$A=\begin{pmatrix}
		1 & -1 & 0 \\ 2 & 0 & 1 \\ 1 & a & 0
	\end{pmatrix}$,且存在非零向量$\alpha$使得$A\alpha=2\alpha$,求$\alpha$.
\end{example}
下面我们说明这两个定义的关系.实际上,假设$A$是$\sigma$在基$\alpha_1,\cdots,\alpha_n$下的表示矩阵,且
$\xi=(\alpha_1,\cdots,\alpha_n)X$,我们有
\begin{align*}
	\sigma(\xi)=\lambda\xi &\Leftrightarrow \sigma(\alpha_1,\cdots,\alpha_n)X=\lambda(\alpha_1,\cdots,\alpha_n)X \\
						   &\Leftrightarrow (\alpha_1,\cdots,\alpha_n)AX=(\alpha_1,\cdots,\alpha_n)(\lambda X) \\
						   &\Leftrightarrow AX=\lambda X
\end{align*}
因此$\lambda$同时是线性变换和矩阵的特征值,与基的选取无关,但$X$与基的选取有关.

我们在之前已经分析了求解特征值的方法,即求解$f(\lambda)=|\lambda E-A|$的根. 我们称其为矩阵$A$的特征多项式.
其$k$重根称为$k$重特征值(或称代数重数),对应的特征子空间维数称几何重数.

我们展开特征多项式得到以下定理:
\begin{theorem}
	对于$n$级矩阵$A=(a_{ij})$,记
	$$f(\lambda)=|\lambda E-A|=a_0\lambda^n+a_1\lambda^{n-1}+\cdots+a_{n-1}\lambda+a_n.$$
	则$a_0=1$,$a_n=(-1)^n|A|$,且$a_k$等于所有$k$级主子式之和乘以$(-1)^k$.
\end{theorem}
这一定理的证明无需掌握,并且关于特征多项式的进一步讨论也将在线性代数II中涉及.这里我们主要掌握两个特例,
即由韦达定理,我们有$\sum\limits_{i=1}^{n}\lambda_i=\sum\limits_{i=1}^{n}a_{ii}$,
$\prod\limits_{i=1}^{n}\lambda_i=|A|$,
即特征值按重数求和为矩阵的迹(即矩阵对角线元素之和),特征值按重数求积为矩阵行列式.
这一结论在解决某些问题时有一定作用.

\subsection{特征值的基本性质}
关于特征值,我们有如下基本性质,证明较为基本,可以自行完成:

设$\lambda$是线性空间$V(\mathbf{F})$上的线性变换$\sigma$的特征值,$\xi$是$\sigma$属于$\lambda$的特征向量,则

1. $k\lambda$是$k\sigma$的特征值,$\lambda^m$是$\sigma^m$的特征值,且$\xi$仍是相应特征向量.

2. 若$f(x)=a_nx^n+a_{n-1}x^{n-1}+\cdots+a_1x+a_0$是$\mathbf{F}$上的多项式,则$f(\sigma)(\xi)=f(\lambda)\xi$.

3. 设$\lambda$是$n$阶矩阵$A$的特征值,$A$可逆,则$\lambda^{-1}$是$A^{-1}$的特征值,$|A|\lambda^{-1}$是$A$的伴随矩阵
$A^*$的特征值,且特征向量不变.
\begin{example}
	回答以下问题:

	\textup{(1)}设$A$为三阶矩阵,$A^2-A-2E=O$,$|A|=2$,求$|A^*+3E|$\textup{;}
	
	\textup{(2)}设$A$为三阶矩阵,其特征值为$1$,$-2$,$-1$,求$|A|$,$A^*+3E$的特征值,$(A^{-1})^2+2E$的特征值
	以及$|A^2-A+E|$\textup{;}
	
	\textup{(3)}设$\alpha=(1,0,-1)^\mathrm{T}$,且$A=\alpha\alpha^\mathrm{T}$,求$|6E-A^n|$\textup{;}
	
	\textup{(4)}设$A$为三阶矩阵,其特征值为$-1$,$-1$,$5$,求$A_{11}+A_{22}+A_{33}$\textup{;}
	
	\textup{(5)}设$A$为三阶实对称矩阵,$A^2=A$且$r(A)=2$,求$|A+2E|$.
\end{example}

下面一个例子也是重要的结论,实际上在行列式专题已给出类似结论,但我们现在从特征值角度考虑这一结论:
\begin{example}
	回答以下两个问题:
	
	\textup{(1)}设$A,B$均为$n$阶矩阵,证明:$\lambda\neq 0$是$AB$的特征值,则$\lambda$也是$BA$的特征值\textup{;}
	
	\textup{(2)}设$A\in M_{m\times n}(\mathbf{C}),B\in M_{n\times m}(\mathbf{C})$,证明:
	$$\begin{pmatrix}
		AB & O \\ B & O
	\end{pmatrix}\sim\begin{pmatrix}
		O & O \\ B & BA
	\end{pmatrix},$$
	并由此推出$AB$和$BA$非零特征值相同,且$m=n$时有$|\lambda E-AB|=|\lambda E-BA|$.
\end{example}
不难发现(2)是(1)的推广.下面这一例子也是一些经典的结论,应当熟悉.
\begin{example}
	对下列矩阵$A$的特征值,能做出怎样的断言?
	
	\textup{(1)}$A$可逆/$A$不可逆/$E+A$可逆/$4E+A$不可逆\textup{;}
	
	\textup{(2)}$\det(E-A^2)=0$\textup{;}
	
	\textup{(3)}$AA^\mathrm{T}=A^\mathrm{T}A=E$(正交)/$A^2=E$(对合)/$A^2=A$(幂等)/$A^k=0$(幂零)\textup{;}
	
	\textup{(4)}$A=\lambda_0E+B$($\lambda_0$为常数,且已知$B$的$n$个特征值为$\lambda_1,\lambda_2,\cdots,\lambda_n$)\textup{;}
	
	\textup{(5)}$A$为对角块矩阵,即$A=\begin{pmatrix}
		A_1 &  &  &  \\  & A_2 &  &  \\  &  & \ddots &  \\  &  &  & A_m
	\end{pmatrix}$(与线代$\textup{II}$中不变子空间有关).
\end{example}
在上一小节我们讨论了特征值之和为迹的事实,实际上关于迹以及相关的幂零矩阵的讨论在专题三中已有涉及,
现在可以回过头去再看一些性质的证明.除此之外,我们还可以给幂零矩阵一个等价的定义:
\begin{theorem}
	一个方阵为幂零矩阵当且仅当其所有特征值均为$0$.
\end{theorem}
这一定理“仅当”部分证明比较基本,只需用到上述某一特征值性质即可,但“当”的部分无需掌握证明,需要用到线性代数II的
哈密顿-凯莱定理(事实上很多更进一步的讨论都要基于这一定理).
\subsection{特征向量的基本性质}
这一部分的定理与下一节中可对角化的等价条件直接相关,实际上有了本节的定理,可对角化条件是很显然的.
\begin{theorem}
	线性映射$\sigma$的不同特征值$\lambda_1,\cdots,\lambda_m$对应的特征向量$\xi_1,\cdots,\xi_m$线性无关.
\end{theorem}
\begin{theorem}
	线性映射$\sigma$的不同特征值$\lambda_1,\cdots,\lambda_m$对应的特征子空间$V_{\lambda_1},\cdots,V_{\lambda_m}$的和是直和,
	即$\dim(V_{\lambda_1}+V_{\lambda_2}+\cdots+V_{\lambda_m})=\sum_{j=1}^{m}\dim V_{\lambda_j}$.
\end{theorem}
以上两个定理的证明可以参考教材定理7.7及其推论,实际上二者等价,只需证出其中一个,另一个就是显然的.
两个定理有如下推论:

1. 若$\lambda_1,\cdots,\lambda_m$是线性映射$\sigma$互异的特征值,则$V_{\lambda_i}\cap\sum\limits_{j\neq i}V_{\lambda_j}=\{0\}
(i=1,\cdots,m)$,则一个特征向量不能属于多个特征值.这一推论来源于直和的一个等价条件,专题一的习题中有涉及.

2. $\sigma$的不同特征值$\lambda_1,\cdots,\lambda_m$对应的特征子空间$V_{\lambda_1},\cdots,V_{\lambda_m}$的基向量
合在一起构成的向量组线性无关,且是$V_{\lambda_1}+V_{\lambda_2}+\cdots+V_{\lambda_m}$的基.

接下来这个定理说明了代数重数和几何重数之间的关系:
\begin{theorem}
	$n$维线性空间$V(\mathbf{F})$的线性变换$\sigma$的每个特征值$\lambda_i$的重数(代数重数)大于等于其特征子空间$V_{\lambda_i}$的维数
	(几何重数).
\end{theorem}
这一定理的证明比较复杂,版本也很多,有兴趣的同学可以了解.

\vspace{2ex} 
\centerline{\heiti \Large 内容总结}

\vspace{2ex} 

\centerline{\heiti \Large 习题}
\vspace{2ex} 
{\kaishu }
\begin{flushright}
    \kaishu

\end{flushright}
\centerline{\heiti A组}
\begin{enumerate}
	\item 
\end{enumerate}
\centerline{\heiti B组}
\begin{enumerate}
	\item 
\end{enumerate}
\centerline{\heiti C组}
\begin{enumerate}
	\item 
\end{enumerate}
\chapter{相似标准形}

\section{上三角矩阵}

\section{对角矩阵}

\section{分块对角矩阵}

\chapter{多项式的进一步讨论}

\section{特征多项式与极小多项式}

\section{哈密顿-凯莱定理}

\section{多项式与标准形}

\chapter{若当标准形}

\section{若当标准形的存在}

\section{若当标准形的求法}

\section{若当标准形的应用}
\chapter{内积空间}

\section{内积和范数}

\section{标准正交基}

\section{正交补}

\vspace{2ex} 
\centerline{\heiti \Large 内容总结}

\vspace{2ex} 

\centerline{\heiti \Large 习题}
\vspace{2ex} 
{\kaishu }
\begin{flushright}
    \kaishu

\end{flushright}
\centerline{\heiti A组}
\begin{enumerate}
	\item 
\end{enumerate}
\centerline{\heiti B组}
\begin{enumerate}
	\item 
\end{enumerate}
\centerline{\heiti C组}
\begin{enumerate}
	\item 
\end{enumerate}
\chapter{内积空间上的算子(I)}

\section{正交矩阵和酉矩阵}

\section{正定矩阵}

\chapter{内积空间上的算子(II)}

\section{自伴算子和正规算子}

\section{谱定理}

\chapter{极分解与奇异值分解}
\chapter{实空间上的算子}
\chapter{行列式(II)}
\chapter{线性代数与解析几何基础}
\chapter{二次型}

\section{双线性函数}

\section{二次型的标准形}

\section{惯性定理}

\vspace{2ex} 
\centerline{\heiti \Large 内容总结}

\vspace{2ex} 

\centerline{\heiti \Large 习题}
\vspace{2ex} 
{\kaishu }
\begin{flushright}
    \kaishu

\end{flushright}
\centerline{\heiti A组}
\begin{enumerate}
	\item 
\end{enumerate}
\centerline{\heiti B组}
\begin{enumerate}
	\item 
\end{enumerate}
\centerline{\heiti C组}
\begin{enumerate}
	\item 
\end{enumerate}
\chapter{线性代数与多元微积分}

\section{向量函数的导数}

\section{行列式的导数}

\section{雅可比行列式}

\end{document}
