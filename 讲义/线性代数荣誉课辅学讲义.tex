\makeatletter
   \def\input@path{{..}}
\makeatother

\documentclass[
    % colors = false,
    geometry = 16k,
]{LALUbook}

\usepackage{booktabs} % Excel 导出的大表格
\usepackage{rotating}
\usepackage{extarrows}

\usepackage{float}
\usepackage{diagbox}
\usepackage{caption}

\usepackage{pgfplots}
\usetikzlibrary{cd, arrows, arrows.meta, calc, intersections, decorations.pathreplacing, patterns, decorations.markings}
\pgfplotsset{compat=newest}

\usepackage[xindy, splitindex]{imakeidx}
\makeindex[
    columns=1,
    program=truexindy,
    intoc=true,
    options=-M texindy -I xelatex -C utf8,
    title={名词索引}
] % 名词索引
\makeindex[
    columns=3,
    program=truexindy,
    intoc=true,
    options=-M numeric-sort -M latex -M latex-loc-fmts -M makeindex -I xelatex -C utf8,
    name=sym,
    title={符号索引}
] % 符号索引

% 嵌套 enumerate 环境的 label
\setlist[enumerate,2]{label=(\arabic*)}
\setlist[enumerate,3]{label=\roman*.}

\usepackage{xparse}
\NewDocumentCommand{\term}{m}{{\sffamily\heiti\bfseries{#1}}}

% 标题格式
% \ResetChapterNumberingStyle 重置章节编号为正常样式
% \SetLUChapterNumberingStyle 设置章节编号为未竟专题样式
% 参数为下一章节编号减一
\NewDocumentCommand{\ResetChapterNumberingStyle}{m}{%
\setcounter{chapter}{#1}
\ctexset{
    chapter = {format={\centering\Huge\bfseries},name={第,讲},number=\arabic{chapter}},
    section = {format={\raggedright\Large\bfseries},name={,},number={\arabic{chapter}.\arabic{section}}},
    subsection = {format={\raggedright\large\bfseries},name={,},number={\arabic{chapter}.\arabic{section}.\arabic{subsection}}},
    subsubsection = {format={\raggedright\normalsize\bfseries},name={,},number={\arabic{chapter}.\arabic{section}.\arabic{subsection}.\arabic{subsubsection}}},
}
\renewcommand{\thechapter}{\arabic{chapter}}
}
\NewDocumentCommand{\SetLUChapterNumberingStyle}{m}{%
\setcounter{chapter}{#1}
\ctexset{
    chapter = {format={\centering\Huge\bfseries},name={未竟专题,},number=\zhnumber{\arabic{chapter}}},
    section = {format={\raggedright\Large\bfseries},name={,},number={\texorpdfstring{\arabic{chapter}$\boldsymbol{\varepsilon}$}{\arabic{chapter}ε}.\arabic{section}}},
    subsection = {format={\raggedright\large\bfseries},name={,},number={\texorpdfstring{\arabic{chapter}$\boldsymbol{\varepsilon}$}{\arabic{chapter}ε}.\arabic{section}.\arabic{subsection}}},
    subsubsection = {format={\raggedright\normalsize\bfseries},name={,},number={\texorpdfstring{\arabic{chapter}$\boldsymbol{\varepsilon}$}{\arabic{chapter}ε}.\arabic{section}.\arabic{subsection}.\arabic{subsubsection}}},
}
\renewcommand{\thechapter}{\arabic{chapter}$\boldsymbol{\varepsilon}$}
}

\ResetChapterNumberingStyle{0}

\title{\heiti 浙江大学 2023--2024 学年 \\ 线性代数荣誉课辅学讲义}
\author{2023--2024 学年线性代数 I/II(H)辅学授课 \\ 吴一航 \quad \verb|yhwu_is@zju.edu.cn|}

\AtEndPreamble{\hypersetup{
    pdfauthor = {吴一航},
    pdftitle = {线性代数荣誉课辅学讲义},
}}

\begin{document}
\frontmatter

% 封面,代替 \maketitle
\includepdf[
    pages={1},
    noautoscale=true,
    trim=0 0 0 10mm,
    clip,
]{./figs/cover-16k.pdf}

\songti

{% 插入空页
\null
\thispagestyle{empty}
\clearpage
\thispagestyle{empty}
\vspace*{\fill}
\begin{center}
    \Large 致每一个阳光下闪烁着七彩光芒的泡沫

    \vspace{2ex}

    \large \textit{To every bubble glittering with colorful lights under the sun}
\end{center}
\vspace*{\fill}

}
\clearpage
\setcounter{page}{1}
\chapter*{序}

\section*{一些初衷}

我为这本讲义起了一个大胆的标题,它来源于浙江大学竺可桢学院线性代数II(H)课程选用的教材《线性代数应该这样学》(英文原版名:《Linear Algebra Done Right》)。我们带着半娱乐性质地将最后两个单词像矩阵求逆一样(见封面设计)进行了颠倒,得到了本书的英文名:《Linear Algebra Left Undone》。

接下来我们遇到了一个问题:中文名应该是什么呢?郑俊达同学提供了一个可行解:《线性代数:未竟之美》。转念一想,这一标题不能更契合我们的编写初衷。事实上,我们认为现行的大部分线性代数或高等代数教材具有如下问题,它们也困扰了笔者和许多读者的学习,我们也给出了解决的方案:
\begin{enumerate}
    \item 从线性代数的角度来看,它们的讲解顺序不够自然,大部分教材都从行列式起步,缺乏引入地给出各种概念,使得读者无法理解线性代数的本质。可以说这些教材应当更名``行列式与矩阵计算'',因为线性代数着重研究的线性空间和线性映射反而成为了边缘内容。因此我们采取了更好的讲解思路,更能体现线性代数的美感而非延续高中填鸭式的数学教育——事实上那根本称不上数学,那样的讲授思路根本不够``数学'',失去了数学本身的自然之美,而且使得读者误解数学、厌恶数学;

    \item 浙江大学竺可桢学院两学期线性代数课程选择的《大学数学:代数与几何》和《线性代数应该这样学》教材采用了从抽象空间引入的方式,更贴近本质。但实践过程中许多同学会对``为什么要一开始就学习这些抽象内容''缺乏概念,特别是《线性代数应该这样学》对于工科同学而言``数学味道太浓'',因此最后可能学习效果还不如填鸭式地灌输解题方法。因此我们在讲义中相当于为教材做了很多的注脚,并且优化了整体设计,提供了大量例题习题,都是为了能更自然地引入抽象内容,让读者知道我们为何要学习这些内容,这些内容当年在数学家眼中最自然的状态是什么,这样才能使得抽象的概念易于被初学者接受;

    \item 我们的例题和习题编排也是精心设计过的,不会出现大部分教材使用过程中``上课讲的、作业做的和考试考的脱节''的情况,这一问题不只是很多数学基础课教学的问题,也是国内各个专业都存在的教学问题,笔者也深受其害,所以编写例题习题特别注重对概念和定理的理解、对方法的掌握,不会出现教材中说什么知识很重要但没有例子体会很重要的这种抽象情况,并且大量的习题贴近所学知识也贴近考试,让读者通过习题更好地掌握知识而非反而迷惑不知道自己学了什么,才能更好地体会线性代数的美感而非感受到题海的压迫;

    \item 除了自然的美感外,更重要的是还有``未竟''的美感。线性代数是一个古老而年轻的学科。它发轫于早先对线性方程组的研究,经历了漫长的几何和代数的交错作用,最后又在近世代数的发展过程中被严格化。直到现在,一些相关的内容,例如线性代数群的研究尚且方兴未艾,在现代数学的种种支线当中也有着重要的应用。另外,它的方法论,尤其是其对代数结构的研究在现代数学中也具备着代表性。因此,我们希望呈现一个更广阔的线性代数观,从线性代数出发,对它的现代发展和它在现代数学的各个分支的应用进行一些导论性的介绍,这一方面是为了使得平淡的叙述更加有趣,另一方面也是为了回答一个疑问:线性代数到底有什么用?我们相信,这是许多初学者都有的一个问题,回答这个问题既需要对线性代数的深入学习,也需要有一个现代数学的全局观,这也就带来了本书的另一个部分,未竟专题,也是我们这本书的标题来源。
\end{enumerate}

古人有三不朽:立德、立功、立言,著书立说即为立言。虽说我完全不可能因为编写了一本基础课的讲义而有如此崇高的地位。但在我心里,我已经通过这本讲义将我的热情、我的想法传达给了不少的读者,这样也无愧于我在浙江大学的本科四年。未来或许这本讲义会淹没在历史的风尘中,但我想只要它的某行文字曾经给予读者一丝丝的启发,或更实际地帮助了读者得到了心仪的分数,我想它就是有价值的,我本人的价值也得到了一定的实现。

\section*{本书的面世}

自2021年秋参与浙江大学竺可桢学院学研部(现竺可桢学院学业指导中心)组织的朋辈辅学活动以来,我即将第三次参与线性代数荣誉课的辅学活动。犹记讲义最初的简陋版本,那时为了辅学的期中、期末准备的简单复习提要,里面因为本人时间有限甚至缺少了特征值与特征向量的内容。那时的讲义基本都是知识点的罗列,缺少了许多重要的例题和证明,犹记第一次拿起这个讲义站上讲台的时候,我深刻体会到了这一讲义的不足,因此那次的授课整体而言较为玄学,比较干瘪。因此在2022年再次参加辅学授课时,我借着疫情放开考试延迟的机会分了六个大专题写出了一本相对完整的适合于《大学数学:代数与几何》的复习讲义。里面的讲解比较全面,习题也十分丰富,可以说在复习资料中已经能算过得去的一版了。

但我并不满足于此,我希望这本讲义能成为一本真正的完整的讲义,能兼具配套学习、考试复习的功能,并且在保证体系严谨完整的前提下有更优化的讲解逻辑。因此在2023年的暑假,我基于原先的复习版本进行扩展重排。在这一版本中,我将原先复习资料中的粗略描述都换为了严谨的完整叙述,并且反复打磨讲解顺序,从而更自然地将另一本教材《线性代数应该这样学》的内容自然融合,并且添加了大量的remark更适合于初学者学习。更重要的是,我们中间添加了许多文字叙述,一方面自然引入我们要讲解的内容,这对于初学者而言是很重要的insight,另一方面反复强调我们的行文逻辑,对推进逻辑做适当总结,使得读者能更快地形成体系,同时也补充了很多拓展内容,一些是为了方便读者更自然地理解抽象内容,有一些是契合``未竟之美''的标题,让读者能体会到数学的美感,体会到学习线性代数后我们知识的边界可以推广到多远。

\section*{参考文献}

本讲义作为浙江大学竺可桢学院线性代数荣誉课的辅学讲义,因此核心思路来源于我们选择的教材《大学数学:代数与几何(第二版)》(居余马,李海中)、《大学数学:代数与几何学习辅导》(林翠琴,居余马)、《线性代数应该这样学(第三版)》([美]Sheldon Axler)。

在编写与修订的过程中,我也参考了其他非常多优质的教材或辅导资料,如《高等代数(第二版)》(丘维声)、《高等代数:学习指导书》(丘维声)、《高等代数学(第四版)》(谢启鸿,姚慕生,吴泉水)、《高等代数学(第四版)配套学习用书》(谢启鸿,姚慕生)、《线性代数辅导讲义》(汤家凤)、《高等代数强化讲义》(李扬)以及《高等代数考研:高频真题分类精解300例》等,在复数域的引入部分我也简单参考了王晓光老师的《复变函数讲义(2023版)》,在矩阵计算等专题则部分参考了《数值分析》(Timothy Sauer)、《矩阵分析》(Roger A.Horn,Charles R.Johnson)等计算数学著作。

最后如果读者学完本讲义后对代数学有浓厚的兴趣,非常推荐读者学习后续的抽象代数课程。这里推荐与我同级的图灵班同学编写的\href{https://frightenedfoxcn.github.io/notes/series/alg-for-cs/}{《写给计算机系学生的代数》}作进一步的了解,我们许多高级专题都对这一讲义有引用。

\section*{致谢}

我或许首先需要感谢2022年疫情放开之下的寒冬,没有这学期线性代数考试的延期,我也不会有如此充裕的时间整理出本讲义较为完整的底稿,也就没有这一完整讲义的面世。

我还需要感谢同级的王和钧同学,感谢他当年push我写出了最初版本的复习提要。我要感谢比我低一级的郭苗苗同学,感谢她当年反复邀请我走上讲台实现梦想,虽然可能第一次授课效果一般,但这对于后来我不断打磨授课方式,打磨讲义有非常重要的意义。我也应当感谢竺可桢学院学研部(现竺可桢学院学业指导中心)给我提供了一个辅学的平台,让我通过讲义将我的热情能传达给更多的读者。

我想我也应该特别感谢数学科学学院的吴志祥、谈之奕和刘康生老师,他们在我线性代数入门过程中做了重要的引路人的工作,讲义中许多讲解思路也来源于他们精彩的授课。我也应该特别感谢数学科学学院的王晓光老师,他在复变函数课程以及讲义上的热情以及倾注的心血启发我也应该将我的学习思路和经验通过讲义传达给更多人,并且启发我思考如何从更高的观点、更自然的角度引导读者学习新知识,享受追求真理的过程。

感谢每一位读者和支持者,没有你们的支持,我也不会有如此的热情坚持编写这本讲义,正是有了大家的支持才有了接下来越来越好的版本的面世。特别感谢林前旭同学,他将这本讲义推广到了\href{https://mp.weixin.qq.com/s/nOQ0xzJ0mX2_8JclcKiWdA}{浙江大学微信公众号},这对我而言是极大地鼓舞。

最后,我需要感谢在编写的过程中对本讲义提供了直接的支持的同学:
\begin{itemize}
    \item 梅敏炫同学主编了讲义内积部分;
    \item 高天健同学对线性空间、矩阵、行列式、特征值以及内积等部分提供了宝贵的新讲解思路;
    \item 王普同学主编了线性代数与微积分部分;
    \item 郑俊达同学主编了解析几何部分;
    \item 周健均同学编写了行列式计算进阶以及奇异值分解的应用部分;
    \item 朱熙哲、谢集、郑俊达、郑涵文、李英琦同学负责了答案的初次编写,金政羽、张晋恺、江舜尧、任朱明、赵嘉瑞同学负责了历年卷答案以及正文答案的二次编写;
    \item 王鹤翔同学设计了本讲义的封面;
    \item 李英琦同学全权负责了本讲义的格式设计以及插图;
    \item 刘泓健同学负责了讲义的未竟专题部分。
\end{itemize}
除此之外,还有为我们的\href{https://github.com/yhwu-is/Linear-Algebra-Left-Undone}{GitHub仓库}提出 issue 以及提交 PR 的其他同学。没有他们的支持,这本讲义不会有今天的完整程度。

\begin{flushright}
    \kaishu
    吴一航 \\
    浙江大学计算机科学与技术学院 \\
    \verb|yhwu_is@zju.edu.cn| \\
    2023 年 8 月
\end{flushright}

\chapter*{致读者}

\section*{本书特色}

自有底稿成书的念头以来,笔者就十分希望本书能够摆脱市面上大部分线性代数或高等代数教材固有的一些不够友好的编写风格,力图呈现一本能让读者眼前一亮的讲义. 因此本讲义在编写过程中笔者不断创新讲授思路,大胆摒弃传统的编排风格,总体而言本讲义有如下几大特色:

\begin{enumerate}
    \item 本讲义兼具教材、笔记、复习提纲等多种功能:
          \begin{enumerate}
              \item 说它像是教材,因为我们保留了完整的讲授体系,所有的思路都是反复打磨确认过的,保证了整体逻辑的完整和自然;

              \item 说它像是笔记,因为这其中我们特别注重一些细节性的内容,这些内容在教材或授课中可能会因为太过平凡被忽略,但在初学中是很重要的,例如我们对求解线性空间像与核、求解线性映射矩阵表示的很多讨论都是基于笔者在初学时出现的困惑增添了很多的细节,力求读者在初学阶段就能减少因为这些细节带来的困惑;

              \item 说它像是复习提纲,因为在编写过程中我们的很多内容都会分条列出,并且笔者特别注意了编写的逻辑连贯性,阅读起来思路比一般教材主线更清晰. 除此之外每讲最后还有内容总结,并且经常会提供思维导图或是文字描述逻辑等便于读者快速掌握完整的思想体系.
          \end{enumerate}

    \item 本讲义提供了丰富的例题和习题,几乎能覆盖到所有重要的概念、定理和方法,同时我们也为这些题目提供了详细的解答,考虑了读者的阅读体验. 我们的例题编排特别考虑了初学者在学习过程中可能遇到的困难,特别设置了很多适合于加深对概念、定理以及基本方法理解的例题. 习题我们也是精心挑选,选择难度适中、有助于理解的经典题目,一些技巧性过强而脱离线性代数本质的题目我们也会删去或给读者一定的提示. 因此我们的讲义特别重视教学和习题和考试的一致性,这对于初次学习而言也是非常重要的,也是很多教材没有精心编排而忽略的,事实上这会特别影响读者阅读体验;

    \item 在本讲义的编排过程中,我们摒弃了传统的讲授思路. 首先我们选择《大学数学:代数与几何》以及《线性代数应该这样学》作为参考教材,它们都是从抽象空间出发研究的,相比于一般的线性代数或高等代数教材更能深入本质. 但我们也考虑到过于抽象的引入对初学者十分不友好,所以我们不断地强调我们的讲授逻辑,重视自然地引入概念,自然地推进对概念的研究,最后引申至这些概念对于我们之后的研究的重要性. 因此编排中我们不断优化内容编排顺序,也添加了足量的补充内容,目的就是使得读者能够更自然地接受而非填鸭式地囫囵吞枣,能够真正体会到数学的自然之美而非在抽象的描述或是繁杂的技巧中迷失了方向,我想这对于每一个数学学习者而言都是非常关键的.
\end{enumerate}

\section*{阅读建议}

我们为以下四类读者提供如下阅读建议:
\begin{enumerate}
    \item 初学线性代数过程中的读者:那么请坐稳扶好,备好配套的《大学数学:代数与几何》以及《线性代数应该这样学》. 这本讲义是很好的学习笔记,其中我们有大量的remark帮助读者理解教材中可能觉得很平凡的内容,也有大量编排合理的例题和习题帮助读者巩固知识. 初学过程中很推荐读者阅读我们反复强调的一些逻辑和一些补充的内容从而尽快形成学习体系,这对于数学学习是非常关键的——把握了主线,剩下的就只有一些细节留待补充. 当然一些较难的习题不一定在第一次就要掌握,因此可以根据自己的接受程度合理选择;

    \item 希望重新学习加深理解的读者:我想这本讲义是非常适合第二次更为深入学习理解的,当然第二次学习可以适当略过一些基础和细节内容,但本讲义中很多深入的讨论、独特的思路和有意义的联系一定对你第二次学习有所裨益;

    \item 在学习其他方面知识时回顾线性代数基础的读者:无论是基础已经遗忘很多或是还有一定印象的读者,本讲义都可以为您提供帮助,因为本书逻辑完整,并且从基础讲起,非常有助于简要回顾一些概念帮助后续研究学习其他内容,相比于一些教材填鸭式的讲解更适合于在简短的内容中迅速把握住重点;

    \item 复习考试的读者:如前述本书特色所说,本讲义有大量的通过分点列举总结的内容,每节最后也有比较完整的内容总结和逻辑梳理,我们也准备了足量的例题和习题供读者参考. 但因为本书是从最基础的讲起,并且非常重视一些细节,因此复习时读者可以略过一些过于基础和细节的内容,也可以选择性参考讲义中给出的证明等,习题也可以优先选择难度适中的,因为考试中不会出现很难的题目.
\end{enumerate}

\section*{例题与习题}

笔者坚信,没有适量习题练习是很难在初学时较为清晰地掌握线性代数这些抽象的思路以及运算技巧的,因此本讲义提供了足量的例题与习题便于读者及时巩固学习的概念,并掌握一些常用的技巧,拓展一些实用的结论,同时一些例题和习题也是讲义完整逻辑中不可或缺的一环.

讲义中的例题有部分是直接概念性的,因为考虑到有一些概念初学时太过抽象,或者一些公式较为复杂,需要及时联系以熟悉使用. 还有一些例题是非常经典的问题,其中的思想在很多其它习题中都会使用到. 基本上在每个重要概念/定理/方法介绍后笔者都会准备合适的例子,并且都会直接在题目后给出答案.

讲义的习题均设置了A、B、C三组,从低至高区分了难度,读者可以根据自己的实际需求选择合适难度的习题进行巩固提升. 所有的习题都在习题答案分册中提供了解答或思路(教材习题可能直接引用),因此读者在思考中遇到困难时可以参考其中的思路,但我们并不推荐直接参考答案将所有习题粗略过一遍,这样的学习效果十分有限.

\section*{授课建议}

非常欢迎在本讲义的基础上节选或改编出适合辅学授课的讲义,但请注意以下几点:
\begin{enumerate}
    \item 请遵循\href{https://creativecommons.org/licenses/by-nc-sa/4.0/deed.zh}{知识共享署名-非商业性使用-相同方式共享 4.0 国际许可协议};

    \item 本讲义由于笔者本人风格以及目的所在,因此内容较为细致,在授课时您应当有选择性地节选内容在课堂上讲授,一些细节性的内容可以在课后让学生自行阅读,否则在有限时间内很难讲授较为完整的体系;

    \item 同理,在授课过程中您可以选取对您的授课思路有帮助的经典例题或习题进行讲授. 授课中题在精不在多,您应当根据自己的授课风格和时间安排合理地选择题目,以有助于学生理解以及掌握基本方法为宗旨.
\end{enumerate}

\section*{最后的话}

我们十分清楚,现在阅读这段话的你可能从小到大都对数学缺乏兴趣,也可能在未来与数学之间不会再有很多的交集. 我想很多同学都是经过填鸭式的应试教育而来,如果并非生来热爱,那种传统的教学方式只能是不断地毁灭式打击学生的数学学习兴趣.

在序言中笔者也提到,这本讲义希望还原数学本原的自然之美,因此从引入到推进到引申,特别是``未竟之美''部分,我们都尽可能地从自然的角度出发,然后不断深入,让读者能够看到人类目前研究的模糊边界,能够看到数学的无限魅力. 或许从小我们便接受过教育,说数学或许不能帮你买菜,但其中的``思维方式''才是最重要的. 我想,通过本讲义由浅及深的自然推进,读者大概可以体会当年无数数学家在探索数学本原时的或许``灵光一现''又或许``站在巨人肩膀''背后的思维方式. 更重要的是,这就是追求真理的过程,是一代代数学家用自己有限的生命逼近宇宙无穷,通向崇高理念世界的过程. 我想读者无论是在学习哪个专业,这都是非常重要的精神品质.

尽管我们在编写过程中尽可能地考虑到了读者的阅读体验,但我们也不可能做到面面俱到,如果内容编排上有什么不合理的地方,或者有什么地方不够清晰,欢迎您将您的阅读体验通过邮件或直接在本讲义所在的 \href{https://github.com/yhwu-is/Linear-Algebra-Left-Undone}{GitHub仓库}提交Issue. 如果您希望加入我们的编写团队,将这一讲义传承下去,也欢迎您通过GitHub仓库提交Pull Request.

愿诸君热爱数学,热爱对真理的追求.

\begin{flushright}
    \kaishu
    吴一航 \\
    浙江大学计算机科学与技术学院 \\
    \verb|yhwu_is@zju.edu.cn| \\
    2023 年 8 月
\end{flushright}


\clearpage
\pdfbookmark[0]{目录}{contents}
\tableofcontents

\addtolength{\parskip}{.5em}

\mainmatter
\setcounter{page}{1} % 将页码计数设置为 1
\section*{1 预备知识}
\addcontentsline{toc}{section}{1 预备知识}

\vspace{2ex}

\centerline{\heiti A组}
\begin{enumerate}
    \item
    \item
    \item
        \begin{align*}
            A=&
            \begin{pmatrix}
                1 & 1  & 1 & 4  & -3 \\
                2 & 1  & 3 & 5  & -5 \\
                1 & -1 & 3 & -2 & -1 \\
                3 & 1  & 5 & 6  & -7
            \end{pmatrix}
            \rightarrow
            \begin{pmatrix}
                1 & 1  & 1 & 4  & -3 \\
                0 & -1 & 1 & -3 & 1  \\
                0 & -2 & 2 & -6 & 2  \\
                0 & -2 & 2 & -6 & 2
            \end{pmatrix}
            \rightarrow \\
             &
            \begin{pmatrix}
                1 & 1 & 1  & 4 & -3 \\
                0 & 1 & -1 & 3 & -1 \\
                0 & 0 & 0  & 0 & 0  \\
                0 & 0 & 0  & 0 & 0
            \end{pmatrix}
            \rightarrow
            \begin{pmatrix}
                1 & 0 & 2  & 1 & -2 \\
                0 & 1 & -1 & 3 & -1 \\
                0 & 0 & 0  & 0 & 0  \\
                0 & 0 & 0  & 0 & 0
            \end{pmatrix}.
        \end{align*}
        令 $ x_3 = k_1, x_4 = k_2, x_5 = k_3$,有 $x_1 = -2k_1 - k_2 + 2k_3$,$x_2 = k_1 + -3k_2 + k_3 $ ,
        则
        \begin{equation*}
            X = (x_1, x_2, x_3, x_4, x_5)^\mathrm{T} = k_1 \begin{pmatrix} -2 \\ 1 \\ 1 \\ 0 \\ 0 \end{pmatrix}
            + k_2 \begin{pmatrix} -1 \\ -3 \\ 0 \\ 1 \\ 0 \end{pmatrix}
            + k_3 \begin{pmatrix} 2 \\ 1 \\ 0 \\ 0 \\ 1 \end{pmatrix}
            \quad(k_1, k_2, k_3 \in \mathbf{R})
        \end{equation*}
    \item
        \begin{align*}
            \bar{A}=&
            \begin{pmatrix}
                1 & -1 & 2 & -2  & 3  & 1 \\
                2 & -1 & 5 & -9  & 8  & -1 \\
                3 & -2 & 7 & -11 & 11 & 0 \\
                1 & -1 & 1 & -1  & 3  & 3
            \end{pmatrix}
            \rightarrow
            \begin{pmatrix}
                1 & -1 & 2  & -2  & 3 & 1 \\
                0 & 1  & 1  & -5  & 2 & -3 \\
                0 & 1  & 1  & -5  & 2 & -3 \\
                0 & 0  & -1 &  1  & 0 & 2
            \end{pmatrix}
            \rightarrow \\
             &
            \begin{pmatrix}
                1 & -1 & 2  & -2  & 3 & 1 \\
                0 & 1  & 1  & -5  & 2 & -3 \\
                0 & 0  & -1 &  1  & 0 & 2 \\
                0 & 0  & 0  &  0  & 0 & 0
            \end{pmatrix}
            \rightarrow
            \begin{pmatrix}
                1 & 0 & 0 & -4 & 5 & 4 \\
                0 & 1 & 0 & -4 & 2 & 1 \\
                0 & 0 & 1 & -1 & 0 & -2 \\
                0 & 0 & 0 &  0 & 0 & 0
            \end{pmatrix}.
        \end{align*}
        令 $ x_4 = k_1, x_5 = k_2, x_6 = k_3$,有 $x_1 = 4k_1 - 5k_2 - 4k_3$,$x_2 = 4k_1 - 2k_2 + k_3, x_3 = k_1 + 2k_3 $ ,
        则
        \begin{equation*}
            X = (x_1, x_2, x_3, x_4, x_5, x_6)^\mathrm{T} = k_1 \begin{pmatrix} 4 \\ 4 \\ 1 \\ 1 \\ 0 \\ 0 \end{pmatrix}
            + k_2 \begin{pmatrix} -5 \\ -2 \\ 0 \\ 0 \\ 1 \\ 0 \end{pmatrix}
            + k_3 \begin{pmatrix} -4 \\ 1 \\ 2 \\ 0 \\ 0 \\ 1 \end{pmatrix}
            \quad(k_1, k_2, k_3 \in \mathbf{R})
        \end{equation*}
    \item 见教材P33例3. 无解.
\end{enumerate}

\centerline{\heiti B组}
\begin{enumerate}
    \item
\end{enumerate}

\centerline{\heiti C组}
\begin{enumerate}
    \item
\end{enumerate}

\clearpage

\chapter*{未竟专题一\  \ 预备思想}
\addcontentsline{toc}{chapter}{未竟专题一\ \ 预备思想}

在第一讲中我们介绍了一些预备知识,但是在正式开始我们的学习旅程前,我希望在这里先讨论一些和知识本身关系不大的话题,也就是一些学习这门课的一些数学思想的准备,目的主要是给刚刚进入大学的同学一个思维上升的台阶,以便更好地接受接下来抽象的内容.

\section*{如何书写数学证明}
\addcontentsline{toc}{section}{如何书写数学证明}

\section*{代数结构的引入}
\addcontentsline{toc}{section}{代数结构的引入}

在第一讲中,

\section*{公理化思想与布尔巴基学派}
\addcontentsline{toc}{section}{公理化思想与布尔巴基学派}

\subsection*{公理化思想}

\subsection*{布尔巴基学派}

\vspace{2ex}
\centerline{\heiti \Large 内容总结}


\vspace{2ex}
\centerline{\heiti \Large 习题}

\vspace{2ex}
{\kaishu 教育不是灌输,而是点燃火焰。}
\begin{flushright}
    \kaishu
    ——苏格拉底
\end{flushright}

\centerline{\heiti A组}
\begin{enumerate}
    \item
\end{enumerate}

\centerline{\heiti B组}
\begin{enumerate}
    \item
\end{enumerate}

\centerline{\heiti C组}
\begin{enumerate}
    \item
\end{enumerate}

\chapter{线性空间}

本讲我们将开始回答第 1 讲最后留下的问题,即线性方程组有唯一解、无穷解或无解的本质原因. 这段旅程或许有些漫长,中间会有很多的铺垫,我们将从其中最为基础的概念——线性空间出发进行探讨.

回忆高斯-若当消元法,方程组中每一行或一列都可以视为向量. 我们可以先看下面这个例子:
\begin{example}{}{线性空间引入}
    考虑如下两个方程组
    \begin{multicols}{2}
        \begin{enumerate}
            \item $\begin{cases}
                          x_1+x_2+x_3=0   \\
                          x_1+2x_2+3x_3=0 \\
                          2x_1+3x_2+4x_3=0
                      \end{cases}$

            \item $\begin{cases}
                          x_1+x_2+x_3=0   \\
                          x_1+2x_2+3x_3=0 \\
                          x_1+3x_2+4x_3=0
                      \end{cases}$
        \end{enumerate}
    \end{multicols}
    不难解得,第一个方程组有无穷解,第二个方程组有唯一解. 从高斯-若当消元法的过程来看,第一个方程组的简化阶梯矩阵出现了全零行,其原因是显而易见的:因为方程组第一行和第二行相加正好是第三行,因此可以直接消去第三行,即三行的系数矩阵的三个行向量
    \[\alpha_1=(1,1,1),\enspace\alpha_2=(1,2,3),\enspace\alpha_3=(2,3,4)\]
    满足$\alpha_1+\alpha_2=\alpha_3$. 而第二个方程组系数矩阵行向量间没有类似的可互相消去的关系.
\end{example}

从上面这一例子中我们可以看出,方程组的解与系数矩阵的行向量之间的关系密切相关. 因此我们会有一个很自然的想法,即我们需要研究向量之间的关联. 受第1讲基本代数结构的启发,我们应当自然地想到我们需要引入一个代数结构,从而使得我们可以统一地研究向量间的关联,这一代数结构便是线性空间.

\section{线性空间的定义}

\term{线性空间}\index{xianxingkongjian@线性空间 (linear space)}是我们接触的第一个核心概念,作为一种代数结构,它需要在非空集合$V$上定义运算. 我们写下其定义,然后给出解读:
\begin{definition}{线性空间}{}
    设$V$是一个非空集合,$\mathbf{F}$是一个数域,我们定义两种运算,其中第一个运算是我们熟知的加法$+$. 在线性空间的定义中,我们要求$\langle V\colon+\rangle$构成Abel群,即其中元素满足如下运算律:
    \begin{enumerate}
        \item (加法结合律) $\alpha+(\beta+\gamma)=(\alpha+\beta)+\gamma,\enspace\forall \alpha,\beta,\gamma \in V$;

        \item (加法单位元) $\exists 0 \in V$使得$\forall\alpha\in V$ 有 $\alpha + 0 = 0 + \alpha = \alpha$;

        \item (逆元) $\forall\alpha\in V,\enspace \exists \beta \in V$,有$\alpha+\beta=\beta+\alpha=0$,记$\beta=-\alpha$;

        \item (交换律) $\forall\alpha, \beta\in V,\enspace \alpha+\beta=\beta+\alpha$.
    \end{enumerate}

    第二种运算和之前学习的其他代数结构不同,我们需要首先引入一个数域$\mathbf{F}$,接下来在$\mathbf{F}\times V$上定义取值于$V$的数乘运算,即$\mathbf{F}\times V$中的每个元素$(\lambda,\alpha)\mapsto \lambda\alpha\in V$,并且数乘运算满足以下性质:$\forall \alpha,\beta \in V,\enspace\forall \lambda,\mu\in\mathbf{F}$以及$\mathbf{F}$上的乘法单位元1,有
    \begin{enumerate}
        \item (数乘单位元) $1\cdot \alpha=\alpha$;

        \item (数乘结合律) $\lambda(\mu\alpha)=(\lambda\mu)\alpha$;

        \item (左分配律) $(\lambda+\mu)\alpha=\lambda\alpha+\mu\alpha$;

        \item (右分配律) $\lambda(\alpha+\beta)=\lambda\alpha+\lambda\beta$.
    \end{enumerate}

    基于此,我们完整定义了一个线性空间,我们一般称集合$V$关于上述两种运算在域$\mathbf{F}$上构成一个线性空间,简称为$V$在域$\mathbf{F}$上的线性空间,记作$V(\mathbf{F})$. 如果$\mathbf{F}$是实(复)数域,则称$V$为实(复)数域上的线性空间.
\end{definition}
关于线性空间的定义,我们还有如下说明:
\begin{enumerate}
    \item 线性空间还有一个重要的概念是运算封闭,即线性空间中的元素进行加法或数乘运算后,得到的元素仍然是属于线性空间的. 这一点是定义要求的,加法封闭是 Abel 群的要求,因为 Abel 群要求加法运算定义为映射 $V\times V\to V$,因此$V$中两个元素相加后必须仍在$V$中(事实上这是代数系统的共性),数乘注意前述定义中数乘运算``取值于$V$''的要求,即它是 $\mathbf{F}\times V\to V$ 的映射;

    \item 在正文中如没有特殊说明,我们都默认 $\mathbf{F}$ 是数域(特征 $0$ 的域,如 $\mathbf{R}$ 和 $\mathbf{C}$),有限域上的情况我们会放在\nameref{chap:有限域矩阵}中进行讨论.

    \item 特别注意线性空间定义在非空集合上,事实上根据加法构成Abel群的要求,最小的线性空间也必须至少包含加法单位元(可以记为$V=\{\vec{0}\}$). 需要注意的一点是,这里为了区分$V$中的零元和数域中的数0,我们将$V$中零元加粗,此后熟悉记号之后我们可能不再作区分,请读者务必仔细理解区分.

    \item 注意数乘结合律并不是平凡的,事实上 $\lambda(\mu\alpha)$ 是先对 $\alpha$ 数乘 $\mu$,然后数乘 $\lambda$,而 $(\lambda\mu)\alpha$ 是先让 $\lambda$ 和 $\mu$ 做了域 $\mathbf{F}$ 上的乘法,然后再数乘 $\alpha$,这两者的计算过程显然是不同的. 左右分配律也可以做这样的分解,左分配律中等号左侧的加法是 $\mathbf{F}$ 上的加法,右侧的加法是 $V$ 上的加法,右分配律等号两侧的加法都是 $V$ 上的加法. 总而言之,区分清楚这里的加法和数乘在运算定义不再是普通数的加法、数乘的时候非常关键,之后我们会看到这样的例子.

    \item 结合我们上一讲对公理化的研究,事实上我们到目前为止也只定义了上面的加法、数乘运算和几条规则,我们需要忘记其他任何规则,由此出发进行推导出一些看似显然但公理没有直接给出的重要运算性质:
          \begin{enumerate}
              \item 由于加法运算构成Abel群,因此加法零元和逆元是唯一的,从而我们可以定义减法运算为加上一个元素的逆,即$\alpha-\beta=\alpha+(-\beta)$;

              \item 事实上,根据公理中的性质,我们可以逐步得到$\lambda(\alpha-\beta)+\lambda\beta=\lambda((\alpha-\beta)+\beta)=\lambda((\alpha+(-\beta))+\beta)=\lambda(\alpha+((-\beta)+\beta))=\lambda(\alpha+\vec{0})=\lambda\alpha$,两边分别加$-(\lambda\beta)$即可以得到
                    \begin{equation}\label{eq:2:线性空间运算性质1}
                        \lambda(\alpha-\beta)=\lambda\alpha-\lambda\beta.
                    \end{equation}
                    上面推导过程中第一个等号来源于数乘分配律,第二个等号来源于减法的定义(加上逆元),第三个等号来源于加法结合律,第四个等号来源于逆元的定义(加起来等于向量加法零元$\vec{0}$),最后一个等号来源于加法单位元的定义. 事实上这一过程是非常清晰的.

                    除此之外,$(\lambda-\mu)\alpha+\mu\alpha=(\lambda-\mu+\mu)\alpha=\lambda\alpha$,两边分别加$-(\mu\alpha)$即可以得到
                    \begin{equation}\label{eq:2:线性空间运算性质2}
                        (\lambda-\mu)\alpha=\lambda\alpha-\mu\alpha.
                    \end{equation}
                    事实上,\autoref{eq:2:线性空间运算性质1} 和\autoref{eq:2:线性空间运算性质2} 可以视为数乘运算对减法也满足分配律(但我们必须时刻牢记在心,数的减法是常规的,向量的减法是加上向量的逆元).

              \item 在\autoref{eq:2:线性空间运算性质1} 中分别令$\alpha=\beta$和$\alpha=\vec{0}$,在\autoref{eq:2:线性空间运算性质2} 分别令$\lambda=\mu$和$\lambda=0$有如下四条性质:
                    \begin{enumerate}
                        \item $\lambda\cdot \vec{0}=\vec{0}$;

                        \item $\lambda(-\beta)=-(\lambda\beta)$;

                        \item $0\cdot \alpha=\vec{0}$;

                        \item $(-\mu)\alpha=-(\mu\alpha)$.
                    \end{enumerate}
                    我们详细证明前两条如何根据公理一步步推导得到,后两条请读者依照此自行证明.
                    \begin{proof}
                        \begin{enumerate}
                            \item 在\autoref{eq:2:线性空间运算性质1} 中令$\alpha=\beta$,则$\lambda(\alpha-\alpha)=\lambda\alpha-\lambda\alpha$,根据减法定义有$\alpha-\alpha=\alpha+(-\alpha)=\vec{0}$,且$\lambda\alpha-\lambda\alpha=\lambda\alpha+(-(\lambda\alpha))=\vec{0}$,因此$\lambda\cdot \vec{0}=\vec{0}$.

                            \item 在\autoref{eq:2:线性空间运算性质1} 中令$\alpha=\vec{0}$有$\lambda(\vec{0}-\beta)=\lambda\vec{0}-\lambda\beta$,根据减法定义有$\vec{0}-\beta=\vec{0}+(-\beta)=-\beta$(第二个等号来源于加法单位元性质),且$\lambda\vec{0}-\lambda\beta=\vec{0}-\lambda\beta=\vec{0}+(-(\lambda\beta))=-(\lambda\beta)$(第一个等号来源于刚刚证明的$\lambda\cdot \vec{0}=\vec{0}$,第二个等号来源于减法的定义,第三个等号来源于加法单位元性质),因此$\lambda(-\beta)=-(\lambda\beta)$.
                        \end{enumerate}
                    \end{proof}
                    特别地,当$\mu=1$时有$(-1)\alpha=-\alpha$. 即$-1$数乘一个元素可以得到该元素的逆元(虽然代入一般平面向量这一点非常显然,但是我们只能基于公理一步步推导得到这一显然的性质).

              \item 若$\lambda\alpha=\vec{0}$,则$\lambda=0$或$\alpha=\vec{0}$,这一点也是显然的,因为如果$\lambda\neq 0$,则$\lambda^{-1}$存在,从而$\alpha=1\alpha=(\lambda^{-1}\lambda)\alpha=\lambda^{-1}(\lambda\alpha)=\lambda^{-1}\vec{0}=\vec{0}$(这里的每一个等号都是能找到对应的,请读者自行判断).

                    最后,综合上述性质我们有方程$\lambda\beta+\lambda_1\alpha_1+\lambda_2\alpha_2+\cdots+\lambda_r\alpha_r=\vec{0}$在$\lambda\neq 0$时的解为$\beta=-\lambda^{-1}\lambda_1\alpha_1-\lambda^{-1}\lambda_2\alpha_2-\cdots-\lambda^{-1}\lambda_r\alpha_r$. 我们放在习题中供读者练习.
          \end{enumerate}
\end{enumerate}

或许读者会疑惑为什么线性空间会要求上述这 $8$ 条性质(加法、数乘各 $4$ 条). 首先需要强调的是技术性层面的结果,这里的加法交换律是可以被其他 $7$ 条推出的,然后注意到逆元的定义依赖于单位元,因此去除掉加法交换律以及单位元存在后其余的 $6$ 条彼此独立,因为我们可以找到这样的例子使得一个集合满足除了其中任意一条公理之外的其它所有公理. 下面我们首先给出加法交换律可以被其它定律推出的证明,然后在 B 组习题\ref{item:2:去除一条线性空间公理}中给出的一些集合满足除了其中任意一条公理之外的其它所有公理例子供读者验证.

\begin{proof}
{
    \allowdisplaybreaks
    设 $\alpha,\beta\in V$, 则有
    \begin{align*}
        \tag{加法单位元}
        \alpha + \beta &= (0 + \alpha) + (\beta + 0)\\
        \tag{逆元}
        &= ((-\alpha + \alpha) + \alpha) + (\beta + (\beta + (-\beta)))\\
        \tag{加法结合律}
        &= -\alpha + ((\alpha + \alpha) + (\beta + \beta)) + (-\beta)\\
        \tag{乘法单位元}
        &= -\alpha + ((1\cdot\alpha + 1\cdot\alpha) + (1\cdot\beta + 1\cdot\beta)) + (-\beta)\\
        \tag{左分配律}
        &= -\alpha + (2\cdot\alpha + 2\cdot\beta) + (-\beta)\\
        \tag{右分配律}
        &= -\alpha + 2\cdot (\alpha + \beta) + (-\beta)\\
        \tag{左分配律}
        &= -\alpha + ((\alpha + \beta) + (\alpha + \beta)) + (-\beta)\\
        \tag{加法结合律}
        &= ((-\alpha + \alpha) + \beta) + (\alpha + (\beta + (-\beta)))\\
        \tag{逆元}
        &= (0 + \beta) + (\alpha + 0)\\
        \tag{加法单位元}
        &= \beta + \alpha
    \end{align*}
}
\end{proof}

接下来我们讨论这些公理蕴含的想法. 不难发现线性空间中定义的运算规则与我们高中学习的平面向量的加法和数乘是非常类似的,我们回顾未竟专题一关于公理化的讨论,实际上这些公理就可以视为从简单的向量加法和数乘抽象出来的一些规则. 而公理的诞生应当是要尽可能简洁——线性空间公理的独立性表明了这一点,而且有足够的表达力——这一点我们将来基于这一定义不断推出线性空间的性质时就会发现非常足够,事实上你现在就能通过我们上面证明的运算性质初步感知到这一点. 因此皮亚诺在 1888 年正式给出这一定义并沿用至今. 但我们需要知道他的工作也是基于前人(如格拉斯曼)的工作不断修正而来的,只是当这些公理被直接摆在我们眼前时,这一自然的不断推进的公理化过程变得很突兀. 回到现实问题,我们需要熟知、记忆这 $8$ 条性质,当然其实难度并不大,Abel 群 $4$ 条性质都有名称标注,数乘运算也是易于记忆的结合律和分配律加单位元性质. 当然,有的同学可能会怀疑,既然加法交换律可以被推出,那么为什么还要写出来呢?我们姑且当作是历史原因——阿贝尔群是一个如此自然美观的结构,为什么不使用呢?当然也有一些深层次的解释这里不便给出.

最后,公理化定义的一个重要作用是使得我们可以不仅仅在向量集合的背景下定义线性空间,这使得我们可以将对于很多结构的研究都转化为对于线性空间的研究. 在下一节中我们将给出一些经典的线性空间的例子,它们分别代表着线性空间在某些方面的抽象,希望读者仔细体会.

\section{线性空间的例子}

我们从最平凡的例子开始,即向量空间 $\mathbf{F}^n$ 和与之有着显然相关性的多项式空间 $\mathbf{F}[x]_{n+1}$.
\begin{example}{平凡的常识}{}
    \begin{enumerate}
        \item $\mathbf{F}^n = \{(a_1,a_2,\cdots,a_n)^\mathrm{T} \mid a_i \in \mathbf{F}\}$ 关于向量的加法和数乘构成线性空间,具体展开而言便是 $\forall (a_1,a_2,\cdots,a_n)^\mathrm{T},(b_1,b_2,\cdots,b_n)^\mathrm{T} \in \mathbf{F}^n, \enspace \forall \lambda \in \mathbf{F}$,有
        \begin{gather*}
            (a_1,a_2,\cdots,a_n)^\mathrm{T} + (b_1,b_2,\cdots,b_n)^\mathrm{T} = (a_1+b_1,a_2+b_2,\cdots,a_n+b_n)^\mathrm{T},\\
            \lambda(a_1,a_2,\cdots,a_n)^\mathrm{T} = (\lambda a_1,\lambda a_2,\cdots,\lambda a_n)^\mathrm{T}.
        \end{gather*}
        \item $\mathbf{F}[x]_{n+1}=\{a_0+a_1x+\cdots+a_nx^n \mid a_i\in\mathbf{F}\}$关于多项式的加法和数乘构成线性空间,具体展开而言便是
        \[
            (p_1+p_2)(x)=p_1(x)+p_2(x),\enspace(\lambda p)(x)=\lambda p(x),\enspace\forall p_1,p_2,p\in\mathbf{F}[x]_{n+1},\enspace\forall \lambda\in\mathbf{F}.
        \]
        这也能解释常见记号的含义:$(k_1p_1+k_2p_2)(x)=k_1p_1(x)+k_2p_2(x)$.
        但我们需要注意 $\mathbf{F}[x]'_{n+1}=\{a_0+a_1x+\cdots+a_nx^n \mid a_i\in\mathbf{F}, a_n\neq 0\}$ 不构成线性空间.
    \end{enumerate}
\end{example}

\begin{proof}
    $\mathbf{F}^n$ 的例子过于平凡,不再验证. 我们针对多项式的例子逐条验证公理.
    \begin{enumerate}
        \item $\forall p_1(x), p_2(x), p_3(x) \in \mathbf{F}[x]_{n+1}=\{a_0+a_1x+\cdots+a_nx^n \mid a_i\in\mathbf{F}\}$,有
              \begin{align*}
                      & (p_1(x) + p_2(x)) + p_3(x)                                                                                \\
                  ={} & ((a_{10} + a_{11}x + \cdots  + a_{1n}x^n) + (a_{20} + a_{21}x + \cdots  + a_{2n}x^n))                     \\
                  +{} & (a_{30} + a_{31}x + \cdots  + a_{3n}x^n)                                                                  \\
                  ={} & ((a_{10} + a_{20}) + (a_{11} + a_{21}) x + \cdots  + (a_{1n} + a_{2n}) x^n)                               \\
                  +{} & (a_{30} + a_{31}x + \cdots  + a_{3n}x^n)                                                                  \\
                  ={} & (((a_{10} + a_{20}) + a_{30}) + ((a_{11} + a_{21}) + a_{31})x + \cdots + ((a_{1n} + a_{2n}) + a_{3n})x^n) \\
                  ={} & ((a_{10} + (a_{20} + a_{30})) + (a_{11} + (a_{21} + a_{31}))x + \cdots + (a_{1n} + (a_{2n} + a_{3n}))x^n) \\
                  ={} & (a_{10} + a_{11}x + \cdots  + a_{1n}x^n)                                                                  \\
                  +{} & ((a_{20} + a_{21}x + \cdots  + a_{2n}x^n) + (a_{30} + a_{31}x + \cdots  + a_{3n}x^n))                     \\
                  ={} & p_1(x) + (p_2(x) + p_3(x))
              \end{align*}
              注意,在证明过程中,我们用了形式的加法定义(逐次数将系数相加),并诉诸域 $\mathbf{F}$ 上的结合律,这种诉诸基域性质的方式在以后的证明中会经常碰上.

        \item 取定 $p_0(x) = 0 \in V$ 则有 $\forall p(x) \in \mathbf{F}[x]_{n+1}, p(x) + p_0(x) = p_0(x) + p(x)$.

        \item $\forall p(x) = a_0 + a_1x + \cdots + a_nx^n \in \mathbf{F}[x]_{n+1}, \exists p^*(x) = -a_0 - a_1x - \cdots - a_nx^n \in \mathbf{F}[x]_{n+1}, p(x) + p^*(x) = p^*(x) + p(x) = p_0(x) = 0$.

        \item $\forall p_1(x), p_2(x) \in \mathbf{F}[x]_{n+1}$有
              \begin{align*}
                  p_1(x) + p_2(x)
                   & = (a_{10} + a_{11}x + \cdots + a_{1n}x^n) + (a_{20} + a_{21}x + \cdots + a_{2n}x^n) \\
                   & = (a_{10} + a_{20}) + (a_{11} + a_{21})x + \cdots + (a_{1n} + a_{2n})x^n            \\
                   & = (a_{20} + a_{10}) + (a_{21} + a_{11})x + \cdots + (a_{2n} + a_{1n})x^n            \\
                   & = (a_{20} + a_{21}x + \cdots + a_{2n}x^n) + (a_{10} + a_{11}x + \cdots + a_{1n}x^n) \\
                   & = p_2(x) + p_1(x).
              \end{align*}

        \item 取定 $\lambda = 1 \in \mathbf{F},\forall p(x) \in \mathbf{F}[x]_{n+1}, \lambda \cdot p(x) = p(x)$.

        \item $\forall \lambda, \mu \in \mathbf{F}, p(x) \in \mathbf{F}[x]_{n+1}$有
              \begin{align*}
                  \lambda(\mu p(x)) & = \lambda(\mu(a_0 + a_1x + \cdots + a_nx^n)) = \lambda(\mu a_0 + \mu a_1x + \cdots + \mu a_nx^n)                  \\
                                    & = \lambda \mu a_0 + \lambda \mu a_1x + \cdots + \lambda \mu a_nx^n = (\lambda \mu) (a_0 + a_1x + \cdots + a_nx^n) \\
                                    & = (\lambda \mu)p(x).
              \end{align*}

        \item $\forall \lambda, \mu \in \mathbf{F}, p(x) \in \mathbf{F}[x]_{n+1}$有
              \begin{align*}
                  (\lambda + \mu) p(x)
                   & = (\lambda + \mu)(a_0 + a_1x + \cdots + a_nx^n)                                         \\
                   & = (\lambda + \mu)a_0 + (\lambda + \mu)a_1x + \cdots + (\lambda + \mu)a_nx^n             \\
                   & = \lambda a_0 + \mu a_0 + \lambda a_1x + \mu a_1x+ \cdots + \lambda a_nx^n + \mu a_nx^n \\
                   & = \lambda(a_0 + a_1x + \cdots + a_nx^n) + \mu(a_0 + a_1x + \cdots + a_nx^n)             \\
                   & = \lambda p(x) + \mu p(x).
              \end{align*}
              这里的第二行到第三行并没有诉诸对单项式的分配律,而是利用了性质 6 和域 $\mathbf{F}$ 上的分配律.

        \item $\forall p_1(x), p_2(x) \in \mathbf{F}[x]_{n+1}, \lambda \in \mathbf{F}$有
              \begin{align*}
                      & \lambda(p_1(x) + p_2(x))                                                                        \\
                  ={} & \lambda((a_{10} + a_{11}x + \cdots + a_{1n}x^n) + (a_{20} + a_{21}x + \cdots + a_{2n}x^n))      \\
                  ={} & \lambda((a_{10} + a_{20}) + (a_{11} + a_{21})x + \cdots + (a_{1n} + a_{2n})x^n)                 \\
                  ={} & \lambda(a_{10} + a_{20}) + \lambda(a_{11} + a_{21})x + \cdots + \lambda(a_{1n} + a_{2n})x^n     \\
                  ={} & \lambda(a_{10} + a_{11}x + \cdots + a_{1n}x^n) + \lambda(a_{20} + a_{21}x + \cdots + a_{2n}x^n) \\
                  ={} & \lambda p_1(x) + \lambda p_2(x).
              \end{align*}
    \end{enumerate}
    但是对\[\mathbf{F}[x]'_{n+1}=\{a_0+a_1x+\cdots+a_nx^n \mid a_i\in\mathbf{F}, a_n\neq 0\}\]不构成线性空间,其原因在于我们加法不封闭,例如我们取$\mathbf{F}[x]'_{n+1}$中的两个元素$x^n$和$-x^n$,它们的和为$0$,不再满足$\mathbf{F}[x]'_{n+1}$中关于$a_n\neq 0$的条件,因此运算不封闭,不构成线性空间.
\end{proof}

为什么说多项式空间与向量有很显然的联系,是因为我们可以把次数不高于 $n-1$ 次的多项式的所有系数 $a_i(i=1,\ldots,n-1)$ 拼成向量 $\alpha=(a_1,\ldots,a_{n-1})$,因此多项式和向量实际上是很类似的,所以这一例子是平凡的,并且应当作为常识,因为日后会非常常见. 并且请特别注意不构成线性空间的例子,这里我们使用运算不封闭这一条件否认,这是非常常用的,在习题中我们还会见到这样的例子.

除此之外,这里有一个记号上的问题需要澄清,很多教材(包括本书)用 $\mathbf{F}[x]_{n+1}$ 表示次数不超过 $n$ 的多项式的集合,也有一些教材使用 $\mathcal{P}_n(\mathbf{F})$ 表示相同的集合. 其中核心的区别在于下标 $n$ 和多项式次数的关联,读者在面对不同的教材或者试题时应当看清楚相应的定义. 有时我们可能会看到没有下标的 $\mathbf{F}[x]$ 和 $\mathcal{P}(\mathbf{F})$,这两者分别表示全体任意次数多项式的集合.

下面我们将讨论``不平凡''的例子,所谓的不平凡的原因来源于线性空间定义的三个要素:集合 $V$、数域 $\mathbf{F}$ 和两种运算. 我们将逐一讨论这三个要素的变化对线性空间的影响. 首先是集合中元素的抽象,即线性空间定义中 $V$ 中的元素可以不是向量,也不是类似于上例中多项式的常规的、平凡的形式,可以是非常抽象的与我们认知中``向量''一词相去甚远的,例如:
\begin{example}{向量层面的抽象}{函数和数列线性空间}
    \begin{enumerate}
        \item 设$V=C[a,b]$为定义在闭区间$[a,b]$上的连续实值函数全体,定义$V(\mathbf{R})$,其中加法运算定义为
              \[(f+g)(x)=f(x)+g(x),\enspace\forall f,g\in C[a,b].\]
              数乘运算定义为
              \[(\lambda f)(x)=\lambda f(x),\enspace\forall f\in C[a,b],\enspace\forall \lambda\in\mathbf{R}.\]
              则$C[a,b]$构成实数域上的线性空间.

        \item 设$V$是以$0$为极限的实数数列全体,定义$V(\mathbf{R})$,其中两个数列的加法和数乘定义为
              \[\{a_n\}+\{b_n\}=\{a_n+b_n\},\enspace\lambda\{a_n\}=\{\lambda a_n\},\enspace\forall \{a_n\},\{b_n\}\in V,\enspace\forall \lambda\in\mathbf{R},\]
              则$V$构成实数域上的线性空间.
    \end{enumerate}
\end{example}

本例的证明我们不再像多项式的例子一样展开,因为方法类似. 我们只需要强调 $C[a,b]$ 中的零元就是 $f(x) = 0, \enspace \forall x \in [a,b]$ 的恒等于 $0$ 的函数,而极限为 $0$ 的数列中的零元就是每一项都为 $0$ 的全零数列.

接下来我们考虑数域上的抽象,即线性空间定义中的数域 $\mathbf{F}$ 可以很多变,并且会导致线性空间性质的不同:
\begin{example}{数域层面的抽象}{}
    复数集$\mathbf{C}$是数域$\mathbf{C}$或数域$\mathbf{R}$上的线性空间,实数集$\mathbf{R}$是实数域$\mathbf{R}$上的线性空间,但实数集$\mathbf{R}$不是复数域$\mathbf{C}$上的线性空间. 除此之外,全体实数$\mathbf{R}$定义在整数集$\mathbf{Z}$上不构成线性空间.

    总结而言,$\mathbf{C}(\mathbf{R})$和$\mathbf{C}(\mathbf{C})$以及$\mathbf{R(R)}$都构成线性空间,但$\mathbf{R}(\mathbf{C})$和$\mathbf{R}(\mathbf{Z})$不构成线性空间.
\end{example}

这一例子表明,同一集合可以在不同数域上构成不同的线性空间. 事实上,在下一讲接触维数的定义后,我们将知道构成线性空间的 $\mathbf{C}(\mathbf{R})$ 和 $\mathbf{C}(\mathbf{C})$ 二者的维数是不一样的(见\autoref{ex:不同数域的维数}). 当然,不同的集合也可以在同一个数域上构成不同的线性空间,例如$\mathbf{C(R)}$和$\mathbf{R(R)}$.

关于数域我们需要强调的并不多,只有一条非常关键的,那就是数域在线性空间$V(\mathbf{F})$中的作用只与数乘运算有关,加法运算与数域没有关系,只和集合$V$中的元素有关,这一点根据定义是显然的. 因此在验证线性空间的过程中,例如$\mathbf{C}(\mathbf{R})$和$\mathbf{C}(\mathbf{C})$之间的差别只在于数域不同,故加法性质是共同满足的,只需各自验证数乘的性质.

\begin{proof}
    这里我们验证$\mathbf{C}(\mathbf{R})$和$\mathbf{C}(\mathbf{C})$都构成线性空间,$\mathbf{R(R)}$证明类似. 我们应当对八条性质逐条验证,但我们在第一讲已经说明了全体复数构成一个域,因此$\mathbf{C}(\mathbf{C})$自动满足线性空间的所有条件,此处不再赘述. 除此之外,$\mathbf{C}(\mathbf{R})$的加法运算与实数无关(回顾线性空间定义,实数只用来参与数乘运算),因此加法Abel群事实上与$\mathbf{C}(\mathbf{C})$一致,都是群$\langle \mathbf{C}\colon+\rangle$,此处也不再验证. 因此这里只验证$\mathbf{C}(\mathbf{R})$数乘运算是否满足线性空间定义的要求:
    \begin{enumerate}
        \item 取定 $1 \in \mathbf{R}, \forall \alpha = a+b\i \in \mathbf{C},\enspace a, b \in \mathbf{R},\enspace 1 \cdot \alpha = 1 \cdot (a+b\i) = a+b\i = \alpha$.

        \item $\forall \lambda, \mu \in \mathbf{R},\enspace \alpha = a+b\i \in \mathbf{C},\enspace a, b \in \mathbf{R}$,
              \begin{align*}
                  \lambda(\mu \alpha) = \lambda(\mu (a+b\i)) = \lambda(\mu a+\mu b\i) = \lambda \mu a + \lambda \mu b\i = (\lambda \mu)(a+b\i) = (\lambda \mu)\alpha.
              \end{align*}

        \item $\forall \lambda, \mu \in \mathbf{R},\enspace \alpha = a+b\i \in \mathbf{C},\enspace a, b \in \mathbf{R}$,
              \begin{align*}
                  (\lambda + \mu) \alpha
                   & = (\lambda a + \lambda b\i) + (\mu a + \mu b\i)            \\
                   & = \lambda(a+b\i)+\mu(a+b\i) = \lambda \alpha + \mu \alpha.
              \end{align*}

        \item $\forall \lambda \in \mathbf{R},\enspace \alpha_1 = a_1+b_1\i, \alpha_2 = a_2+b_2\i \in \mathbf{C},\enspace a_i, b_i \in \mathbf{R},\enspace i = 1, 2$,
              \begin{align*}
                  \lambda(\alpha_1+\alpha_2)
                   & = \lambda((a_1+b_1\i)+(a_2+b_2\i)) = \lambda((a_1+a_2)+(b_1+b_2)\i)                             \\
                   & = \lambda(a_1+a_2)+\lambda(b_1+b_2)\i = (\lambda a_1+\lambda b_1\i)+(\lambda a_2+\lambda b_2\i) \\
                   & = \lambda(a_1+b_1\i)+\lambda(a_2+b_2\i) = \lambda \alpha_1 + \lambda \alpha_2.
              \end{align*}
    \end{enumerate}
    由此我们证明了 $\mathbf{C}(\mathbf{R})$ 构成线性空间.

    对于$\mathbf{R(R)}$的验证同理,下面考察$\mathbf{R}(\mathbf{C})$,它与$\mathbf{R(R)}$定义的集合一致,都是实数集合$\mathbf{R}$,因此加法性质共同满足,因此我们只需验证数乘运算不满足要求即可. 事实上仍然是运算不封闭导致的,回顾封闭的要求$\mathbf{F}\times V\to V$,现在$\mathbf{R(C)}$中的$V=\mathbf{R}$,数域$\mathbf{F}=\mathbf{C}$,我们取复数$i$与实数$1$进行数乘,结果为$i\notin\mathbf{R}$,破坏了封闭性,故$\mathbf{R(C)}$不构成线性空间.
\end{proof}

最后我们考察 $\mathbf{R(Z)}$,事实上我们会发现运算封闭以及与运算规律都是满足的,那为何不是线性空间呢?事实上这里开了个小玩笑,这是因为 $\mathbf{Z}$ 不是数域!我们知道最小的数域是有理数域 $\mathbf{Q}$,因而 $\mathbf{Z} \subsetneq \mathbf{Q}$ 不是数域,因此 $\mathbf{R(Z)}$ 不构成线性空间.

实际上抽象代数中有一个与线性空间类似的概念,但数乘运算可以使用环,不强制要求域,这称为环上的模,我们给出其定义:
\begin{definition}{模的定义}{}
    设 $R$ 是一个环,$M$ 是一个加法Abel群,若定义了数乘运算 $R \times M \to M$,满足
    \begin{enumerate}
        \item $\forall \alpha \in M, \enspace 1\alpha = \alpha$;
        \item $\forall r, s \in R, \forall \alpha \in M, \enspace r(s\alpha) = (rs)\alpha$;
        \item $\forall r, s \in R, \forall \alpha \in M, \enspace (r+s)\alpha = r\alpha + s\alpha$;
        \item $\forall r \in R, \forall \alpha, \beta \in M, \enspace r(\alpha + \beta) = r\alpha + r\beta$.
    \end{enumerate}
    则称 $M$ 是环 $R$ 上的左模,或左 $R$-模,简记为 $_R M$. 对称地,也可以定义右模,即上述定义环 $R$ 中的元素是乘在模 $M$ 元素右边的,记作 $M_R$. 如果有环 $R, S$,$M$ 是左 $R$-模,又是右 $S$-模,且
    \[
        \forall r \in R, s \in S, \alpha \in M, (r \alpha) s = r (\alpha s)
    \]
    则称它为 $(R, S)$-双模,记作 $_R M_S$.
\end{definition}

不难验证 $\mathbf{R}$ 是 $(\mathbf{Z},\mathbf{Z})$-双模.

最后,我们考虑运算的抽象,即线性空间定义中的加法和数乘可以不是我们熟悉的数之间的运算:
\begin{example}{运算层面的抽象}{运算与同构}
    $V$是正实数全体$\mathbf{R}^+$,定义$V(\mathbf{R})$上的加法和数乘为
    \begin{gather*}
        a\oplus b=ab \\
        \lambda\circ a=a^\lambda
    \end{gather*}
    则如上定义的$V(\mathbf{R})$是$\mathbf{R}$上的线性空间.
\end{example}

这一例子给出了一个非常不自然的运算定义,当然证明时仍然是按照运算封闭和八条性质逐条验证即可,我们留作下一讲的 B 组习题\ref{item:3:正实数线性空间}供读者验证,这里给出另一种基于同构的理解其构成线性空间的方式.

我们可以定义映射$\varphi\colon\mathbf{R}\to\mathbf{R}^+$,其中$\phi(x)=e^x$,则$\varphi$是一个双射(这是一个有反函数$\ln x$的可逆映射). 我们如何理解这个双射呢?实际上在两个集合之间建立双射就说明两个集合之间的元素具有一一对应的关系,例如每个$\mathbf{R}$中的元素$a$对应唯一的一个$\mathbf{R}^+$中的元素$e^a$,反之每个$\mathbf{R}^+$中的元素$b$对应唯一的一个$\mathbf{R}$中的元素$\ln b$. 例如$\mathbf{R}$中的元素$0$唯一对应于$\mathbf{R}^+$中的1,而$\mathbf{R}^+$中的1又反过来唯一与$\mathbf{R}$中的$0$对应,事实上两个集合中所有元素都可以像$0$和$1$这样两两凑对(例如$(1,e)$也能凑对,可以记为$(a,\varphi(a))$),这就是双射``一一对应''的意义.

除此之外,映射$\varphi$还有另一个重要的性质. 我们不难发现,
\begin{gather*}
    \varphi(x+y)=e^{x+y}=e^x\cdot e^y=\varphi(x)\oplus\varphi(y),\enspace\forall x,y\in\mathbf{R} \\
    \varphi(\lambda x)=e^{\lambda x}=(e^x)^\lambda=\lambda\circ\varphi(x),\enspace\forall x,\lambda\in\mathbf{R}
\end{gather*}

这两条性质将进一步加深``一一对应''的含义. 第一条关于加法的性质实际上说明了,在$\mathbf{R}$中任意两个元素$x$和$y$做$\mathbf{R}$中的加法的结果,与$x$和$y$``凑对''的元素$\varphi(x)$和$\varphi(y)$做$\mathbf{R}^+$中的加法的结果也是``凑对''的,也就是说,$\mathbf{R}$中两个元素做了加法,正好对应于$\mathbf{R}^+$中的两个元素也做了加法. 例如,$\mathbf{R}$中元素$2$和$3$做加法后结果为$5$,而$\mathbf{R}^+$中元素$e^2$和$e^3$做$\mathbf{R}^+$中加法后结果为$e^5$,所有参与加法的三个元素是完全对应的. 事实上数乘也是如此,对$\mathbf{R}$中元素的数乘运算也对应于$\mathbf{R}^+$中对应的元素做数乘运算. 具有这样保持两个代数结构运算的性质的映射$\varphi$我们称之为``同态'',特别地,若代数结构为线性空间,则称之为``线性映射''. 更进一步地,若映射$\varphi$是双射,则称之为``同构'',特别地,若代数结构为线性空间,则称之为``线性同构''(或者在线性代数的讨论场景下,不产生歧义时简称同构). 我们在此给出一个初步的定义:
\begin{definition}{}{同构}
    设$V(\mathbf{F})$和$W(\mathbf{F})$是数域$\mathbf{F}$上的线性空间,若映射$\varphi\colon V(\mathbf{F})\to W(\mathbf{F})$满足
    \begin{gather*}
        \varphi(\alpha+\beta)=\varphi(\alpha)+\varphi(\beta),\enspace\forall\alpha,\beta\in V(\mathbf{F}) \\
        \varphi(\lambda\alpha)=\lambda\varphi(\alpha),\enspace\forall\alpha\in V(\mathbf{F}),\enspace\forall\lambda\in\mathbf{F}
    \end{gather*}
    则称$\varphi$是从$V(\mathbf{F})$到$W(\mathbf{F})$的一个\textbf{同态},若$\varphi$是双射,则称$\varphi$是从$V(\mathbf{F})$到$W(\mathbf{F})$的一个\textbf{同构},若$\varphi$是从$V(\mathbf{F})$到$W(\mathbf{F})$的一个同构,则称线性空间$V(\mathbf{F})$和$W(\mathbf{F})$是\textbf{同构的},记为$V(\mathbf{F})\cong W(\mathbf{F})$.
\end{definition}

与之前的讨论对应的,我们需要注意\autoref{def:同构} 中,$\alpha+\beta$中的加法是线性空间$V$中的加法,而$\varphi(\alpha)+\varphi(\beta)$中的加法是线性空间$W$中的加法,数乘也是类似的. 这一点在定义中没有直接体现,但它是``同态''的本质,即保持运算,因此非常重要. 例如\autoref{ex:运算与同构} 中两个线性空间的加法数乘运算的定义差别很大,但我们可以构造出一个同构映射$\varphi$,使得$\varphi$保持了两个空间加法和数乘的完美对应.

我们将在后续章节中进一步讨论同构的性质,现在我们可以总结经过这一例子对于同构的认识:事实上同构就是两个线性空间之间元素有了一一对应的关系,并且其中一个线性空间的元素做了加法和数乘的运算,另一个线性空间对应的元素也做了加法和数乘运算(这两个运算分别是两个线性空间中对应的),因此这样两个线性空间事实上可以看成是``一模一样的'',因为我们可以集合上做元素的对应,然后它们运算也是有对应关系,这已经将线性空间``在集合上定义运算''的所有要素集齐,故可以说是``一模一样''的. 因此通过构建了$\varphi$这一映射,我们发现在这一奇特的运算定义下,$\mathbf{R}^+$事实上拥有了和$\mathbf{R}$一样的结构,所以它自然就是一个线性空间,本质上无需多余的验证. 当然,如果觉得上面这段话无法完全说服你,那在后续正式讨论同构之后我们再回过头看想必会有更深的理解.

总而言之,在上例以及习题中我们可以看到很多特殊的线性空间,它们集合中的元素不一定是数或向量,运算也不一定是熟知的数的运算和向量的数乘,对这些空间我们需要学会熟练判断,从而加深对``在集合上定义运算''的理解. 除此之外我们也简单引入了同构的概念遮蔽了两个线性空间之间运算的不同,因此即使运算变得面目全非,$\mathbf{R}^+$仍然构成了线性空间. 需要注意的是,同构是线性空间中非常重要的概念,我们将在后续详细讨论.

\section{线性子空间}

对于一个代数结构,一个很自然地问题是:这个代数结构的子集是否也具有相同的结构呢?因此我们将介绍线性子空间的定义:
\begin{definition}{线性子空间}{} \index{xianxingkongjian!zi@线性子空间 (linear subspace), 子空间 (subspace)}
    设$W$是线性空间$V(\mathbf{F})$的非空子集,如果$W$对$V$中的运算也构成域$\mathbf{F}$上的线性空间,则称$W$是$V$的\term{线性子空间}(简称\term{子空间}).
\end{definition}

请注意定义中的非空子集的要求,建议验证子空间时先验证非空. 接下来自然的问题便是,什么时候$V$的子集$W$对$V$中的运算也构成域$\mathbf{F}$上的线性空间?事实上这一条件是惊人地简单与美观的:
\begin{theorem}{}{子空间判别}
    数域 $\mathbf{F}$ 上的线性空间$V(\mathbf{F})$的非空子集$W$为$V$的子空间的充分必要条件是$W$对于$V(\mathbf{F})$的线性运算封闭.
\end{theorem}

这表明只要子空间非空且其中的元素满足对原空间的加法和数乘运算封闭即可构成原空间的子空间. 这一定理的证明也非常简单,必要性显然,因为子空间也是线性空间,而构成线性空间必须满足运算封闭,故不再赘述. 对于充分性,实际上要证明的是满足运算封闭的子集一定仍然能保持线性空间的八条性质. 我们只需要作如下思考:
\begin{enumerate}
    \item 结合律、分配律以及数乘运算律是一定满足的,例如我们回顾加法结合律的定义$a+(b+c)=(a+b)+c,\enspace\forall a,b,c\in V$,由于这一性质对于任意$V$中元素成立,则若$a,b,c\in W\subseteq V$也必有这一性质成立(更通俗而言就是子集$W$中的元素也是$V$中的,因此必然受$V$中运算性质的限制);

    \item 于是现在还剩下加法单位元和逆元仍不能保证存在,因为这不仅与运算法则相关,更与集合中元素的存在相关——取子集可能使得加法单位元和逆元被拿掉. 但在定理要求的数乘封闭性下这是不可能的:由于$\mathbf{F}$是数域,因此所有有理数都是其子集,因此$0,-1\in\mathbf{F}$. $\forall \alpha\in V$,我们由于数乘封闭性可知,$0\cdot\alpha=0\in W$,$(-1)\cdot\alpha=-\alpha\in W$,因此$W$中也有加法单位元和逆元.
\end{enumerate}

具体的证明只需要将上面的描述形式化即可,这里不再赘述. 下面我们来看两个常见的例子体会子空间的判别方法:
\begin{example}{}{常见子空间}
    回答下列关于子空间的判定问题:
    \begin{enumerate}
        \item \label{item:2:常见子空间:1}
              说明$\mathbf{R}[x]_2$是$\mathbf{R}[x]_3$的子空间;

        \item \label{item:2:常见子空间:2}
              判断$W_1=\left\{(x,y,z) \;\middle|\; \dfrac{x}{3}=\dfrac{y}{2}=z\right\},\enspace W_2=\{(x,y,z) \mid x+y+z=1,\enspace x-y+z=1\}$是否为$\mathbf{R}^3$的子空间;

        \item \label{item:2:常见子空间:3}
              试说明齐次线性方程组$AX=0$的解集是线性空间$\mathbf{F}^n$的一个子空间(我们称之为齐次线性方程组的\term{解空间}),但非齐次线性方程组的解不再构成线性空间.
    \end{enumerate}
\end{example}

\begin{solution}
    \begin{enumerate}
        \item 只需证明$\mathbf{R}[x]_2 \subseteq \mathbf{R}[x]_3$,以及$\mathbf{R}[x]_2$对$\mathbf{R}[x]_3$中的加法和数乘封闭即可.

              $\forall v \in \mathbf{R}[x]_2$,可被写作$v=a+bx,a,b \in \mathbf{R}$. 又有$\mathbf{R}[x]_3=\{a+bx+cx^2,a,b,c \in \mathbf{R}\}$,取$c=0$,有$v=a+bx \in \mathbf{R}[x]_3$,因此$\mathbf{R}[x]_2 \subseteq \mathbf{R}[x]_3$.

              对于$\mathbf{R}[x]_3$中的加法和数乘:
              \[mv_1+nv_2=m(a_1+b_1x)+n(a_2+b_2x)=(ma_1+na_2)+(mb_1+nb_2)x \in \mathbf{R}[x]_3\]
              所以$\mathbf{R}[x]_2$是$\mathbf{R}[x]_3$的子空间.

        \item 对于此类给出条件求解子空间的问题,实际上很容易理解,因为$W_1$实际上就是所有满足方程组
              \[\begin{cases}
                      x-3z=0 \\
                      y-2z=0
                  \end{cases}\]
              的向量,而$W_2$实际上就是所有满足方程组
              \[\begin{cases}
                      x+y+z=1 \\
                      x-y+z=1
                  \end{cases}\]
              的向量. 因此我们只需要验证这两个方程组的解集是否对加法和数乘封闭即可.

              对 $W_1$: 引入参数$t$,
              \[W_1=\left\{(3t,2t,t) \;\middle|\; \frac{x}{3} = \frac{y}{2} = z = t\right\}\]
              对于$\forall v_1, v_2 \in W_1, v_1 = (3t_1, 2t_1, t_1), v_2 = (3t_2, 2t_2, t_2)$,有
              \begin{align*}
                  av_1 + bv_2 & = (3at_1 + 3bt_2, 2at_1 + 2bt_2, at_1 + bt_2)           \\
                              & = (3(at_1 + bt_2), 2(at_1 + bt_2), at_1 + bt_2) \in W_1
              \end{align*}
              故$W_1$封闭,是 $\mathbf{R}^3$ 的子空间.

              对 $W_2$: 有反例. 取 $u_1 = (1, 0, 0), u_2 = (0, 0, 1) \in W_2$,但 $W_1 + W_2 = (1, 0, 1)$ 不满足 $x + y + z = 1$,故 $W_2$ 不封闭,不是 $\mathbf{R}^3$ 的子空间.

        \item 设齐次线性方程组 $AX=0$ 的解构成的集合是 $W_1$,$\forall X_1, X_2 \in W_1$,有 $AX_1 = AX_2 = 0$,所以 $\forall a, b \in \mathbf{F}$,
              \[A(a X_1 + b X_2) = A(a X_1) + A(b X_2) = a AX_1 + b AX_2 = 0\]
              故 $W_1$ 封闭,是 $\mathbf{F}^n$ 的子空间.

              设非齐次线性方程组 $AX = \beta,\enspace \beta \in \mathbf{F}^m,\enspace \beta \neq 0$ 的解构成的集合是 $W_2$,$\forall X_1, X_2 \in W_2$,有 $AX_1 = AX_2 = \beta$,所以 $A(X_1 + X_2) = AX_1 + AX_2 = 2\beta \neq \beta$. 故 $W_2$ 不封闭,不是 $\mathbf{F}^n$ 的子空间.
    \end{enumerate}
\end{solution}

上例中 \ref*{item:2:常见子空间:2} 表明过原点的直线/平面构成三维空间的子空间,不过原点的无法保持线性性. 事实上 \ref*{item:2:常见子空间:2} 和 \ref*{item:2:常见子空间:3} 在表述同一个问题,\ref*{item:2:常见子空间:2} 从几何角度描述了 \ref*{item:2:常见子空间:3} 中齐次/非齐次线性方程组的解集. 事实上,在定义了子空间后,如果一个线性空间的子集也构成线性空间,我们就可以对其进行同样的研究. 这一想法在我们后续的内容中十分重要, 现在需要大家先熟知子空间的定义和判别.

最后我们需要注意一个名词的定义. 线性空间有两个子空间称为平凡子空间,即仅含零元的子集$\{0\}$和其自身$V$. 而其它子空间称为非平凡子空间.

\section{线性表示 \quad 线性扩张}

在高中平面向量的学习中我们知道,两个单位向量$(1,0)$和$(0,1)$可以表示出整个平面的所有向量,高中我们也称这样的向量为平面向量的基底. 接下来我们将二维平面扩展至任意线性空间,同样讨论有关于``表示''、``基底''的问题.

我们首先来看线性组合和线性表示的概念:
\begin{definition}{}{}
    设$V(\mathbf{F})$是一个线性空间,$\alpha_i\in V,\enspace\lambda_i\in \mathbf{F}\enspace(i=1,2,\ldots,m)$,则向量$\alpha=\lambda_1\alpha_1+\lambda_2\alpha_2+\cdots+\lambda_m\alpha_m$称为向量组$\alpha_1,\alpha_2,\ldots,\alpha_m$在域$\mathbf{F}$的线性组合,或说$\alpha$在域$\mathbf{F}$上可用向量组$\alpha_1,\alpha_2,\ldots,\alpha_m$线性表示.
\end{definition}
这和我们高中所学的用向量的基底表示其他向量是完全一致的. 基于此,我们给出线性扩张的定义:
\begin{definition}{线性扩张}{} \index{xianxingkuozhang@线性扩张 (linear span)}
    设$S$是线性空间$V(\mathbf{F})$的非空子集,我们称
    \[ \spa(S)=\{\lambda_1\alpha_1+\cdots+\lambda_k\alpha_k \mid \lambda_1,\ldots,\lambda_k\in\mathbf{F},\enspace\alpha_1,\ldots,\alpha_k\in S,\enspace k\in\mathbf{N}_+\} \]
    为$S$的\term{线性扩张},即$S$中所有有限子集在域$\mathbf{F}$上的一切线性组合组成的$V(\mathbf{F})$的子集.
\end{definition}
注意,有的教材中也中使用 $L$ 表示线性扩张. 考虑到本讲义记号统一性,我们采用更加常用并且不会与之后其它定义的记号冲突的$\spa$.

下面的定理告诉我们可以通过线性扩张构造子空间:
\begin{theorem}{}{线性扩张构造子空间}
    线性空间$V(\mathbf{F})$的非空子集$S$的线性扩张$\spa(S)$是$V$中包含$S$的最小子空间.
\end{theorem}
仍然利用平面向量进行直观的理解,平面(也显然在平面向量加法和数乘下构成线性空间)$\mathbf{R}^2$可以由向量$(1,0)$和$(0,1)$扩张而成. 由这一定理的结果我们可以将一个向量组的线性扩张称为向量组的张成空间. 这一定理的证明思想非常重要,因此在此给出:

\begin{proof}
    \begin{enumerate}
        \item 首先我们证明$\spa(S)$是$V$的子空间.
              \begin{enumerate}
                  \item $\spa(S)$非空:由于$S$非空,且$S\subseteq\spa(S)$显然成立:取$\lambda=1,\enspace\forall s\in S,\enspace \lambda s=s\in\spa(S)$. 因此$\spa(S)$非空;

                  \item 设$\alpha,\beta\in\spa(S)$,则存在$\lambda_1,\ldots,\lambda_k\in\mathbf{F},\enspace \alpha_1,\ldots,\alpha_k\in S,\enspace\mu_1,\ldots,\mu_l\in\mathbf{F},\enspace\beta_1,\ldots,\beta_l\in S$,使得
                        \begin{gather*}
                            \alpha=\lambda_1\alpha_1+\cdots+\lambda_k\alpha_k \\
                            \beta=\mu_1\beta_1+\cdots+\mu_l\beta_l
                        \end{gather*}
                        因此我们可以得到$\spa(S)$
                        \begin{enumerate}
                            \item 关于加法封闭:$\alpha+\beta=\lambda_1\alpha_1+\cdots+\lambda_k\alpha_k+\mu_1\beta_1+\cdots+\mu_l\beta_l\in\spa(S)$;

                            \item 关于数乘封闭:$\lambda\alpha=\lambda\lambda_1\alpha_1+\cdots+\lambda\lambda_k\alpha_k\in\spa(S)$(数域关于乘法运算封闭,故$\lambda\lambda_i\in\mathbf{F},\enspace i=1,\ldots,k$).
                        \end{enumerate}
              \end{enumerate}
              综上,$\spa(S)$是$V$的子空间;

        \item 接下来我们证明$\spa(S)$是包含$S$的最小子空间. 设$W$是$V$的任一子空间,我们只需证明$\spa(S)\subseteq W$.

              事实上,类似于前面$S\subseteq\spa(S)$的证明我们有$S\subseteq W$,故$S$中元素都在$W$中. 且由\autoref{thm:子空间判别} 可知子空间中元素一定关于加法、数乘封闭,因此$\forall {\alpha}=\lambda_1\alpha_1+\cdots+\lambda_k\alpha_k\in\spa(S)$. 由于$\alpha_1,\ldots,\alpha_k\in S\subseteq W,\enspace\lambda_1,\ldots,\lambda_k\in\mathbf{F}$,因此$\alpha\in W$,从而$\spa(S)\subseteq W$,由此得证.
    \end{enumerate}
\end{proof}

上述证明的重要性在于,我们在这一个证明中练习了子集的证明方法、子空间的充要条件以及对于``最小''问题证明的一般方法. 希望读者能掌握其中的每一个思想与技巧. 此外,这一定理有很强的直观性,因为线性扩张实际上就是将子集中的元素进行无穷次重复的线性组合,将所有可能经过线性运算获得的向量都生成了,因此线性扩张的结果一定保障了线性运算的封闭.

\begin{summary}

    本讲我们追随着第一讲最末尾关于线性方程组为什么无解、有唯一解或无穷解的问题,展开我们对线性方程组一般理论的讨论. 我们首先通过一个例子引入我们为什么要研究线性空间——因为我们需要了解向量之间的关联,从直觉上这与线性方程组解的情况是有关系的. 我们给出了线性空间的定义——其核心仍然是在集合上定义满足一定条件的运算,事实上就是对我们高中就熟知的向量加法数乘规则的抽象,然后我们讨论了基于这一公理化的定义我们可以得到的性质. 我们介绍了线性空间的子空间的定义与判别方法,引入了线性表示、线性扩张的概念并说明了我们如何通过线性扩张得到子空间——实际上这些内容放在向量空间中都可以找到直观的几何解释,因此理解起来并不困难.

    事实上,这一讲的内容是比较抽象的,因为线性空间的定义实际上就是将我们熟知的向量加法数乘运算抽象出来,从而适用于所有有类似结构的集合,因此读者在学习时可能会自动带入一些高中平面向量的直观,然后发现显然的问题不用证,复杂的问题摸不着头脑,但读者应当在未竟专题一中训练了基于定义和公理的数学证明思想,我们也尽力给出大量经典的例子,将推导过程写得非常详细,所以整体而言思路应当是清晰易懂的.

\end{summary}

\begin{exercise}
    \exquote[毕达哥拉斯]{在数学的天地里,重要的不是我们知道什么,而是我们怎么知道什么.}

    \begin{exgroup}
        \item 检验下列集合对指定的加法和数乘运算是否构成实数域上的线性空间.
        \begin{enumerate}
            \item 有理数集$\mathbf{Q}$对普通的数的加法和乘法;

            \item 集合$\mathbf{R}^2$对通常的向量加法和如下定义的数量乘法:$\lambda\cdot(x,y)=(\lambda x,y)$;

            \item $\mathbf{R}_+^n$(即$n$元正实数向量)对如下定义的加法和数乘运算:
                  \begin{gather*}
                      (a_1,\ldots,a_n)+(b_1,\ldots,b_n)=(a_1b_1,\ldots,a_nb_n) \\
                      \lambda\cdot(a_1,\ldots,a_n)=(a_1^\lambda,\ldots,a_n^\lambda)
                  \end{gather*}

            \item 集合 $V$ 为区间 $[a, b]$ 上所有函数值 $\geqslant 0$ 的实变量函数,即:
                  \[V=\{f \mid f(x) \geqslant 0, \forall x \in [a, b] \}\]
                  对通常的函数加法和数与函数的乘法,即:
                  \begin{gather*}
                      (f \oplus g)(x) = f(x) + g(x) \\
                      (\lambda \circ f)(x) = \lambda f(x)
                  \end{gather*}

            \item \[V_1= \{f \mid x \in \mathbf{R}, f(x) \in \mathbf{R}, f(-x)=-f(x)\}\]
                  \[V_2= \{f \mid x \in \mathbf{R}, f(x) \in \mathbf{R}, f(0)=1, f(-x)=f(x)\}\]
                  对题(4)所定义的加法和数量乘法.

            \item $V = \{f \mid x \in \mathbf{R}, f(x) \in \mathbf{C}, f(-x)= \overline{f(x)} \}$.
                  对题(4)所定义的加法和数量乘法.
        \end{enumerate}

        \begin{answer}
            \begin{enumerate}
                \item 有理数集 $Q$ 关于实数乘法不封闭,不构成实数域上的线性空间.

                \item $\mathbf{R}^2$ 关于通常向量加法构成交换群,封闭性也显然成立. 再看数乘.
                      \begin{enumerate}
                          \item $\exists \lambda=1$ 使得 $\lambda\cdot(x,y)=(\lambda x,y)=(x,y)$.

                          \item $\lambda(\mu\cdot(x,y))=\lambda\cdot(\mu x,y)=((\lambda\mu)x,y)=(\lambda\mu)\cdot(x,y)$.

                          \item $(\lambda+\mu)\cdot(x,y)=((\lambda+\mu)x,y)=(\lambda x,y)+(\mu x,y)$. 因此$(\lambda+\mu)\cdot(x,y)=\lambda\cdot(x,y)+\mu\cdot(x,y)$ 成立.

                          \item $\lambda((x_1,y_1)+(x_2,y_2))=\lambda\cdot(x_1+x_2,y_1+y_2)=(\lambda x_1+\lambda x_2,y_1+y_2)=(\lambda x_1,y_1)+(\lambda x_2,y_2)$,因此$\lambda((x_1,y_1)+(x_2,y_2))=(\lambda x_1,y_1)+(\lambda x_2,y_2)$.

                          \item (封闭性)$\forall \lambda \in \mathbf{R},\lambda\cdot(x,y)=(\lambda x,y)\in \mathbf{R}^2$,封闭性满足.

                                综上,$\mathbf{R}^2$ 关于通常向量加法与该数乘构成实数域上的向量空间.
                      \end{enumerate}

                \item \begin{enumerate}
                          \item 对于加法,显然,封闭性,结合律,交换律成立. 存在加法单位元 $(1,1,\ldots,1)$ 有
                                \begin{align*}
                                    (1,1,\ldots,1)+(a_1,a_2,\ldots,a_n) & = (a_1,a_2,\ldots,a_n)+(1,1,\ldots,1) \\
                                                                        & = (a_1,a_2,\ldots,a_n)
                                \end{align*}
                                由于为正实数向量,则对于 $(a_1,\ldots,a_n)$,存在唯一的逆元 $\left(\dfrac 1{a_1},\ldots,\dfrac 1{a_n}\right)$,使得 $(a_1,\ldots,a_n)+\left(\dfrac 1{a_1},\ldots,\dfrac 1{a_n}\right)=(1,\ldots,1)$.

                          \item 对于数乘,显然有封闭性成立,乘法单位元为 $\lambda_0=1$. 又有
                                \begin{enumerate}
                                    \item \begin{align*}
                                                  & \lambda(\mu\cdot(a_1,\ldots,a_n))            \\
                                              ={} & \lambda\cdot(a_1^\mu,\ldots,a_n^\mu)         \\
                                              ={} & ((a_1^\mu)^\lambda,\ldots,(a_n^\mu)^\lambda) \\
                                              ={} & (a_1^{\lambda\mu},\ldots,a_n^{\lambda\mu})
                                          \end{align*}
                                          因此 $\lambda(\mu\cdot(a_1,\ldots,a_n))=(\lambda\mu)\cdot(a_1,\ldots,a_n)$ 成立.

                                    \item \begin{align*}
                                                  & (\lambda+\mu)\cdot(a_1,\ldots,a_n)                         \\
                                              ={} & (a_1^{\mu+\lambda},\ldots,a_n^{\mu+\lambda})               \\
                                              ={} & (a_1^\lambda a_1^\mu,\ldots,a_n^\lambda a_n^\mu)           \\
                                              ={} & (a_1^\lambda,\ldots,a_n^\lambda)+(a_1^\mu,\ldots,a_n^\mu),
                                          \end{align*}
                                          因此 $(\lambda+\mu)\cdot(a_1,\ldots,a_n)=\lambda\cdot(a_1,\ldots,a_n)+\mu(a_1,\ldots,a_n)$,第一个加号为数的加法,第二个加号为定义的向量加法.

                                    \item $\lambda\cdot((a_1,\ldots,a_n)+(b_1,\ldots,b_n))=\lambda\cdot(a_1b_1,\ldots,a_nb_n)=(a_1^\lambda b_1^\lambda,\ldots,a_n^\lambda b_n^\lambda)=(a_1^\lambda,\ldots,a_n^\lambda)+(b_1^\lambda,\ldots,b_n^\lambda)$,因此 $\lambda\cdot((a_1,\ldots,a_n)+(b_1,\ldots,b_n))=\lambda\cdot(a_1,\ldots,a_n)+\lambda\cdot(b_1,\ldots,b_n)$.
                                \end{enumerate}
                                综上有 $\mathbf{R}_+^n$ 对如下加法和数乘构成实数域线性空间.
                      \end{enumerate}


                        \item 当 $\lambda<0$ 时,$(\lambda\circ f)(x)=\lambda f(x)\leqslant 0$,是函数值 $\le0$ 的实变量函数,则 $\lambda f(x)\not\in V$,即关于数乘不封闭,不构成线性空间.

                        \item $V_1$ 是奇函数集合,只需验证 $V_1$ 对加法和数乘封闭即可. 这显然成立. 则 $V_1$ 构成线性空间. 对于 $V_2$:当 $\lambda\neq 1$,有 $(\lambda\circ f)(0)=\lambda f(0)=\lambda\neq 1$. 则 $(\lambda\circ f)(x)\in V_2$,$V_2$ 不封闭,不构成线性空间.

                        \item 先验证 $V$ 非空:有 $f(x)=0,\forall x\in \mathbf{R}$,则 $f(x)\in V$,即 $V$ 非空. 再验证封闭性:对于 $(f\oplus g)(x)$,有 $(f\oplus g)(-x)=f(-x)+g(-x)=\overline{f(x)}+\overline{g(x)}=\overline{f(x)+g(x)}=\overline{(f\oplus g)(x)}$. 对于 $(\lambda\circ f)(x)$,有 $(\lambda\circ f)(-x)=\lambda f(-x)=\lambda\overline{f(x)}$. 由于 $\lambda \in \mathbf{R}$,则 $\lambda \overline{f(x)}=\overline{\lambda f(x)}=\overline{(\lambda\circ f)(x)}$. 因此 $V$ 关于 $\mathbf{R}$ 的函数加法和数乘封闭. 再给出加法零元 $f(x)=0$,数乘单位元 $\lambda=1$. 其余性质还请读者自行验证. 总之,$V$ 构成 $\mathbf{R}$ 上线性空间.
            \end{enumerate}
        \end{answer}

        \item 判断下列子集是否为给定线性空间的子空间:
        \begin{enumerate}
            \item $W = \{(x_1,\ldots,x_n) \in F^n \mid a_1 x_1+\cdots +a_n x_n =0\}$, 其中 $a_1, \ldots, a_n$ 为域 $F$ 中的固定数量.

            \item $W_1 = \{(x,1,0) \in \mathbf{R}^3 \}$, $W_2 = \{(x,y,0) \in \mathbf{R}^3\}$.

            \item $W_1 = \{(x,y,z) \in \mathbf{R}^3 \mid x-3y+z = 0\}$, $W_2 = \{(x,y,z) \in \mathbf{R}^3 \mid x-3y+z = 1\}$.

            \item $W_1 = \left\{(x,y,z) \in \mathbf{R}^3 \middle|\ \dfrac{x}{2} = \dfrac{y-4}{1} = \dfrac{z-1}{-3}\right\}$,

                  $W_2 = \{(x,y,z) \in \mathbf{R}^3 \mid x-y = 0, x+y+z = 0\}$.

            \item $W_1 = \{p(x) \in R[x] \mid p(1) = 0\}$, $W_2 = \{P(x) \in R[x]_n \mid p(1) = p(0)\}$(此题主要就是要判断满足一定条件的多项式是否构成子空间).

            \item $W = \{f \in F(-\infty, +\infty) \mid f(-x)=f(x), \forall x \in R\}$, 其中 $F(-\infty, +\infty)$ 是所有定义在 $(-\infty, +\infty)$ 上的实值函数对通常的函数加法及数与函数的乘法在实数域上构成的线性空间.
        \end{enumerate}

        \begin{answer}
            \begin{enumerate}
                \item 对于集合 \( W = \{(x_1, \ldots, x_n) \in F^n \mid a_1x_1 + \cdots + a_nx_n = 0\} \):

                首先,我们来判断该集合是否为线性空间的子空间. 判断一个集合是否为线性空间的子空间需要满足以下三个条件:

                \begin{enumerate}
                    \item \text{零向量在其中}:

                    当 \( x_1 = x_2 = \cdots = x_n = 0 \) 时,方程 \( a_1x_1 + \cdots + a_nx_n = 0 \) 显然成立,因此零向量 \( (0, 0, \ldots, 0) \) 属于 \( W \).

                    \item \text{封闭性(加法)}:

                    假设 \( \mathbf{v} = (x_1, \ldots, x_n) \) 和 \( \mathbf{w} = (y_1, \ldots, y_n) \) 属于 \( W \),即:
                    \[
                    a_1x_1 + \cdots + a_nx_n = 0 \quad \text{且} \quad a_1y_1 + \cdots + a_ny_n = 0.
                    \]
                    那么,对于 \( \mathbf{v} + \mathbf{w} = (x_1 + y_1, \ldots, x_n + y_n) \),有:
                    \[
                    a_1(x_1 + y_1) + \cdots + a_n(x_n + y_n) = (a_1x_1 + \cdots + a_nx_n) + (a_1y_1 + \cdots + a_ny_n) = 0 + 0 = 0.
                    \]
                    因此,\( \mathbf{v} + \mathbf{w} \in W \).

                    \item \text{封闭性(数乘)}:

                    对于任意 \( \mathbf{v} = (x_1, \ldots, x_n) \in W \) 和任意 \( \lambda \in F \),有:
                    \[
                    a_1(\lambda x_1) + \cdots + a_n(\lambda x_n) = \lambda (a_1x_1 + \cdots + a_nx_n) = \lambda \cdot 0 = 0.
                    \]
                    因此,\( \lambda \mathbf{v} \in W \).
                \end{enumerate}

                由于上述三个条件均满足,集合 \( W \) 是 \( F^n \) 的一个子空间.

                \item 对于集合 \( W_1 = \{(x, 1, 0) \in \mathbb{R}^3\} \) 和 \( W_2 = \{(x, y, 0) \in \mathbb{R}^3\} \):

                \begin{enumerate}
                    \item \text{对于 \( W_1 \)}:

                    \begin{itemize}
                        \item \text{零向量在其中}:零向量 \( (0, 0, 0) \) 不在 \( W_1 \) 中,因为 \( W_1 \) 中的第二个分量始终为 1,因此零向量不在 \( W_1 \) 中.
                        \item 因为零向量不在 \( W_1 \) 中,所以 \( W_1 \) 不是子空间.
                    \end{itemize}

                    \item \text{对于 \( W_2 \)}:

                    \begin{itemize}
                        \item \text{零向量在其中}:当 \( x = 0, y = 0 \) 时,向量 \( (0, 0, 0) \in W_2 \).
                        \item \text{封闭性(加法)}:假设 \( \mathbf{v} = (x_1, y_1, 0) \in W_2 \) 和 \( \mathbf{w} = (x_2, y_2, 0) \in W_2 \),则:
                        \[
                        \mathbf{v} + \mathbf{w} = (x_1 + x_2, y_1 + y_2, 0) \in W_2.
                        \]
                        \item \text{封闭性(数乘)}:对于 \( \lambda \in \mathbb{R} \) 和 \( \mathbf{v} = (x, y, 0) \in W_2 \),有:
                        \[
                        \lambda \mathbf{v} = (\lambda x, \lambda y, 0) \in W_2.
                        \]
                    \end{itemize}
                \end{enumerate}

                因此,\( W_2 \) 是 \( \mathbb{R}^3 \) 的一个子空间.

                \item 对于集合 $W_1 = \{(x, y, z) \in \mathbb{R}^3 \mid x - 3y + z = 0\}$ 和 $W_2 = \{(x, y, z) \in \mathbb{R}^3 \mid x - 3y + z = 1\}$:
                \begin{enumerate}
                    \item \text{零向量在其中}:当 $x = 0, y = 0, z = 0$ 时,方程 $x - 3y + z = 0$ 显然成立,因此零向量 $(0, 0, 0) \in W_1$.
                    \item \text{封闭性(加法)}:假设 $\mathbf{v} = (x_1, y_1, z_1) \in W_1$ 和 $\mathbf{w} = (x_2, y_2, z_2) \in W_1$,则:
                    \[
                    (x_1 - 3y_1 + z_1) = 0 \quad \text{且} \quad (x_2 - 3y_2 + z_2) = 0.
                    \]
                    对于 $\mathbf{v} + \mathbf{w} = (x_1 + x_2, y_1 + y_2, z_1 + z_2)$,有:
                    \[
                    (x_1 + x_2) - 3(y_1 + y_2) + (z_1 + z_2) = (x_1 - 3y_1 + z_1) + (x_2 - 3y_2 + z_2) = 0 + 0 = 0.
                    \]
                    因此,$\mathbf{v} + \mathbf{w} \in W_1$.
                    \item \text{封闭性(数乘)}:对于任意 $\lambda \in \mathbb{R}$ 和 $\mathbf{v} = (x, y, z) \in W_1$,有:
                    \[
                    \lambda x - 3(\lambda y) + \lambda z = \lambda(x - 3y + z) = \lambda \cdot 0 = 0.
                    \]
                    因此,$\lambda \mathbf{v} \in W_1$.
                \end{enumerate}
                综上所述,$W_1$ 是 $\mathbb{R}^3$ 的一个子空间.
                \begin{enumerate}
                    \item \text{零向量不在其中}:若 $(0, 0, 0) \in W_2$,则必须满足 $0 - 3 \cdot 0 + 0 = 1$,这显然不成立,因此零向量不在 $W_2$ 中.
                \end{enumerate}
                因为零向量不在 $W_2$ 中,所以 $W_2$ 不是子空间.

                \item 对于集合 $W_1 = \left\{(x, y, z) \in \mathbb{R}^3 \mid \frac{x}{2} = \frac{y - 4}{1} = \frac{z - 1}{-3} \right\}$ 和 $W_2 = \{(x, y, z) \in \mathbb{R}^3 \mid x - y = 0, x + y + z = 0\}$:

                这个集合的方程表示一个直线的参数方程. 通常直线不是子空间,因为它不通过原点. 因此,$W_1$ 不是子空间.

                \begin{enumerate}
                    \item \text{零向量在其中}:当 $x = 0, y = 0, z = 0$ 时,方程 $x - y = 0$ 和 $x + y + z = 0$ 都成立,因此零向量 $(0, 0, 0) \in W_2$.
                    \item \text{封闭性(加法)}:假设 $\mathbf{v} = (x_1, y_1, z_1) \in W_2$ 和 $\mathbf{w} = (x_2, y_2, z_2) \in W_2$,即:
                    \[
                    x_1 - y_1 = 0, \quad x_1 + y_1 + z_1 = 0
                    \]
                    \[
                    x_2 - y_2 = 0, \quad x_2 + y_2 + z_2 = 0
                    \]
                    对于 $\mathbf{v} + \mathbf{w} = (x_1 + x_2, y_1 + y_2, z_1 + z_2)$,有:
                    \[
                    (x_1 + x_2) - (y_1 + y_2) = (x_1 - y_1) + (x_2 - y_2) = 0 + 0 = 0,
                    \]
                    \[
                    (x_1 + x_2) + (y_1 + y_2) + (z_1 + z_2) = (x_1 + y_1 + z_1) + (x_2 + y_2 + z_2) = 0 + 0 = 0.
                    \]
                    因此,$\mathbf{v} + \mathbf{w} \in W_2$.
                    \item \text{封闭性(数乘)}:对于任意 $\lambda \in \mathbb{R}$ 和 $\mathbf{v} = (x, y, z) \in W_2$,有:
                    \[
                    \lambda x - \lambda y = \lambda (x - y) = \lambda \cdot 0 = 0,
                    \]
                    \[
                    \lambda x + \lambda y + \lambda z = \lambda (x + y + z) = \lambda \cdot 0 = 0.
                    \]
                    因此,$\lambda \mathbf{v} \in W_2$.
                \end{enumerate}
                综上所述,$W_2$ 是 $\mathbb{R}^3$ 的一个子空间.

                \item 对于集合 $W_1 = \{p(x) \in \mathbb{R}[x] \mid p(1) = 0\}$ 和 $W_2 = \{P(x) \in \mathbb{R}[x]_n \mid p(1) = p(0)\}$:

                \begin{enumerate}
                    \item \text{零多项式在其中}:零多项式 $p(x) = 0$ 满足 $p(1) = 0$,因此零多项式 $p(x) = 0 \in W_1$.
                    \item \text{封闭性(加法)}:假设 $p(x), q(x) \in W_1$,即 $p(1) = 0$ 且 $q(1) = 0$,那么对于 $p(x) + q(x)$,有:
                    \[
                    (p(x) + q(x))(1) = p(1) + q(1) = 0 + 0 = 0,
                    \]
                    因此 $p(x) + q(x) \in W_1$.
                    \item \text{封闭性(数乘)}:对于任意实数 $\lambda$ 和 $p(x) \in W_1$,有:
                    \[
                    (\lambda p(x))(1) = \lambda p(1) = \lambda \cdot 0 = 0,
                    \]
                    因此 $\lambda p(x) \in W_1$.
                \end{enumerate}
                综上所述,$W_1$ 是一个子空间.

                \begin{enumerate}
                    \item \text{零多项式在其中}:零多项式 $p(x) = 0$ 显然满足 $p(1) = p(0)$,因此零多项式 $p(x) = 0 \in W_2$.
                    \item \text{封闭性(加法)}:假设 $p(x), q(x) \in W_2$,即 $p(1) = p(0)$ 且 $q(1) = q(0)$,那么对于 $p(x) + q(x)$,有:
                    \[
                    (p(x) + q(x))(1) = p(1) + q(1), \quad (p(x) + q(x))(0) = p(0) + q(0).
                    \]
                    因为 $p(1) = p(0)$ 和 $q(1) = q(0)$,所以 $p(1) + q(1) = p(0) + q(0)$,即 $p(x) + q(x) \in W_2$.
                    \item \text{封闭性(数乘)}:对于任意实数 $\lambda$ 和 $p(x) \in W_2$,有:
                    \[
                    (\lambda p(x))(1) = \lambda p(1), \quad (\lambda p(x))(0) = \lambda p(0).
                    \]
                    因为 $p(1) = p(0)$,所以 $\lambda p(1) = \lambda p(0)$,即 $\lambda p(x) \in W_2$.
                \end{enumerate}
                综上所述,$W_2$ 是一个子空间.

                \item 对于集合 $W = \{f \in F(-\infty, +\infty) \mid f(-x) = f(x), \forall x \in \mathbb{R}\}$:

                \begin{enumerate}
                    \item \text{零函数在其中}:零函数 $f(x) = 0$ 显然满足 $f(-x) = f(x)$,因此零函数 $f(x) = 0 \in W$.
                    \item \text{封闭性(加法)}:假设 $f(x), g(x) \in W$,即 $f(-x) = f(x)$ 且 $g(-x) = g(x)$,那么对于 $f(x) + g(x)$,有:
                    \[
                    (f + g)(-x) = f(-x) + g(-x) = f(x) + g(x) = (f + g)(x),
                    \]
                    因此 $f(x) + g(x) \in W$.
                    \item \text{封闭性(数乘)}:对于任意实数 $\lambda$ 和 $f(x) \in W$,有:
                    \[
                    (\lambda f)(-x) = \lambda f(-x) = \lambda f(x) = (\lambda f)(x),
                    \]
                    因此 $\lambda f(x) \in W$.
                \end{enumerate}
                综上所述,$W$ 是一个子空间.

            \end{enumerate}
        \end{answer}
    \end{exgroup}

    \begin{exgroup}
        \item 证明:已知线性空间$V(\mathbf{F})$,$\lambda,\lambda_1,\ldots,\lambda_r\in\mathbf{F}$,$\beta,\alpha_1,\ldots,\alpha_r\in V$,有$\lambda\beta+\lambda_1\alpha_1+\lambda_2\alpha_2+\cdots+\lambda_r\alpha_r=\vec{0}$在$\lambda\neq 0$时的解为$\beta=-\lambda^{-1}\lambda_1\alpha_1-\lambda^{-1}\lambda_2\alpha_2-\cdots-\lambda^{-1}\lambda_r\alpha_r$.

        \item \label{item:2:去除一条线性空间公理}
        下面是一些集合满足除了其中任意一条公理之外的其它所有公理的例子,它们说明线性空间定义中各条公理是独立的:
        \begin{enumerate}
            \item 加法结合律

            考虑集合 $\mathbf{F} = \mathbf{R}, V = \{(x, y)\in\mathbf{R}^2 \mid xy = 0\}$,即 $x$ 轴和 $y$ 轴的并,在其上自定义加法和乘法
            \begin{align*}
                (x_1, x_2) + (y_1, y_2) &= \begin{cases}
                    (0, 0), & (x_1+x_2)(y_1+y_2) \neq 0 \\
                    (x_1+x_2, y_1+y_2), & (x_1+x_2)(y_1+y_2) = 0
                \end{cases} \\
                k\cdot (x_1, x_2) &= (k x_1, k x_2)
            \end{align*}
            请验证其满足除了加法结合律以外的所有要求.

            \item 加法逆元

            因为如果去掉加法单位元要求会导致逆元无法定义,所以跳过仅仅去掉加法单位元的情况. 考虑 $\mathbf{F} = \mathbf{R}$,构造这样一个集合:向实数集中人为添加一个元素 $0'$,保持其它数的加法,称这个集合为 $V$,然后对 $0'$ 参与的所有加法运算定义
            \[
                0' + \alpha = \alpha + 0' = \alpha
            \]
            注意到此时 $0' + 0 = 0 + 0' = 0$,也就是说 $0'$ 是真正的加法单位元,对 $\alpha \neq 0$,沿用实数集中``正常''的乘法. 对 $0'$ 定义数乘
            \[
                k \cdot 0' = 0'
            \]
            请验证除了逆元以外其它的公理都得到了满足,但是对于一般的 $\alpha$,我们无法找到 $\beta$ 使得 $\alpha + \beta = 0'$.

            \item 数乘单位元

            考虑 $\mathbf{F} = V = \mathbf{R}$. 在实数集中保持加法不变,令 $k\cdot \alpha = 0$. 请验证它满足除了乘法单位元以外的所有要求.

            \item 数乘结合律

            仍然保持实数上的加法不变,考虑 $\mathbf{F}=\mathbf{C}, V=\mathbf{R}$,定义乘法
            \[
                k\cdot \alpha = (\Re k) \alpha
            \]
            其中 $\Re k$ 表示复数 $k$ 的实部,请验证它满足除了乘法结合律以外的所有要求.

            \item 左分配律

            考虑 $\mathbf{F}=V=\mathbf{R}$. 保持实数集中加法不变,定义
            \[
                k\cdot \alpha = \alpha
            \]
            请验证它除了左分配律以外的公理都满足.

            \item 右分配律

            (笔者暂时没有想到更好的例子,所以这个例子会略微抽象)考虑 $\mathbf{F}=\mathbf{C}, V=\mathbf{C}^2$. 保持 $\mathbf{C}^2$ 上的加法不变,对 $k\in\mathbf{C}, \alpha = (z_1, z_2)$ 定义乘法
            \[
                k\cdot\alpha = \begin{cases}
                    (k z_1, k z_2), & z_1 = 0~\text{或}~\Im(z_2 / z_1) \geqslant 0\\
                    (\overline{k} z_1, \overline{k} z_2), & \text{其它}
                \end{cases}
            \]
            这里 $\Im$ 表示复数的虚部,请验证它满足除右分配律以外的所有公理.
        \end{enumerate}
        \begin{answer}

        \end{answer}

        \item 设$V$是一个线性空间,$W$是$V$的子集,证明:$W$是$V$的子空间$\iff \spa(W)=W$.
        \begin{answer}
            $W$ 是 $V$ 的子空间等价于 $\forall \alpha_1,\ldots,\alpha_k\in W$,$\forall \lambda_1,\ldots,\lambda_k\in \mathbf{F}$($\mathbf{F}$ 是 $V$ 对应数域)有 $\lambda_1\alpha_1+\cdots+\lambda_k\alpha_k\in W$. 根据线性扩张的定义,以上描述等价于 $\spa(W)\subseteq W$,又 $\spa(W)$ 是 $W$ 的线性扩张,即 $W\subseteq\spa(W)$,故$\spa(W)\subseteq W \iff \spa(W)=W$. 综上 $W$ 是 $V$ 的子空间,得证.
        \end{answer}
    \end{exgroup}

    \begin{exgroup}
        \item 设$\mathbf{E}$是域$\mathbf{F}$的一个子域.
        \begin{enumerate}
            \item 证明:$\mathbf{F}$关于自身的加法和乘法构成一个$\mathbf{E}$上的向量空间,并举一例;

            \item 举例说明:$\mathbf{E}\enspace(\mathbf{E}\neq \mathbf{F})$不是$\mathbf{F}$上的线性空间;

            \item 证明:若$V$是$\mathbf{F}$上的一个线性空间,则$V$也是$\mathbf{E}$上的一个线性空间.
        \end{enumerate}
        \begin{answer}
            \begin{enumerate}
                \item 此处仅验证数乘封闭性,其余性质留给读者. $\forall \alpha\in \mathbf{F},\lambda\in \mathbf{E}$. 由于$\mathbf{E}\subseteq \mathbf{F}$,因此$\lambda\in \mathbf{F}$. 由于 $\mathbf{F}$ 本身是封闭的,故$\lambda\alpha\in F$,则 $F$ 构成 $\mathbf{E}$ 上的线性空间. 例: $\mathbf{C}$ 构成 $\mathbf{R}$ 上的线性空间.

                \item 例如:$\mathbf{R}$ 不是 $\mathbf{C}$ 上的线性空间. 因为 $\forall a\in\mathbf{R}$,有 $i\cdot a=a\cdot i\not\in\mathbf{R}$. 故数乘不封闭,不构成线性空间.

                \item $\forall\lambda\in \mathbf{E}$ 有 $\lambda\in \mathbf{F}$,则 $V$ 关于 $\mathbf{E}$ 的数乘运算肯定是封闭的,其余性质与在 $\mathbf{F}$ 上一致. 又 $\mathbf{E}$ 本身也封闭,则 $V$ 也是 $\mathbf{E}$ 上的一个线性空间,得证.
            \end{enumerate}
        \end{answer}

        \item 考虑在第一章定义的有限域 $\mathbf{F}_4$ 和 $\mathbf{Z}_2$. 证明:$\mathbf{Z}_2$ 可以看作 $\mathbf{F}_4$ 的一个子域. 并给出 $\mathbf{F}_4$ 在 $\mathbf{Z}_2$ 上的线性空间结构. 验证 $\mathbf{Z}_p$ 一定是 $\mathbf{F}_{p^n}$ 的一个子域.
    \end{exgroup}
\end{exercise}

\chapter{有限维线性空间}

在第二讲开头的\autoref{ex:线性空间引入} 中,我们讨论了齐次线性方程组解的个数与方程组系数矩阵行向量间没有可互相消去的关系之间的联系. 本节我们将这种``可互相消去的关系''进行形式化定义. 另一方面,在第二讲最后探讨线性扩张的概念时,一个很自然的问题便是:一个有限维线性空间最少可以由多少个向量线性扩张而来?循此路径,我们将在本讲探寻线性空间的最基本的结构属性.

\section{线性相关性}

\subsection{线性相关性的定义}

本节我们将形式化定义在引言中我们提到的``可相互消去的关系''——线性相关性,同时这一定义也可以解决引言中提到的关于有限维线性空间至少需要多少个向量张成的问题.
\begin{definition}{线性相关性}{} \index{xianxingxiangguan@线性相关 (linearly dependent)} \index{xianxingwuguan@线性无关 (linearly independent)}
    设$V(\mathbf{F})$是一个线性空间,$\alpha_1,\alpha_2,\ldots,\alpha_m\in V$,若存在不全为0的$\lambda_1,\lambda_2,\ldots,\lambda_m\in\mathbf{F}$,使得
    \[\lambda_1\alpha_1+\lambda_2\alpha_2+\cdots+\lambda_m\alpha_m=0\]
    成立,则称$\alpha_1,\alpha_2,\ldots,\alpha_m$\term{线性相关},否则称\term{线性无关}(即系数只能为0).
\end{definition}
很显然,\autoref{ex:线性空间引入} 中的方程组1系数矩阵的三个行向量$\alpha_1,\alpha_2,\alpha_3$满足$\alpha_1+\alpha_2-\alpha_3=0$,因此满足线性相关的定义,方程组2的系数矩阵三个行向量$\beta_1,\beta_2,\beta_3$的线性组合则只有$0\cdot\beta_1+0\cdot\beta_2+0\cdot\beta_3$等于0,因此符合线性无关的定义.

事实上,直接由定义我们还可以导出以下关于零向量的结论:
\begin{enumerate}
    \item 线性空间中单个向量$\alpha$线性相关的充要条件是$\alpha$为零向量;

    \item 任何含零向量的向量组都线性相关.
\end{enumerate}

需要注意的是,很多时候线性相关和线性无关的证明就是基于定义,请务必牢牢掌握. 我们先来看几个基本的例子:
\begin{example}{}{}
    \begin{enumerate}[label=(\arabic*)]
        \item 判断$\mathbf{R}^3$中向量$(1,1,0),(0,1,1),(1,0,-1)$的线性相关性;

        \item 判断$\mathbf{R}^3$中向量$(1,-3,1),(-1,2,-2),(1,1,3)$的线性相关性;

        \item \label{item:3:线性相关性:3}
              判断$\mathbf{R}[x]_3$中$p_1(x)=1+x,\ p_2(x)=1-x,\ p_3(x)=x+x^2$的线性相关性;

        \item 判断连续函数全体构成的线性空间中$1,\ \sin^2x,\ \cos^2x$的线性相关性;

        \item \label{item:3:线性相关性:5}
              判断连续函数全体构成的线性空间中$1,\ 2^x,\ 2^{-x}$的线性相关性.
    \end{enumerate}
\end{example}

\begin{solution}
    \begin{enumerate}
        \item 根据定义,应求解方程
              \[\lambda_1(1,1,0) + \lambda_2(0,1,1) + \lambda_3(1,0,-1) = 0,\]
              即
              \[ \begin{cases}
                      \lambda_1 + \lambda_3 = 0 \\
                      \lambda_1 + \lambda_2 = 0 \\
                      \lambda_2 - \lambda_3 = 0
                  \end{cases} \]
              解得基础解系$k(1,-1,-1)^\mathrm{T}$,所以存在非零解,向量组线性相关.

        \item 求解方程
              \[\lambda_1(1,-3,1) + \lambda_2(-1,2,-2) + \lambda_3(1,1,3) = 0,\]
              即
              \[ \begin{cases}
                      \lambda_1 - \lambda_2 + \lambda_3 = 0    \\
                      -3\lambda_1 + 2\lambda_2 + \lambda_3 = 0 \\
                      \lambda_1 - 2\lambda_2 + 3\lambda_3 = 0
                  \end{cases} \]
              解得 $\lambda_1 = \lambda_2 = \lambda_3 = 0$,此向量组线性无关.

        \item 求解方程
              \begin{align*}
                  \lambda_1p_1(x) + \lambda_2p_2(x) + \lambda_3p_3(x)                           & = 0 \\
                  (\lambda_1 + \lambda_2) + (\lambda_1 - \lambda_2 + \lambda_3)x + \lambda_3x^2 & = 0
              \end{align*}
              所以需要求解方程组
              \[ \begin{cases}
                      \lambda_1 + \lambda_2 = 0             \\
                      \lambda_1 - \lambda_2 + \lambda_3 = 0 \\
                      \lambda_3 = 0
                  \end{cases} \]
              解得 $\lambda_1 = \lambda_2 = \lambda_3 = 0$,此向量组线性无关.

        \item 易知 $-1 + \sin^2x + \cos^2x = 0$,对应系数为 $-1, 1, 1$,不全为零,所以此向量组线性相关.

        \item 求解方程
              \[\lambda_1 + \lambda_2·2^{-x} + \lambda_3·2^x = 0\]
              很明显会发现仅凭此方程是难以求解的,方程数目不足. 注意到此方程应该对于任意的 $x$ 均成立,所以取 $x = 0, x = 1, x = -1$,得到方程组
              \[ \begin{cases}
                      \lambda_1 + \lambda_2 + \lambda_3 = 0              \\[1ex]
                      \lambda_1 + \dfrac{1}{2}\lambda_2 + 2\lambda_3 = 0 \\[1ex]
                      \lambda_1 + 2\lambda_2 + \dfrac{1}{2}\lambda_3 = 0
                  \end{cases} \]
              解得 $\lambda_1 = \lambda_2 = \lambda_3 = 0$,此向量组线性无关.
    \end{enumerate}
\end{solution}
注意上述 \ref*{item:3:线性相关性:3} 到 \ref*{item:3:线性相关性:5} 题为不能代入特殊的$x$值来说明,例如 \ref*{item:3:线性相关性:3} 令$x=0$得到线性相关的做法是错误的,因为\ref*{item:3:线性相关性:3} 中线性空间就是多项式构成的线性空间,其中的元素就是多项式,不能代入值. 注意 \ref*{item:3:线性相关性:5} 是特殊题型,需要构造更多的方程来求解这一问题.

\subsection{线性相关性的定理}

实际上,除了定义之外,线性相关性还有大量的等价描述. 我们将在本节介绍常见的等价描述,它们是理解线性空间结构等后续内容的基础,因此希望读者对以下结论及其证明十分熟练并且要有深刻的理解. 我们的主线思路是从不同方面理解线性相关性:
\begin{enumerate}
    \item 从线性组合看(定义)

          向量组线性相关$\iff$它们有系数不全为0的线性组合等于零向量;

          向量组线性无关$\iff$它们只有系数全为0的线性组合才会等于零向量.

    \item 从线性表示看
          \begin{theorem}{}{}
              线性空间$V(\mathbf{F})$中的向量组$\alpha_1,\alpha_2,\ldots,\alpha_m\enspace(m \geqslant 2)$线性相关的充分必要条件是$\alpha_1,\alpha_2,\ldots,\alpha_m$中有一个向量可由其余向量在域$\mathbf{F}$上线性表示.
          \end{theorem}
          这一定理等价描述为,向量组线性无关的充分必要条件是其中的向量无法互相表示. 这是显然的,因为向量组能互相表示利用定义可以轻松写出非零系数的线性表示. 总结一下即为:

          向量组线性相关$\iff$其中至少有一个向量可以由其余向量线性表示;

          向量组线性无关$\iff$其中每一个向量都不能由其余向量线性表示.

    \item 从齐次线性方程组看
          \begin{example}{}{}
              判断 $\mathbf{R}^3$ 中向量组 $\{\alpha_1, \alpha_2, \alpha_3\}$ 和 $\{\beta_1, \beta_2, \beta_3\}$ 的线性相关性, 其中

              \[
                  \alpha_1 = (1, 1, 0), \quad \alpha_2 = (0, 1, 1), \quad \alpha_3 = (1, 0, -1),
              \]
              \[
                  \beta_1 = (1, -3, 1), \quad \beta_2 = (-1, 2, -2), \quad \beta_3 = (1, 1, 3).
              \]
          \end{example}
          \begin{solution}
              容易看出, $\alpha_3 = \alpha_1 - \alpha_2$, 所以 $\alpha_1, \alpha_2, \alpha_3$ 线性相关. 但对 $\beta_1, \beta_2, \beta_3$ 不易看出是否有线性关系, 要按定义来判别. 设

              \begin{equation} \label{eq:线性相关性:1}
                  x_1 \beta_1 + x_2 \beta_2 + x_3 \beta_3 = 0
              \end{equation}

              即

              \[
                  x_1(1, -3, 1) + x_2(-1, 2, -2) + x_3(1, 1, 3) = (0, 0, 0).
              \]

              这个向量方程等价于以下的三个元线性齐次方程组:

              \[
                  \begin{cases}
                      x_1 - x_2 + x_3 = 0    \\
                      -3x_1 + 2x_2 + x_3 = 0 \\
                      x_1 - 2x_2 + 3x_3 = 0
                  \end{cases},
              \]

              容易解得这个方程组只有零解 $x_1 = x_2 = x_3 = 0$. 即只有全为零的 $x_1, x_2, x_3$ 才使得\autoref{eq:线性相关性:1} 成立, 故 $\beta_1, \beta_2, \beta_3$ 线性无关.

              一般若 $\beta_1 = (a_1, b_1, c_1), \beta_2 = (a_2, b_2, c_2), \beta_3 = (a_3, b_3, c_3)$, 则 $\beta_1, \beta_2, \beta_3$ 线性相关(线性无关)的充要条件是三元线性齐次方程组

              \[
                  \begin{cases}
                      a_1 x_1 + a_2 x_2 + a_3 x_3 = 0 \\
                      b_1 x_1 + b_2 x_2 + b_3 x_3 = 0 \\
                      c_1 x_1 + c_2 x_2 + c_3 x_3 = 0
                  \end{cases}
              \]

              有非零解(只有零解). 用此法可得 $\mathbf{R}^n$ 中任何 4 个向量, $\mathbf{R}^n$ 中任何 $n + 1$ 个向量都线性相关.

              总的来说,我们有以下结论:

              列向量组$\alpha_1,\alpha_2,\ldots,\alpha_m$线性相关$\iff$齐次线性方程组$x_1\alpha_1+x_2\alpha_2+\cdots+x_m\alpha_m=0$有非零解;

              列向量组$\alpha_1,\alpha_2,\ldots,\alpha_m$线性无关$\iff$齐次线性方程组$x_1\alpha_1+x_2\alpha_2+\cdots+x_m\alpha_m=0$只有零解.
          \end{solution}
    \item 从向量组与它的部分组的关系看
          \begin{example}{}{}
              如果向量组 $\{ \alpha_1,\alpha_2,\ldots,\alpha_n \}$ 线性无关,则其任意子集也线性无关,如果向量组 $\{ \alpha_1,\alpha_2,\ldots,\alpha_n \}$ 线性相关,则其任意包含它的向量组也线性相关.
          \end{example}
          \begin{proof}
              我们先证明前者,不失一般性,设子集为 $\{ \alpha_{i_1},\alpha_{i_2},\ldots,\alpha_{i_k} \}$, 其中 $1 \leqslant i_1 < i_2 < \cdots < i_k \leqslant n$, 若存在不全为零的 $\lambda_1,\lambda_2,\ldots,\lambda_k$ 使得

              \[
                  \lambda_1 \alpha_{i_1} + \lambda_2 \alpha_{i_2} + \cdots + \lambda_k \alpha_{i_k} = 0,
              \]

              则将上式扩充为

              \[
                  \lambda_1 \alpha_{i_1} + \lambda_2 \alpha_{i_2} + \cdots + \lambda_k \alpha_{i_k}  + 0 \alpha_{i_{k+1}} + \cdots + 0 \alpha_{i_n} = 0,
              \]
              其中 $\lambda_{k+1} = \lambda_{k+2} = \cdots = \lambda_n = 0$, 且 $\lambda_1,\lambda_2,\ldots,\lambda_k$ 不全为零,(即将不在子集中的其它元素以$0$作为系数加到方程中,这样就找到了一个满足线性相关定义的式子)这与 $\{ \alpha_1,\alpha_2,\ldots,\alpha_n \}$ 线性无关矛盾. 故 $\{ \alpha_{i_1},\alpha_{i_2},\ldots,\alpha_{i_k} \}$ 线性无关.

              相同的方法证明后者,设 $\{ \alpha_1,\alpha_2,\ldots,\alpha_n \}$ 线性相关,则存在不全为零的 $\lambda_1,\lambda_2,\ldots,\lambda_n$ 使得

              \[
                  \lambda_1 \alpha_1 + \lambda_2 \alpha_2 + \cdots + \lambda_n \alpha_n = 0,
              \]

              则对于任意包含它的向量组,我们也可以将多出来的向量系数取$0$,这样就找到了一个满足线性相关定义的式子,因此包含它的向量组也线性相关.
          \end{proof}
          如果向量组的一个部分组线性相关,那么整个向量组也线性相关;

          如果向量组线性无关,那么它的任何一个部分组也线性无关.

    \item 从向量组线性表示一个向量的方式看
          \begin{theorem}{}{线性无关等价表示唯一}
              若向量组$\alpha_1,\alpha_2,\ldots,\alpha_m$线性无关,而向量组$\beta,\alpha_1,\alpha_2,\ldots,\alpha_m$线性相关,则$\beta$可由$\alpha_1,\alpha_2,\ldots,\alpha_m$线性表示,且表示法唯一.
          \end{theorem}
          这一定理证明十分经典,特别是唯一性的证明需要掌握,因此此处我们给出证明:

          \begin{proof}
              由于向量组$\beta,\alpha_1,\alpha_2,\ldots,\alpha_m$线性相关,故存在不全为0的$\lambda_0,\lambda_1,\ldots,\lambda_m$使得
              \begin{equation}\label{eq:3:线性无关等价定理}
                  \lambda_0\beta+\lambda_1\alpha_1+\lambda_2\alpha_2+\cdots+\lambda_m\alpha_m=0,
              \end{equation}
              其中$\lambda_0$必不为0,因为如果将$\lambda_0=0$代入\autoref{eq:3:线性无关等价定理},则由于向量组$\alpha_1,\alpha_2,\ldots,\alpha_m$线性无关,必有$\lambda_1=\lambda_2=\cdots=\lambda_m=0$,与$\lambda_0,\lambda_1,\ldots,\lambda_m$不全为0的假设矛盾.

              因此我们有
              \[\beta=-\frac{\lambda_1}{\lambda_0}\alpha_1-\frac{\lambda_2}{\lambda_0}\alpha_2-\cdots-\frac{\lambda_m}{\lambda_0}\alpha_m.\]
              由此我们知道$\beta$可由$\alpha_1,\alpha_2,\ldots,\alpha_m$线性表示. 接下来我们证明表示方式的唯一性. 假设有两种表示方法:
              \begin{gather*}
                  \beta=\mu_1\alpha_1+\mu_2\alpha_2+\cdots+\mu_m\alpha_m, \\
                  \beta=\nu_1\alpha_1+\nu_2\alpha_2+\cdots+\nu_m\alpha_m.
              \end{gather*}
              两式相减可得
              \[0=(\mu_1-\nu_1)\alpha_1+(\mu_2-\nu_2)\alpha_2+\cdots+(\mu_m-\nu_m)\alpha_m.\]
              由于$\alpha_1,\alpha_2,\ldots,\alpha_m$线性无关,因此$\mu_i-\nu_i=0\enspace(i=1,2,\ldots,m)$,即$\mu_i=\nu_i\enspace(i=1,2,\ldots,m)$,因此表示方式唯一.
          \end{proof}

          事实上关于这一定理我们有一个直接的推论
          \begin{corollary}{}{}
              若向量组外另一向量可由这一组向量线性表示,则
              \begin{enumerate}
                  \item \label{item:3:线性无关等价表示唯一:1}
                        向量组线性无关$\iff$表示方式唯一;

                  \item \label{item:3:线性无关等价表示唯一:2}
                        向量组线性相关$\iff$表示方式有无穷多种.
              \end{enumerate}
          \end{corollary}
          推论的证明非常简单,此处考虑到读者可能处于初学阶段,给出证明范例:

          \begin{proof}
              我们设向量组为$\alpha_1,\alpha_2,\ldots,\alpha_m$,向量组外的向量为$\beta$. 对于 \ref*{item:3:线性无关等价表示唯一:1},向量组线性无关$\implies$表示方式唯一就是\autoref{thm:线性无关等价表示唯一} 的直接结论,因此我们只需考虑表示方式唯一$\implies$向量组线性无关. 利用反证法,假设向量组线性相关,则存在不全为0的$\lambda_1,\lambda_2,\ldots,\lambda_m$使得
              \begin{equation}\label{eq:3:线性无关等价推论1}
                  0=\lambda_1\alpha_1+\lambda_2\alpha_2+\cdots+\lambda_m\alpha_m.
              \end{equation}
              由于$\beta$可由$\alpha_1,\alpha_2,\ldots,\alpha_m$线性表示,因此存在$\mu_1,\mu_2,\ldots,\mu_m$使得
              \begin{equation}\label{eq:3:线性无关等价推论2}
                  \beta=\mu_1\alpha_1+\mu_2\alpha_2+\cdots+\mu_m\alpha_m.
              \end{equation}
              事实上,我们只需将\autoref{eq:3:线性无关等价推论1} 两边乘以任意的$k\in\mathbf{F}$($\mathbf{F}$为向量组所在线性空间定义的数域),然后加到\autoref{eq:3:线性无关等价推论2} 的两边即可得到
              \[\beta=(\mu_1+k\lambda_1)\alpha_1+(\mu_2+k\lambda_2)\alpha_2+\cdots+(\mu_m+k\lambda_m)\alpha_m.\]
              因此表示方式不唯一(且有无穷多种),与假设矛盾,因此向量组线性无关. 事实上这一证明也将 \ref*{item:3:线性无关等价表示唯一:2} 中向量组线性相关$\implies$表示方式有无穷多种证明给出,\ref*{item:3:线性无关等价表示唯一:2} 的另一边同样用反证法可以回到 \ref*{item:3:线性无关等价表示唯一:1} 的证明,由此推论得证.
          \end{proof}
\end{enumerate}

\section{基与维数}

\subsection{引入:向量组的秩与极大线性无关组}

在上一节中我们介绍了很基本的线性无关的等价表述,现在我们回到我们的主线,即我们希望解决有限维线性空间至少需要多少个向量张成的问题,接下来的讨论将逐步逼近问题的答案.
\begin{lemma}{}{线性相关性引理}
    设$\alpha_1,\alpha_2,\ldots,\alpha_m$线性相关,则有$j\in\{1,2,\ldots,m\}$使得:
    \begin{enumerate}
        \item \label{item:3:线性相关性引理:1}
              $\alpha_j \in \spa(\alpha_1,\alpha_2,\ldots,\alpha_{j-1})$;

        \item \label{item:3:线性相关性引理:2}
              从$\alpha_1,\alpha_2,\ldots,\alpha_m$中删去向量$\alpha_j$,剩余向量张成空间仍等于$\spa(\alpha_1,\alpha_2,\ldots,\alpha_m)$.
    \end{enumerate}
\end{lemma}
可能大家看见 \ref*{item:3:线性相关性引理:1} 的记号可能又有些许陌生了,但只需简单回顾线性扩张的定义,我们知道证明\ref*{item:3:线性相关性引理:1} 就是证明$\alpha_j$可以被$\alpha_1,\alpha_2,\ldots,\alpha_{j-1}$线性表示. 这一结论初看和\autoref{thm:线性无关等价表示唯一} 很类似,但细看发现不太一样:我们要求必须有一个向量可以由排列在它前面的向量线性表示,而非被其余所有向量线性表示. 因此这一结论并不平凡,证明的过程中也有一个技巧,我们给出证明供读者参考学习:

\begin{proof}
    由于$\alpha_1,\alpha_2,\ldots,\alpha_m$线性相关,因此存在不全为0的$\lambda_1,\lambda_2,\ldots,\lambda_m$使得
    \[\lambda_1\alpha_1+\lambda_2\alpha_2+\cdots+\lambda_m\alpha_m=0.\]

    设$j$是$1,2,\ldots,m$中使得$\lambda_j\neq 0$的最大者,则有
    \begin{equation}\label{eq:3:线性相关性引理}
        \alpha_j=-\frac{\lambda_1}{\lambda_j}\alpha_1-\frac{\lambda_2}{\lambda_j}\alpha_2-\cdots-\frac{\lambda_{j-1}}{\lambda_j}\alpha_{j-1}.
    \end{equation}
    因此$\alpha_j$可由$\alpha_1,\alpha_2,\ldots,\alpha_{j-1}$线性表示,即$\alpha_j\in\spa(\alpha_1,\alpha_2,\ldots,\alpha_{j-1})$,故 \ref*{item:3:线性相关性引理:1} 得证.

    接下来我们证明 \ref*{item:3:线性相关性引理:2}. 首先$\spa(\alpha_1,\ldots,\alpha_{j-1},\alpha_{j+1},\ldots,\alpha_m)\subseteq\spa(\alpha_1,\alpha_2,\ldots,\alpha_m)$是显然的,因为任意被$\alpha_1,\ldots,\alpha_{j-1},\alpha_{j+1},\ldots,\alpha_m$线性表示的向量实际上也是被
    \[\alpha_1,\alpha_2,\ldots,\alpha_m\]
    线性表示了,只是$\alpha_j$前的系数恒为0.

    然后证明另一边包含关系,即
    \[\spa(\alpha_1,\alpha_2,\ldots,\alpha_m)\subseteq\spa(\alpha_1,\ldots,\alpha_{j-1},\alpha_{j+1},\ldots,\alpha_m).\]
    任取$\beta\in\spa(\alpha_1,\ldots,\alpha_m)$,则存在$\mu_1,\mu_2,\ldots,\mu_m$使得
    \[\beta=\mu_1\alpha_1+\mu_2\alpha_2+\cdots+\mu_m\alpha_m.\]
    将$\alpha_j$用\autoref{eq:3:线性相关性引理} 表示,代入上式可得任意$\spa(\alpha_1,\ldots,\alpha_m)$中的向量都可以由\[\alpha_1,\ldots,\alpha_{j-1},\alpha_{j+1},\ldots,\alpha_m\]线性表示,因此$\beta\in\spa(\alpha_1,\alpha_2,\ldots,\alpha_{j-1},\alpha_{j+1},\ldots,\alpha_m)$,故引理得证.
\end{proof}

事实上 \ref*{item:3:线性相关性引理:1} 中证明最核心的步骤就是取$j$是$1,2,\ldots,m$中使得$\lambda_j\neq 0$的最大者,这一最大者是一定存在的,因为首先存在$\lambda_i\neq 0$,其次$\lambda_i\neq 0$的个数是有限的,因此一定存在最大者. 这一证明的技巧十分重要,通俗的记忆方法为``从右往左检查,找到第一个不为0的系数(即最大的不为0的系数)''. 我们给出一个推论,推论的证明思想就是如此,我们放在习题中供读者练习:
\begin{corollary}{}{}
    $\alpha_1,\alpha_2,\ldots,\alpha_m$线性相关(其中$\alpha_1\neq 0$)的充要条件是存在一个向量$\alpha_i\enspace(1<i\neq m)$可由$\alpha_1,\alpha_2,\ldots,\alpha_{i-1}$线性表示,且表示法唯一.
\end{corollary}
事实上这一推论也可以作为线性无关的等价表述之一.

接下来我们继续我们的主线思路,事实上\autoref{lem:线性相关性引理} 的 \ref*{item:3:线性相关性引理:2} 给我们了一个很重要的启示,即对于线性相关的向量组,我们丢弃其中某些(可以被其他向量线性表示)的向量后,张成的空间是不变的. 因此我们可以重复丢弃这样的向量,并仍然保持张成空间不变. 一个自然的问题是,这样丢弃的操作直到什么时候停止呢?

事实上答案也是非常自然的,即我们最后一次从向量组中丢弃向量(并保证张成的空间不变)后,剩余的向量组恰好线性无关时即可停止丢弃. 原因非常简单,因为如果这最后一次不丢弃,则根据\autoref{lem:线性相关性引理} 我们一定还能选出一个向量,使得丢弃这一向量后仍能保持张成空间不变. 但一旦丢弃向量后向量组线性无关,这时一定不能继续丢弃,例如这时剩余的线性无关向量组为$\beta_1,\ldots,\beta_m$,这时丢弃其中任意一个$\beta_i,\enspace i\in\{1,2,\ldots,m\})$,则原向量组张成的空间中,至少$\beta_i$无法被剩余向量组线性表示(否则$\beta_i$可以被$\beta_1,\ldots,\beta_{i-1},\beta_{i+1},\ldots,\beta_m$线性表示,则$\beta_1,\ldots,\beta_m$必线性相关),因此我们一定不能继续丢弃.

在上述过程中我们可以引入两个重要的概念,即向量组的秩和极大线性无关组:
\begin{definition}{}{}
    设向量组$S=\{\alpha_1,\alpha_2,\ldots,\alpha_m\}$张成的线性空间为$V$,若存在$S$的一个线性无关向量组$B=\{\alpha_{k1},\alpha_{k2},\ldots,\alpha_{kr}\}$,使得$V=\spa(B)$,则称$B$为$S$的一个\term{极大线性无关组}\index{jidaxianxing@极大线性无关组 (maximal linearly independent system)},并称极大线性无关组的长度$r=r(S)$为$S$的\term{秩}\index{zhi@秩 (rank)}.
\end{definition}
定义中``极大''一词我们只需简单思考前述过程即可明白其含义,因为我们要求丢弃后的向量组一旦线性无关就要停止继续丢弃向量,因此这一剩余向量组的长度一定是所有线性无关向量组中最大的.

要注意的是,极大线性无关组在本讲义以及其它教材(如丘维声老师的《高等代数》)中的定义都有所不同,实际上不同的版本只是为了顺应不同讲解思路而提出的,本质上并无区别,相信读者在完全理解本节内容后能认识到这一点.

由此我们关于有限维线性空间至少需要多少个向量张成的问题有了初步的解答,即如果我们已知这一线性空间是可以由某一向量组张成的,那么这一向量组的秩(即极大线性无关组的长度)就是张成空间需要的最少向量个数. 可能初看这一段话,其中出现的``极大''和``最小''容易导致思维的混乱,但我们可以用一句话清晰地总结:极大线性无关组的长度就是张成空间需要的最少向量个数(如果仍然混乱,我们可以回忆丢弃向量的过程:我们不断丢弃向量得到``最小''的仍然满足张成空间不变的向量组,而这一向量组必须是所有线性无关向量组中最长的,因为向量组丢到线性无关后不能再丢了).

\subsection{向量组的性质}

事实上,我们会有一个自然的疑问,即极大线性无关组的长度是否唯一?我们在丢弃向量的时候,如果向量的排序不同,我们丢弃的次序也可能不同,因此我们最终得到的极大线性无关组是有可能不同的. 但长度不同表明向量组的秩不唯一,这样向量组的秩就失去了很多研究价值——数学喜欢唯一确定的,例如数学分析中表达式的极限不唯一我们会称其极限不存在;又例如定积分的值如果可以是不唯一的,那么我们一定会重新思考积分的定义,否则面积、体积甚至物理中的很多问题都会产生意义不明的多解.

因此我们需要尝试证明极大线性无关组的长度是唯一的,我们从下面这一非常重要的定理开始:
\begin{theorem}{}{线性表示}
    设$V(\mathbf{F})$中向量组$ \beta_1,\beta_2,\ldots,\beta_s $的每个向量可由另一向量组$\alpha_1,\alpha_2,\ldots,\alpha_r$线性表示. 若$s>r$,则$ \beta_1,\beta_2,\ldots,\beta_s $线性相关.
\end{theorem}
这一定理的等价(逆否)命题为,$ \beta_1,\beta_2,\ldots,\beta_s $线性无关则必有$s\leqslant r$.

这一定理可通俗概括为:多的向量组可以被少的向量组线性表示,多的一定线性相关. 反过来说,线性无关的向量只能被等长或更长的向量组线性表示. 定理的证明思想上非常简单,但写起来可能有些许复杂,我们给出证明:

\begin{proof}
    设 $\beta_j = \displaystyle\sum_{i = 1}^r \lambda_{ij} \alpha_i,\enspace \lambda_{ij} \in \mathbf{F},\enspace j = 1, 2, \ldots, s$. 由线性相关的定义,再设
    \[x_1\beta_1 + x_2\beta_2 + \cdots + x_s\beta_s = 0,\]
    即
    \[\sum_{j = 1}^s x_j\beta_j = \sum_{j = 1}^s x_j\left(\sum_{i = 1}^r \lambda_{ij} \alpha_i\right) = \sum_{i = 1}^r \left(\sum_{j = 1}^s \lambda_{ij}x_j\right)\alpha_i = 0\]
    事实上我们现在只需证明存在一组不全为零的 $x_1, x_2, \ldots, x_s$ 使得上式成立即可,因为这就是线性相关的定义. 因此我们只需要找出这么一组不全为零的数即可,怎么寻找呢?我们发现若 $\alpha_i$ 前的系数均取 0,则此方程必然成立,我们看看这种情况下能不能找到一组解,事实上此时有
    \[\sum_{j = 1}^s \lambda_{ij}x_j = 0,\enspace i = 1, \ldots, r.\]
    此为关于 $x_1, x_2, \ldots, x_s$ 的齐次线性方程组,其方程个数 $r$ 小于未知数数量 $s$,因此此方程组必然有非零解,于是我们就找到了一组不全为零的 $x_1, x_2, \ldots, x_s$ 使式子成立,故 $ \beta_1,\beta_2,\ldots,\beta_s $ 线性相关.
\end{proof}

事实上,\autoref{thm:线性表示} 因其重要性又被称为源泉定理,因为我们可以基于此得到大量的推论,下面我们将给出几个简单的作为代表,习题中会出现更为复杂的应用:
\begin{example}{}{线性表示推论}
    证明以下\autoref*{thm:线性表示} 的推论:
    \begin{enumerate}[label=(\arabic*)]
        \item 若向量组$B_1$可以被向量组$B_2$线性表示,则有$r(B_1)\leqslant r(B_2)$;

        \item \label{item:3:线性表示推论:2}
              设$B_1$和$B_2$是两个线性无关向量组,若$B_1$可以被$B_2$线性表示,$B_2$也可以被$B_1$线性表示,则$B_1$和$B_2$长度相等.
    \end{enumerate}
\end{example}

\begin{proof}
    \begin{enumerate}
        \item $B_1$ 可被其极大线性无关组 $A_1$ 表示,$B_2$ 可被其极大线性无关组 $A_2$ 表示,所以原条件等价于 $A_1$ 可以被 $A_2$ 线性表示. 而由极大线性无关组的定义,$A_1, A_2$ 中的向量个数分别是 $r(B_1), r(B_2)$,根据\autoref{thm:线性表示},有 $r(B_1)\leqslant r(B_2)$.

        \item 因为 $B_1, B_2$ 可以互相表示,所以 $r(B_1) \leqslant r(B_2),r(B_2) \leqslant r(B_1)$,所以 $r(B_1) = r(B_2)$. 又极大线性无关组的秩就是其向量个数,所以 $B_1, B_2$ 长度相等.
    \end{enumerate}
\end{proof}

事实上,\autoref{ex:线性表示推论} \ref*{item:3:线性表示推论:2} 中两个向量组$B_1$和$B_2$可以互相表示也可以称$B_1$和$B_2$等价. 这里的等价和\autoref{def:等价关系} 中描述的等价关系一致,即向量组等价同样满足自反性、对称性和传递性,即
\begin{enumerate}
    \item 自反性:任意向量组 $B$ 本身总是与自己等价,即向量组本身可以由本身表示;

    \item 对称性:设向量组 $B_1$ 等价于向量组 $B_2$,则向量组 $B_2$ 等价于向量组 $B_1$,因为它们可以相互表示;

    \item 传递性:设向量组 $B_1$ 等价于向量组 $B_2$,向量组 $B_2$ 等价于向量组 $B_3$,则向量组 $B_1$ 等价于向量组 $B_3$. 因为 $B_1$ 和 $B_2$ 可以相互表示,$B_2$ 和 $B_3$ 可以相互表示就有 $B_1$ 和 $B_3$ 可以相互表示.
\end{enumerate}
三个条件的成立是显然的,我们不再赘述,接下来我们基于等价向量组的定义给出\autoref{thm:线性表示} 的进一步结论,直至证明向量组的秩唯一:
\begin{corollary}{}{}
    关于等价的向量组,我们有如下结论:
    \begin{enumerate}
        \item 向量组与其极大线性无关组等价;

        \item 向量组的任意两个极大线性无关组等价;

        \item 向量组的任意两个极大线性无关组长度相等,即向量组的秩唯一.
    \end{enumerate}
\end{corollary}

\begin{proof}
    \begin{enumerate}
        \item 依据极大线性无关组的定义,并且注意到极大线性无关组是原向量组的子集即可;

        \item 设向量组 $B$ 的任意两个极大线性无关组为 $A_1, A_2$,由定义得 $B$ 可被 $A_1$ 表示,也可被 $A_2$ 表示,而 $A_1 \subseteq B, A_2 \subseteq B$,所以 $A_1, A_2$ 可以相互表示;

        \item 由上知向量组的任意两个极大线性无关组是等价的,结合\autoref{ex:线性表示推论} \ref*{item:3:线性表示推论:2} 即可得到二者长度相等,由向量组的秩的定义可知其唯一.
    \end{enumerate}
\end{proof}

由此我们证明了向量组的秩是唯一的,因此这一定义对我们将来的研究非常友好.

\subsection{基与维数}

在前几小节中,我们讨论了这一问题:给定向量组$B$,我们能否选出一个长度最小的向量组$B_1$使其张成的空间与$B$能张成的空间相同. 接下来我们讨论更一般化的情形,即我们不给定向量组$B$,直接讨论能张成一个线性空间的线性无关向量组.
\begin{definition}{}{}
    若线性空间$V(\mathbf{F})$的有限子集$B=\{\alpha_1,\alpha_2,\ldots,\alpha_n\}$线性无关,且$\spa(B) = V$,则称$B$为$V$的一组基,并称$n$为$V$的维数,记作$\dim V = n$.
\end{definition}

关于基与维数的定义,我们有以下几点需要强调:
\begin{enumerate}
    \item \label{item:3:基与维数:1}
          我们有一个自然的问题:有限维线性空间是否一定有基,若是,则上述定义的基和维数对所有有限维线性空间都是存在的. 事实上结论是显然的. 根据定义,有限维线性空间$V$一定能被其某一有限子集$S$张成,我们根据求取极大线性无关组的算法取出$S$的极大线性无关组$B$,则$B$一定是$V$的基.

    \item 由 \ref*{item:3:基与维数:1} 我们发现,基的存在依赖于极大线性无关组的存在,二者只是在定义上有差别:极大线性无关组是一个向量组的最短等价向量组,而基是张成线性空间的最短向量组. 但二者本质统一,实际上极大线性无关组就是它能张成的线性空间的一组基,其长度(向量组的秩)也就是线性空间的维数.

    \item 有限维线性空间的基不一定唯一,但它们的长度必定唯一(即维数唯一). 这一推导和向量组的秩唯一完全一致. 我们可以假设有限维线性空间$V$有两组基$B_1$和$B_2$,根据基的定义(即它们可以张成$V$,也就是可以表示出$V$中的所有向量). 因此$B_1$中的每一个向量都可以由$B_2$线性表示,反之亦然,因此$B_1$和$B_2$等价,由此我们可以得到$B_1$和$B_2$的长度相等,即因此有限维线性空间维数唯一.

    \item 我们还需要提及一个概念:自然基. 例如三维空间的自然基为$\vec{e}_1=(1,0,0),\vec{e}_2=(0,1,0),\vec{e}_3=(0,0,1)$. $n$维空间也有类似的推广(即$n$个只有一位为 1 其余全为 0 的向量. 此后若没有特殊说明,$\vec{e}_i$就表示$\mathbf{R}^n$第$i$位为1,其余位置为0的自然基). 对于多项式我们则将$1,x,x^2,\ldots$称为自然基,矩阵、函数等构成的线性空间也有相关的常用的基.

    \item \label{item:3:基与维数:5}
          基与维数的意义可以由这个性质反映出来:对一 $n$ 维线性空间 $V$ 而言,其中的任意 $n + 1$ 个向量必然线性相关,而其中的任意 $n - 1$ 个向量必然无法张成空间 $V$,这也是\autoref{thm:线性表示} 的直接推论.
\end{enumerate}

事实上,定义出基和维数之后我们对线性空间的研究方式就更明朗了:我们从开始的令人眼花缭乱的 8 条运算性质,利用这些线性运算的特点导出线性扩张与子空间的关联,然后经过线性相关性的讨论最终得到线性空间的本质结构实际上就是可以由基经过一系列线性运算扩张而来,因此我们对线性空间的研究很多时候只需要研究其基和维数即可,由此我们的抽象上升一层,即我们不需要观察线性空间中无限个向量,事实上只需要研究有限个向量的性质即可对整个线性空间有较为全面的了解. 实际上这一思想与之后我们得到矩阵等讨论是密切相关的,因此在我们整个向着对线性方程组解的结构的讨论的路径中也称得上是一块关键的里程碑.

我们经常会遇到验证线性空间的基的问题(求解基的题目最后往往也需要验证你写出的向量组确实是基),我们主要有如下两个角度:
\begin{enumerate}
    \item 根据定义,我们只需验证基的两个条件:线性无关和张成空间. 线性无关利用定义即可,张成空间则需要验证任意向量都可以由基线性表示.

    \item 若我们能确认线性空间$V$的维数$\dim V$,那么我们只需找到$\dim V$个线性无关的向量即可,因为它们必然是$V$的基. 这一结论的证明是容易的,在下面的例题中我们给出一个更一般的结论的证明供读者参考.
\end{enumerate}

\begin{example}{}{}
    在$n$维线性空间$V$中,$n$个向量$\alpha_1,\ldots,\alpha_n$线性无关的充要条件是它们可以线性表示出$V$中的任意向量.
\end{example}

\begin{proof}
    充分性:因为 $\alpha_1,\ldots,\alpha_n$ 可以线性表示出 $V$ 中的任意向量,所以 $V$ 的一组基 $\beta_1, \ldots, \beta_n$ 也能由 $\alpha_1,\ldots,\alpha_n$ 表示. 而由基的性质,$\alpha_1,\ldots,\alpha_n$ 又能被 $\beta_1, \ldots, \beta_n$ 表示,所以这两个向量组等价,$\alpha_1,\ldots,\alpha_n$ 的秩就是 $n$,所以 $\alpha_1,\ldots,\alpha_n$ 线性无关.

    必要性:由 \ref*{item:3:基与维数:5} 可知,$\forall \beta \in V, \alpha_1, \ldots, \alpha_n, \beta$ 必线性相关,又 $\alpha_1,\ldots,\alpha_n$,由\autoref{thm:线性无关等价表示唯一} 可知,$\beta$ 可以被 $\alpha_1,\ldots,\alpha_n$ 唯一表示,因此 $V$ 中的任意向量都可以被 $\alpha_1,\ldots,\alpha_n$ 线性表示.
\end{proof}

除此之外,我们也在此给出一些求解或验证线性空间的基和维数的基本例题,在习题以及后续章节中会有更多的例子.
\begin{example}{}{不同数域的维数}
    证明:线性空间$\mathbf{C}(\mathbf{C})$维数为1,不同于线性空间$\mathbf{C}(\mathbf{R})$维数为2.
\end{example}

\begin{proof}
    对于线性空间$\mathbf{C}(\mathbf{C})$中的任一向量 $a + b\i$,其总可以被向量 $\alpha = 1$表示,数乘系数为 $\lambda = a + b\i$,所以$\mathbf{C}(\mathbf{C})$维数为1;对于线性空间$\mathbf{C}(\mathbf{R})$,向量组 $\alpha = 1, \beta = \i$线性无关,且任一向量 $u = a + b\i = a·\alpha + b·\beta$,可被 $\alpha, \beta$表示,所以$\mathbf{C}(\mathbf{R})$维数为2.
\end{proof}

\begin{example}{}{线性方程组的解的维数}
    设 $V_1, V_2$ 分别是 $\mathbf{R}$ 上齐次线性方程组 $x_1 = x_2 = \cdots = x_n = 0$ 和 $x_1 + x_2 + \cdots + x_n = 0$ 的解空间,求 $V_1, V_2$ 的基与维数.
\end{example}

在\autoref{ex:常见子空间}的\ref{item:2:常见子空间:3}中我们已经知道,齐次线性方程组的所有解构成一个线性空间(即解空间),这一例子就是希望读者能够通过求解基和维数的进一步理解解空间的基本结构.

\begin{solution}
    直接求解两个线性方程组. $V_1$ 对应的线性方程组我们只需要将连等号拆开成多个方程即可:
    \[\begin{cases}
        x_1 - x_2 = 0, \\
        x_2 - x_3 = 0, \\
        \cdots \\
        x_{n-1} - x_n = 0.
    \end{cases}\]
    其它的方程例如 $x_1 - x_3 = 0$ 等都可以由这些方程表示出来,因此消元过程中会被直接消去,我们只需要求解上述方程组即可. 很容易看出方程组的解为
    \[(x_1,x_2,\cdots,x_n)^\mathrm{T} = k(1,1,\ldots,1)^\mathrm{T}\]
    其中 $k\in\mathbf{R}$,根据线性扩张的定义,
    \[V_1 = \spa\{(1,1,\ldots,1)\}\]
    所以 $V_1$ 的一组基为 $(1,1,\ldots,1)$,$\dim V_1 = 1$. 接下来是 $V_2$,它对应的线性方程组的解可以直接写出为
    \begin{align*}
        &\quad\ (x_1,x_2,\ldots,x_n)^\mathrm{T} \\
        &= k_1(-1,1,0,\ldots,0)^\mathrm{T} + k_2(-1,0,1,\ldots,0)^\mathrm{T} + \cdots + k_{n-1}(-1,0,\ldots,0,1)^\mathrm{T}
    \end{align*}
    其中 $k_1,k_2,\ldots,k_{n-1}\in\mathbf{R}$,根据线性扩张的定义
    \[V_2 = \spa\{(-1,1,0,\ldots,0)^\mathrm{T},(-1,0,1,\ldots,0)^\mathrm{T},\ldots,(-1,0,\ldots,0,1)^\mathrm{T}\}\]
    并且这些向量都是线性无关的,故而我们得到 $V_2$ 的一组基为
    \[(-1,1,0,\ldots,0)^\mathrm{T},(-1,0,1,\ldots,0)^\mathrm{T},\ldots,(-1,0,\ldots,0,1)^\mathrm{T},\]
    进一步有 $\dim V_2 = n - 1$.
\end{solution}

我们发现,上述齐次线性方程组的解空间实际上就是由基础解系作为一组基张成的线性空间,因为上述方程的基础解系是线性无关向量组. 事实上这一结论是具有一般性的,即任意齐次线性方程组的解空间都可以由基础解系作为一组基张成,这一结论我们会在\autoref{thm:齐次维数}中给出证明.

\begin{example}{}{}
    证明:$1,(x-5)^2,(x-5)^3$是$\mathbf{R}[x]_4$的子空间$U$的一组基,其中$U$定义为
    \[U=\{p\in\mathbf{R}[x]_4 \mid p'(5)=0\}.\]
\end{example}

\begin{proof}
    易知$1,(x-5)^2,(x-5)^3$线性无关,且这三个向量都属于子空间$U$,下证$U$的维数是 3.

    设 $p = a + bx + cx^2 + dx^3 \in U$. 因为 $p'(5) = 0$,即 $b + 10c +75d = 0$,所以将$b$代入后有 $p = a + c·(-10x + x^2) + d·(-75x + x^3)$,因此$U$中任意向量可被$1, -10x + x^2, -75x + x^3$表示,所以$U$的维数是 3,进而由基的性质可知$1,(x-5)^2,(x-5)^3$是$U$的一组基.
\end{proof}

我们在后续讨论中经常会涉及子空间和原空间之间的关联,特别是它们的基之间的关联,下面这一定理能很好地满足我们的需求:
\begin{theorem}{}{}
    如果$W$是$n$维线性空间$V$的一个子空间,则$W$的基可以扩充为$V$的基.
\end{theorem}
这一定理的应用非常广泛,我们未来会经常使用扩基的方式进行研究. 这一定理的正确性我们将在介绍极大线性无关组的方法后通过算法的形式给出,即我们会说明这样的扩基一定能通过一个简单的流程得到.

实际上还有关于向量组的秩、基与维数有关的很多结论,事实上都可以由前述的定理推导而来,很多结论事实上都非常自然,我们将习题中展示. 考虑到本讲概念、定理内容多而杂. 我们在本讲最后也会给出一个思维导图,读者可以参考.

最后我们再说明有限维线性空间和无限维线性空间的定义,本课程研究的内容都在有限维线性空间,如果少数时间拓展至无限维空间我们会给出说明:
\begin{definition}{}{}
    $V(\mathbf{F})$称为有限维线性空间,如果$V$中存在一个有限子集$S$使得$\spa(S)=V$,反之称为无限维线性空间.
\end{definition}

\begin{example}{}{}
    证明:$\mathbf{R}[x]_3$是有限维线性空间,$\mathbf{R}[x]$是无限维线性空间.
\end{example}

\begin{proof}
    \begin{enumerate}
        \item 显然$\mathbf{R}[x]_3$的有限子集$S=\{1,x,x^2\}$可以张成$\mathbf{R}[x]_3$,因此$\mathbf{R}[x]_3$是有限维线性空间;

        \item 对于$\mathbf{R}[x]$,我们只需证明其任意有限子集都无法张成其本身. 我们取其任意有限子集,则其中多项式元素的次数一定有最大值,我们记为$m$,那么$z^{m+1}$以及更高次数的无法被表示,因此$\mathbf{R}[x]$是无限维线性空间.
    \end{enumerate}
\end{proof}

事实上,基于这一例子我们也可以说明定义在$[a,b]$上的连续实值函数全体$C[a,b]$在实数域上构成的线性空间是无限维的,因为$C[a,b]$包含全体多项式,而全体多项式都无法被有限个向量表示,更不用说全体连续函数了. 当然,事实上\autoref{ex:函数和数列线性空间} 中的另一个数列的例子也构成无限维线性空间,这一点我们将在行列式一节作为习题,因为使用范德蒙行列式的结论更为便捷.

\subsection{极大线性无关组的求法}

我们在前述讲解中实际上已经给出一个求解极大线性无关组的方法,即不断丢弃线性相关的向量,最后一次从向量组中丢弃向量(并保证张成的空间不变)后,剩余的向量组恰好线性无关时即可停止丢弃. 然而这样的方法显然缺乏章法:我们也很难一眼看出哪个向量是线性相关的,也很难直接判断出剩余向量是否线性无关,因此我们需要一种更为程序化的方法. 为了得到这一方法,我们需要首先证明如下引理:

\begin{lemma}{初等行变换不改变列的线性相关性}{初等行变换不改变列的线性相关性}
    对一个矩阵做三类初等行变换均不改变矩阵的列的线性相关性.
\end{lemma}

\begin{proof}
    我们只证明倍加变换的性质,其余两种变换的性质留作习题. 设矩阵为

    \[\begin{pmatrix}
            a_{11} & a_{12} & \cdots & a_{1n} \\
            a_{21} & a_{22} & \cdots & a_{2n} \\
            \vdots & \vdots & \ddots & \vdots \\
            a_{m1} & a_{m2} & \cdots & a_{mn}
        \end{pmatrix}\]

    对其进行倍加变换,即将第 $i$ 行的 $k$ 倍加到第 $j$ 行上,即第 $j$ 行变为 $a_{j1} + ka_{i1}, a_{j2} + ka_{i2}, \ldots, a_{jn} + ka_{in}$. 记原先的列向量为 $S_A = \{\alpha_1, \alpha_2, \ldots, \alpha_n\}$,新的列向量为 $S_B = \{\beta_1, \beta_2, \ldots, \beta_n\}$.

    \begin{enumerate}
        \item $S_A$ 线性相关 $\implies$ $S_B$ 线性相关:若 $S_A$ 线性相关,则存在不全为零的数 $x_1, x_2, \ldots, x_n$ 使得 $x_1\alpha_1 + x_2\alpha_2 + \cdots + x_n\alpha_n = 0$. 由于 $S_B$ 与 $S_A$ 仅有第 $j$ 行不同,所以我们仅需要判断第 $j$ 行在系数 $x_1, x_2, \ldots, x_n$ 下的线性组合是否为 0 即可. 事实上我们有 $x_1(a_{j1} + ka_{i1}) + x_2(a_{j2} + ka_{i2}) + \cdots + x_n(a_{jn} + ka_{in}) = x_1a_{j1} + x_2a_{j2} + \cdots + x_na_{jn} + k(x_1a_{i1} + x_2a_{i2} + \cdots + x_na_{in}) = 0$,故 $S_B$ 线性相关.
        \item $S_A$ 线性无关 $\implies$ $S_B$ 线性无关:若 $S_A$ 线性无关,使用反证法,假设 $S_B$ 线性相关,但我们知道$S_A$ 实际上就是 $S_B$ 的第 $j$ 行减去第 $i$ 行的 $k$ 倍,因此 $S_B$ 变换到 $S_A$ 可以视为做了一个倍加变换,所以根据证明的第 1 点可知 $S_A$ 线性相关,出现矛盾.
    \end{enumerate}
\end{proof}

接下来的例子使用上面的引理,为极大线性无关组的求解提供了一种``通用而简便的方法''.

\begin{example}{}{}
    已知 $\mathbf{R}^4$ 的一个子集 $S = \{a_1, a_2, a_3, a_4\}$, 其中
    \[
        a_1 = (1,1,0,1), \quad a_2 = (0,1,2,4), \quad
        a_3 = (2,1,-2,2), \quad a_4 = (0,1,1,1).
    \]
    试求 $\spa(S)$ 的维数及其一组基$B$。
\end{example}

\begin{solution}

    显然这一问题的关键就是求出 $S$ 的一组极大线性无关组,因为 $S$ 的任一极大线性无关组 $B$ 都是 $\spa(S)$ 的基. 为了达到这一目标,我们首先将四个向量按列排成矩阵,即

    \begin{equation} \label{eq:极大线性无关组:1}
        \begin{pmatrix}
            1 & 0 & 2  & 0 \\
            1 & 1 & 1  & 1 \\
            0 & 2 & -2 & 1 \\
            1 & 4 & -2 & 1
        \end{pmatrix}
    \end{equation}

    对上述矩阵作初等行变换,所得的阶梯矩阵为

    \begin{equation} \label{eq:极大线性无关组:2}
        \begin{pmatrix}
            1 & 0 & 2  & 0 \\
            0 & 1 & -1 & 0 \\
            0 & 0 & 0  & 1 \\
            0 & 0 & 0  & 0
        \end{pmatrix}
    \end{equation}

    根据上面的引理,我们知道矩阵 \eqref{eq:极大线性无关组:1} 中的列的线性相关性与矩阵 \eqref{eq:极大线性无关组:2} 中的列的线性相关性是一致的,而 \eqref{eq:极大线性无关组:2} 中的经过初等变换的线性相关关系非常容易看出来:我们直接找每一行第一个非零元素所在的列对应的向量即可得到极大线性无关组,即 $S$ 的一个极大线性无关组为 $\{a_1, a_2, a_4\}$. 故 $a_1, a_2, a_4$ 是 $\spa(S)$ 的一组基,$\spa(S)$ 的维数为 3.
\end{solution}

我们总结以上方法的关键步骤,抽象出一般的极大线性无关组的求法:
\begin{lemma}{极大线性无关组的求法}{极大线性无关组的求法}

    我们将题目给定的向量按列排成矩阵,然后将矩阵作初等变换化成阶梯矩阵,找到每一行第一个非零元素所在的列,提取出原矩阵对应列的向量即可.
\end{lemma}

实际上如果能一眼看出结果的也不必如此麻烦(当然题目直接要求极大线性无关组还是应当写具体过程的). 除此之外,显然的一点是注意极大线性无关组是不唯一的,上面给出的程式化的方法得到的结果只是其中一个(例如选择 $a_1,a_3,a_4$ 显然也是一组极大线性无关组).

学会求解极大线性无关组后,我们还能解决一个重要的问题,就是如何扩张一个线性无关向量组成为线性空间的一组基. 之前我们只说明了这样的扩张是存在的,但具体如何取到并没有给出. 虽然在未来实际应用中我们大部分时候可能只需要扩充一两个向量就行,很多时候我们随手取或者依靠之后的行列式等工具就很好解决. 但实践中我们发现很多同学在教材没给出固定算法的情况下完全无法接受``随手取''这样的描述,因此在此笔者还是给出一种虽然暴力但一定有效的算法.

设线性空间$V$维数为$n$,我们已有的线性无关向量组为$B=\{\alpha_1,\alpha_2,\ldots,\alpha_s\},\enspace s<n$. 我们的目标是将这个向量组扩充为$V$的一组基$B'=\{\alpha_1,\alpha_2,\ldots,\alpha_s,\alpha_{s+1},\ldots,\alpha_n\}$,我们的算法如下:
\begin{enumerate}
    \item 首先,如果$V$不是$\mathbf{F}^n$空间,我们取$B$在$V$的任意一组基下的坐标(如果有自然基最好取自然基方便计算);

    \item 任取$V$的一组基$B_0=\{\beta_1,\beta_2,\ldots,\beta_n\}$,这组基和前面取的是否一致无所谓(看了后面的例子就明白了). 我们得到了一个新的向量组$B_1=\{\alpha_1,\alpha_2,\ldots,\alpha_s,\beta_1,\beta_2,\ldots,\beta_n\}$;

    \item 求$B_1$的极大线性无关组即可,特别注意最后选向量的时候不能把$\alpha_1,\alpha_2,\ldots,\alpha_s$扔掉了,只能扔后面的向量,因为我们求的是从$\alpha_1,\alpha_2,\ldots,\alpha_s$扩充来的一组基;

    \item 最后将我们上面得到的坐标结合第一步取的$V$的基得到由$B$扩充而来的一组基.
\end{enumerate}

\begin{example}{}{}
    设$V=\mathbf{R}[x]_4$,我们已有向量组$B=\{1+x,x^3+x^2+3x\}$,请将其扩充为$V$的一组基.
\end{example}

\begin{solution}
    按讲义中的方法,取定 $\mathbf{R}[x]_4$ 的自然基,给出$B$中向量的坐标
    \[\alpha_1 = (1, 1, 0, 0), \alpha_2 = (0, 3, 1, 1).\]
    再取一组基
    \[\beta_1 = (1, 0, 0, 0), \beta_2 = (0, 1, 0, 0), \beta_3 = (0, 0, 1, 0), \beta_4 = (0, 0, 0, 1).\]
    将这 6 个向量排列成矩阵,求解极大线性无关组.
    \[ \begin{pmatrix}
            1 & 0 & 1 & 0 & 0 & 0 \\
            1 & 3 & 0 & 1 & 0 & 0 \\
            0 & 1 & 0 & 0 & 1 & 0 \\
            0 & 1 & 0 & 0 & 0 & 1
        \end{pmatrix}
        \xLongrightarrow{\text{初等行变换}}
        \begin{pmatrix}
            1 & 0 & 1  & 0 & 0 & 0  \\
            0 & 1 & 0  & 0 & 0 & 1  \\
            0 & 0 & -1 & 1 & 0 & -3 \\
            0 & 0 & 0  & 0 & 1 & -1
        \end{pmatrix} \]
    则可取 $\alpha_1, \alpha_2, \beta_1, \beta_3$作为$V$的一组基,即$\{1 + x, 3x + x^2 + x^3, 1, x^2\}$.
\end{solution}

\section{向量的坐标} \label{sec:向量的坐标}

坐标的概念实际上我们已经熟悉,例如高中所学的平面向量的坐标表示就是向量在二维平面的基$(1,0),(0,1)$下的坐标表示. 我们现在将这个概念拓展到更一般的线性空间:
\begin{definition}{}{}
    设$B=\{\beta_1,\beta_2,\ldots,\beta_n\}$是$n$维线性空间$V(\mathbf{F})$的一组基,如果$V$中元素$\alpha$表示为$\alpha=a_1\beta_1+a_2\beta_2+\cdots+a_n\beta_n$,则其系数组$a_1,a_2,\ldots,a_n$称为$\alpha$在基$B$下的坐标,记为$\alpha_B=(a_1,a_2,\ldots,a_n)$.
\end{definition}

\begin{example}{}{}
    分别求$p(x)=a_0+a_1x+a_2x^2$在基$B_1=\{1,x,x^2\}$和$B_2=\{1,x-1,(x-1)^2\}$下的坐标.
\end{example}

\begin{solution}
    通过待定系数法解方程即可.

    在 $B_1$ 下:$(a_0, a_1, a_2)$. 在 $B_2$ 下:$(a_0 + a_1 + a_2, a_1 + 2a_2, a_2)$.
\end{solution}
关于向量的定义我们有以下几点需要强调:
\begin{enumerate}
    \item \hypertarget{基底的矩阵写法}若向量$\alpha$在基$\beta_1,\beta_2,\ldots,\beta_n$下的坐标为$\alpha_B=(a_1,a_2,\ldots,a_n)$,则我们也可以写为
          \[\alpha=(\beta_1, \beta_2, \ldots, \beta_n)\begin{pmatrix} a_1 \\ a_2 \\ \vdots \\ a_n \end{pmatrix}\]
          这一记号与\autoref{eq:矩阵左乘列向量}中给出的规则是一致的(虽然基一般不是数域$\mathbf{F}$或者向量空间$\mathbf{F}^m$中的元素,但形式上是一致的).

    \item 回顾坐标的定义,事实上取向量的坐标相当于将一个$n$维线性空间$V$中的元素$\alpha$转化为$\mathbf{F}^n$中的元素$\alpha_B$. 因此我们可以将求向量坐标的过程视为一个映射$\varphi_B$(下标 $B$ 代表我们取的基),即从原线性空间$V(\mathbf{F})$到$\mathbf{F}^n$的坐标映射,那么我们很容易看到以下结果:
          \begin{enumerate}
              \item 同态性:设$n$维线性空间$V$的一组基为$B=\{\beta_1,\ldots,\beta_n\}$,则对于任意的$\alpha,\beta\in V$,有
                    \begin{gather*}
                        \alpha=a_1\beta_1+\cdots+a_n\beta_n,\quad\beta=b_1\beta_1+\cdots+b_n\beta_n,\\
                        \alpha+\beta=(a_1+b_1)\beta_1+\cdots+(a_n+b_n)\beta_n,\\
                        \lambda\alpha=(\lambda a_1)\beta_1+\cdots+(\lambda a_n)\beta_n,
                    \end{gather*}
                    则有
                    \begin{gather*}
                        \varphi_B(\alpha+\beta)=(a_1+b_1,\ldots,a_n+b_n)=(a_1,\ldots,a_n)+(b_1,\ldots,b_n)=\varphi_B(\alpha)+\varphi_B(\beta),\\
                        \varphi_B(\lambda\alpha)=(\lambda a_1,\ldots,\lambda a_n)=\lambda(a_1,\ldots,a_n)=\lambda\varphi_B(\alpha).
                    \end{gather*}
                    因此坐标映射保持原向量与坐标运算一致,即是一个同态映射.

              \item 双射性:即坐标与向量是一一对应的:一个坐标可以确定唯一的向量,一个向量在基下表示的系数(即向量的坐标)也必然唯一(因为基是线性无关的).
          \end{enumerate}
          由此我们发现,坐标映射$\varphi_B$实际上是一个同构映射,这表明任意一个$n$维线性空间$V(\mathbf{F})$都与$\mathbf{F}^n$同构,这表明任意的线性空间的元素都可以与一个向量空间中的一般向量(即它的坐标)一一对应,并且它们之间的运算也是完全一致的. 因此我们研究任意的$n$维线性空间都可以转化为研究$\mathbf{F}^n$这一非常基本的空间. 这是十分有趣的事情,因为我们当时以抽象公理定义线性空间便是希望将几何上普通向量的性质抽象出来也能应用于其它的集合,现在我们发现任意线性空间都可以视为一个普通几何上的向量组成的线性空间. 于是,这些被公理化纳入线性空间研究范畴的集合,例如多项式、矩阵等,都可以在坐标的视角下视为普通向量组成的线性空间,从而可以更加方便地研究它们的性质.

    \item 由以上讨论我们可以知道:我们对各种各样的$n$维线性空间的研究都可以首先通过坐标转化为$\mathbf{F}^n$中的元素进行研究,例如
          \begin{example}{}{转化为坐标}
              求$\mathbf{R}[x]_4$中向量组$\{p_1=x^3-x^2+2x+4,p_2=3x^2+x+2,p_3=3x^3+7x+14,p_4=x^3-x^2+2x,p_5=2x^3+x^2+5x+6\}$的极大线性无关组.
          \end{example}
          \begin{solution}
              我们首先将所有多项式先转化为坐标,然后就会发现和\autoref{ex:求解极大线性无关组} 完全一致,最后将坐标转回多项式即可.
          \end{solution}

          事实上,将任意的线性空间转化为$\mathbf{F}^n$研究的思想是非常重要的,因为这可以带来进一步的抽象,即我们甚至可以遮蔽线性空间基的特点,只关注其维数进行研究,这与此后线性空间的同构以及矩阵表示都有密不可分的联系. 事实上,我们一直都在使用这一基本思想,我们每次设线性空间有一组基$\alpha_1,\ldots,\alpha_n$时,事实上我们只关注其维数$n$而遮蔽了基的特点:它可以是向量,可以是多项式,可以是矩阵、函数等等,但这些都不重要,我们都可以将这些元素视为几何空间$\mathbf{F}^n$中的向量,获得更直观的理解,从而可以忽视一些使我们理解困难的细节.

    \item 容易验证$\mathbf{R}^n$中的向量在自然基下的坐标实际上就是向量本身,例如$(x,y,z)=xe_1+ye_2+ze_3$,故在$\mathbf{R}^3$自然基下的坐标仍然为$(x,y,z)$,需要牢记,有时可以加速解题.
\end{enumerate}

\begin{summary}

    本节内容相对而言概念和定理非常多,涉及的题型也很多,因此我们在这里给出一个思维导图,供读者捋顺思路(读者也可以将其他看到过的,例如习题中的命题进一步加入思维导图).
    \begin{figure}[htbp]
        \centering
        \includegraphics[scale=0.6]{figs/3-1.png}
    \end{figure}

    事实上,与其他内容风格不一样的是,本讲中很大一部分的定理我们都给出了证明,一方面是为了提升阅读体验,防止在初学时就被多个``显然''等词汇困惑,另一方面也是希望读者能够从这些比较规范的证明中得到一些证明的技巧.

    也许读到这里很多读者都会有些迷惑与焦急——为什么我们仿佛在学习很多看起来十分抽象而且似乎没什么实际应用的知识呢?或许我们需要在这里给读者一个``定心丸''. 事实上,我们在上一讲中定义的线性空间运算法则就是从一般向量加法数乘运算法则抽象而来的最为抽象和基本的内容,我们仅仅建立在这一基础上,伴随着线性表示、线性扩张、线性相关等概念的提出,导出了(有限维)线性空间都具有一种统一的本质结构描述——基和维数,由此我们从抽象的运算规则走到了比较具体的结构. 在此基础上,我们基本上将单个线性空间的研究完成,之后我们将会讨论线性空间之间的关系——一方面可以定义线性空间之间的运算,我们将在下一讲详细介绍,另一方面可以建立两个线性空间之间的某种映射,在关于这种映射的讨论中我们会发现线性空间的本质结构是维数,甚至基之间的差异都可以完全被遮蔽(只需通过本讲介绍的坐标即可),然后我们对线性空间的认识便可以从某种比较抽象的结构走向大家熟悉的一定长度的向量,接下来便可以定义更为具象的矩阵. 这一路上我们实际上是从最为抽象的内容逐步定义概念,说明定理,走向具象的内容. 不同于一般线性代数从行列式、矩阵开始,这样的思路一定能让读者对线性代数有更为深刻的认识.

\end{summary}

\begin{exercise}
    \exquote[柯西]{给我五个系数,我将画出一头大象;给我六个系数,大象将会摇动尾巴.}

    \begin{exgroup}
        \item 下列命题是否正确?若正确请证明,否则举出反例.
        \begin{enumerate}
            \item 若 $\alpha_1, \ldots, \alpha_m \ (m > 2)$ 线性相关,则其中每一向量都是其余向量的线性组合;

            \item 若 $\alpha_1, \ldots, \alpha_m$ 线性无关,则其中每一向量不是其余向量的线性组合,这个命题的等价命题应如何叙述?

            \item $\alpha_1, \ldots, \alpha_m \ (m > 2)$ 线性无关的充要条件是任意两个向量都线性无关;

            \item  若 $\alpha_1, \alpha_2$ 线性相关,$ \beta_1, \beta_2$ 线性相关,则 $\alpha_1 + \beta_1, \alpha_2 + \beta_2$ 也线性相关;

            \item 若 $\alpha_1, \ldots, \alpha_n$ 线性无关,则 $\alpha_1 + \alpha_2, \alpha_2 + \alpha_3, \ldots, \alpha_{n-1} + \alpha_n, \alpha_n + \alpha_1$ 也线性无关;

            \item 若 $\alpha_1, \alpha_2, \alpha_3$ 线性相关,则 $\alpha_1 + \alpha_2, \alpha_2 + \alpha_3, \alpha_3 + \alpha_1$ 也线性相关;

            \item 设 $B = \{\alpha_1, \alpha_2, \alpha_3\}$ 是 $ \mathbf{R}^3 $ 的一组基,非零向量 $\alpha_0 \in \mathbf{R}^3$,则 $\{\alpha_0 + \alpha_1, \alpha_0 + \alpha_2, \alpha_0 + \alpha_3\}$ (其中三个向量均是非零向量) 也是 $\mathbf{R}^3$ 的一组基;

            \item  设 $B = \{\alpha_1, \alpha_2\}$ 是 $\mathbf{R}^2$ 的一组基,则 $\{\alpha_1 + \alpha_2, \alpha_1 - \alpha_2\}$ 也是 $\mathbf{R}^2$ 的一组基;

            \item 一个有限维线性空间内只含有有限个子空间;

            \item  若 $W_1, W_2$ 是 $ \mathbf{R}^n $ 的两个子空间,$B_1, B_2$ 分别是 $W_1, W_2$ 的基,则存在 $ \mathbf{R}^n $ 的一组基 $B$,使得 $B \supseteq B_1 \cup B_2$。

        \end{enumerate}
        \begin{answer}
            \begin{enumerate}
                \item 错. 反例:$\alpha_1=(1,0),\alpha_2=(2,0),\alpha_3=(0,1)$,则 $\alpha_1,\alpha_2,\alpha_3$ 线性相关而 $\alpha_3$ 不是 $\alpha_1.\alpha_2$ 的线性组合.

                \item 对. 该命题的等价命题(逆否命题)是:若存在一个向量是其余向量的线性组合,则 $\alpha_1,\ldots,\alpha_m$ 线性相关. 这正是定理 3.1 的内容,因而成立.

                \item 错. 反例:$\alpha_1=(1,0),\alpha_2=(0,1),\alpha_3=(1,1)$,则 $\alpha_1,\alpha_2,\alpha_3$ 两两无关,而三者线性相关. 可证两两无关是向量组无关的必要条件.

                \item 错. 反例:$\alpha_1=(1,0),\alpha_2=(0,0),\beta_1=(0,0),\beta_2=(0,1)$,有 $\alpha_1,\alpha_2$ 相关,$\beta_1,\beta_2$ 相关,而 $\alpha_1+\beta_1$ 与 $\alpha_2+\beta_2$ 线性无关.

                \item 错. 若 $\alpha_1,\ldots,\alpha_n$ 线性无关,有
                    \[\lambda_1\alpha_1+\cdots+\lambda_n\alpha_n\implies\lambda_1=\lambda_2=\cdots=\lambda_n=0,\]
                    判断 $\alpha_1+\alpha_2,\alpha_2+\alpha_3,\ldots,\alpha_n+\alpha_1$ 是否无关. 设
                    \[\lambda_1'(\alpha_1+\alpha_2)+\cdots+\lambda_n'(\alpha_n+\alpha_1)=0,\]
                    则
                    \[(\lambda_n'+\lambda_1')\alpha_1+(\lambda_1'+\lambda_2')\alpha_2+\cdots+(\lambda_{n-1}'+\lambda_{n}')\alpha_n=0,\]
                    则
                    \[\begin{cases} \begin{aligned}
                                \lambda_n'+\lambda_1'       & = 0               \\
                                                            & \vdotswithin{ = } \\
                                \lambda_{n-1}'+\lambda_{n}' & = 0               \\
                            \end{aligned} \end{cases}.\]
                    解该方程可得 $\lambda_n'=(-1)^n\lambda_1'$,因此当 $n$ 为偶数时,上述方程组有非零解,则向量组相关,而当 $n$ 为奇数时,向量组无关. 综上,该命题不成立.

                \item 对. 由定理 3.1,不妨设 $\alpha_3$ 可由 $\alpha_1,\alpha_2$ 线性表示,则 $\alpha_1+\alpha_2,\alpha_2+\alpha_3,\alpha_3+\alpha_1$ 均可由 $\alpha_1,\alpha_2$ 线性表示,再由定理 3.3 可知,$\alpha_1+\alpha_2,\alpha_2+\alpha_3,\alpha_3+\alpha_1$ 线性相关.

                \item 错. 反例:取 $\alpha_0=\alpha_1-\alpha_2-\alpha_3$,则 $\alpha_0+\alpha_1=(\alpha_0+\alpha_2)+(\alpha_0+\alpha_3)$,三者线性相关,不是 $\mathbf{R}^3$ 的基.

                \item 对. 判断 $\alpha_1+\alpha_2$ 与 $\alpha_1-\alpha_2$ 是否无关.
                    \[\lambda_1(\alpha_1+\alpha_2)+\lambda_2(\alpha_1-\alpha_2)=0,\]
                    则有$(\lambda_1+\lambda_2)\alpha_1+(\lambda_1-\lambda_2)\alpha_2=0$,则 $\lambda_1+\lambda_2=0,\lambda_1-\lambda_2=0\implies \lambda_1=\lambda_2=0$,因此线性无关且个数等于维数,是一组基.

                \item 错. 反例:$\mathbf{R}^2$ 中过原点的直线 $L_0$ 是 $\mathbf{R}^2$ 的一个子空间. 显然这样的直线有无数条.

                \item 错. 反例:$\mathbf{R}^3$ 中,子空间 $W_1=\spa(e_1,e_2)$,$W_2=\spa(e_1+e_2,e_3)$,则 $B_1\cup B_2=\{e_1,e_2,e_3,e_1+e_2\}$,显然 $\mathbf{R}^3$ 中的任一组基都不可能包含四个元素.
            \end{enumerate}
        \end{answer}

        \item 证明:如果向量组线性相关,把每个向量去掉$m$个位置一致的分量,得到的缩短组仍线性相关;如果向量组线性无关,把每个向量添加$m$个位置一致的分量,得到的缩短组仍线性无关;
        \begin{answer}
            \begin{enumerate}
                \item 若向量组线性相关,则对应该方程组有无穷多解. 去掉 $m$ 个分量,相当于删去该方程组中的任意 $m$ 行方程,依然有无穷多解. 这是因为对于原方程组的任意一个解,将其带入被削减后的方程组也依然成立. 故线性相关得证.

                \item 若向量组线性无关,对应原方程组仅有唯一解,也就是全零解. 增加 $m$ 个分量相当于增加 $m$ 个方程,依然只有唯一解,因为若出现非零解,代入原方程组对应的方程中不会成立,矛盾. 故线性无关得证.
            \end{enumerate}
        \end{answer}

        \item $a$取何值时,$\beta_1=(1,3,6,2)^\mathrm{T},\beta_2=(2,1,2,-1)^\mathrm{T},\beta_3=(1,-1,a,-2)^\mathrm{T}$线性无关?
        \begin{answer}
            方程组:$x_1\beta_1+x_2\beta_2+x_3\beta_3=0$ 系数矩阵
            \[\begin{pmatrix}
                    1 & 2  & 1  \\
                    3 & 1  & -1 \\
                    6 & 2  & a  \\
                    2 & -1 &
                    -2\end{pmatrix}\rightarrow\begin{pmatrix}
                    1 & 2   & 1   \\
                    0 & -5  & -4  \\
                    0 & -10 & a-6 \\
                    0 & -6  & -4
                \end{pmatrix}\rightarrow\begin{pmatrix}
                    1 & 2  & 1   \\
                    0 & -5 & -4  \\
                    0 & 0  & a+2 \\
                    0 & 0  & 0
                \end{pmatrix},\]
            仅全零解的条件是 $a\neq-2$,此时向量组线性无关.
        \end{answer}

        \item 设$\alpha_1,\alpha_2,\ldots,\alpha_n\in\mathbf{F}^n$. 证明:$\alpha_1,\alpha_2,\ldots,\alpha_n$线性无关的充要条件是$\mathbf{F}^n$中任一向量都可以由它们线性表示.
        \begin{answer}
            \begin{enumerate}
                \item 必要性:$\alpha_1,\ldots,\alpha_n$ 线性无关,对于 $F^n$ 中的任一向量 $\beta$, $\alpha_1,\ldots,\alpha_n,\beta$ 的向量个数大于维数 $n$,则线性相关. 由定理 3.2,$\beta$ 可被 $\alpha_1,\ldots,\alpha_n$ 唯一表示.

                \item 充分性:由于 $F^n$ 中任意向量均可被 $\alpha_1,\ldots,\alpha_n$ 线性表示,并且向量个数等于维数. 则 $\alpha_1,\ldots,\alpha_n$ 是 $F^n$ 的一组基. 则 $\alpha_1,\ldots,\alpha_n$ 线性无关.

                      $^*$ 更详细的证明:对于 $F^n$ 的一组基 $e_1,\ldots,e_n$,其可被 $\alpha_1,\ldots,\alpha_n$ 表示. 若 $\alpha_1,\ldots,\alpha_n$ 线性相关,不妨设 $\alpha_n$ 可被 $\alpha_1,\ldots,\alpha_{n-1}$ 表示,则有 $e_1,\ldots,e_n$ 可被 $\alpha_1,\ldots,\alpha_{n-1}$ 表示. 由于 $e_1,\ldots,e_n$ 线性无关. 根据定理 3.3,$n\leqslant n-1$,矛盾. 因此得证.
            \end{enumerate}
        \end{answer}

        \item 设$S_1=\{\alpha_1,\ldots,\alpha_s\},S_2=\{\beta_1,\ldots,\beta_t\}$是向量空间$V$的两个线性无关的子集,证明:$\alpha_1,\ldots,\alpha_s,\beta_1,\ldots,\beta_t$线性无关$\iff \spa(S_1)\cap \spa(S_2)=\{0\}$.
        \begin{answer}
            \begin{enumerate}
                \item 必要性:对于 $\forall v\in\spa(S_1) \cap \spa(S_2)$ 有 $v=a_1\alpha_1+\cdots+a_s\alpha_s=b_1\beta_1+\cdots+b_t\beta_t$. 由于 $\alpha_1,\ldots,\alpha_s,\beta_1,\ldots,\beta_t$ 线性无关 $\implies a_1=\cdots=a_s=b_1=\cdots=b_t=0$,则 $v=0$,即 $\spa(S_1)\cap\spa(S_2)=\{0\}$.

                \item 充分性:考虑反证法. 如果 $\alpha_1,\ldots,\alpha_s,\beta_1,\ldots,\beta_t$ 线性相关,则存在不全为零的系数使得 $a_1\alpha_1+\cdots+a_s\alpha_s+b_1\beta_1\cdots+b_t\beta_t=0$. 因此存在一个向量 $v=a_1\alpha_1+\cdots+a_s\alpha_s=-(b_1\beta_1\cdots+b_t\beta_t)\neq 0$ 且 $v\in\spa(S_1),v\in\spa(S_2)$. 即存在非零向量 $v$ 属于 $\spa(S_1),v\in\spa(S_2)$,矛盾!则充分性得证.
            \end{enumerate}
        \end{answer}

        \item 完成 \autoref{lem:初等行变换不改变列的线性相关性} 其它两种初等行变换的证明.
        \begin{answer}
            设矩阵为
            \[
                \begin{pmatrix}
                    a_{11} & a_{12} & \cdots & a_{1n} \\
                    a_{21} & a_{22} & \cdots & a_{2n} \\
                    \vdots & \vdots & \ddots & \vdots \\
                    a_{m1} & a_{m2} & \cdots & a_{mn}
                \end{pmatrix},
            \]

            \begin{enumerate}
                \item 对其进行倍乘行变换,即将第 $i$ 行乘以 $c$,那么第 $i$ 行会变为 $c a_{i1}, c a_{i2}, \ldots, c a_{in}$,而其他行不变. 记原先矩阵的列向量为 $S_A = \{\alpha_1,\alpha_2,\ldots,\alpha_n\}$,变换后的矩阵列向量为 $S_B = \{\beta_1,\beta_2,\ldots,\beta_n\}$.

                    先证 $S_A$ 线性相关 $\implies S_B$ 线性相关:若 $S_A$ 线性相关,则存在不全为零的系数 $x_1, x_2, \ldots, x_n$ 使得 $x_1 \alpha_1 + x_2 \alpha_2 + \cdots + x_n \alpha_n = 0$. 由于 $S_B$ 与 $S_A$ 只有第 $i$ 行不同,我们仅需要判断第 $i$ 行中的元素在系数 $x_1, x_2, \ldots, x_n$ 下的线性组合是否为 $0$ 即可. 事实上我们有 $x_1 c a_{i1} + x_2 c a_{i2} + \cdots + x_n c a_{in} = c(x_1 a_{i1} + x_2 a_{i2} + \cdots + x_n a_{in}) = 0$,故 $S_B$ 线性相关.

                    再证 $S_B$ 线性相关 $\implies S_A$ 线性相关:若 $S_B$ 线性相关,由于倍乘行变换保证 $c \neq 0$,故从 $S_B$ 到 $S_A$ 相当于做一次将第 $i$ 行乘以 $\frac{1}{c}$ 的行变换,同理可得 $S_A$ 线性相关.

                    因此,对矩阵做倍乘行变换不改变矩阵的列的线性相关性.

                \item 对其进行对换行变换,即将第 $i$ 行与第 $j$ 行交换,其他行不变. 记原先矩阵的列向量为 $S_A = \{\alpha_1,\alpha_2,\ldots,\alpha_n\}$,变换后的矩阵列向量为 $S_B = \{\beta_1,\beta_2,\ldots,\beta_n\}$.

                    先证 $S_A$ 线性相关 $\implies S_B$ 线性相关:若 $S_A$ 线性相关,则存在不全为零的系数 $x_1, x_2, \ldots, x_n$ 使得 $x_1 \alpha_1 + x_2 \alpha_2 + \cdots + x_n \alpha_n = 0$. 由于 $S_B$ 与 $S_A$ 只有第 $i$ 行与第 $j$ 行不同,我们仅需要判断第 $i$ 行与第 $j$ 行中的元素在系数 $x_1, x_2, \ldots, x_n$ 下的线性组合是否均为 $0$ 即可. 事实上由原先第 $i$($j$)行的元素在 $x_1, x_2, \ldots, x_n$ 下的线性组合为 $0$ 即可得到变换后第 $j$($i$)行的元素的线性组合为 $0$,故 $S_B$ 线性相关.

                    再证 $S_B$ 线性相关 $\implies S_A$ 线性相关:若 $S_B$ 线性相关,从 $S_B$ 到 $S_A$ 相当于再做一次相同的对换行变换,同理有 $S_A$ 线性相关.

                    因此,对矩阵做对换行变换不改变矩阵的列的线性相关性.
            \end{enumerate}
        \end{answer}

        \item 已知$\alpha_1=(1,2,4,3),\alpha_2=(1,-1,-6,6),\alpha_3=(-2,-1,2,-9),\alpha_4=(1,1,-2,7),\beta=(4,2,4,a)$.
        \begin{enumerate}
            \item 求子空间$\spa(\alpha_1,\alpha_2,\alpha_3,\alpha_4)$的维数和一组基;

            \item 求$a$的值使得$\beta\in W$,并求$\beta$在 (1) 所选基下的坐标.
        \end{enumerate}
        \begin{answer}
            \begin{enumerate}
                \item 也就是求 $\alpha_1,\alpha_2,\alpha_3,\alpha_4$ 的极大线性无关组. 利用讲义中所述求法:方程组
                      \[ a_1\alpha_1+\cdots+a_4\alpha_4=0 \]
                      对应系数矩阵 $\begin{pmatrix}
                              1 & 1  & -2 & 1  \\
                              2 & -1 & -1 & 1  \\
                              4 & -6 & 2  & -2 \\
                              3 & 6  & -9 & 7\end{pmatrix}$. 化简为行阶梯型:$\begin{pmatrix}
                              1 & 1  & -2 & 1  \\
                              0 & -3 & 3  & -1 \\
                              0 & 0  & 0  & 3  \\
                              0 & 0  & 0  & 0\end{pmatrix}$,因此 $\alpha_1,\alpha_2,\alpha_3,\alpha_4$ 有非零解,这四个向量线性相关. (其实此处已知矩阵秩为 3,即维数是 3).

                      再选取 $\alpha_1,\alpha_2,\alpha_4$ 来求解方程 $a_1\alpha_1+a_2\alpha_2+a_4\alpha_4=0$:
                      \[\begin{pmatrix}
                              1 & 1  & 1  \\
                              2 & -1 & 1  \\
                              4 & -6 & -2 \\
                              3 & 6  & 7\end{pmatrix}\rightarrow\begin{pmatrix}
                              1 & 1  & 1  \\
                              0 & -3 & -2 \\
                              0 & 0  & 2  \\
                              0 & 0  & 0\end{pmatrix},\]
                      因此该方程组只有全零解,即 $\alpha_1,\alpha_2,\alpha_4$ 是$\alpha_1,\alpha_2,\alpha_3,\alpha_4$的极大线性无关组. 则 $\spa(\alpha_1,\alpha_2,\alpha_4)$ 的维数是 3. 一组基是 $\alpha_1,\alpha_2,\alpha_4$ .

                \item 也就是 $a_1\alpha_1+\cdots+a_4\alpha_4=\beta$有解:增广矩阵 $\begin{pmatrix}
                              1 & 1  & -2 & 1  & 4 \\
                              2 & -1 & -1 & 1  & 2 \\
                              4 & -6 & 2  & -2 & 4 \\
                              3 & 6  & -9 & 7  & a\end{pmatrix}$化为 $\begin{pmatrix}
                              1 & 1  & -2 & 1         & 4   \\
                              0 & -3 & 3  & -1        & -6  \\
                              0 & 0  & 0  & -\frac 83 & 8   \\
                              0 & 0  & 0  & 0         & a-9\end{pmatrix}$,若方程有解,则 $a=9$. 求坐标,取 $x_3=0$ 代入得 $\beta=4\alpha_1+3\alpha_2-3\alpha_4$.
            \end{enumerate}
        \end{answer}

        \item 求解子空间 $V_1 = \{(a,0,b) \mid a,b \in \mathbf{R}\}$ 和 $V_2 = \{(a,2a,b) \mid a,b \in \mathbf{R}\}$ 的基和维数.
        \begin{answer}

        \end{answer}

        \item 证明:$B=\{1,x-a,(x-a)^2\}\enspace(a\neq 0)$是$\mathbf{R}[x]_3$的一组基,并求$\mathbf{R}[x]_3$的自然基$\{1,x,x^2\}$中每个向量关于基$B$的坐标.
        \begin{answer}
            只需证明 $B$ 线性无关即可. $\lambda_1+\lambda_2(x_a)+\lambda_3(x-a)^2$求导,增加方程数得到
            \begin{gather*}
                \lambda_2+2\lambda_3x=0, \\
                2\lambda_3=0,
            \end{gather*}
            则 $\lambda_1=\lambda_2=\lambda_3$,线性无关得证. 又 $B$ 中向量个数等于 $R[x]_3$ 维数. 则 $B$ 是一组基. $1=1 \cdot 1+0 \cdot (x-a)+0\times(x-a)^2$ ,即 $(1,0,0)$;$x=a \cdot 1+1 \cdot (x-a)+0\times(x-a)^2$ ,即 $(a,1,0)$;$x^2=a^2 \cdot 1+2a \cdot (x-a)+1\times(x-a)^2$ ,即 $(a^2,2a,1)$.
        \end{answer}

        \item 已知向量组$A=\{\alpha_1,\alpha_2,\alpha_3\},\enspace B=\{\alpha_1,\alpha_2,\alpha_3,\alpha_4\},\enspace C=\{\alpha_1,\alpha_2,\alpha_3,\alpha_5\}$的秩分别为$r(A)=r(B)=3,\enspace r(C)=4$. 证明:$\{\alpha_1,\alpha_2,\alpha_3,\alpha_5-\alpha_4\}$的秩为4.
        \begin{answer}
            等价于证明 $\alpha_1,\alpha_2,\alpha_3,\alpha_5-\alpha_4$ 线性无关. 即求解
            \begin{equation}
                \lambda_1\alpha_1+\lambda_2\alpha_2+\lambda_3\alpha_3+\lambda_4(\alpha_5-\alpha_4)=0. \tag{*} \label{eq:3:A.10}
            \end{equation}

            由于 $r(A)=r(B)=3$ 可得 $A$ 线性无关. $B$ 线性相关. 由定理 3.2 得 $\alpha_4$ 可由 $\alpha_1,\alpha_2,\alpha_3$ 唯一表示:$\alpha_4=\mu_1\alpha_1+\mu_2\alpha_2+\mu_3\alpha_3$. 则代入 (\ref*{eq:3:A.10}) 式. 有
            \[(\lambda_1-\mu_1\lambda_4)\alpha_1+(\lambda_2-\mu_2\lambda_4)\alpha_2+(\lambda_3-\mu_3\lambda_4)\alpha_3+\lambda_4\alpha_5=0,\]
            因为 $r(C)=4$,$\alpha_1,\alpha_2,\alpha_3,\alpha_5$ 线性无关. 有 $\lambda_4=0,\lambda_1=\mu_1\lambda_4=0,\lambda_2=\mu_2\lambda_4=0,\lambda_3=\mu_3\lambda_4=0$. 故 $\alpha_1,\alpha_2,\alpha_3,\alpha_5-\alpha_4$ 线性无关. 原题得证.
        \end{answer}

        \item 设向量组$\alpha_1,\alpha_2,\ldots,\alpha_s$的秩为$r$. 在其中任取$m$个向量$\alpha_{i1},\alpha_{i2},\ldots,\alpha_{im}$,证明:向量组$\alpha_{i1},\alpha_{i2},\ldots,\alpha_{im}$的秩$\geqslant r+m-s$.
        \begin{answer}
            相当于从 $\alpha_1,\ldots,\alpha_s$ 向量中选取 $s-m$ 个向量丢弃,剩余向量的秩:
            \[r(\alpha_{i1},\ldots,\alpha_{im})\geqslant r-(s-m) =r+m-s.\]
        \end{answer}

        \item 已知$\alpha_1,\alpha_2,\ldots,\alpha_n$线性无关,而$\alpha_1,\alpha_2,\ldots,\alpha_n,\beta,\gamma$线性相关. 证明:要么$\beta,\gamma$可以由$\alpha_1,\alpha_2,\ldots,\alpha_n$线性表示,要么$\alpha_1,\alpha_2,\ldots,\alpha_n,\beta$与$\alpha_1,\alpha_2,\ldots,\alpha_n,\gamma$等价.
        \begin{answer}
            方程:$\lambda_1\alpha_1+\cdots+\lambda_n\alpha_n+\lambda_{n+1}\beta+\lambda_{n+2}\gamma=0$,显然 $\lambda_{n+1},\lambda_{n+2}$ 不全为零. 否则与 $\alpha_1,\ldots,\alpha_n$ 线性无关矛盾.
            \begin{enumerate}
                \item 若 $\lambda_{n+1}=0,\lambda_{n+2}\neq 0$,则 $\gamma$ 可被 $\alpha_1,\ldots,\alpha_n$ 表示. 若 $\lambda_{n+1}\neq0,\lambda_{n+2}=0$, 则 $\beta$ 可被 $\alpha_1,\ldots,\alpha_n$ 表示.

                \item 若 $\lambda_{n+1}\lambda_{n+2}\neq 0$ ,则有
                      \begin{gather*}
                          \beta=-\frac 1{\lambda_{n+1}}(\lambda_1\alpha_1+\cdots+\lambda_n\alpha_n+\lambda_{n+2}\gamma), \\
                          \gamma=-\frac 1{\lambda_{n+2}}(\lambda_1\alpha_1+\cdots+\lambda_n\alpha_n+\lambda_{n+1}\beta).
                      \end{gather*}
                    两组向量可以相互表示. 两者等价. 综上原题得证.
            \end{enumerate}
        \end{answer}
    \end{exgroup}

    \begin{exgroup}
        \item 已知$\alpha_1\neq 0$,则$\alpha_1,\alpha_2,\ldots,\alpha_n$线性相关的充要条件是存在$i\enspace(2 \leqslant i \leqslant n)$使得$\alpha_i$可由$\alpha_1,\alpha_2,\ldots,\alpha_{i-1}$线性表示,且表示法唯一.
        \begin{answer}
            充分性显然成立,下证必要性:由于 $\alpha_1,\ldots,\alpha_n$ 线性相关,则存在 $m$,其能使得 $\alpha_1,\ldots,\alpha_m$ 线性无关的最大下标,有 $1\leqslant m<n$. 因此 $i=m+1$,$\alpha_1,\ldots,\alpha_{i-1}$ 线性无关,$\alpha_1,\ldots,\alpha_i$ 线性相关. 可得 $\alpha_i$ 可被 $\alpha_1,\ldots,\alpha_{i-1}$ 唯一表示.
        \end{answer}

        \item 证明以下两个结论:
        \begin{enumerate}
            \item 设$U$和$W$都是$V$的非零子空间,如果$U\subseteq W$,那么$\dim U \leqslant \dim W$;

            \item 设$U$和$W$都是$V$的非零子空间,$U\subseteq W$,且$\dim U = \dim W$,则$U = W$.
        \end{enumerate}
        \begin{answer}
            \begin{enumerate}
                \item 设 $U$ 的一组基为 $u_1,\ldots,u_m$,$W$ 的一组基为 $w_1,\ldots,w_n$. 由于 $U\subseteq W$,则 $u_1,\ldots,u_m$ 可由 $w_1,\ldots,w_n$ 线性表示,且 $u_1,\ldots,u_m$ 线性无关. 由定理 3.3 的等价命题可得 $m\leqslant n$,则 $m=\dim U\leqslant\dim W=n$ 的得证.

                \item 因为 $\dim U=\dim W$,则 $u_1,\ldots,u_m$ 也是 $W$ 的一组基. 则 $W$ 的任意向量均可由 $u_1,\ldots,u_m$ 表示,可得 $W\subseteq U$,而 $U\subseteq W$,故有 $U=W$ 得证.
            \end{enumerate}
        \end{answer}

        \item 设向量组$\alpha_1,\alpha_2,\ldots,\alpha_n$线性无关. 证明:在向量组$\beta,\alpha_1,\alpha_2,\ldots,\alpha_n$中至多有一个向量$\alpha_i\enspace(1 \leqslant i \leqslant r)$可被其前面的$i$个向量$\beta,\alpha_1,\alpha_2,\ldots,\alpha_{i-1}$线性表示.
        \begin{answer}
            反证法. 若存在两个向量 $\alpha_i,\alpha_j$ 可被前面的向量表示,即
            \begin{gather*}
                \alpha_i=\lambda_0\beta+\lambda_1\alpha_1+\cdots+\lambda_{i-1}\alpha_{i-1}, \\
                \alpha_j=\mu_0\beta+\mu_1\alpha_1+\cdots+\mu_{j-1}\alpha_{j-1}.
            \end{gather*}
            如果 $\lambda_0$ 或者 $\mu_0$ 为 0,则有向量组中的 $\alpha_i$ 或 $\alpha_j$ 可被其他向量线性表示,则该向量组相关,这与条件矛盾. 若 $\lambda_0$ 与 $\mu_0$ 均不为 0,等式可化为
            \begin{gather*}
                \frac 1{\lambda_0}\alpha_i=\beta+\frac{\lambda_1}{\lambda_0}\alpha_1+\cdots+\frac{\lambda_{i-1}}{\lambda_0}\alpha_{i-1}, \\
                \frac 1{\mu_0}\alpha_j=\beta+\frac{\mu_1}{\mu_0}\alpha_1+\cdots+\frac{\mu_{j-1}}{\mu_0}\alpha_{j-1}.
            \end{gather*}
            不妨设 $i>j$. 相减得
            \[\frac 1{\lambda_0}\alpha_i=\left(\frac{\lambda_1}{\lambda_0}-\frac{\mu_1}{\mu_0}\right)\alpha_1+\cdots+\left(\frac{\lambda_i}{\lambda_0}+\frac 1{\mu_0}\right)\alpha_j+\frac{\lambda_{i-1}}{\lambda_0}\alpha_{i-1},\]
            则 $\alpha_i$ 可被其他向量线性表示,因此向量组线性相关,与条件矛盾. 综上,至多有一个向量 $\alpha_i$ 可被前面的相邻线性表示.
        \end{answer}

        \item 证明:$1,e^{\lambda_1\cdot x},e^{\lambda_2\cdot x}$($\lambda_1\neq\lambda_2$且均不为0)线性无关.
        \begin{answer}
            考虑使用求导构造更多方程.
            \[\begin{cases}
                    k_0+k_1\cdot e^{\lambda_1 x}+k_2\cdot e^{\lambda_2 x}=0               \\
                    k_1\lambda_1\cdot e^{\lambda_1 x}+k_2\lambda_2\cdot e^{\lambda_2 x}=0 \\
                    k_1\lambda_1^2\cdot e^{\lambda_1 x}+k_2\lambda_2^2\cdot e^{\lambda_2 x}=0
                \end{cases},\]
          由后两式可知$k_1k_2(\lambda_1-\lambda_0)=0$. 又 $\lambda_1\neq\lambda_2$,故$k_1=k_2=0$,代回第一式得 $k_0=0$,则 $1,e^{\lambda_1x},e^{\lambda_2x}$ 线性无关,得证.
        \end{answer}

        \item 设线性空间$V(\mathbf{F})$中,向量$\beta$是$\alpha_1,\ldots,\alpha_r$的线性组合,但不是$\alpha_1,\ldots,\alpha_{r-1}$的线性组合. 证明:$\spa(\alpha_1,\ldots,\alpha_{r-1},\alpha_r)=\spa(\alpha_1,\ldots,\alpha_{r-1},\beta)$.
        \begin{answer}
            只需证明 $\alpha_r$ 可以被 $\alpha_1,\ldots,\alpha_{r-1},\beta$ 表示即可. 由于 $\beta$ 是 $\alpha_1,\ldots,\alpha_{r-1}$ 的线性组合,若 $\lambda_r=0$,则 $\beta$ 是 $\alpha_1,\ldots,\alpha_{r-1}$ 的线性组合. 这与条件矛盾. 因此 $\alpha_r=-\vspace{1ex}\dfrac 1{\lambda_r}(\lambda_1\alpha_1+\cdots+\lambda_{r-1}\alpha_{r-1}-\beta)$,则这两组向量等价. $\spa(\alpha_1,\ldots,\alpha_{r-1},\alpha_r)=\spa(\alpha_1,\ldots,\alpha_{r-1},\beta)$ 得证.
        \end{answer}

        \item \label{item:3:正实数线性空间}
        设$\mathbf{R}^+$是所有正实数组成的集合,加法和数乘定义如下:
        \[ \forall a,b \in \mathbf{R}_+,\enspace k\in \mathbf{R}\colon\enspace a\oplus b = ab,\enspace k\odot a = a^k \]
        则 $\mathbf{R}^+$关于这一加法和数乘构成一个实线性空间. 求$\mathbf{R}^+$的一组基.
        \begin{answer}
            分析该实线性空间,可以看出加法单位元为 1,数乘单位元为 1. 我们给出一组基:$e$,其中 $e$ 为自然对数的底数. 当然, 2,3 或者 10 都可以作为一组基. 接下来我们验证 $e$ 是 $\mathbf{R}^+$ 的基:$\forall a\in \mathbf{R}^+,\exists k=\ln a\in\mathbf{R}$,满足 $k\odot e=e^k=a$,则 $\spa(e)=\mathbf{R}^+$ 成立. 由于该向量组只有一个元素,且并非设该空间的零元 1,则 $e$ 是线性无关的. 得证.
        \end{answer}
    \end{exgroup}

    \begin{exgroup}
        \item 已知$m$个向量$\alpha_1,\alpha_2,\ldots,\alpha_m$线性相关,但其中任意$m-1$个都线性无关,证明:
        \begin{enumerate}
            \item 若$k_1\alpha_1+\cdots+k_m\alpha_m=0$,则$k_1,\ldots,k_m$全为0或全不为0;

            \item 若以下等式成立
                  \begin{align*}
                      k_1\alpha_1+\cdots+k_m\alpha_m & =0 \\
                      l_1\alpha_1+\cdots+l_m\alpha_m & =0
                  \end{align*}
                  其中$l_1\neq 0$,证明:$\dfrac{k_1}{l_1}=\cdots=\dfrac{k_m}{l_m}$.
        \end{enumerate}
        \begin{answer}
            \begin{enumerate}
                \item 若 不全为 0. 不妨设设至少有 $k_i=0$,则有 $k_1\alpha_1+\cdots+k_{i-1}\alpha_{i-1}+k_{i+1}\alpha_{i+1}+\cdots+k_m\alpha_m=0$,并且系数不全为 0. 因此 $\alpha_1,\ldots,\alpha_{i-1},\alpha_{i+1},\ldots,\alpha_m$ 这 $m-1$ 个向量相关,与题设矛盾. 则原题得证.

                \item $l_1\neq 0$,则 $l_2,\ldots,l_m$ 均不为 0.
                      \begin{enumerate}
                          \item 若 $k_1=\cdots=k_m=0$. 原式显然成立.

                          \item 若 $k_1,\ldots,k_m$ 不全为 0. 则
                                \begin{gather*}
                                    l_1(k_1\alpha_1+\cdots+k_m\alpha_m)=0, \\
                                    k_1(l_1\alpha_1+\cdots+l_m\alpha_m)=0.
                                \end{gather*}
                                两式相减,得
                                \[(k_2l_1-k_1l_2)\alpha_2+\cdots+(k_ml_1-k_1l_m)\alpha_m=0,\]
                                因为 $\alpha_2,\ldots,\alpha_m$ 线性无关. 则以上系数均为 0. 故$\dfrac {k_2}{l_2}=\dfrac {k_1}{l_1},\ldots,\dfrac {k_m}{l_m}=\dfrac {k_1}{l_1}$. 得证.
                      \end{enumerate}
            \end{enumerate}
        \end{answer}

        \item (替换定理)设$\alpha_1,\alpha_2,\ldots,\alpha_r$线性无关,且可以被$\beta_1,\beta_2,\ldots,\beta_n$线性表示,则可以将$\beta_1,\beta_2,\ldots,\beta_n$中的$r$个向量替换成$\alpha_1,\alpha_2,\ldots,\alpha_r$后得到与$\beta_1,\beta_2,\ldots,\beta_n$等价的新向量组(注:可以使用数学归纳法证明).
        \begin{answer}
            利用递推法:当 $r=1$ 时,由于 $\alpha_1$ 线性无关,可得 $\alpha_1\neq 0$. 设 $\alpha_1=\lambda_1\beta_1+\cdots+\lambda_n\beta_n$,则至少存在一个 $\lambda_i\neq 0$,不妨设 $\lambda_1\neq 0$,因此有 $\beta_1=-\vspace{1ex}\dfrac 1{\lambda_1}(-\alpha_1+\lambda_2\beta_2+\cdots+\lambda_n\beta_n)$,故 $\beta_1,\ldots,\beta_n$ 与 $\alpha_1,\beta_2,\ldots,\beta_n$ 等价.

            当 $r=2$ 时,由于 $\alpha_1,\alpha_2$ 无关. 有 $\alpha_1\alpha_2\neq 0$. 根据 $r=1$ 的情况,不妨设 $\beta_1,\ldots,\beta_n$ 与 $\alpha_1,\beta_2,\ldots,\beta_n$ 等价. 因此 $\alpha_2$ 可由 $\alpha_1,\beta_2,\ldots,\beta_n$ 表出:
            \[\alpha_2=\mu_1\alpha_1+\mu_2\beta_2+\cdots+\mu_n\beta_n.\]
            由于 $\alpha_1,\alpha_2$ 无关,故 $\mu_2,\ldots,\mu_n$ 至少有一个非零的数. 不妨设 $\mu_2\neq 0$,同上可得$\alpha_1,\alpha_2,\beta_3,\ldots,\beta_n$ 就与 $\alpha_1,\beta_2,\ldots,\beta_n$ 等价,也与$\beta_1,\beta_2,\ldots,\beta_n$ 等价. 综上,通过递推可知,对正整数 $r$,上述结论依然成立.
        \end{answer}

        \item 设线性空间$V=\mathbf{F}^n$. 证明:
        \begin{enumerate}
            \item 存在$V$的子空间$W$,使得$W$的任一非零向量的分量均不为0;

            \item 若$V$的子空间$W$的任一非零向量的分量均不为0,则$\dim W=1$;

            \item 若$V$的子空间$W$的任一非零向量的零分量个数均不超过$r$,则$\dim W \leqslant r+1$.
        \end{enumerate}
        \begin{answer}
            \begin{enumerate}
                \item $\alpha=(1,1,\ldots,1)^{\mathrm{T}}, W=\spa(\alpha)$,显然 $W$ 是满足条件的一维子空间.

                \item 考虑反证法:若 $\dim W>1$,则 $W$ 中存在线性无关的两向量. 由条件,
                      \[a_1,\ldots,a_n,b_1,\ldots,b_n\neq 0,\]
                      可设 $a_1=kb_1,k\in F$,因此 $\alpha-k\beta=(0,a_2-kb-2,\ldots,a_n-kb_n)^{\mathrm{T}}\in W$. 且由于 $\alpha,\beta$ 无关,$\alpha-k\beta\neq 0$ 但存在分量为 0,这与条件矛盾. 故 $\dim W=1$.

                \item 考虑反证法:若 $\dim W>r+1$,则存在 $r+2$ 个线性无关的向量,设为
                      \begin{align*}
                          \alpha_1     & = (a_{11},a_{12},\ldots,a_{1(r+1)},\ldots,a_{1n})^{\mathrm{T}},                 \\
                                       & \vdotswithin{ = }                                                              \\
                          \alpha_{r+2} & = (a_{(r+2)1},a_{(r+2)2},\ldots,a_{(r+2)(r+1)},\ldots,a_{(r+2)n})^{\mathrm{T}}.
                      \end{align*}
                      取这些向量的前 $r+1$ 个分量组成新的向量组:
                      \begin{align*}
                          \beta_1     & = (a_{11},a_{12},\ldots,a_{1(r+1)})^{\mathrm{T}},             \\
                                      & \vdotswithin{ = }                                            \\
                          \beta_{r+2} & = (a_{(r+2)1},a_{(r+2)2},\ldots,a_{(r+2)(r+1)})^{\mathrm{T}}.
                      \end{align*}
                      由于 $\beta_1,\ldots,\beta_{r+2}$ 是 $r+2$ 个 $r+1$ 维向量,其必然线性相关,则存在不全为 0 的系数 $\lambda_1,\ldots,\lambda_{r+2}$,$\lambda_1\beta_1+\cdots+\lambda_{r+2}\beta_{r+2}=0$.  由于 $\alpha_1,\ldots,\alpha_{r+2}$ 线性无关知:$\lambda_1\alpha_1+\cdots+\lambda_{r+2}\alpha_{r+2}\neq 0$,且其属于 $W$. 但其前 $r+1$ 个分量均为 0,这与条件矛盾. 故 $\dim W\leqslant r+1$ 得证.
            \end{enumerate}
        \end{answer}

        \item 设 $\mathbf{Q}(\sqrt[3]{2}) = \{a+b\sqrt[3]{2}+c\sqrt[3]{4}\mid a,b,c\in\mathbf{Q}\}$,证明 $\mathbf{Q}(\sqrt[3]{2})$ 是 $\mathbf{Q}$ 上的线性空间并求其维数.
        \begin{answer}
            先证明 $\mathbf{Q}(\sqrt[3]{2})$ 是 $\mathbf{Q}$ 上的线性空间:

            $\mathbf{Q}(\sqrt[3]{2})$ 对通常的加法构成交换群以及其封闭性是显然的,在此不再赘述,下阐述数乘性质:
            \begin{enumerate}
                \item $\forall v = a+b\sqrt[3]{2}+c\sqrt[3]{4} \in \mathbf{Q}(\sqrt[3]{2}), \enspace 1 (a+b\sqrt[3]{2}+c\sqrt[3]{4}) = a+b\sqrt[3]{2}+c\sqrt[3]{4}$;

                \item $\forall v = a+b\sqrt[3]{2}+c\sqrt[3]{4} \in \mathbf{Q}(\sqrt[3]{2}), \forall \lambda, \mu \in \mathbf{Q}, \enspace \lambda(\mu v) = (\lambda \mu) v$;

                \item $\forall v = a+b\sqrt[3]{2}+c\sqrt[3]{4} \in \mathbf{Q}(\sqrt[3]{2}), \forall \lambda, \mu \in \mathbf{Q}, \enspace (\lambda + \mu) v = (\lambda + \mu)a + (\lambda + \mu) b\sqrt[3]{2} + (\lambda + \mu) c\sqrt[3]{4} = \lambda(a+b\sqrt[3]{2}+c\sqrt[3]{4}) + \mu(a+b\sqrt[3]{2}+c\sqrt[3]{4}) = \lambda v + \mu v$;

                \item $\forall v = a_1 + b_1\sqrt[3]{2} + c_1\sqrt[3]{4}, u = a_2 + b_2\sqrt[3]{2} + c_2\sqrt[3]{4} \in \mathbf{Q}(\sqrt[3]{2}), \forall \lambda \in \mathbf{Q}, \enspace \lambda(v+u) = \lambda (a_1 + a_2) + \lambda (b_1 + b_2)\sqrt[3]{2} + \lambda (c_1 + c_2)\sqrt[3]{4} = \lambda (a_1 + b_1\sqrt[3]{2} + c_1\sqrt[3]{4}) + \lambda (a_2 + b_2\sqrt[3]{2} + c_2\sqrt[3]{4}) = \lambda v + \lambda u$;

                \item (封闭性)$\forall v = a+b\sqrt[3]{2}+c\sqrt[3]{4} \in \mathbf{Q}(\sqrt[3]{2}), \forall \lambda \in \mathbf{Q}, \enspace \lambda v = \lambda a + \lambda b\sqrt[3]{2} + \lambda c\sqrt[3]{4} \in \mathbf{Q}(\sqrt[3]{2})$.
            \end{enumerate}
            故 $\mathbf{Q}(\sqrt[3]{2})$ 是 $\mathbf{Q}$ 上的线性空间.

            下求 $\mathbf{Q}(\sqrt[3]{2})$ 的维数. 考虑找 $\mathbf{Q}(\sqrt[3]{2})$ 的一组基,我们给出 $\{1, \sqrt[3]{2}, \sqrt[3]{4}\}$,下证其为 $\mathbf{Q}(\sqrt[3]{2})$ 的一组基:

            \begin{enumerate}
                \item (线性无关)令 $k_1 + k_2 \sqrt[3]{2} + k_3 \sqrt[3]{4} = 0$,其中 $k_1, k_2, k_3 \in \mathbf{Q}$,由于左边只有第一项为有理数,故有 $k_1 = 0$,进而有 $\sqrt[3]{2} (k_2 + k_3 \sqrt[3]{2}) = 0$,又可得到 $k_2 = 0$,并且 $k_3 = 0$. 故 $\{1, \sqrt[3]{2}, \sqrt[3]{4}\}$ 线性无关.

                \item (张成空间)由 $\mathbf{Q}(\sqrt[3]{2})$ 的定义易得.
            \end{enumerate}
            综上,$\mathbf{Q}(\sqrt[3]{2})$ 是 $\mathbf{Q}$ 上的线性空间,且 $\dim \mathbf{Q}(\sqrt[3]{2}) = 3$.

        \end{answer}

        \item 设$\mathbf{K} \subseteq \mathbf{F} \subseteq \mathbf{E}$是三个数域,已知$\mathbf{F}$作为$\mathbf{K}$上的线性空间是$n$维的,$\mathbf{E}$作为$\mathbf{F}$上的线性空间是$m$维的,证明:$\mathbf{E}$作为$\mathbf{K}$上的线性空间是$mn$维的.
        \begin{answer}
            取 $\mathbf{F}(\mathbf{K})$ 的一组基 $f_1, f_2, \ldots, f_n$,$\mathbf{E}(\mathbf{F})$ 的一组基 $e_1, e_2, \ldots, e_m$,下证向量组 $B = \{f_1 e_1, f_2 e_1, \ldots, f_n e_1, f_1 e_2, f_2 e_2, \ldots, f_n e_2, \ldots, f_1 e_m, f_2 e_m, \ldots, f_n e_m\}$ 是 $\mathbf{E}(\mathbf{K})$ 的一组基:

            \begin{enumerate}
                \item (线性无关)令 $k_{11} f_1 e_1 + k_{12} f_2 e_1 + \cdots + k_{1n} f_n e_1 + k_{21} f_1 e_2 + k_{22} f_2 e_2 + \cdots + k_{2n} f_n e_2 + \cdots + k_{m1} f_1 e_m + k_{m2} f_2 e_m + \cdots + k_{mn} f_n e_m = 0$,其中 $k_{ij} \in \mathbf{K}$,则
                    \[
                        \sum_{i=1}^m \left(\sum_{j=1}^n k_{ij} f_j \right) e_i = 0.
                    \]
                    由于 $e_1, e_2, \ldots, e_n$ 是 $\mathbf{E}(\mathbf{F})$ 的一组基 $e_1, e_2, \ldots, e_m$,我们有
                    \[
                        \sum_{j=1}^n k_{ij} f_j = 0, \enspace \forall i = 1, 2, \ldots, m.
                    \]
                    而 $f_1, f_2, \ldots, f_n$ 又是 $\mathbf{F}(\mathbf{K})$ 的一组基,故 $k_{ij} = 0$,即向量组 $B$ 线性无关.

                \item (张成空间)$\forall e = \mu_1 e_1 + \mu_2 e_2 + \cdots + \mu_m e_m \in \mathbf{E} \enspace (\mu_1, \mu_2, \ldots, \mu_m \in \mathbf{F})$,由于 $\mu_i \in \mathbf{F}$,故存在 $k_{ij} \in \mathbf{K}$ 使得 $\mu_i = k_{i1} f_1 + k_{i2} f_2 + \cdots + k_{in} f_n$,于是
                    \[
                        e = \sum_{i=1}^m \left(\sum_{j=1}^n k_{ij} f_j \right) e_i
                          = \sum_{i=1}^m \sum_{j=1}^n k_{ij} f_j e_i.
                    \]
                    故 $e \in \spa B$.
            \end{enumerate}
            综上,$B$ 是 $\mathbf{E}(\mathbf{K})$ 的一组基,故 $\dim \mathbf{E}(\mathbf{K}) = mn$.
        \end{answer}

        \item 延续上一讲对于 $\mathbf{F_4}(\mathbf{Z}_2)$ 的讨论,尝试求 $\mathbf{F_4}$ 在 $\mathbf{Z}_2$ 上的一组基及其维数,以及其中每个元素的坐标表示.
        \begin{answer}

        \end{answer}
    \end{exgroup}
\end{exercise}

\chapter{线性空间的运算}

在前述章节中我们对(有限维)线性空间中的基本概念以及研究的基本问题进行了了解. 事实上,很多时候我们还需要研究不同线性空间进行运算后得到的新集合的性质,本节我们将详细展开讨论这一问题.

\section{线性空间的交、并、和}

\begin{definition}{}{}
    设$W_1,W_2$是线性空间$V(\mathbf{F})$的两个子空间,则
    \begin{align*}
        W_1 \cap W_2 & =\{\alpha \mid \alpha\in W_1 \text{~且~} \alpha\in W_2\}            \\
        W_1 \cup W_2 & =\{\alpha \mid \alpha\in W_1 \text{~或~} \alpha\in W_2\}            \\
        W_1 + W_2    & =\{\alpha_1+\alpha_2 \mid \alpha_1\in W_1,\enspace\alpha_2\in W_2\}
    \end{align*}
    分别称为$W_1$和$W_2$的交、并、和.
\end{definition}

交与并的定义实际上与集合交与并的定义类似,而和的定义可能有些许反直觉. 我们可以通过一个例子来体会为什么要定义子空间的和.
\begin{example}{}{子空间运算}
    在$\mathbf{R}^3$中,我们设
    \[\alpha_1=(0,0,1),\ \alpha_2=(0,1,0),\ \alpha_3=(1,0,0).\]
    令$\mathbf{R}^3$子空间$W_1=\spa(\alpha_1,\alpha_2)$,$W_2=\spa(\alpha_1,\alpha_3)$,则$W_1$实际上是$yOz$平面,$W_2$是$xOz$平面,因此我们根据交与并的概念(实际上就是集合取交集和并集)得到$W_1 \cap W_2=\spa(\alpha_1)$(即$z$坐标轴).

    进一步考察并集,事实上显然$W_1 \cup W_2$得到的集合表示$xOz$和$yOz$平面上所有的点. 事实上我们发现,$W_1 \cup W_2$得到的集合关于向量加法、数乘运算并不封闭,例如只需取$\alpha_2+\alpha_3=(0,1,1)$就不在$W_1 \cup W_2$中,因此不再是$\mathbf{R}^3$的子空间.

    接下来我们考察二者之和. 事实上$W_1+W_2=\mathbf{R}^3$. 原因在于
    \begin{enumerate}
        \item $\forall \beta\in W_1 + W_2$,由子空间和的定义可知有$\beta=\beta_1+\beta_2$,其中$\beta_1\in W_1\subseteq \mathbf{R}^3$,$\beta_2\in W_2\subseteq \mathbf{R}^3$,由于$\mathbf{R}^3$是线性空间,其中元素关于加法运算封闭,因此$\beta=\beta_1+\beta_2\in \mathbf{R}^3$,即$W_1+W_2\subseteq \mathbf{R}^3$;

        \item $\mathbf{R}^3$中任一向量$(x,y,z)$总能写成$(x,y,z)=(0,y,z)+(x,0,0)$的形式,其中$(0,y,z)$在$W_1$中,$(x,0,0)$在$W_2$中,因此根据子空间和的定义可知$\mathbf{R}^3\subseteq W_1 + W_2$成立.
    \end{enumerate}
    综上,我们得到$W_1+W_2=\mathbf{R}^3$.
\end{example}

从上面证明$W_1+W_2=\mathbf{R}^3$的过程中我们可以提炼出证明子空间的和等于某一空间的一般方法:本质而言仍然是证明集合相等,因此证明两边包含即可. 证明子空间的和属于某一空间是平凡的,如上述证明的第一部分;第二部分证明某一空间属于子空间和只需要将该空间中任意向量都可分解为各个子空间中向量的和即可.

事实上,根据\autoref{ex:子空间运算} 我们发现,子空间$W_1$和$W_2$的交与和仍然是线性空间,但是它们的并不是线性空间. 事实上,我们可以证明如下定理:
\begin{theorem}{}{}
    设$W_1,W_2$是线性空间$V(\mathbf{F})$的两个子空间,则
    \begin{enumerate}
        \item $W_1 \cap W_2$是$V$的子空间;

        \item $W_1 + W_2$是$V$的子空间;

        \item $W_1 \cup W_2$为$V$的子空间$\iff W_1 \subseteq W_2$或$W_2 \subseteq W_1 \iff W_1 \cup W_2=W_1+W_2$.
    \end{enumerate}
\end{theorem}

\begin{proof}
    我们从子空间的定义出发证明这一定理.
    即验证 $W_1 \cap W_2$ 满足子空间的三个条件:


    \begin{itemize}
        \item
        由于 $W_1, W_2$ 都是 $V$ 的子空间,零向量 $\mathbf{0} \in W_1$ 且 $\mathbf{0} \in W_2$。因此,$\mathbf{0} \in W_1 \cap W_2$。

        \item
        对于任意的 $x, y \in W_1 \cap W_2$,有 $x \in W_1$ 且 $x \in W_2$,$y \in W_1$ 且 $y \in W_2$。
        由于 $W_1$ 和 $W_2$ 都是 $V$ 的子空间,所以 $x + y \in W_1$ 且 $x + y \in W_2$。
        因此,$x + y \in W_1 \cap W_2$。

        \item
        对于任意的 $x \in W_1 \cap W_2$ 和任意的标量 $\lambda \in \mathbf{F}$,有 $x \in W_1$ 且 $x \in W_2$。
        由于 $W_1$ 和 $W_2$ 都是 $V$ 的子空间,所以 $\lambda x \in W_1$ 且 $\lambda x \in W_2$。
        因此,$\lambda x \in W_1 \cap W_2$。
    \end{itemize}

    所以 $W_1 \cap W_2$ 是 $V$ 的子空间。


    下证$W_1 + W_2$ 是 $V$ 的子空间:
    \begin{itemize}
        \item
        由于 $W_1$ 和 $W_2$ 是 $V$ 的子空间,$\mathbf{0} \in W_1$ 且 $\mathbf{0} \in W_2$。
        因此,$\mathbf{0} = \mathbf{0} + \mathbf{0} \in W_1 + W_2$。

        \item
        对于任意的 $u_1, u_2 \in W_1 + W_2$,存在 $x_1, x_2 \in W_1$ 和 $y_1, y_2 \in W_2$,使得 $u_1 = x_1 + y_1$,$u_2 = x_2 + y_2$。
        则
        $$
        u_1 + u_2 = (x_1 + y_1) + (x_2 + y_2) = (x_1 + x_2) + (y_1 + y_2).
        $$
        由于 $W_1$ 和 $W_2$ 是 $V$ 的子空间,$x_1 + x_2 \in W_1$ 且 $y_1 + y_2 \in W_2$,
        因此,$u_1 + u_2 \in W_1 + W_2$。

        \item
        对于任意的 $u \in W_1 + W_2$ 和标量 $\lambda \in \mathbf{F}$,存在 $x \in W_1$ 和 $y \in W_2$,使得 $u = x + y$。
        则
        $$
        \lambda u = \lambda (x + y) = \lambda x + \lambda y.
        $$
        由于 $W_1$ 和 $W_2$ 是 $V$ 的子空间,$\lambda x \in W_1$ 且 $\lambda y \in W_2$,
        因此,$\lambda u \in W_1 + W_2$。
    \end{itemize}

    因此,$W_1 + W_2$ 也是 $V$ 的子空间。
\end{proof}
我们只在此证明定理的前两条,第三条我们留作习题供读者练习,因为在考试中有出现过. 前两条还可以进行推广,即$V$的有限个子空间的交与和仍然是$V$的子空间.

除此之外,这一定理也告诉我们为什么需要研究子空间的和而更少研究子空间的并:因为子空间的和仍然是线性空间. 直观理解实际上就是和的定义中出现了两个子空间的向量的加法,而构成子空间的核心就是运算封闭,因此这一定义为子空间的和仍构成子空间提供了保证,因此这一定义也是十分自然的.

下面我们来看一个例子,在例子中我们将给出求子空间的和与交的一般方法:
\begin{example}{}{}
    设 $\alpha_1 = (1, 0, -1, 0)$,$\alpha_2 = (0, 1, 2, 1)$,$\alpha_3 = (2, 1, 0, 1)$,是四维实行向量空间 $V$ 中的向量,他们张成的子空间为 $V_1$;又设向量 $\beta_1 = (-1, 1, 1, 1)$,$\beta_2 = (1, -1, -3, -1)$,$\beta_3 = (-1, 1, -1, 1)$ 张成的子空间为 $V_2$,求 $V_1$ 和 $V_2$ 的交与和的基.
\end{example}

\begin{solution}
    \begin{enumerate}
        \item 方法一.  $V_1 +V_2$ 是由 $\alpha_i$ 和 $\beta_i$ 生成的,因此只需要求出这 $6$ 个向量的极大线性无关组即可. 将这 $6$ 个向量按列分块方式拼成矩阵,并用初等行变换将其化为阶梯形矩阵:
              \begin{align*}
                  \begin{pmatrix}
                      1  & 0 & 2 & -1 & 1  & -1 \\
                      0  & 1 & 1 & 1  & -1 & 1  \\
                      -1 & 2 & 0 & 1  & -3 & -1 \\
                      0  & 1 & 1 & 1  & -1 & 1
                  \end{pmatrix}
                  \xrightarrow{}
                  \begin{pmatrix}
                      1 & 0 & 2 & -1 & 1  & -1 \\
                      0 & 1 & 1 & 1  & -1 & 1  \\
                      0 & 2 & 2 & 0  & -2 & -2 \\
                      0 & 0 & 0 & 0  & 0  & 0
                  \end{pmatrix} \\
                  \xrightarrow{}
                  \begin{pmatrix}
                      1 & 0 & 2 & -1 & 1  & -1 \\
                      0 & 1 & 1 & 1  & -1 & 1  \\
                      0 & 0 & 0 & -2 & 0  & -4 \\
                      0 & 0 & 0 & 0  & 0  & 0
                  \end{pmatrix}
              \end{align*}

              所以就可以取 $\alpha_1$,$\alpha_2$,$\beta_1$ 为 $V_1 + V_2$ 的基(不唯一).

              下面再来取 $V_1\cap V_2$ 的基,首先注意到 $\alpha_1$,$\alpha_2$ 是 $V_1$ 的基(从上面的矩阵即可看出),又不难验证 $\beta_1$,$\beta_2$ 是 $V_2$ 的基,$V_2$ 中的向量可以表示为 $\beta_1$,$\beta_2$ 的线性组合. 假设 $t_1\beta_1 + t_2\beta_2$ 属于 $V_1$,则向量组 $\alpha_1, \alpha_2, t_1\beta_1 + t_2\beta_2$ 和向量组 $\alpha_1, \alpha_2$ 的秩相等(因为 $\alpha_1, \alpha_2$ 是 $V_1$ 的基). 因此,我们可以用矩阵方法来求出参数 $t_1, t_2$. 注意到
              \[ \begin{pmatrix}
                      1  & 0 & -t + t_2   \\
                      0  & 1 & t_1 - t_2  \\
                      -1 & 2 & t_1 - 3t_2 \\
                      0  & 1 & -t_1 - t_2
                  \end{pmatrix} \xrightarrow{} \begin{pmatrix}
                      1 & 0 & -t + t_2  \\
                      0 & 1 & t_1 - t_2 \\
                      0 & 2 & -2t_2     \\
                      0 & 0 & 0
                  \end{pmatrix} \xrightarrow{} \begin{pmatrix}
                      1 & 0 & -t + t_2  \\
                      0 & 1 & t_1 - t_2 \\
                      0 & 0 & -2t_1     \\
                      0 & 0 & 0
                  \end{pmatrix} \]

              所以可以得出当且仅当 $t_1 = 0$ 时 $t_1\beta_1 + t_2\beta_2$ 属于 $V_1$,所以 $V_1 \cap V_2$ 的基可取为 $\beta_2$.

        \item 方法二. 求 $V_1 + V_2$ 的基同方法一,现用解线性方程组的方法来求 $V_1 \cap V_2$ 的基. 因为 $\alpha_1$,$\alpha_2$ 是 $V_1$ 的基,$\beta_1$,$\beta_2$ 是 $V_2$ 的基,故对任一向量 $\gamma \in V_1 \cap V_2$,$\gamma = x_1\alpha_1 + x_2\alpha_2 = -x_3\beta_1 - x_4\beta_2$. 因此,求向量 $\gamma$ 等价于求解线性方程组

              \[ x_1\alpha_1 + x_2\alpha_2 + x_3\beta_1 + x_4\beta_2 = 0. \]

              上述线性方程的通解是 $(x_1, x_2, x_3, x_4) = k(-1, 1, 0, 1)$,从而 $\gamma = -k(\alpha_1 - \alpha_2) = -k\beta_2 (k \in \mathbf{R})$,于是 $\beta_2$ 是 $V_1 \cap V_2$ 的基.
    \end{enumerate}
\end{solution}

我们不难发现,两个线性空间的和的求法就是将两个空间的基合并后求极大线性无关组,而交的求法则更具技巧性. 当然这里使用的是简单的向量空间的例子,如果是一般的线性空间,则可以先转化为基下的坐标然后使用上面的方法求解.

\section{覆盖定理}

\begin{theorem}{覆盖定理}{覆盖定理} \index{fugaidingli@覆盖定理}
    设$V_1,V_2,\ldots,V_s$是线性空间$V$的$s$个非平凡子空间,证明:$V$中至少存在一个向量不属于$V_1,V_2,\ldots,V_s$中的任何一个,即$V_1 \cup V_2 \cup \cdots \cup V_s\subsetneq V$.
\end{theorem}

覆盖定理表明任何一个线性空间都不能被自身有限个非平凡子空间通过并得到. 初看可能有些不够自然,但我们可以从简单的几何意义获得直观的理解:有限条直线的并不可能是一个平面. 下面我们利用数学归纳法进行证明.

\begin{proof}
    \begin{enumerate}
        \item 当$s=2$时,由于$V_1,V_2$是非平凡子空间,因此$V$中存在$\alpha\notin V_1$. 若$\alpha\notin V_2$,则结论已经成立. 若$\alpha\in V_2$,由$V_2$非平凡知存在$\beta\notin V_2$. 我们考虑$\alpha+\beta$和$2\alpha+\beta$,则必有这两个向量都不属于$V_2$(否则有$\beta\in V_2$),并且这两个向量也不能同时属于$V_1$(否则两个向量相减等于$\alpha$也属于$V_1$,矛盾). 这就说明这两个向量中至少有一个既不在$V_1$中也不在$V_2$中,因此结论成立.

        \item 对于$s>2$,假设命题对$s-1$个子空间成立,即$V$中存在向量$\alpha\notin V_1\cup V_2\cup\cdots\cup V_{s-1}$. 若$\alpha\notin V_s$,则结论成立. 若$\alpha\in V_s$,由$V_s$非平凡知存在$\beta\notin V_s$. 我们考虑$\alpha+\beta,2\alpha+\beta,\ldots,s\alpha+\beta$,则与归纳基础中同样的原因,必有这$s$个向量都不属于$V_s$,且这$s$个向量中不可能存在两个向量同属于一个$V_i\enspace(i=1,2,\ldots,s-1)$,因此这$s$个向量中至少有一个不在$V_1\cup V_2\cup\cdots\cup V_s$中,因此结论成立.
    \end{enumerate}
\end{proof}

本质而言$s>2$的情况就是将$s-1$个子空间的并视为一个整体,然后套用$s=2$的情况证明. 若将这一定理的条件限制在$V$为有限维线性空间,我们也可以利用Vandermonde行列式的方法证明,详见\autoref{ex:行列式证明覆盖定理}.我们已经在本章习题A组最后两题为读者准备了覆盖定理的直接证明的题目,下面再给出一个例子供读者应用覆盖定理:
\begin{example}{}{}
    $V_1,V_2,\ldots,V_s$是线性空间$V$的$s$个非平凡子空间,证明:存在$V$的一组基$\alpha_1,\alpha_2,\ldots,\alpha_n$都不在$V_1,V_2,\ldots,V_s$中.
\end{example}

\begin{proof}
    由\nameref{thm:覆盖定理},$V$中存在向量$\alpha_1\notin V_1\cup V_2\cup\cdots\cup V_s$. 继续取$\alpha_2\notin V_1\cup V_2\cup\cdots\cup V_s\cup\spa(\alpha_1)$,则一定有$\alpha_1,\alpha_2$线性无关. 继续取$\alpha_3\notin V_1\cup V_2\cup\cdots\cup V_s\cup\spa(\alpha_1,\alpha_2)$,则一定有$\alpha_1,\alpha_2,\alpha_3$线性无关. 以此类推,最终得到一组基$\alpha_1,\alpha_2,\ldots,\alpha_n$都不在$V_1,V_2,\ldots,V_s$中.
\end{proof}

\section{维数公式}

\begin{theorem}{线性空间维数公式}{线性空间维数公式}
    设$W_1,W_2$是线性空间$V(\mathbf{F})$的两个子空间,则
    \[\dim W_1+\dim W_2=\dim(W_1+W_2)+\dim(W_1\cap W_2).\]
\end{theorem}
上式称为子空间的维数公式,区别于下一专题中的线性映射基本定理的维数公式. 这一定理的证明思想非常重要,因此此处我们给出证明.

\begin{proof}
    设$\dim W_1=s,\enspace \dim W_2=t,\enspace \dim(W_1\cap W_2)=r$. 设$W_1\cap W_2$的一组基为$\alpha_1,\alpha_2,\ldots,\alpha_r$,则可以扩充为$W_1$的一组基,记为$\alpha_1,\alpha_2,\ldots,\alpha_r,\beta_1,\ldots,\beta_{s-r}$;也可以扩充为$W_2$的一组基,记为$\alpha_1,\alpha_2,\ldots,\alpha_r,\gamma_1,\ldots,\gamma_{t-r}$. 则我们有
    \[W_1+W_2=\spa(\alpha_1,\ldots,\alpha_r,\beta_1,\ldots,\beta_{s-r},\gamma_1,\ldots,\gamma_{t-r})\]
    (如果对这一步有疑问可以回顾\autoref{ex:子空间运算} 中的证明). 由此,我们要证$\dim (W_1+W_2)=s+t-r$,只需证$\alpha_1,\ldots,\alpha_r,\beta_1,\ldots,\beta_{s-r},\gamma_1,\ldots,\gamma_{t-r}$线性无关. 为此,我们设
    \begin{equation}\label{eq:4:维数公式证明1}
        a_1\alpha_1+\cdots+a_r\alpha_r+b_1\beta_1+\cdots+b_{s-r}\beta_{s-r}+c_1\gamma_1+\cdots+c_{t-r}\gamma_{t-r}=0,
    \end{equation}
    即
    \begin{equation}\label{eq:4:维数公式证明2}
        a_1\alpha_1+\cdots+a_r\alpha_r+b_1\beta_1+\cdots+b_{s-r}\beta_{s-r}=-c_1\gamma_1-\cdots-c_{t-r}\gamma_{t-r}.
    \end{equation}
    显然,\autoref{eq:4:维数公式证明2} 等号两端的向量分别属于$W_1$和$W_2$,因此它们都属于$W_1\cap W_2$,因此都可以被$W_1\cap W_2$的基线性表示,即
    \[-c_1\gamma_1-\cdots-c_{t-r}\gamma_{t-r}=d_1\alpha_1+\cdots+d_r\alpha_r,\]
    即
    \begin{equation}\label{eq:4:维数公式证明3}
        c_1\gamma_1+\cdots+c_{t-r}\gamma_{t-r}+d_1\alpha_1+\cdots+d_r\alpha_r=0.
    \end{equation}
    由于$\alpha_1,\ldots,\alpha_r,\gamma_1,\ldots,\gamma_{t-r}$是$W_2$的基,因此\autoref{eq:4:维数公式证明3} 所有系数都为0,即$c_1=\cdots=c_{t-r}=d_1=\cdots=d_r=0$. 代入\autoref{eq:4:维数公式证明2} 后,由于$\alpha_1,\ldots,\alpha_r,\beta_1,\ldots,\beta_{s-r}$是$W_1$的基,因此可得$a_1=\cdots=a_r=b_1=\cdots=b_{s-r}=0$,因此,代入\autoref{eq:4:维数公式证明1} 后可知$\alpha_1,\ldots,\alpha_r,\beta_1,\ldots,\beta_{s-r},\gamma_1,\ldots,\gamma_{t-r}$必定线性无关(因为根据前述证明所有系数只能为0),故得证.
\end{proof}

总结而言,这一定理证明用到的思想就是``设小扩大''. 我们设出最小空间$V_1\cap V_2$的基,然后分别扩充为$V_1$和$V_2$的基,然后观察要证明的等式和已知的联系,然后利用\autoref{eq:4:维数公式证明2} 构造等式两边属于不同空间的向量这一技巧即可. 下面是一个证明思想类似的例子,需要用到矩阵的相关知识,暂未学到的同学可以先略过本题:
\begin{example}{}{}
    已知$A,B$分别是数域$\mathbf{F}$上的$l \times k$和$k \times n$矩阵,$X$是$n \times 1$的列向量. 证明:所有满足$ABX=0$的$BX$构成一个线性空间$V$,且$\dim V = r(B) - r(AB)$.
\end{example}

\begin{proof}
    $V$是线性空间只需要说明其中元素关于加法数乘封闭即可,因为这样$V$就是$\mathbf{F}^k$的子空间. 这一证明非常基本,我们在此略过.

    记$V_1=\{X\mid BX=0\},\enspace V_2=\{X\mid ABX=0\}$,则$V_1\subseteq V_2$,因为$\forall X\in V_1$,有$BX=0$,因此$ABX=A0=0$,即$X\in V_2$,因此$V_1\subseteq V_2$. 利用``设小扩大''的思想,取$V_1$的一组基$\alpha_1,\ldots,\alpha_r$,则可以扩充为$V_2$的一组基,记为$\alpha_1,\ldots,\alpha_r,\alpha_{r+1},\ldots,\alpha_m$,则$r=n-r(B)$,$s=n-r(AB)$,于是
    \begin{align*}
        V & =\{BX\mid ABX=0\}                                                \\
          & =\spa(B\alpha_1,\ldots,B\alpha_r,B\alpha_{r+1},\ldots,B\alpha_m) \\
          & =\spa(B\alpha_{r+1},\ldots,B\alpha_m).
    \end{align*}
    下面证明$B\alpha_{r+1},\ldots,B\alpha_m$线性无关. 为此,设
    \[c_{r+1}B\alpha_{r+1}+\cdots+c_mB\alpha_m=0,\]
    则
    \[B(c_{r+1}\alpha_{r+1}+\cdots+c_m\alpha_m)=0,\]
    因此$c_{r+1}\alpha_{r+1}+\cdots+c_m\alpha_m\in V_1$,因此存在$c_1,\ldots,c_r$使得
    \[c_{r+1}\alpha_{r+1}+\cdots+c_m\alpha_n=c_1\alpha_1+\cdots+c_r\alpha_r,\]
    即
    \[c_{r+1}\alpha_{r+1}+\cdots+c_m\alpha_m-c_1\alpha_1-\cdots-c_r\alpha_r=0.\]
    由于$\alpha_1,\ldots,\alpha_m$线性无关,因此
    \[c_{r+1}=\cdots=c_m=c_1=c_2=\cdots=c_r=0,\]
    因此$B\alpha_{r+1},\ldots,B\alpha_m$线性无关,因此$V$的维数为$s-r=(n-r(AB))-(n-r(B))=r(B)-r(AB)$,得证.
\end{proof}

\section{线性空间的直和}

我们将来证明或者利用和空间时,很多时候都是利用和空间定义进行向量分解. 我们特别重视分解唯一时的情形,因为这对我们的研究很有帮助,这时的和即为直和. 严谨而言,我们有如下定义:
\begin{definition}{}{}
    设$W_1,W_2$是线性空间$V(\mathbf{F})$的两个子空间. 若$W_1 \cap W_2=\{0\}$,则$W_1+W_2$叫做$W_1$与$W_2$的\term{直和}\index{zhihe@直和 (direct sum)},记作$W_1\oplus W_2$.

    进一步地,若$V=W_1\oplus W_2$,则称$W_1,W_2$为\term{互补子空间}\index{xianxingkongjian!hubu@互补子空间 (complementary subspaces)},或$W_1$是$W_2$的补空间,或$W_2$是$W_1$的补空间.
\end{definition}

直和有以下等价的命题,我们证明或者利用直和都可以任意选择:
\begin{theorem}{}{直和等价命题}
    对于子空间$W_1,W_2$,下列命题等价:
    \begin{enumerate}
        \item $W_1+W_2$是直和,即$W_1 \cap W_2=\{0\}$;

        \item $W_1+W_2$中的每个向量$\alpha$的分解式$\alpha=\alpha_1+\alpha_2\enspace(\alpha_1\in W_1,\enspace\alpha_2\in W_2)$唯一;

        \item 零向量的分解式$\vec{0}=\alpha_1+\alpha_2 \enspace(\alpha_1\in W_1,\enspace\alpha_2\in W_2)$仅当$\alpha_1=\alpha_2=\vec{0}$时成立;

        \item $\dim (W_1+W_2)=\dim W_1+\dim W_2$.
    \end{enumerate}
\end{theorem}

定理的证明是基本的.
\begin{proof}
    上述命题等价,只需证明 (1) $\implies$ (2) $\implies$ (3) $\implies$ (4) $\implies$ (1).

    先证 (1) $\implies$ (2): 设 $W_1 + W_2$ 中的 $\alpha$ 有两个分解式
    $$
    \alpha = \alpha_1 + \alpha_2 = \beta_1 + \beta_2, \quad \alpha_1, \beta_1 \in W_1, \ \alpha_2, \beta_2 \in W_2,
    $$
    则 $\alpha_1 - \beta_1 = \beta_2 - \alpha_2 \in W_1 \cap W_2 = \{\mathbf{0}\}$,于是得 $\beta_1 = \alpha_1, \beta_2 = \alpha_2$,故 $\alpha$ 的分解式是唯一的。

    其次证 (2) $\implies$ (3): 由 $W_1 + W_2$ 中零向量的分解式唯一性以及 $0 = 0 + 0$,立即得命题 (3) 成立.

    再证 (3) $\implies$ (4): 由命题 (3) 可推出 $W_1 \cap W_2 = \{\mathbf{0}\}$,因为若 $W_1 \cap W_2 \ne \{\mathbf{0}\}$,则存在 $0 \ne a \in W_1 \cap W_2$,使得 $0 = \alpha + (-\alpha)$,(其中 $\alpha \in W_1, -\alpha \in W_2$),这与命题 (3) 相矛盾。再根据维数公式 就得命题 (4)。

    最后由命题 (4) 及维数公式 立即得命题 (1) 成立.
\end{proof}
在实际运用中我们要非常熟悉这些等价条件,因为都可能使用到.

我们也可以定义有限个子空间的直和,即$V=W_1\oplus W_2\oplus\cdots\oplus W_n \iff W_i \cap \sum\limits_{j \neq i}W_j=\{0\}$,即每个子空间与其余子空间的和的交都是$\{0\}$. 等价命题也是上述定理的推广,例如唯一分解、$\vec{0}$的分解以及维数公式推广,此处不再赘述. 除此之外,我们还有一个与多空间直和相关的定理:
\begin{theorem}{}{多空间直和}
    若$V=V_1\oplus V_2,\enspace V_1=V_{11}\oplus\cdots\oplus V_{1s},\enspace V_2=V_{21}\oplus\cdots\oplus V_{2t}$,则
    \[V=V_{11}\oplus\cdots\oplus V_{1s}\oplus V_{21}\oplus\cdots\oplus V_{2t}\]
\end{theorem}
这一定理的证明是很简单的,实际上利用零向量分解唯一即可. 进一步地,我们可以定义无穷个子空间的直和,推广\autoref{thm:直和等价命题}的 $2$,无穷个子空间的直和结果中的元素也可以表达为每个子空间的元素之和,但我们需要额外要求和式中只有有限项非零. 直观地,这其实是为了保证和式的收敛性.

在习题中我们证明直和一般有两种思路,一种是先证和,再证直和,我们来看一个例子(没有学到矩阵的可以先略过):
\begin{example}{}{}
    数域$\mathbf{F}$上所有$n$阶方阵组成的线性空间$V=\mathbf{M}_n(\mathbf{F})$,$V_1$表示所有对称矩阵组成的集合,$V_2$表示所有反对称矩阵组成的集合. 证明:$V_1,V_2$都是$V$的子空间,且$V=V_1\oplus V_2$.
\end{example}

\begin{proof}
    首先证明和. 事实上,对于任意矩阵$A\in V$,有
    \[A=B+C,\enspace B=\frac{1}{2}(A+A^T),\enspace C=\frac{1}{2}(A-A^T),\]
    其中$B$是对称矩阵,$C$是反对称矩阵,即$B\in V_1$,$C\in V_2$,因此$V_1+V_2=V$(因为$V$中任意元素都可以写成$V_1$和$V_2$元素和的形式,根据和的定义可知成立).

    下面证明直和. 我们有如下三种方法:
    \begin{enumerate}
        \item 利用零向量分解唯一:设$O$是$n$阶零矩阵,设$O=B+C$,其中$B$是对称矩阵,$C$是反对称矩阵. 由于$B$是对称矩阵,因此$B^T=B$,由于$C$是反对称矩阵,因此$C^T=-C$,因此
              \[O=O^T=(B+C)^T=B^T+C^T=B-C\]
              解得$B=C=O$,因此零向量分解唯一,故直和得证;

        \item 利用$V_1\cap V_2=\{0\}$:设$A\in V_1\cap V_2$,则$A=A^T=-A$,因此$A=-A$,即$A=O$,因此$V_1\cap V_2=\{0\}$,故直和得证;

        \item 利用$\dim V_1+\dim V_2=\dim V$:这一方法较为复杂,我们简单阐述思想. 设$E_{ij}$是第$i$行第$j$列元素为1,其余元素为0的矩阵,则$V$的一组基为$E_{ij},\enspace i,j=1,2,\ldots,n$,$V_1$的一组基为$E_{ij}+E_{ji},\enspace i<j$和$E_{ii},\enspace i=1,2,\ldots,n$,$V_2$的一组基为$E_{ij}-E_{ji},\enspace i<j$,则$\dim V_1=\dfrac{n(n-1)}{2},\enspace \dim V_2=\dfrac{n(n-1)}{2}$,因此$\dim V_1+\dim V_2=n^2$,因此$\dim V_1+\dim V_2=\dim V$,故直和得证.
    \end{enumerate}
\end{proof}

还有一种证明$V=V_1\oplus V_2$的方式是先令$W=V_1+V_2$,先证明和为直和(即交为$\{0\}$)再证$W=V$即可,下面是一个例子:
\begin{example}{}{}
    设$A$是数域$\mathbf{F}$上的一个$n$阶可逆方阵,$A$的前$r$个行向量组成的矩阵为$B$,后$n-r$个行向量组成的矩阵为$C$,$n$元线性方程组$BX=0$与$CX=0$的解空间分别为$V_1,V_2$. 证明:$\mathbf{F}^n=V_1\oplus V_2$.
\end{example}

\begin{proof}
    先记$W=V_1+V_2$,若$\alpha\in V_1\cap V_2$,则$B\alpha=C\alpha=0$,所以
    \[A\alpha=\begin{pmatrix}
            B \\
            C
        \end{pmatrix}\alpha=0,\]
    由于$A$可逆,因此$\alpha=0$,即$V_1\cap V_2=\{0\}$,因此$V_1+V_2$是直和,因此只需证$W=\mathbf{F}^n$即可. 事实上,我们知道$r(B)=r,r(C)=n-r$,因此$\dim V_1=n-r,\enspace \dim V_2=n-(n-r)=r$,所以
    \[\dim W=\dim V_1+\dim V_2=n-r+r=n=\dim \mathbf{F}^n,\]
    又$W=V_1\oplus V_2\subseteq \mathbf{F}^n$,因此$W=\mathbf{F}^n$,故得证.
\end{proof}

最后我们要提醒读者注意的是,有限维线性空间的一个子空间的补空间并不唯一,如下面的例子:
\begin{example}{}{}
    在$\mathbf{R}^3$中,$W_1=\spa(\alpha_1)$,则其补空间根据直和的维数公式可知为2,记为$W_2=\spa(\alpha_2,\alpha_3)$. 实际上只需要$\alpha_1,\alpha_2,\alpha_3$线性无关即可,事实上这样的选择是有无穷种的,因为$W_1$本质表示一条直线,故$W_2$是不包含$W_1$且不与$W_1$平行的平面即可,这样$\alpha_2,\alpha_3$是$W_2$任意一组基都可以.
\end{example}

\section{商空间}

这一节我们需要引入代数中一个非常重要的运算,即某个代数结构的商. 在第一讲中我们给出了一个非常重要的概念——等价类,这里的目标是在一个线性空间 $V$ 中定义出等价关系,即利用\autoref{thm:等价类的性质},以某种方式将一个线性空间中的所有向量划分为几个不相交的等价类的并集,最后在此基础上定义出线性空间的商运算.

\subsection{从等价关系出发}

为了定义出这个等价关系,我们首先需要确定的是它应当满足怎样的性质,否则对于如何定义这一等价关系我们将毫无头绪. 这一问题的出发点事实上就隐藏在\autoref{ex:有限域} 后关于``相容''的讨论中. 在那里,我们要求从一般整数加法和乘法继承下来的模$n$剩余类上的加法和乘法是良定义的,类比于这里的目标,则是希望线性空间中等价类的集合上(也就是商集)定义的加法和数乘运算,可以自然地继承一般的向量加法和乘法,并且保证相容性.

下面我们开始形式化地把上面抽象的描述转化为表达式. 我们需要在线性空间$V$中定义一个等价关系$R$,得到一个等价类构成的商集:
\[V/R=\{\overline{v_1},\overline{v_2},\ldots,\overline{v_n},\ldots\},\]
其中$\overline{v_i}$表示$v_i\in V$所在的等价类,取$v_i$为代表元. 需要注意的是,这里的等价类不一定是有限个,因此最后还是省略号.

接下来我们需要定义等价类之间的运算,我们希望自然继承线性空间的加法和数乘运算,故我们定义商集上的加法和数乘运算满足对于任意的$\overline{v_1},\overline{v_2}\in V/R$和$\lambda\in\mathbf{F}$,有
\begin{equation} \label{eq:10:商集运算}
    \begin{gathered}
        \overline{v_1}+\overline{v_2}=\overline{v_1+v_2},\\
        \lambda\overline{v_1}=\overline{\lambda v_1}.
    \end{gathered}
\end{equation}
需要注意的是,等号左边的运算是商集$V/R$上的,右边是线性空间$V$上的,这里不再像模$n$剩余类上的运算定义那样使用不同的符号(例如加法改成$\oplus$)是出于习惯,以及相信读者学习到今天应当适应了为记号上方便做出的牺牲(比如对于任何线性空间,即使加法不是定义成最一般的加法,但也写成加号的形式).

最后我们需要上面的运算保证相容性,也就是说,对于任意的$v_1,v_2,u_1,u_2\in V$,如果$v_1Rv_2$,$u_1Ru_2$则$(v_1+u_1)R(v_2+u_2)$,$\lambda v_1R\lambda v_2$,展开写为:
\begin{gather*}
    \overline{v_1+u_1}=\overline{v_2+u_2},\\
    \overline{\lambda v_1}=\overline{\lambda v_2}.
\end{gather*}

推进到这里或许我们还是很难看出如何定义这一等价关系,但我们可以从简单的角度入手,逐步观察这些等价类的特点. 我们可以首先考虑线性空间的零向量所在的等价类具有什么性质. 事实上,我们很容易得到如下定理:
\begin{theorem}{}{}
    线性空间$V$的零向量所在的等价类$\overline{0}$一定是$V$的子空间.
\end{theorem}
\begin{proof}
    \begin{enumerate}
        \item 加法封闭:$\forall \alpha,\beta\in\overline{0}$,则$\alpha\,R\,0$,$\beta\,R\,0$,因此根据相容性,$\alpha+\beta\,R\,0$,即$\alpha+\beta\in\overline{0}$;
        \item 数乘封闭:$\forall \alpha\in\overline{0}$,则$\alpha\,R\,0$,$\lambda\in\mathbf{F}$,因此根据相容性,$\lambda\alpha\,R\,0$,即$\lambda\alpha\in\overline{0}$.
    \end{enumerate}
\end{proof}

这一结论非常关键,它使得我们把抽象的等价关系与一个子空间绑定. 我们记这一子空间为$U$,即$\overline{0}=U$,于是下面这一结论也是容易得到的:
\begin{theorem}{}{}
    设$v_1,v_2\in V$,则$v_1Rv_2$当且仅当$v_1-v_2\in U$.
\end{theorem}
\begin{proof}
    \begin{enumerate}
        \item ($\implies$) 直接取$u_1,u_2\in U$,即$u_1\,R\,0$,$u_2\,R\,0$,因此根据相容性,$\overline{v_1}+\overline{u_1}=\overline{v_2}+\overline{u_2}$,移项得$\overline{v_1}-\overline{v_2}=\overline{u_2}-\overline{u_1}$,注意到$\overline{v_1}-\overline{v_2}=\overline{v_1}+(-1)\cdot\overline{v_2}=\overline{v_1}+\overline{-v_2}=\overline{v_1-v_2}$,同理有$\overline{u_2}-\overline{u_1}=\overline{u_2-u_1}$,因此$\overline{v_1-v_2}=\overline{u_2-u_1}=\overline{0}$,即$v_1-v_2\in U$;
        \item 设$v_1-v_2=u\in U$,则$\overline{v_1}=\overline{v_2}+\overline{u}=\overline{v_2}+\overline{0}=\overline{v_2}$,因此$v_1Rv_2$.
    \end{enumerate}
\end{proof}

至此,我们完成了对线性空间$V$上的符合相容性的等价关系$R$的性质的讨论. 我们发现,尽管运算定义和相容性的要求非常抽象,但是在线性空间的背景下,我们成功地将$R$与一个子空间$U$对应起来,这个子空间实际上就是等价类$\overline{0}$. 反过来,当我们需要定义线性空间的等价类的时候,我们可以从一个子空间$U$出发,然后定义等价关系$R$为
\begin{equation} \label{eq:10:线性空间等价关系}
    \forall\alpha,\beta\in V,\enspace\alpha\,R\,\beta\iff \alpha-\beta\in U.
\end{equation}
事实上,验证这一关系的确是等价关系是非常简单的:
\begin{enumerate}
    \item (自反性) $\forall \alpha\in V,\enspace\alpha-\alpha=0\in U$,故$\alpha\,R\,\alpha$;

    \item (对称性) $\forall \alpha,\beta\in V,\enspace\alpha\,R\,\beta\implies \alpha-\beta\in U\implies \beta-\alpha=-(\alpha-\beta)\in U\implies \beta\,R\,\alpha$;

    \item (传递性) $\forall \alpha,\beta,\gamma\in V,\enspace\alpha\,R\,\beta,\enspace\beta\,R\,\gamma\implies \alpha-\beta\in U,\enspace\beta-\gamma\in U\implies \alpha-\gamma=(\alpha-\beta)+(\beta-\gamma)\in U\implies \alpha\,R\,\gamma$.
\end{enumerate}
基于此,下一节开始我们将正式给出商空间的定义. 我们可以首先可以定义商集$V/R$是由等价类构成的集合,在线性空间的背景下,由于我们知道$R$是从一个子空间$U$出发的,因此我们也将商集记为$V/U$. 我们将按照自然的方式定义商集中的元素的加法和数乘运算,并证明实际上商集可以构成线性空间,于是称其为商空间,下面我们开始我们严格的陈述.

\subsection{仿射子集与商空间}

紧接着上一节末尾的思路,我们首先定义线性空间等价关系的商集. 研究商集,事实上首先需要研究等价类的性质. 回顾\autoref{eq:10:线性空间等价关系},我们知道向量$\alpha\in V$所在的等价类为:
\[\overline{\alpha}=\{\beta\in V \mid \beta\,R\,\alpha\}=\{\beta\in V \mid \beta-\alpha\in U\}=\{\beta\in V \mid \beta=\alpha+\gamma,\enspace\gamma\in U\}\]
最后一个集合还可以进一步写成$\{\alpha+\gamma \mid \gamma\in U\}$,我们记为$\alpha+U$,称之为$V$的仿射子集. 我们给出如下完整的定义:
\begin{definition}{仿射子集}{} \index{fangsheziji@仿射子集 (affine subset)}
    设$v\in V$,$U$是$V$的子空间,则$V$的\term{仿射子集}是$V$的形如$v+U$的子集,其中$v+U$定义为
    \[v+U=\{v+u \mid u\in U\}.\]
\end{definition}
我们知道,仿射子集就是我们在线性空间上定义的等价关系的等价类. 基于等价类的性质,我们有如下定理:
\begin{theorem}{}{}
    设$U$是$V$的子空间,$v,w\in V$,则以下陈述等价:
    \begin{enumerate}
        \item $v-w\in U$;
        \item $v+U=w+U$;
        \item $(v+U)\cap(w+U)\neq \varnothing$.
    \end{enumerate}
\end{theorem}

还需要强调的一点是,$(v+U)+(w+U)$与$(v+w)+U$是完全相同的集合,等价性是显然的,我们只需要展开写出仿射子集定义然后证明两个集合互相包含即可. 当然更一般的情形为
\[(v_1+U)+(v_2+U)+\cdots+(v_n+U)=(v_1+v_2+\cdots+v_n)+U.\]
相信读者对``仿射''一词并不完全陌生,仿射变换实际上就是形如\[\vec{y}=A\vec{x}+\vec{b}\]的映射,其中$\vec{y},\vec{x},\vec{b}$为向量,$A$是一个矩阵. 实际上一元向量的情况就对应着一条斜率为$A$截距为$b$的直线. 事实上,若$V$为二维空间(平面),$U$为$V$的一维子空间,则其几何意义就是一条过原点的直线,而集合$v+U$实际上将原集合所有点沿着$v$的方向平移,可以得到截距不为0的直线,这就体现了``仿射''一词的意义. 高维空间则是同理,只是我们很难直观地看到这一点. 因此,我们也可以称仿射子集$v+U$\textbf{\heiti 平行于}$U$. 当然,在我们讨论完对偶后,我们会再来审视仿射子集更深的含义.

下面的例子给出了仿射子集的一种等价描述,基于此我们可以对仿射子集中向量的结构有更进一步的了解:
\begin{example}{}{仿射子集性质}
    证明:$V$的非空子集$A$是$V$的仿射子集当且仅当对所有的$v,w\in A$和$\lambda\in\mathbf{F}$均有$\lambda v+(1-\lambda)w\in A$.
\end{example}

\begin{solution}

\end{solution}

事实上,结合我们之前所说的仿射子集几何意义,这一结论在平面上来看正是我们高中学习的平面向量中学习的三点共线的等价条件的同义表达:
\begin{theorem}{}{}
    设$P,A,B,C$是平面上四点,$P$与$A,B$不共线,则$C$与$A,B$共线等价于存在$\lambda\in\mathbf{R}$使得$\overrightarrow{PC}=\lambda\overrightarrow{PA}+(1-\lambda)\overrightarrow{PB}$.
\end{theorem}
同时我们发现仿射子集实际上是我们在数学分析或微积分学习的凸集的特殊形式,在凸集中我们只要求$\lambda\in[0,1]$,这里我们要求整个数域上的点都要有\autoref{ex:仿射子集性质} 所述的性质. 当然这不是线性代数中研究的内容,感兴趣的同学可以学习凸优化的相关课程进一步了解.

事实上在习题中我们将给出\autoref*{ex:仿射子集性质} 更一般的形式,我们可以回忆\autoref{thm:线性扩张构造子空间},就会发现仿射子集的结构和线性空间保留了一些相似性,即虽然不能像线性空间一样保证加法数乘运算封闭,但仿射子集一定是保证凸组合封闭的集合.

定义了等价类(即仿射子集)并研究了其性质后,我们可以定义相应的商集(即由全体等价类构成的集合),我们称之为商空间:
\begin{definition}{}{}
    设$U$是$V$的子空间,则商空间$V/U$是指所有由\autoref{eq:10:线性空间等价关系} 诱导的等价类构成的集合,即$V$的所有平行于$U$的仿射子集的集合,即
    \[V/U=\{v+U \mid v\in V\}.\]
\end{definition}
我们希望这些这一商集(商空间)真的构成线性空间,因此还需要定义加法和数乘运算. 定义则与上一节中的要求一样,是自然继承向量加法而来的,即满足\autoref{eq:10:商集运算},我们把式中等价类写为仿射子集的形式即可得到如下定义:
\begin{definition}{}{}
    设$U$是$V$的子空间,则商空间$V/U$上的加法和数乘运算定义为:$\forall \alpha,\beta\in V$和$\lambda\in\mathbf{F}$,
    \begin{gather*}
        (\alpha+U)+(\beta+U)=(\alpha+\beta)+U, \\
        \lambda(\alpha+U)=(\lambda\alpha)+U.
    \end{gather*}
\end{definition}
我们很容易根据线性空间8条性质验证商空间在上述加法和数乘运算定义下构成线性空间,在此不再赘述. 特别注意这一线性空间的零向量是特别的,应当为$U$(即$\vec{0}+U$,一定注意不是$\vec{0}$,读者在验证商空间是线性空间时就会发现).

正常而言,在定义了一个线性空间后我们自然地想了解它的基本结构——基和维数,商空间也不例外. 我们有很多的角度来得到相关的结论,这里首先介绍一个直接的方式得到关于维数的结论,其余的方法我们将在介绍完线性映射和对偶空间后讨论.

\begin{theorem}{商空间的维数公式}{商空间的维数公式}
    设$U$是有限维线性空间$V$的子空间,则
    \[\dim V/U=\dim V-\dim U.\]
\end{theorem}

这一定理的形式与维数公式完全类似,都是几个线性空间之间的维数的等式关系,因此我们有理由相信证明思想也会是类似的,即``设小扩大'':

\begin{proof}
    取$U$的一组基$\alpha_1,\alpha_2,\ldots,\alpha_s$,将其扩充为$V$的一组基$\alpha_1,\alpha_2,\ldots,\alpha_s,\alpha_{s+1},\ldots,\alpha_n$. 于是我们要证的转化为$\dim V/U=n-s$,即证明$V/U$的一组基的长度为$n-s$.

    类似于线性映射基本定理的证明,我们可以依靠直觉猜想. 我们猜想$V/U$的一组基为$\{\alpha_{s+1}+U,\alpha_{s+2}+U,\ldots,\alpha_n+U\}$. 这是很自然的想法. 我们只需要验证这组基的两个条件:线性无关和张成性:
    \begin{enumerate}
        \item 线性无关:设$\lambda_{s+1},\lambda_{s+2},\ldots,\lambda_n\in\mathbf{F}$,使得
              \[\lambda_{s+1}(\alpha_{s+1}+U)+\lambda_{s+2}(\alpha_{s+2}+U)+\cdots+\lambda_n(\alpha_n+U)=U.\]
              特别注意这里的零元是$\vec{0}+U=U$,实际上,上式等价于
              \[(\lambda_{s+1}\alpha_{s+1}+\lambda_{s+2}\alpha_{s+2}+\cdots+\lambda_n\alpha_n)+U=U.\]
              根据仿射子集定义,$\lambda_{s+1}\alpha_{s+1}+\lambda_{s+2}\alpha_{s+2}+\cdots+\lambda_n\alpha_n\in U$,因此可以被表示为$U$的基的线性组合,即
              \[\lambda_{s+1}\alpha_{s+1}+\lambda_{s+2}\alpha_{s+2}+\cdots+\lambda_n\alpha_n=\mu_1\alpha_1+\mu_2\alpha_2+\cdots+\mu_s\alpha_s.\]
              于是我们有
              \[\lambda_{s+1}\alpha_{s+1}+\lambda_{s+2}\alpha_{s+2}+\cdots+\lambda_n\alpha_n-\mu_1\alpha_1-\mu_2\alpha_2-\cdots-\mu_s\alpha_s=0.\]
              由于$\alpha_1,\alpha_2,\ldots,\alpha_n$是$V$的一组基,因此我们有$\lambda_{s+1}=\lambda_{s+2}=\cdots=\lambda_n=\mu_1=\mu_2=\cdots=\mu_s=0$. 从而$\alpha_{s+1}+U,\alpha_{s+2}+U,\ldots,\alpha_n+U$线性无关;

        \item 张成空间:$\forall\alpha+U\in V/U$,其中$\alpha\in V$,我们有$\alpha$可以被$V$的基线性表示为
              \[\alpha=\lambda_1\alpha_1+\lambda_2\alpha_2+\cdots+\lambda_n\alpha_n.\]
              于是
              \begin{align*}
                  \alpha+U & =(\lambda_1\alpha_1+\lambda_2\alpha_2+\cdots+\lambda_n\alpha_n)+U         \\
                           & =(\lambda_1\alpha_1+U)+(\lambda_2\alpha_2+U)+\cdots+(\lambda_n\alpha_n+U) \\
                           & =\lambda_1(\alpha_1+U)+\lambda_2(\alpha_2+U)+\cdots+\lambda_n(\alpha_n+U)
              \end{align*}
              因此$V/U$中任意元素均可被$\alpha_{s+1}+U,\alpha_{s+2}+U,\ldots,\alpha_n+U$线性表示,即$\alpha_{s+1}+U,\alpha_{s+2}+U,\ldots,\alpha_n+U$张成$V/U$.
    \end{enumerate}
\end{proof}

由此我们知道了商空间的维数表达式,也在通过证明过程知道了如何得到商空间的一组基. 下面我们来计算一个例子:

\begin{example}{}{}
    设$A$是$\mathbf{R}$上的$2\times 3$矩阵:
    \[A=\begin{pmatrix}
            1 & -1 & 2 \\ 1 & 0 & -1
        \end{pmatrix}.\]
    \begin{enumerate}
        \item 求齐次线性方程组$AX=0$的解空间$W$的一组基;

        \item 求商空间$\mathbf{R}^3/W$的维数和一组基.
    \end{enumerate}
\end{example}

\begin{solution}

\end{solution}

\begin{summary}

    本讲我们首先介绍了线性空间之间的三种运算——交、并、和. 和的概念初次见到可能有些许抽象,但经过一些例子之后我们应当能理解为什么线性空间不同于普通集合,更常用``和''这一运算. 关于并我们给出了一些构成线性空间的条件以及一个重要的覆盖定理,读者了解即可. 关于交与和我们给出了一个维数公式,它不仅结论非常重要,``设小扩大''的证明思想也是在未来非常常用的. 进一步地,我们讨论了直和的概念以及它的等价条件,以及证明直和的两种思路. 我们必须要重视直和这一概念,因为它在未来关于线性变换矩阵约化表示的讨论中起到重要的桥梁作用.

    商运算是代数学中的一种重要的运算,我们从``如何在线性空间中定义相容的等价类上的运算''出发,得到了线性空间上等价关系的定义方式,从而进一步定义了仿射子集和商空间的概念:我相信只要读者理解了等价类的相关内容,商空间也应当是不会太抽象的. 然后我们讨论了商空间的维数公式,这与线性空间交与和的维数公式证明有异曲同工之妙. 当然这一公式的证明有非常多的方式,我们将在线性映射、对偶空间的讨论中再次回顾这一公式.

    除此之外,线性空间之间还有一种基于笛卡尔积的运算,我们将在后续讨论同构的时候作为一个应用讨论,将直和与笛卡尔积联系起来.

\end{summary}

\begin{exercise}
    \exquote[H. 庞加莱(Henri Poincaré)]{在选择了一套恰当的语言之后,我们会惊讶地发现,所有对于已知对象的阐述都能被立刻推广到许多新的对象上:不需要进行任何改写,甚至包括术语,因为在这种语言下,所有的名字也是一致的。}

    \begin{exgroup}
        \item 设$V=\{(a_1,a_2,a_3,a_4) \mid a_1+a_2+a_3+a_4=0\}$,$W=\{(a_1,a_2,a_3,a_4) \mid a_1-a_2-a_3+a_4=0,a_1+a_2+a_3-a_4=0\}$.
        \begin{enumerate}
            \item 证明:$V$和$W$为$\mathbf{R}^4$的子空间;

            \item 分别求$V \cap W$,$V+W$以及$W$的补空间的维数与一组基.
        \end{enumerate}

        \item 设 $f_1=-1+x,\ f_2=1-x^2,\ f_3=1-x^3,\ g_1=x-x^2,\ g_2=x+x^3,\ V_1=\spa(f_1,f_2,f_3),\ V_2=\spa(g_1,g_2)$,求:
        \begin{enumerate}
            \item $V_1+V_2$ 的基和维数;

            \item $V_1 \cap V_2$ 的基和维数;

            \item $V_2$ 在 $\mathbf{R}[x]_4$ 空间的补.
        \end{enumerate}

        \item 在数域$\mathbf{F}$上,已知$V_1,V_2$分别为方程组$x_1+x_2+\cdots+x_n=0$与$x_1=x_2=\cdots=x_n$的解空间. 证明:$\mathbf{F}^n=V_1\oplus V_2$.

        \item 设 $V_1,V_2$ 是线性空间 $V$ 的两个非平凡子空间,证明: $\exists \alpha \in V$,使得$\alpha \notin V_1$ 且 $\alpha \notin V_2$.并在$\mathbf{R}^3$ 中举一例.

        \item 设 $V_1,\ldots,V_m (m > 2)$ 是线性空间$V$的$m$个非平凡子空间,证明:$V$ 中存在一个同时不属于任何一个$V_i(1 \leqslant i \leqslant m)$的向量.并在$\mathbf{R}^3$ 中举一例.
    \end{exgroup}

    \begin{exgroup}
        \item 已知$V_1$是线性方程组\[\begin{cases}
                3x_1+4x_2-5x_3+7x_4=0 \\
                4x_1+11x_2-13x_3+16x_4=0
            \end{cases}\]
        的解空间,$V_2$是线性方程组\[\begin{cases}
                2x_1-3x_2+3x_3-2x_4=0 \\
                7x_1-2x_2+x_3+3x_4=0
            \end{cases}\]
        的解空间,分别求$V_1 \cap V_2$与$V_1+V_2$的基和维数.

        \item 设$W_1,W_2$是线性空间$V(\mathbf{F})$的两个子空间. 证明以下命题等价:
        \begin{enumerate}
            \item $W_1 \cup W_2$为$V$ 的子空间;

            \item $W_1 \subseteq W_2$或$W_2 \subseteq W_1$;

            \item $W_1 \cup W_2=W_1+W_2$.
        \end{enumerate}

        \item 设\[W_1=\left\{\begin{pmatrix}
                x & -x \\ y & z
            \end{pmatrix} \;\middle|\; x,y,z\in \mathbf{F} \right\},W_2=\left\{\begin{pmatrix}
                a & b \\ -a & c
            \end{pmatrix} \;\middle|\; a,b,c\in \mathbf{F} \right\}.\]

        \begin{enumerate}
            \item 证明:$W_1,W_2$是$\mathbf{M}_2(\mathbf{F})$的子空间,并求$\dim W_1,\dim W_2,\dim(W_1+W_2),\dim(W_1\cap W_2)$;

            \item 求$W_1\cap W_2$的一组基,并求$A=\begin{pmatrix}
                          3 & -3 \\ -3 & 1
                      \end{pmatrix}$关于这组基的坐标.
        \end{enumerate}

        \item 设$V$是域$\mathbf{F}$上的$n$维线性空间,$\alpha_1,\alpha_2,\ldots,\alpha_n$是$V$的一组基,且
        \begin{gather*}
            V_1=\spa(\alpha_1+2\alpha_2+\cdots+n\alpha_n) \\
            V_2=\left\{k_1\alpha_1+k_2\alpha_2+\cdots+k_n\alpha_n \;\middle|\; k_1+\dfrac{k_2}{2}+\cdots+\dfrac{k_n}{n}=0\right\}
        \end{gather*}
        证明:
        \begin{enumerate}
            \item $V_2$是$V$的子空间;

            \item $V=V_1\oplus V_2$.
        \end{enumerate}

        \item 设$\mathbf{F}$为数域,$V_1=\{A\in\mathbf{F}^{n\times n} \mid A^\mathrm{T}=A\},\enspace
            V_2=\{A\in\mathbf{F}^{n\times n} \mid A^\mathrm{T}=-A\},\enspace V_3=\{A\in\mathbf{F}^{n\times n} \mid A\text{~为上三角矩阵}\}$.
        \begin{enumerate}
            \item 证明:$V_1,V_2,V_3$都是$\mathbf{F}^{n\times n}$的子空间;

            \item 证明:$\mathbf{F}^{n\times n}=V_1+V_3$但不为直和,$\mathbf{F}^{n\times n}=V_2\oplus V_3$.
        \end{enumerate}

        \item 已知$V_1,V_2$是有限维线性空间$V$的子空间,且$\dim(V_1+V_2)=\dim(V_1 \cap V_2)+1$. 证明:要么$V_1 \subseteq V_2$,要么$V_2 \subseteq V_1$.

        \item 证明:和$\sum\limits_{i=1}^{s}V_i$为直和的充要条件是$V_i \cap \sum\limits_{j=1}^{i-1}V_j=\{0\},\enspace i=1,2,\ldots,s$.

        \item 判断下列说法是否正确:
        \begin{enumerate}
            \item 若$V \subseteq V_1 \cup V_2 \cup \cdots \cup V_s$,则$V=(V_1 \cap V)\cup(V_2 \cap V)\cup\cdots\cup(V_s \cap V)$;

            \item 若$V \subseteq V_1+V_2+\cdots +V_s$,则$V=(V_1 \cap V)+(V_2 \cap V)+\cdots+(V_s \cap V)$.
        \end{enumerate}

        \item 设$V$为有限维线性空间,$V_1$为其非零子空间. 证明:存在唯一的子空间$V_2$,使得$V=V_1\oplus V_2$的充要条件为$V_1=V$.

       \item 设$A_1$和$A_2$均为$V$的仿射子集,证明:$A_1\cap A_2$是$V$的仿射子集或空集(可推广至任意交).
    \end{exgroup}

    \begin{exgroup}
        \item 设$V$是域$\mathbf{F}$上的$n$阶对称矩阵关于矩阵加法和数乘运算构成的线性空间,令
        \[U=\{A\in V \mid \tr(A)=0\},\enspace W=\{\lambda E \mid \lambda\in\mathbf{F}\}.\]
        \begin{enumerate}
            \item 证明:$U,W$为$V$的子空间;

            \item 分别求$U,W$的一组基和维数;

            \item 证明:$V=U\oplus W$.
        \end{enumerate}

        \item 设$W_0,W_1,W_2,\ldots,W_s$是线性空间$V$的$s+1$个非平凡子空间,且$W_0 \subseteq W_1 \cup W_2 \cup \cdots \cup W_s$. 证明:必存在$i$使得$W_0\subseteq W_i$.

        \item 设$v_1,\ldots,v_m\in V$. 令
        \[A=\{\lambda_1v_1+\cdots+\lambda_mv_m \mid \lambda_1,\ldots,\lambda_m\in\mathbf{F}\text{~且~}\lambda_1+\cdots+\lambda_m=1\}.\]
        证明:
        \begin{enumerate}
            \item $A$是$V$的仿射子集;

            \item $V$的每个包含$v_1,\ldots,v_m$的仿射子集均包含$A$;

            \item 存在某个$v\in V$和$V$的子空间$U$使得$A=v+U$且$\dim U\leqslant m-1$.
        \end{enumerate}

        \item (加强的覆盖定理) 设 $A_1, A_2, \ldots, A_n$ 是域 $F$ 上向量空间 $V$ 的仿射子集,证明 $A_1\cup A_2\cup \cdots \cup A_n$ 不能覆盖 $V$,即存在 $\alpha \in V$ 但是 $\forall 1\leqslant i\leqslant n, \alpha \notin A_n$
    \end{exgroup}
\end{exercise}

\phantomsection
\section*{5 线性映射}
\addcontentsline{toc}{section}{5 线性映射}

\vspace{2ex}

\centerline{\heiti A组}
\begin{enumerate}
    \item
\end{enumerate}

\centerline{\heiti B组}
\begin{enumerate}
    \item
\end{enumerate}

\centerline{\heiti C组}
\begin{enumerate}
    \item
\end{enumerate}

\clearpage

\phantomsection
\section*{6 线性映射基本定理}
\addcontentsline{toc}{section}{6 线性映射基本定理}

\vspace{2ex}

\centerline{\heiti A组}
\begin{enumerate}
    \item
\end{enumerate}

\centerline{\heiti B组}
\begin{enumerate}
    \item \begin{enumerate}
              \item 证明:$ \forall p_1(x), p_2(x) \in \mathbf{R}[x]_n,\enspace \forall k_1, k_2 \in \mathbf{R} $,有
                    \begin{align*}
                        \sigma(k_1 p_1 + k_2 p_2) & = (k_1 p_1(x) + k_2 p_2(x))'                                    \\
                                                  & = k_1 p_1'(x) + k_2 p_2'(x) = k_1 \sigma(p_1) + k_2 \sigma(p_2)
                    \end{align*}
                    因此 $ \sigma $ 是 $ \mathbf{R}[x]_n $ 上的线性变换.

              \item \begin{gather*}
                        \begin{aligned}
                            \im \sigma & = \spa(\sigma(1), \sigma(x), \sigma(x^2), \ldots, \sigma(x^{n - 1})) \\
                                       & = \spa(1, 2x, \ldots, (n - 1) x^{n - 2})                             \\
                                       & = \spa(1, x, \ldots, x^{n - 2})
                        \end{aligned} \\
                        r(\sigma) = n - 1
                    \end{gather*}
                    可知 $ \sigma $ 不是单射,因此不可逆.

              \item 由 $ \sigma(p(x)) = 0 $ 可知 $ p(x) = c $(常数). 因此 $ \ker \sigma = \spa(1),\enspace \dim \ker \sigma = 1 $.

              \item \begin{gather*}
                        r(\sigma) + \dim \ker \sigma = (n - 1) + 1 = n \\
                        \im \sigma + \ker \sigma = \spa(1, x, \ldots, x^{n - 2}) = \im \sigma \neq \mathbf{R}[x]_n
                    \end{gather*}
          \end{enumerate}

    \item \begin{enumerate}
              \item 可能. 例如 $ \sigma(x, y) = (x + y, x + y) $.
                    \begin{gather*}
                        \im \sigma = \spa(\vec{e}_1 + \vec{e}_2) \\
                        \ker \sigma = \spa(\vec{e}_1 - \vec{e}_2) \\
                        \im \sigma \cap \ker \sigma = \{\vec{0}\}
                    \end{gather*}

              \item 可能. 例如 $ \sigma(x, y, z) = (x - y, x - y, x - y) $.
                    \begin{gather*}
                        \im \sigma = \spa((1, 1, 1)) \\
                        \ker \sigma = \spa((1, 1, 1), (1, 1, 0)) \\
                        \im \sigma \subseteq \ker \sigma
                    \end{gather*}

              \item 可能. 例如 $ \sigma(x, y) = (x - y, x - y) $.
                    \begin{gather*}
                        \im \sigma = \ker \sigma = \spa((1, 1))
                    \end{gather*}

              \item 可能. 例如 $ \sigma(x, y) = (x, x - y) $.
                    \begin{gather*}
                        \im \sigma = \mathbf{R}^2 \\
                        \ker \sigma = \{\vec{0}\} \\
                        \ker \sigma \subseteq \im \sigma
                    \end{gather*}
          \end{enumerate}

    \item \begin{enumerate}
              \item 错误. 一个反例为 $ \sigma(x_1, x_2) = (x_1 - x_2, x_1 - x_2) $,则 $ \im \sigma + \ker \sigma = \spa((1, 1)) $.

              \item 正确. $ \im \sigma + \ker \sigma \subseteq V $,此时 $ \dim(\im \sigma + \ker \sigma) = \dim \im \sigma + \dim \ker \sigma - 0 = \dim V $,所以 $ \im \sigma + \ker \sigma = V $.

              \item \label{item:6:B:2:3}
                    错误. $ \forall \alpha \in V $ 有
                    \[ (\sigma_1 + \sigma_2)(\alpha) = \sigma_1(\alpha) + \sigma_2(\alpha) \in \sigma_1(V) + \sigma_2(V) \]
                    所以
                    \[ (\sigma_1 + \sigma_2)(V) \subseteq \sigma_1(V) + \sigma_2(V) \]
                    但上式中等号不一定成立. 反例:$ V = \mathbf{R}^3 $ 上的线性变换 $ \sigma_1, \sigma_2 $ 关于 $ \mathbf{R}^3 $ 的基 $ \vec{e}_1, \vec{e}_2, \vec{e}_3 $ 的像分别为
                    \begin{gather*}
                        \sigma_1(\vec{e}_1) = \vec{e}_1,\enspace \sigma_1(\vec{e}_2) = \sigma_1(\vec{e}_3) = \vec{e}_2 \\
                        \sigma_2(\vec{e}_1) = \vec{e}_1,\enspace \sigma_2(\vec{e}_2) = \sigma_2(\vec{e}_3) = \vec{e}_3 \\
                        \sigma_1(V) = \spa(\vec{e}_1, \vec{e}_2),\enspace \sigma_2(V) = \spa(\vec{e}_1, \vec{e}_3) \\
                        \sigma_1(V) + \sigma_2(V) = \spa(\vec{e}_1, \vec{e}_2, \vec{e}_3) = \mathbf{R}^3
                    \end{gather*}
                    而
                    \begin{gather*}
                        (\sigma_1 + \sigma_2)(\vec{e}_1) = \vec{e}_1 + \vec{e}_1 = 2 \vec{e}_1 \\
                        (\sigma_1 + \sigma_2)(\vec{e}_2) = (\sigma_1 + \sigma_2)(\vec{e}_3) = \vec{e}_2 + \vec{e}_3 \\
                        (\sigma_1 + \sigma_2)(V) = \spa(\vec{e}_1, \vec{e}_2 + \vec{e}_3) \neq \mathbf{R}^3
                    \end{gather*}

              \item 错误. $ (I - \sigma)(V) + \sigma(V) \subseteq V $,等号不一定成立,原因同 \ref*{item:6:B:2:3},此时只需将 $ I - \sigma $ 视作 $ \sigma_1 $,将 $ \sigma $ 视作 $ \sigma_2 $.
          \end{enumerate}

    \item 证明:\begin{enumerate}
              \item 由于 $ \ker \sigma = \im \sigma $,由 $ \dim \im \sigma + \dim \ker \sigma = \dim V $ 可得.

              \item 设 $ \beta_1, \ldots, \beta_n $ 为 $ V $ 的一组基,则
                    \[ \im \sigma = \spa(\sigma(\beta_1), \ldots, \sigma(\beta_n)) = \ker \sigma \]
                    设 $ \sigma(\beta_1), \ldots, \sigma(\beta_{\frac{n}{2}}) $ 为 $ \im \sigma $ 的基,则可以证明
                    \[ \sigma(\beta_1), \ldots, \sigma(\beta_{\frac{n}{2}}), \beta_1, \ldots, \beta_{\frac{n}{2}} \]
                    线性无关,且 $ \sigma $ 在此基下的矩阵即为所求的形式.
          \end{enumerate}

    \item \begin{enumerate}
              \item 在. 因为
                    \[ \sigma(\alpha_1) = -2 \sigma(\sigma_2) + \sigma(\alpha_3) = \sigma(-2 \alpha_2 + \alpha_3) \]
                    同构映射 $ \sigma $ 可逆. 所以
                    \[ \alpha_1 = -2 \alpha_2 + \alpha_3 \in \spa(\alpha_2, \alpha_3) \]
          \end{enumerate}

    \item 我们仅对 $ n = 3 $ 的情况给出证明. % TODO P117/7

          先证 $ \sigma $ 是线性映射. $ \forall p(x), q(x) \in \mathbf{R}[x]_3,\enspace \forall k_1, k_2 \in \mathbf{R} $ 有
          \begin{align*}
                  & \sigma(k_1 p(x) + k_2 q(x))                                                 \\
              ={} & (k_1 p(c_1) + k_2 q(c_1), k_1 p(c_2) + k_2 q(c_2), k_1 p(c_3) + k_2 q(c_3)) \\
              ={} & k_1(p(c_1), p(c_2), p(c_3)) + k_1(q(c_1), q(c_2), q(c_3))                   \\
              ={} & k_1 \sigma(p(x)) + k_2 \sigma(q(x))
          \end{align*}

          再证 $ \sigma $ 是双射,即 $ \forall (d_1, d_2, d_3) \in \mathbf{R}^3 $,存在唯一的
          \[ p(x) = a + bx + cx^2 \in \mathbf{R}[x]_3 \]
          使 $ \sigma(p(x)) = (d_1, d_2, d_3) $. 根据
          \[ \sigma(p(x)) = (p(c_1), p(c_2), p(c_3)) \]
          以及 $ \sigma(p(x)) = (d_1, d_2, d_3) $,有
          \[ \begin{cases}
                  a + bc_1 + cc_1^2 = d_1 \\
                  a + bc_2 + cc_2^2 = d_2 \\
                  a + bc_3 + cc_3^3 = d_3
              \end{cases} \]
          方程组是关于未知元 $ a, b, c $ 的三元线性非齐次方程组,其中 $ c_1, c_2, c_3 $ 是互异的实常数. 用高斯消元法,易将其增广矩阵变换为下列阶梯形矩阵,即
          \begin{align} % TODO 增广矩阵
                             & \begin{pmatrix}
                                   1 & c_1 & c_1^2 & \Bigm| & d_1 \\
                                   1 & c_2 & c_2^2 & \Bigm| & d_2 \\
                                   1 & c_3 & c_3^3 & \Bigm| & d_3
                               \end{pmatrix} \notag                                                                    \\
              \implies \quad & \begin{pmatrix}
                                   1 & c_1 & c_1^2     & \Bigm| & d_1                                                         \\[1ex]
                                   0 & 1   & c_2 + c_1 & \Bigm| & \dfrac{d_1 - d_2}{c_1 - c_2}                                \\[2ex]
                                   0 & 0   & c_3 - c_2 & \Bigm| & \dfrac{d_3 - d_1}{c_3 - c_1} - \dfrac{d_2 - d_1}{c_2 - c_1}
                               \end{pmatrix} \label{item:6:B:6:1}
          \end{align}
          阶梯性矩阵 \ref*{item:6:B:6:1}(其中 $ c_3 - c_2, c_3 - c_1, c_2 - c_1 $ 均为非零常数)对应的方程组有唯一解 $ a, b, c $,即存在唯一的
          \[ p(x) = a + bx + cx^2 \in \mathbf{R}[x]_3 \]
          使得 $ \sigma(p(x)) = (d_1, d_2, d_3) $ 成立. 所以 $ \sigma $ 是线性双射,即 $ \mathbf{R}[x]_3 $ 到 $ \mathbf{R}^3 $ 的同构映射.
\end{enumerate}

\centerline{\heiti C组}
\begin{enumerate}
    \item 证明:必要性:$ \forall \alpha \in V $,由 $ \sigma $ 可逆,存在唯一的 $ \beta \in V $ 使得 $ \sigma(\beta) = \alpha $ 且 $ \beta = \beta_1 + \beta_2 $,其中 $ \beta_1 \in W_1 $,$ \beta_2 \in W_2 $. 于是
          \[ \alpha = \sigma(\beta) = \sigma(\beta_1) + \sigma(\beta_2) \in \sigma(W_1) + \sigma(W_2) \]
          所以 $ V \subseteq \sigma(W_1) + \sigma(W_2) $.

          $ \sigma(W_1), \sigma(W_2) $ 都是 $ V $ 的子空间,它们的和也是 $ V $ 的子空间. 所以 $ \sigma(W_1) + \sigma(W_2) \subseteq V $,故 $ V = \sigma(W_1) + \sigma(W_2) $.

          充分性:$ \forall \alpha \in V = \sigma(W_1) + \sigma(W_2) $,$ \exists \alpha_i \in \sigma(W_i) $ 且 $ \exists \beta_i \in W_i $ 使 $ \alpha_i = \sigma(\beta_i) $($ i = 1, 2 $)使得
          \[ \alpha = \alpha_1 + \alpha_2 = \sigma(\beta_1) + \sigma(\beta_2) = \sigma(\beta_1 + \beta_2) = \sigma(\beta) \]
          其中 $ \beta = \beta_1 + \beta_2 \in W_1 + W_2 = V $,所以 $ \sigma $ 为满射.

          由于 $ n $ 维线性空间上的线性映射为满射时也必为单射,从而必是双射,所以 $ \sigma $ 可逆.

    \item 证明:先证右边. 由于 $ \sigma(V_1) \subseteq V_2 $,所以 $ (\tau \sigma)(V_1) \subseteq \tau(V_2) $,如下图所示. 因此
          \[ \dim(\tau \sigma)(V_1) \leqslant \dim \tau(V_2) \]
          即 $ r(\tau \sigma) \leqslant r(\tau) $.
          \begin{figure}[H]
              \centering
              \includegraphics[scale=.5]{figs/6C.2.1.jpg}
          \end{figure}
          又因为 $ (\tau \sigma)(V_1) = \tau(\sigma(V_1)) $,所以又有
          \[ \dim(\tau \sigma)(V_1) \leqslant \dim \sigma(V_1) \]
          即 $ r(\tau \sigma) \leqslant r(\sigma) $.

          再证左边. 由线性映射维数公式,
          \begin{gather*}
              r(\tau) + \dim \ker \tau = n \\
              r(\tau \sigma) + \dim \ker(\tau \sigma) = m
          \end{gather*}
          又 $ \dim \ker(\tau \sigma) \leqslant \dim \ker \tau $,所以
          \[ m - r(\tau \sigma) = \dim \ker(\tau \sigma) \leqslant \dim \ker \tau \]
          代入线性映射维数公式,得 $ \dim \ker \tau = n - r(\tau) \geqslant m - r(\sigma) $,即
          \begin{align*}
              r(\tau \sigma) & \geqslant m + r(\tau) - n         \\
                             & \geqslant r(\sigma) + r(\tau) - n
          \end{align*}

    \item 证明:由于 $ \forall \beta \in (\sigma + \tau)(V_1),\enspace \exists \alpha \in V_1 $ 使 $ \beta = (\sigma + \tau)(\alpha) = \sigma(\alpha) + \tau(\alpha) \in \sigma(V_1) + \tau(V_1) $,所以
          \[ (\sigma + \tau)(V_1) \subseteq \sigma(V_1) + \tau(V_1) \]
          因此
          \begin{align*}
              \dim(\sigma + \tau)(V_1) & \leqslant \dim(\sigma(V_1) + \tau(V_1))     \\
                                       & \leqslant \dim \sigma(V_1) + \dim \tau(V_1)
          \end{align*}

    \item 证明:\begin{enumerate}
              \item \label{item:6:C:4:1}
                    $ \forall \sigma \in \mathcal{L}(V, V) $,则 $ I - \sigma \in \mathcal{L}(V, V) $. $ \forall \alpha \in (I - \sigma)(V),\enspace \exists \beta \in V $,有
                    \begin{gather*}
                        \alpha = (I - \sigma)(\beta) = \beta - \sigma(\beta) \\
                        \sigma(\alpha) = \sigma(\beta - \sigma(\beta)) = \sigma(\beta) - \sigma^2(\beta)
                    \end{gather*}
                    而由于 $ \sigma^2 = \sigma $,所以 $ \sigma(\alpha) = \vec{0} $,于是 $ \alpha \in \ker \sigma $,因此 $ (I - \sigma)(V) \subseteq \ker \sigma $.

              \item 利用 $ r(\sigma + \tau) \leqslant r(\sigma) + r(\tau) $ 和 $ r(\sigma) + \dim \ker \sigma = n $,由 \ref*{item:6:C:4:1} 可得
                    \begin{equation} \label{eq:6:C:4:2:1}
                        r(I - \sigma) + r(\sigma) \leqslant n
                    \end{equation}
                    又因为
                    \begin{equation} \label{eq:6:C:4:2:2}
                        r(I - \sigma) + r(\sigma) \geqslant r(I - \sigma + \sigma) = r(I) = n
                    \end{equation}
                    于是由\autoref{eq:6:C:4:2:1} 和\autoref{eq:6:C:4:2:2} 即可得到 $ r(I - \sigma) + r(\sigma) = n $.
          \end{enumerate}

    \item 证明:\begin{enumerate}
              \item \label{item:6:C:5:1}
                    $ \forall \alpha \in \im \sigma,\enspace \exists \beta \in V $ 使得 $ \sigma(\beta) = \alpha $. 由 $ \sigma^2 = \theta $ 可得 $ \sigma(\alpha) = \sigma^2(\beta) = \vec{0} $,因此 $ \alpha \in \ker \sigma $,从而 $ \im \sigma \subseteq \ker \sigma $. 于是我们得到
                    \[ n = \dim \im \sigma + \dim \ker \sigma \geqslant 2 \dim \im \sigma \]
                    即 $ \dim \im \sigma \leqslant \dfrac{n}{2} $.

              \item 由 \ref*{item:6:C:5:1} 可知,方程组 $ AX = \vec{0} $ 的基础解系含有 $ n - r(A) = \dim \ker \sigma \geqslant \dfrac{n}{2} $ 个解向量,所以结论成立.
          \end{enumerate}

    \item 假设 $ \alpha_1, \alpha_2, \ldots, \alpha_n \in \mathbf{F} $ 是 $ \mathbf{F} $ 作为数域 $ \mathbf{K} $ 上的线性空间的一组基,$ \beta_1, \ldots, \beta_m \in \mathbf{E} $ 是 $ \mathbf{E} $ 作为数域 $ \mathbf{F} $ 上的线性空间的一组基,则现在对于任意的 $ \beta \in \mathbf{E} $,都存在 $ k_1, \ldots, k_m \in \mathbf{F} $ 使得 $ \beta = \displaystyle\sum_{i = 1}^{m} k_i \beta_i $.

          而对于每一个 $ k_i \in \mathbf{F},\enspace i = 1, 2, \ldots, m $,存在 $ l_{ij} \in \mathbf{K},\enspace j = 1, 2, \ldots, n $ 使得 $ k_i = \displaystyle\sum_{j = 1}^{n} l_{ij} a_j $. 于是
          \[ \beta = \sum_{i = 1}^{m} \sum_{j = 1}^{n} l_{ij} \alpha_j \beta_i \]
          这说明对任意的 $ \beta \in \mathbf{E} $,都可由
          \[ \alpha_1 \beta_1, \ldots, \alpha_1 \beta_m, \alpha_2 \beta_1, \ldots, \alpha_n \beta_m \]
          这 $ mn $ 个向量线性表出,其中线性表出的系数都属于最小的数域 $ \mathbf{K} $.

          同时,如果假设 $ l_{ij} \in \mathbf{K} $ 满足 $ \displaystyle\sum_{i = 1}^{m} \displaystyle\sum_{j = 1}^{n} l_{ij} \alpha_j \beta_i = 0 $,则
          \[ \sum_{i = 1}^{m} \left(\sum_{j = 1}^{n} l_{ij} \alpha_j\right) \beta_i = 0 \]
          利用 $ \beta_1, \ldots, \beta_m $ 的线性无关性可得 $ \displaystyle\sum_{j = 1}^{n} l_{ij} \alpha_j = 0 $,再结合 $ \alpha_1, \ldots, \alpha_n $ 线性无关可得 $ l_{ij} = 0 $,这就说明 $ \alpha_1 \beta_1, \ldots, \alpha_1 \beta_m, \alpha_2 \beta_1, \ldots, \alpha_n \beta_m $ 在数域 $ \mathbf{K} $ 是线性无关的.

          综上,$ \alpha_1 \beta_1, \ldots, \alpha_1 \beta_m, \alpha_2 \beta_1, \ldots, \alpha_n \beta_m $ 就是 $ \mathbf{E} $ 作为 $ \mathbf{K} $ 的线性空间的一组基,从而这个线性空间是 $ mn $ 维的.
\end{enumerate}

\clearpage

\section*{7 线性映射矩阵表示(I)}
\addcontentsline{toc}{section}{7 线性映射矩阵表示(I)}

\vspace{2ex}

\centerline{\heiti A组}
\begin{enumerate}
    \item 教材 P154/9,此处略.
    \item 教材 P154/10,此处略.
\end{enumerate}

\centerline{\heiti B组}
\begin{enumerate}
    \item $T(\beta_1,\beta_2,\cdots,\beta_n)=(\beta_1,\beta_2,\cdots,\beta_n)\begin{pmatrix}0 & 0 & 0 & \cdots & 0 & a_1 \\ 1 & 0 & 0 & \cdots & 0 & a_2 \\ 0 & 1 & 0 & \cdots & 0 & a_3 \\ \vdots & \vdots & \vdots & & \vdots & \vdots \\ 0 & 0 & 0 & \cdots & 1 & a_n\end{pmatrix}$
    所以 $A=\begin{pmatrix}0 & 0 & 0 & \cdots & 0 & a_1 \\ 1 & 0 & 0 & \cdots & 0 & a_2 \\ 0 & 1 & 0 & \cdots & 0 & a_3 \\ \vdots & \vdots & \vdots & & \vdots & \vdots \\ 0 & 0 & 0 & \cdots & 1 & a_n\end{pmatrix}$ 是 $T$ 关于基 $B$ 的表示矩阵.

    $T$ 是同构 $\Leftrightarrow$ $T$ 是双射 $\Leftrightarrow$ $r(T) = n$,所以 $A$ 满秩即 $a_1\neq 0$ 时 $T$ 是同构映射.
    \item \begin{enumerate}
        \item 略.
        \item 设 $(\sigma(f_1),\sigma(f_2),\sigma(f_3))=(f_1,f_2,f_3)A$,可用待定系数法解得 $A=\begin{pmatrix}-1 & -2 & -2 \\ 3 & 2 & 3 \\ -1 & -1 & -1\end{pmatrix}$.
        \item 用待定系数求出 $f=-2f_1+3f_2$,故 $\sigma(f)=-2\sigma(f_1)+3\sigma(f_2)=-4+3x+2x^2$.
    \end{enumerate}
    \item \begin{enumerate}
        \item 略.
        \item $\lambda \neq -1$ 时,$A,B$ 均可逆,故 $\varphi^{-1}(X)=A^{-1}XB^{-1}$.
        \item 取 $V$ 的一组基 $\begin{pmatrix}1 & 0 \\ 0 & 0\end{pmatrix},\begin{pmatrix}0 & 1 \\ 0 & 0\end{pmatrix},\begin{pmatrix}0 & 0 \\ 1 & 0\end{pmatrix},\begin{pmatrix}0 & 0 \\ 0 & 1\end{pmatrix}$,有
        \[\mathrm{Im}\sigma = \mathrm{span}(\varphi(\alpha_1),\varphi(\alpha_2),\varphi(\alpha_3),\varphi(\alpha_4))=\mathrm{span}(\begin{pmatrix}1 & 2 \\ -1 & -2\end{pmatrix},\begin{pmatrix}-1 & -1 \\ 1 & 1\end{pmatrix})\]
        \[\mathrm{Ker}\sigma = \mathrm{span}(\begin{pmatrix}2 & -3 \\ 0 & 1\end{pmatrix},\begin{pmatrix}1 & 0 \\ 1 & 0\end{pmatrix})\]
        (省略步骤,答案不唯一)
        \item 取 $\begin{pmatrix}1 & 2 \\ -1 & -2\end{pmatrix},\begin{pmatrix}-1 & -1 \\ 1 & 1\end{pmatrix},\begin{pmatrix}0 & 0 \\ 1 & 0\end{pmatrix},\begin{pmatrix}0 & 0 \\ 0 & 1\end{pmatrix}$ 即可.
        
        此时矩阵为 $\begin{pmatrix}2 & -2 & -1 & 0 \\ 4 & -2 & 0 & -1 \\ 0 & 0 & 0 & 0 \\ 0 & 0 & 0 & 0\end{pmatrix}$(答案不唯一).
    \end{enumerate}
    \item 求基的过程与求 $V=\{X\in \mathbf{R}^4\ |\ x_1+x_2+x_3=0\}$ 类似,求 $V$ 的基只需求解方程组 $x_1+x_2+x_3=0$ 即可,得到基础解系 $\begin{pmatrix}-1 \\ 0 \\ 1 \\ 0\end{pmatrix},\begin{pmatrix}-1 \\ 1 \\ 0 \\ 0\end{pmatrix},\begin{pmatrix}0 \\ 0 \\ 0 \\ 1\end{pmatrix}$.
    
    换回本题,有基为 $A_1=\begin{pmatrix}-1 & 0 \\ 1 & 0\end{pmatrix},A_2=\begin{pmatrix}-1 & 1 \\ 0 & 0\end{pmatrix},A_3=\begin{pmatrix}0 & 0 \\ 0 & 1\end{pmatrix}$.
    而 $\sigma(A_1)=A_1+A_2,\sigma(A_2)=A_1+A_2,\sigma(A_3)=2A_3$,可得 $\sigma(A_1+A_2)=2(A_1+A_2),\sigma(A_1-A_2)=0,\sigma(A_3)=2A_3$.

    所以取基 $\{A_1-A_2,A_1+A_2,A_3\}$ 有
    \[(\sigma(A_1-A_2),\sigma(A_1+A_2),\sigma(A_3))=(A_1-A_2,A_1+A_2,A_3)\begin{pmatrix}0 & 0 & 0 \\ 0 & 2 & 0 \\ 0 & 0 & 2\end{pmatrix}\]
    为对角矩阵.
    \item \begin{enumerate}
        \item 求核空间:设 $f(x)=ax^3+bx^2+cx+d$,由 $f(-1)=f(0)=f(1)=0$ 解得 $f(x)=a(x^3-x)$. 故 $N(T)=\mathrm{span}(x^3-x)$.
        
        求像空间:取常用基 $\{1,x,x^2,x^3\}$,我们要求 $T(1),T(x),T(x^2),T(x^3)$ 的极大线性无关组,我们发现 $T(x)=T(x^3)$,故先舍弃 $T(x^3)$,然后令 $k_1T(1)+k_2T(x)+k_3T(x^2)=0$,可解得 $k_1=k_2=k_3=0$,故 $T(1),T(x),T(x^2)$ 线性无关,故 $R(T)=\mathrm{span}(\begin{pmatrix}1 & 1 \\ 1 & 1\end{pmatrix},\begin{pmatrix}0 & 1 \\ -1 & 0\end{pmatrix},\begin{pmatrix}0 & 1 \\ 1 & 0\end{pmatrix})$.
        \item 由上一问有 $\mathrm{dim}N(T)=1,\mathrm{dim}R(T)=3$,又有 $\mathrm{dim}\mathbf{R}[x]_4=4$. 则维数公式成立.
    \end{enumerate}
    \item
        \begin{enumerate}
            \item
                对非齐次线性方程组$AX=\xi_1$,\\
                $\bar{A}=\begin{pmatrix}
                    1 & -1 & -1 & -1 \\
                    -1 & 1 & 1 & 1 \\
                    0 & -4 & -2 & -2
                \end{pmatrix}
                \rightarrow\begin{pmatrix}
                    1 & -1 & -1 & -1 \\
                    0 & 1 & \frac 12 & \frac 12 \\
                    0 & 0 & 0 & 0
                \end{pmatrix}
                \rightarrow\begin{pmatrix}
                    1 & 0 & -\frac 12 & -\frac 12 \\
                    0 & 1 & \frac 12 & \frac 12 \\
                    0 & 0 & 0 & 0
                \end{pmatrix}$,则\\
                $\xi_2=C_1\begin{pmatrix}
                    \frac 12 \\
                    -\frac 12 \\
                    1
                \end{pmatrix} + \begin{pmatrix}
                    -\frac 12 \\
                    \frac 12 \\
                    0
                \end{pmatrix}=\frac 12\begin{pmatrix}
                    C_1 - 1 \\
                    -C_1 + 1 \\
                    2C_1
                \end{pmatrix}$(其中$C_1$为任意常数).\\
                $A^2=\begin{pmatrix}
                    2 & 2 & 0 \\
                    -2 & -2 & 0 \\
                    4 & 4 & 0
                \end{pmatrix}$,对齐次线性方程组$A^2X=\xi_1$,\\
                $\bar{B}=\begin{pmatrix}
                    A^2 & \xi_1
                \end{pmatrix}=\begin{pmatrix}
                    2 & 2 & 0 & 1 \\
                    -2 & -2 & 0 & 1 \\
                    4 & 4 & 0 & -2
                \end{pmatrix}
                \rightarrow\begin{pmatrix}
                    1 & 1 & 0 & -\frac 12 \\
                    0 & 0 & 0 & 0 \\
                    0 & 0 & 0 & 0
                \end{pmatrix}$,\\
                则$A^2X=\xi_1$的通解\\
                $\xi_3=C_2\begin{pmatrix}
                    -1 \\
                    1 \\
                    0
                \end{pmatrix}+C_3\begin{pmatrix}
                    0 \\
                    0 \\
                    1
                \end{pmatrix}+\begin{pmatrix}
                    -\frac 12 \\
                    0 \\
                    0
                \end{pmatrix}=\begin{pmatrix}
                    -C_2 - \frac 12 \\
                    C_2 \\
                    C_3
                \end{pmatrix}$(其中$C_2, C_3$为任意常数).
            \item
                因为$|\xi_1,\xi_2,\xi_3|=\frac 12\begin{vmatrix}
                    -1 & C_1 - 1 & -C_2 - \frac 12 \\
                    1 & -C_1 + 1 & C_2 \\
                    -2 & 2C_1 & C_3
                \end{vmatrix}=-\frac 12\neq 0$,\\
                所以$\xi_1,\xi_2,\xi_3$线性无关.
        \end{enumerate}
    \item \begin{enumerate}
        \item 对新的一组基,使用过渡矩阵进行表达如下:
        \[(\beta_{1}, \beta_{2}, \beta_{3})=(\alpha_{1}, \alpha_{2}, \alpha_{3})\begin{pmatrix}
        2 & 1 & -1 \\
        1 & 1 & 1 \\
        3 & 2 & 1    
        \end{pmatrix}=(\alpha_{1}, \alpha_{2}, \alpha_{3}) C\]
        其中 $C$ 是可逆矩阵,且
        \[(\alpha_{1},\ \alpha_{2},\ \alpha_{3})=(\beta_{1},\ \beta_{2},\ \beta_{3}) C^{-1}\]
        将上式代入已知条件得
        \[\sigma\left(\left(\beta_{1},\ \beta_{2},\ \beta_{3}\right) C^{-1}\right)=\left(\left(\beta_{1},\ \beta_{2},\ \beta_{3}\right) C^{-1}\right) A\]
        容易验证(只需利用线性变换和矩阵的等价性然后利用矩阵乘法结合律即可)上式左端等于 $(\sigma(\beta_{1}, \beta_{2}, \beta_{3})) C^{-1}$,所以
        \[(\sigma(\beta_{1},\ \beta_{2},\ \beta_{3})) C^{-1}=(\beta_{1},\ \beta_{2},\ \beta_{3})(C^{-1} A)\]
        从而得 $\sigma(\beta_{1},\ \beta_{2},\ \beta_{3})=(\beta_{1},\ \beta_{2},\ \beta_{3})(C^{-1} A C)$,故 $\sigma$ 关于基 $\{\beta_{1},\ \beta_{2},\ \beta_{3}\}$ 下对应的矩阵 $B=C^{-1} A C=\begin{pmatrix}2 & 0 & 1 \\ 0 & 2 & 1 \\ 3 & 1 & -1\end{pmatrix}$.
        \item $\sigma$ 的值域是 $A$ 列向量组的极大线性无关组,由于 $A $ 的第 $1$ 列可以由第 $2$ 列和第 $3$ 列线性表示,从而 $\sigma(V)=L(2 \alpha_{1}+\alpha_{2},\ -\alpha_{1}+\alpha_{3})$.$\operatorname{Ker} \sigma$ 是线性方程组 $AX=0$ 的解空间,从而 $\mathrm{Ker} \sigma=\mathrm{span}(\alpha_{1}-2 \alpha_{2}-3 \alpha_{3})$.
        \item 由于 $\alpha_{1}$ 不能由 $2 \alpha_{1}+\alpha_{2}$ 和 $-\alpha_{1}+\alpha_{3}$ 线性表示,可以把 $\sigma(V)$ 的基扩充为 $V$ 的基 $\{\alpha_{1},\ 2 \alpha_{1}+\alpha_{2},\ -\alpha_{1}+\alpha_{3}\}$,$\sigma$ 在这个基下对应的矩阵是 $\begin{pmatrix}0 & 0 & 0 \\ 2 & 5 & -2 \\ 3 & 6 & -2\end{pmatrix}$.
        \item 由于 $\alpha_{1},\ \alpha_{2}$ 不能由 $\alpha_{1}-2 \alpha_{2}-3 \alpha_{3}$ 线性表示,可以把 $\mathrm{Ker} \sigma$ 的基扩充为 $V$ 的基 $\{\alpha_{1},\ \alpha_{2},\ \alpha_{1}-2 \alpha_{2}-3 \alpha_{3}\}$,$\sigma$ 在这个基下对应的矩阵是 $\begin{pmatrix}2 & 2 & 0 \\ 0 & 1 & 0 \\ 1 & 0 & 0\end{pmatrix}$.
    \end{enumerate}
    \item \begin{enumerate}
        \item 因为 $A^k=\begin{pmatrix}\lambda_1^k & 0 \\ 0 & \lambda_2^k\end{pmatrix}$,所以 $f(A)=a_mA^m+a_{m-1}A^{m-1}+\cdots+a_1A+a_0E = \begin{pmatrix}f(\lambda_1) & 0 \\ 0 & f(\lambda_2)\end{pmatrix}$.
        \item $A=PBP^{-1}$,则 $A^2=(PBP^{-1})(PBP^{-1})=PB^2P^{-1}$. 由归纳法得 $A^k=PB^kP^{-1}$,于是
        \[f(A)=a_mPB^mP^{-1}+a_{m-1}PB^{m-1}P^{-1}+\cdots+a_0=P\begin{pmatrix}f(\lambda_1) & 0 \\ 0 & f(\lambda_2)\end{pmatrix}P^{-1}=Pf(B)P^{-1}\]
    \end{enumerate}
\end{enumerate}

\centerline{\heiti C组}
\begin{enumerate}
    \item
\end{enumerate}

\clearpage

\chapter{线性映射矩阵表示(II)}

\section{矩阵转置}
\subsection{基本概念}
实际上,矩阵的转置就是第$i$行变成了第$i$列,或者抽象表达为:
\[A=(a_{ij})_{m \times n},\enspace A^\mathrm{T}=(a'_{ji})_{n \times m},\enspace a_{ij}=a'_{ji}\]
写成矩阵形式就是:
\begin{definition}
    设$A=\begin{pmatrix}
        a_{11} & a_{12} & \cdots & a_{1n} \\
        a_{21} & a_{22} & \cdots & a_{2n} \\
        \vdots & \vdots & \ddots & \vdots \\
        a_{m1} & a_{m2} & \cdots & a_{mn}
    \end{pmatrix}$,称$\begin{pmatrix}
        a_{11} & a_{21} & \cdots & a_{m1} \\
        a_{12} & a_{22} & \cdots & a_{m2} \\
        \vdots & \vdots & \ddots & \vdots \\
        a_{1n} & a_{2n} & \cdots & a_{mn}
    \end{pmatrix}$为矩阵$A$的转置,记作$A^\mathrm{T}$.
\end{definition}

\subsection{基本性质}
\begin{enumerate}
    \item $(A^\mathrm{T})^\mathrm{T}=A$

    \item $(A+B)^\mathrm{T}=A^\mathrm{T}+B^\mathrm{T}$

    \item $(\lambda A)^\mathrm{T}=\lambda A^\mathrm{T},\enspace \lambda \in \mathbf{F}$

    \item $(AB)^\mathrm{T}=B^\mathrm{T}A^\mathrm{T}$,$(A_1A_2\cdots A_n)^\mathrm{T}=A_n^\mathrm{T}\cdots A_2^\mathrm{T}A_1^\mathrm{T}$

    \item $(A^\mathrm{T})^{-1}=(A^{-1})^\mathrm{T}$

    \item $(A^\mathrm{T})^m=(A^m)^\mathrm{T}$
\end{enumerate}

以上证明大都是平凡的,可以自己尝试完成.
\subsection{对阵矩阵与反对称矩阵}
\begin{definition}
    设$A=(a_{ij})_{n \times n}$,如果$\forall i,j=1,2,\ldots,n$均有$a_{ij}=a_{ji}$,
    则称$A$为对称矩阵. 若均有$a_{ij}=-a_{ji}$,则称$A$为反对称矩阵.
\end{definition}
易得$A$为对称矩阵的充要条件为$A=A^\mathrm{T}$,$A$为反对称矩阵的充要条件为$A=-A^\mathrm{T}$.
\begin{example}
    证明以下几点性质:
    \begin{enumerate}
        \item 反对称矩阵主对角元均为0;

        \item $AA^\mathrm{T}$和$A^\mathrm{T}A$均为对称矩阵;

        \item 设$A,B$为$n$阶对称和反对称矩阵,则$AB+BA$是反对称矩阵;

        \item 对称矩阵的乘积不一定对称;

        \item 可逆的对称(反对称)矩阵的逆矩阵也是对称(反对称)矩阵.
    \end{enumerate}
\end{example}

\section{初等矩阵}
\subsection{基本概念与性质}
\begin{definition}
    将单位矩阵$E$做一次初等变换得到的矩阵称为初等矩阵,与三种初等行、列变换对应的三类初等矩阵为:
    \begin{enumerate}
        \item 将单位矩阵第$i$行(或列)乘$c$,得到初等倍乘矩阵$E_i(c)$;

        \item 将单位矩阵第$i$行乘$c$加到第$j$行,或将第$j$列乘$c$加到第$i$列,得到初等倍加矩阵$E_{ij}(c)$;

        \item 将单位矩阵第$i,j$行(或列)对换,得到初等对换矩阵$E_{ij}$.
    \end{enumerate}
\end{definition}
请各位同学以矩阵形式写出以上三类矩阵.注意:
\begin{enumerate}
    \item 倍加变化请一定注意$i$和$j$在行列的情况下的不同;

    \item 三类矩阵不是三个矩阵,例如行列选择不唯一,常数选择不唯一;

    \item 注意三种初等矩阵都是可逆的,且$E_i^{-1}(c)=E_i\left(\dfrac{1}{c}\right)$,$E_{ij}^{-1}(c)=E_{ij}(-c)$,$E_{ij}^{-1}=E_{ij}$;

    \item 三种初等矩阵的转置:$E_i^\mathrm{T}(c)=E_i(c)$,$E_{ij}^\mathrm{T}(c)=E_{ji}(c)$,$E_{ij}^\mathrm{T}=E_{ij}$;
\end{enumerate}

初等矩阵大家非常关心为什么左乘代表行变换,右乘代表列变换.以右乘为例,我们来看矩阵$A$和$B$相乘的任一列结果.我们可以将矩阵$A$
按列做分块矩阵得到$\begin{pmatrix}\alpha_1,\ldots,\alpha_n\end{pmatrix}$,$\alpha_i$即表示$A$的第$i$列.然后矩阵$B$的第$j$列为列向量$(x_1,\ldots,x_n)^\mathrm{T}$,
由于矩阵$A$与$B$相乘结果第$j$列就是$A$与$B$的第$j$列相乘结果(回顾矩阵乘法的计算方式),则有$B$的第$i$列等于
$x_1\alpha_1+\cdots+x_n\alpha_n$即为$A$的全部列向量的线性组合,故右乘矩阵$A$得到矩阵的任一列都是$A$的全部列向量的线性组合,
所以右乘可以代表列变换.注意我这里并没有限制矩阵$B$为初等矩阵或可逆矩阵.

实际上左乘表示行变换可以用类似方法说明,只需按行对$B$分块即可.这一思想是特别重要的,在很多时候如果我们意识到左右乘是对被乘矩阵的行列
重新线性组合,思路会清晰很多.

关于初等矩阵还有一个相当重要的定理:

\begin{theorem}
    任意可逆矩阵都可以被表示为若干个初等矩阵的乘积.
\end{theorem}
定理证明只需要回忆高斯消元法可以将可逆矩阵化为单位矩阵即可.

利用矩阵初等变换我们可以获得本学期需要学习的三个矩阵标准形,因此这一内容虽然很基本但是非常重要:
\begin{enumerate}
    \item 相抵矩阵:本章已学习的内容,在之后会详细说明;
    \item 相似矩阵:若$P$为初等矩阵,对矩阵做$P^{-1}AP$变换即可得到与$A$相似的矩阵;
    \item 相合矩阵:两个矩阵,其中一个可以通过做相同的初等行列变换的到另一个矩阵(若$P$为初等矩阵,
    $P^{\mathrm{T}}AP$就是对$A$做了一次相同的初等行列变换).
\end{enumerate}
请同学们思考:如何从线性映射矩阵表示的角度理解初等变换与标准形的关系?在B组习题中将有练习进行体会
(实际上对矩阵表示的基做``初等变换''就是对表示矩阵做了初等变换,这两种变换行列方向不一致且矩阵互逆).

\section{矩阵的逆}
\subsection{基本概念}
\begin{definition}
    设$A \in \mathbf{M}_n(\mathbf{F})$. 若存在$B \in \mathbf{M}_n(\mathbf{F})$使得$AB=BA=E$,则称矩阵$A$可逆,
    并把$B$称为$A$的逆矩阵,记作 $ B = A^{-1} $.
\end{definition}
注意,逆矩阵定义基于方阵,非方阵没有上述逆矩阵.广义逆矩阵允许非方阵,但那是另一个定义,
我们不需要掌握.对于可逆矩阵,注意以下两个定理:
\begin{theorem}
    可逆矩阵$A$的逆矩阵唯一.
\end{theorem}
\begin{theorem}
    $AB=E \iff A$与$B$互为逆矩阵.
\end{theorem}
这两个定理的证明教材中有,特别注意唯一性的证明,反证法的思路一定要掌握,十分经典.
还需要强调的一点是,逆矩阵来源于逆映射.
\subsection{基本性质}
\begin{enumerate}
    \item 注意没有加法性质(请举出反例),对于数乘有$(\lambda A)^{-1}=\lambda^{-1}A^{-1}$;

    \item $(AB)^{-1}=B^{-1}A^{-1},\enspace (A_1A_2\cdots A_k)^{-1}=A_k^{-1}\cdots A_2^{-1}A_1^{-1}$;

    \item $(A^k)^{-1}=(A^{-1})^k,\enspace A^kA^m=A^{k+m},\enspace (A^k)^m=A^{km}$;

    \item 若$A$和$B$可逆,则$A\neq O$且$B\neq O$能推出$AB\neq O$,并且$A$可逆且$AB=O$可以推出$B=O$,除此之外还有消去律成立,即$A$则有$AB=AC \implies B=C$成立.
\end{enumerate}

还需要熟练掌握可逆矩阵的几个等价条件:
\begin{theorem}
    设$A \in \mathbf{M}_n{\mathbf{F}}$,则下列命题等价:
    \begin{enumerate}
        \item $A$可逆;

        \item $r(A)=n$;

        \item $A$的$n$个行(列)向量线性无关;

        \item 齐次线性方程组$AX=0$只有零解;

        \item $|A|\neq 0$.
    \end{enumerate}
\end{theorem}
\begin{example}
    已知矩阵 $A=\begin{pmatrix}a & b & c \\ d & e & f \\ h & x & y\end{pmatrix}$ 的逆是 $A^{-1}=\begin{pmatrix}-1 & -2 & -1 \\ 2 & 1 & 0 \\ 0 & -3 & -1\end{pmatrix}$,

$B=\begin{pmatrix}a-2b & b-3c & -c \\ d-2e & e-3f & -f \\ h-2x & x-3y & -y\end{pmatrix}$.求矩阵 $X$ 满足:

\[X+\left(B(A^TB^2)^{-1}A^T\right)^{-1}=X\left(A^2(B^TA)^{-1}B^T\right)^{-1}(A+B)\]
\end{example}

\subsection{逆矩阵的求解(基本方法)}
\begin{enumerate}
    \item 利用解线性方程组的方法:假设$AX=b$,使用高斯消元法求解;

    \item 利用初等矩阵的方法(初等行变换为常用方法).
\end{enumerate}

注意,基于初等变换的方法是非常重要的,我们很多时候不要被题目吓到去采用其他
偏门的方法,实际上很多时候拿到一个具体的矩阵求逆,使用的方法就是初等行变换.

\begin{example}
    用上述两种方法求矩阵$A=\begin{pmatrix}1 & -1 & 1 \\ 0 & 1 & 2 \\ 1 & 0 & 4\end{pmatrix}$的逆矩阵.
\end{example}

\subsection{矩阵方程}
\begin{enumerate}
    \item 考虑以下情形(其中出现的矩阵除$X$外均可逆,$X$不一定是列向量):
    \begin{enumerate}[label=(\arabic*)]
        \item $AX=B \implies X=A^{-1}B, \enspace XA=B \implies X=BA^{-1}$;
        \item $AXB=C \implies X=A^{-1}CB^{-1}$;
    \end{enumerate}
    \item 考虑以下情形:$AX=B$但$A$不可逆($X$不一定是列向量),直接高斯消元即可;
    \item 考虑以下求解方式的合理性:
    \begin{enumerate}[label=(\arabic*)]
        \item 若求$A^{-1}$,只需对$(A,E)$只做初等行变换,可以得到$(E,A^{-1})$;
        \item 若求$A^{-1}B$,只需对$(A,B)$只做初等行变换,可以得到$(E,A^{-1}B)$;
        \item 若求$BA^{-1}$,只需对$\begin{pmatrix}
            A \\ B
        \end{pmatrix}$只做初等列变换,可以得到$\begin{pmatrix}
            E \\ BA^{-1}
        \end{pmatrix}$;
        \item 对$\begin{pmatrix}
            A & E \\ E & O
        \end{pmatrix}$的前$n$行与$n$列做相同的行列变换,可以得到$\begin{pmatrix}
            P^\mathrm{T}AP & P^\mathrm{T} \\ P & O
        \end{pmatrix}$.
    \end{enumerate}
\end{enumerate}

\begin{example}
    设$A=\begin{pmatrix}1 & 0 & 0 \\ 1 & 1 & 0 \\ 1 & 1 & 1\end{pmatrix},\
    B=\begin{pmatrix}0 & 1 & 1 \\ 1 & 0 & 1 \\ 1 & 1 & 0\end{pmatrix}$,求矩阵$X$满足:
    \[AXA+BXB=AXB+BXA+A(A-B)\]
\end{example}

\subsection{一些相似的定理}
\begin{theorem} \label{thm:6:线性映射对向量坐标的影响}
    \textbf{\heiti 线性映射对向量坐标的影响}

    设$\sigma \in \mathcal{L}(V_1,V_2)$关于$V_1$和$V_2$的基$B_1$和基$B_2$的矩阵为$A=(a_{ij})_{m \times n}$,
    且$\alpha$与$\sigma(\alpha)$在基$B_1$和基$B_2$下的坐标分别为$X$和$Y$,则$Y=AX$.
\end{theorem}
上述即教材定理4.1,这一定理给出一个向量经过线性映射之后,其坐标的变化. 我们可以用下图表示:

\begin{figure}[h]
    \centering
    \begin{tikzpicture}[>=Stealth]
        \node (V) at (0,0) {$V$};
        \node (W) at (3,0) {$W$};
        \node (Fn) at (0,-3) {$\mathbf{F}^n$};
        \node (Fm) at (3,-3) {$\mathbf{F}^m$};
        \draw[->,thick] (V) -- node[below]{表示矩阵:$A$} (W);
        \draw[<->,thick] (V) -- node[right]{同构} (Fn);
        \draw[<->,thick] (W) -- node[left]{同构} (Fm);
        \draw[->,thick] (Fn) -- node[above]{$\sigma(\alpha)=A\alpha$} (Fm);
    \end{tikzpicture}
\end{figure}

图中我们可以看出通过坐标映射后得到的新映射即为\autoref{thm:6:线性映射对向量坐标的影响} 描述的映射.

在描述下一定理之前,我们首先介绍过渡矩阵(变换矩阵)的概念.
\begin{definition}
    设$B_1=\{\alpha_1,\alpha_2,\ldots,\alpha_n\}$与$B_2=\{\beta_1,\beta_2,\ldots,\beta_n\}$是线性空间
    $V(\mathbf{F})$的任意两组基,$B_2$中每个基向量被基$B_1$表示为
    \[ \left\{
    \begin{array}{rcl}
        \beta_1&=&a_{11}\alpha_1+a_{21}\alpha_2+\cdots+a_{n1}\alpha_n \\
        \beta_2&=&a_{12}\alpha_1+a_{22}\alpha_2+\cdots+a_{n2}\alpha_n \\
        &\vdots& \\
        \beta_n&=&a_{1n}\alpha_1+a_{2n}\alpha_2+\cdots+a_{nn}\alpha_n
    \end{array}
    \right. \]
    将上式用矩阵表示为
    \[(\beta_1,\beta_2,\cdots,\beta_n)=(\alpha_1,\alpha_2,\cdots,\alpha_n)\begin{pmatrix}
        a_{11} & a_{12} & \cdots & a_{1n} \\
        a_{21} & a_{22} & \cdots & a_{2n} \\
        \vdots & \vdots & \ddots & \vdots \\
        a_{n1} & a_{n2} & \cdots & a_{nn}
    \end{pmatrix}\]
    我们将这一矩阵称为即$B_1$变为基$B_2$的变换矩阵(或过渡矩阵).
\end{definition}
简单而言就是将$B_2$中的向量在$B_1$下的坐标按列排列.需要注意表述中是$B_1$变为基$B_2$还是反过来,
这两个矩阵互逆.注意过渡矩阵一定是基与基之间的表示矩阵,并且过渡矩阵一定可逆.
\begin{theorem}
    \textbf{\heiti 基的选择对向量坐标的影响}

    设线性空间$V$的两组基为$B_1$和$B_2$,且基$B_1$到$B_2$的变换矩阵(过渡矩阵)为$A$,如果
    $\xi \in V(\mathbf{F})$,且在$B_1$和$B_2$下的坐标分别为$X$和$Y$,则$Y=A^{-1}X$.
\end{theorem}
上述即教材定理4.10,描述同一个向量在不同基下坐标之间的关系.事实上,这与本节同构关系紧密,因为
同构意味着两个线性空间结构一致,故同构映射可以保持向量组的线性关系不变.在同构关系下,
线性组合对应线性组合,线性无关对应线性无关,线性相关对应线性相关.我们有如下定理:
\begin{theorem}
    设$(\alpha_1,\alpha_2,\ldots,\alpha_n)$是线性无关的向量组,且
    \[(\beta_1,\beta_2,\ldots,\beta_s)=(\alpha_1,\alpha_2,\ldots,\alpha_n)A\]
    则向量组$(\beta_1,\beta_2,\ldots,\beta_s)$的秩等于矩阵$A$的秩.
\end{theorem}
定理的证明需要用到坐标映射是同构映射这一事实,我们不难发现等式左侧向量组与$A$的列向量组是等价的.
事实上我们也可以由此发现,过渡矩阵一定是可逆矩阵.
\begin{theorem}
    已知$\beta_i=a_{1i}\alpha_1+a_{2i}\alpha_2+\cdots+a_{ni}\alpha_n\enspace(i=1,2,\ldots,n)$,
    且$A=(a_{ij})$可逆,则$\alpha_1,\alpha_2,\ldots,\alpha_n$与$\beta_1,\beta_2,\ldots,\beta_n$
    等价.
\end{theorem}
实际上这一定理与上一定理的思想都是类似的,我们可以看一个例题练习一下:
\begin{example}
    已知$\beta_1=\alpha_2+\alpha_3,\enspace\beta_2=\alpha_1+\alpha_3,\enspace\beta_3=\alpha_1+\alpha_2$,
    证明$\alpha_1,\alpha_2,\alpha_3$与$\beta_1,\beta_2,\beta_3$等价.
\end{example}
\begin{theorem}
    \textbf{\heiti 基的选择对映射矩阵的影响}

    设线性变换$\sigma \in \mathcal{L}(V,V)$,$B_1=\{\alpha_1,\ldots,\alpha_n\}$和$B_2=\{\beta_1,\ldots,\beta_n\}$
    是线性空间的$V(\mathbf{F})$的两组基,基$B_1$变为基$B_2$的变换矩阵为$C$,如果$\sigma$在基$B_1$下的矩阵为$A$,
    则$\sigma$关于基$B_2$所对应的矩阵为$C^{-1}AC$.
\end{theorem}
上述即教材定理7.4,研究同一个映射在不同基下表示矩阵之间的关系.实际上我们将在下一专题初等矩阵一节进一步讨论.
这一定理的证明需要用到我们之前描述的两种线性映射矩阵表示的统一性.

\vspace{2ex}
\centerline{\heiti \Large 内容总结}

\vspace{2ex}

\centerline{\heiti \Large 习题}
\vspace{2ex}
{\kaishu }
\begin{flushright}
    \kaishu

\end{flushright}
\centerline{\heiti A组}
\begin{enumerate}
    \item
\end{enumerate}
\centerline{\heiti B组}
\begin{enumerate}
    \item
\end{enumerate}
\centerline{\heiti C组}
\begin{enumerate}
    \item
\end{enumerate}

\section*{9 线性映射矩阵表示(III)}
\addcontentsline{toc}{section}{9 线性映射矩阵表示(III)}

\vspace{2ex}

\centerline{\heiti A组}
\begin{enumerate}
    \item
\end{enumerate}

\centerline{\heiti B组}
\begin{enumerate}
    \item
\end{enumerate}

\centerline{\heiti C组}
\begin{enumerate}
    \item
\end{enumerate}

\clearpage

\phantomsection
\section*{10 矩阵运算进阶(I)}
\addcontentsline{toc}{section}{10 矩阵运算进阶(I)}

\vspace{2ex}

\centerline{\heiti A组}
\begin{enumerate}
    \item 由题意有 $P_2AP_1 = E$,从而有 $A=P_2^{-1}P_1^{-1}=P_2P_1^{-1}$.

    \item 由题意知 $B = E_{ij}A$,所以 $BA^{-1}=E_{ij}$ 从而 $B$ 可逆,同时可得 $AB^{-1}=A(A^{-1}E_{ij}^{-1})=E_{ij}$.

    \item $Q = (\alpha_1+\alpha_2,\alpha_2,\alpha_3)=(\alpha_1,\alpha_2,\alpha_3)\begin{pmatrix}1 & 0 & 0 \\ 1 & 1 & 0 \\ 0 & 0 & 1\end{pmatrix}=P\begin{pmatrix}1 & 0 & 0 \\ 1 & 1 & 0 \\ 0 & 0 & 1\end{pmatrix}$.

          记 $E_{12}(1)=\begin{pmatrix}1 & 0 & 0 \\ 1 & 1 & 0 \\ 0 & 0 & 1\end{pmatrix}$,有 $Q^{-1}AQ=E_{12}(1)^{-1}P^{-1}APE_{12}(1)=\begin{pmatrix}1 & 0 & 0 \\ 0 & 1 & 0 \\ 0 & 0 & 2\end{pmatrix}$.

    \item 我们只求解第一个矩阵的逆,第二个方法类似. 事实上正文例题中也有类似的,本题甚至更简单,因为左下角已经是零矩阵,所以不再需要``打这个洞''. 设$A$和$D$分别是$m$、$n$阶矩阵,我们首先将第二个分块行左乘$-BD^{-1}$加到第一个分块行,目的很明确,就是把右上角的洞打出来:
          \[\left(\begin{array}{cc:cc}
                      A & B & E_m & O \\ O & D & O & E_n
                  \end{array}\right)\rightarrow\left(\begin{array}{cc:cc}
                      A & O & E_m & -BD^{-1} \\ O & D & O & E_n
                  \end{array}\right),\]
          接下来再用$A^{-1}$和$D^{-1}$分别左乘第一分块行和第二分块行,得到
          \[\left(\begin{array}{cc:cc}
                      E_m & O & A^{-1} & -A^{-1}BD^{-1} \\ O & E_n & O & D^{-1}
                  \end{array}\right),\]
          由此可得原矩阵的逆就是上述虚线右侧的$\begin{pmatrix}
                  A^{-1} & -A^{-1}BD^{-1} \\ O & D^{-1}
              \end{pmatrix}$.
\end{enumerate}

\centerline{\heiti B组}
\begin{enumerate}
    \item 此处仅给出答案,具体过程略.
          \begin{enumerate}
              \item $\begin{pmatrix}a_{21} & a_{22} & a_{23} \\ a_{11} & a_{12} & a_{13} \\ a_{31} & a_{32} & a_{33}\end{pmatrix}$.

              \item $\begin{pmatrix}-a_{11} & -a_{12} & -a_{13} \\ a_{21} & a_{22} & a_{23} \\ a_{31} & a_{32} & a_{33}\end{pmatrix}$.

              \item $\begin{pmatrix}-a_{13} & -a_{12} & a_{11} \\ a_{23} & a_{22} & -a_{21} \\ a_{33} & a_{32} & -a_{31}\end{pmatrix}$.

              \item $\begin{pmatrix}-a_{11}-a_{12} & -a_{12}-a_{13} & -a_{13}-a_{11} \\ a_{21}+a_{22} & a_{22}+a_{23} & a_{23}+a_{21} \\ a_{31}+a_{32} & a_{32}+a_{33} & a_{33}+a_{31}\end{pmatrix}$.
          \end{enumerate}

    \item \begin{enumerate}
              \item 略.

              \item 设 $A=(a_{ij})_{3\times 2}$,$e_{ij}$ 为 $\mathbf{R}^{3\times 2}$ 的自然基. 因为 $PAQ = \begin{pmatrix}a_{12}+a_{22} & 0 \\ a_{22} & 0 \\ 0 & 0\end{pmatrix}$,所以 $\sigma(e_{12}) = e_{11},\sigma(e_{22}) = e_{11}+e_{21},\sigma(e_{11})=\sigma(e_{21})=\sigma(e_{31})=\sigma(e_{32})=0$.

                    于是 $\ker\sigma = \spa (e_{11},e_{21},e_{31},e_{32}),\im\sigma = \spa (e_{11},e_{11}+e_{21})$.

              \item 令 $B_1=\{e_{12},e_{22},e_{11},e_{21},e_{31},e_{32}\},B_2=\{e_{11},e_{11}+e_{21},e_{12},e_{22},e_{31},e_{32}\}$,则均为 $\mathbf{R^{3\times 2}}$ 的基,且 $\sigma(\varepsilon)=(\eta)\begin{pmatrix}E_2 & 0 \\ 0 & 0\end{pmatrix}$.
          \end{enumerate}

    \item 见教材 P147/例 5
\end{enumerate}

\centerline{\heiti C组}
\begin{enumerate}
    \item 使用数学归纳法. 当 $n=1$ 时,$A=\begin{pmatrix}a\end{pmatrix}(a\neq 0)$,$B$ 取任意一阶矩阵均成立;假设 $n-1$ 阶成立,$A = \begin{pmatrix}A_1 & \alpha \\ \beta & a_{nn}\end{pmatrix}$,其中 $A_1$ 为 $n-1$ 阶矩阵且存在 $n-1$ 阶下三角矩阵 $B_1$ 使得 $B_1A_1$ 为上三角矩阵,则有
          \[\begin{pmatrix}B_1 & O \\ O & 1\end{pmatrix}\begin{pmatrix}A_1 & \alpha \\ O & a_{nn}-\beta A^{-1}\alpha\end{pmatrix} = \begin{pmatrix}B_1A_1 & B_1\alpha \\ O & a_{nn}-\beta A_1^{-1}\alpha\end{pmatrix}\]为上三角矩阵.
          而 \[\begin{pmatrix}E_{n-1} & O \\ -\beta A_1^{-1} & 1\end{pmatrix}\begin{pmatrix}A_1 & \alpha \\ \beta & a_{nn}\end{pmatrix}=\begin{pmatrix}A_1 & \alpha \\ O & a_{nn}-\beta A_1^{-1} \alpha\end{pmatrix}\]
          故 $B=\begin{pmatrix}B_1 & O \\ O & 1\end{pmatrix}\begin{pmatrix}E_{n-1} & O \\ -\beta A_1^{-1} & 1\end{pmatrix}$ 符合条件($B_1$ 为下三角矩阵,故 $B$ 也是).

    \item 证明:$ \forall \alpha \in W,\enspace A_{12} \alpha = \vec{0} $.
          \begin{align*}
              \begin{pmatrix} O_{k \times 1} \\ \alpha \end{pmatrix}
               & = A^{-1}A \begin{pmatrix} O_{k \times 1} \\ \alpha \end{pmatrix} = A^{-1} \begin{pmatrix} A_{11} & A_{12} \\ A_{21} & A_{22} \end{pmatrix} \begin{pmatrix} O_{k \times 1} \\ \alpha \end{pmatrix} = A^{-1} \begin{pmatrix} O_{l \times 1} \\ A_{22} \alpha \end{pmatrix} \\
               & = \begin{pmatrix} B_{11} & B_{12} \\ B_{21} & B_{22} \end{pmatrix} \begin{pmatrix} O_{l \times 1} \\ A_{22} \alpha \end{pmatrix} = \begin{pmatrix} B_{12} A_{22} \alpha \\ B_{22} A_{22} \alpha \end{pmatrix}
          \end{align*}
          故 $ B_{12} A_{22} \alpha = \vec{0} $,故我们可以推测如下定义:$ \sigma \in \mathcal{L}(W,U),\enspace \sigma(\alpha) = A_{22} \alpha $.
          只需证明 $ \sigma $ 是单射且满射即可.

          单射:$ \sigma(\alpha) = \sigma(\beta) \implies A_{22}(\alpha - \beta) = \vec{0} $. 又有 $ A_{12} \alpha = A_{12} \beta = \vec{0} \implies A_{12}( \alpha - \beta) = \vec{0} $. 故 $ \begin{pmatrix} A_{12} \\ A_{22} \end{pmatrix} (\alpha - \beta) = \vec{0} $. 由于 $ \begin{pmatrix} A_{12} \\ A_{22} \end{pmatrix} $ 列满秩($ A $ 可逆),故 $ \alpha = \beta $.

          满射:$ \forall \gamma \in U,\enspace B_{12} \gamma = \vec{0} $.
          \begin{align*}
              \begin{pmatrix} O_{l \times 1} \\ \gamma \end{pmatrix}
               & = A A^{-1} \begin{pmatrix} O_{l \times 1} \\ \gamma \end{pmatrix} = A \begin{pmatrix} B_{11} & B_{12} \\ B_{21} & B_{22} \end{pmatrix} \begin{pmatrix} O_{l \times 1} \\ \gamma \end{pmatrix}      \\
               & = A \begin{pmatrix} O_{k \times 1} \\ B_{22} \gamma \end{pmatrix} = \begin{pmatrix} A_{11} & A_{12} \\ A_{21} & A_{22} \end{pmatrix} \begin{pmatrix} O_{k \times 1} \\ B_{22} \gamma \end{pmatrix} \\
               & = \begin{pmatrix} A_{12} B_{22} \gamma \\ A_{22} B_{22} \gamma \end{pmatrix}
          \end{align*}
          故 $ \exists B_{22} \gamma \in W,\enspace A_{22} B_{22} \gamma = \gamma \in U $.
\end{enumerate}

\clearpage

\section*{11 矩阵的秩}
\addcontentsline{toc}{section}{11 矩阵的秩}

\vspace{2ex}

\centerline{\heiti A组}
\begin{enumerate}
    \item 取 $\mathbf{R^4}$ 标准基 $\varepsilon_1,\varepsilon_2,\varepsilon_3,\varepsilon_4$.
    那么 $(\alpha_1,\alpha_2,\alpha_3,\alpha_4)=(\varepsilon_1,\varepsilon_2,\varepsilon_3,\varepsilon_4)A,(\beta_1,\beta_2,\beta_3,\beta_4)=(\varepsilon_1,\varepsilon_2,\varepsilon_3,\varepsilon_4)B.$
    其中 \[A=\begin{pmatrix}1 & 1 & 1 & 1 \\ 1 & 1 & -1 & -1 \\ 1 & -1 & 1 & -1 \\ 1 & -1 & -1 & 1\end{pmatrix},B=\begin{pmatrix}1 & 2 & 1 & 0 \\ 1 & 1 & 1 & 1 \\ 0 & 3 & 0 & -1 \\ 1 & 1 & 0 & -1\end{pmatrix}.\] 
    由此可知 \[(\beta_1,\beta_2,\beta_3,\beta_4)=(\varepsilon_1,\varepsilon_2,\varepsilon_3,\varepsilon_4)B=(\alpha_1,\alpha_2,\alpha_3,\alpha_4)A^{-1}B.\]
    过渡矩阵 \[A^{-1}B=\dfrac{1}{4}\begin{pmatrix}3 & 7 & 2 & -1 \\ 1 & -1 & 2 & 3 \\ -1 & 3 & 0 & -1 \\ 1 & -1 & 0 & -1\end{pmatrix}\]
    另外,容易求得 $\xi$ 在 $\alpha_1,\alpha_2,\alpha_3,\alpha_4$ 下的坐标为 $\begin{pmatrix}0 \\ \frac{1}{2} \\ \frac{1}{2} \\ 0\end{pmatrix}$
    \item 证明:考虑矩阵 $A$ 的行向量组的极大线性无关组,若添加的一行可由其极大线性无关组线性表示,则秩不变. 否则秩增加 $1$. 
    \item 证明:设 $A$ 的行向量组为 $\{\alpha_1,\alpha_2,\cdots,\alpha_s\}$,$r(A)=r$; $B$ 的行向量组为 $\{\alpha_1,\alpha_2,\cdots,\alpha_m\},r(B)=k$.
    不妨设:$B$ 的行向量组的极大线性无关组为 $\{\alpha_1,\alpha_2,\cdots,\alpha_k,\alpha_{i_1},\cdots,\alpha_{i_{r-k}}\}$,其中 $\{\alpha_{i_1},\cdots,\alpha_{i_{r-k}}\}$(共 $r-k$ 个向量)是包含在 $\{\alpha_{m+1},\cdots,\alpha_s\}$(共 $s-m$ 个向量)之中的. 显然有
    \[r-k \leq s-m,\]
    即
    \[r(B)=k\ge r+m-s=r(A)+m-s.\]
\end{enumerate}

\centerline{\heiti B组}
\begin{enumerate}
    \item 已知 $r(A+B) \leq r(A)+r(B)$,把 $B$ 写成 $-B$ 则有 $r(A-B) \leq r(A)+r(-B)=r(A)+r(B)$. 不等式右半部分得证.
    
    另外,$r(A)=r(A-B+B) \leq r(A-B)+r(B)$,从而 $r(A-B) \ge r(A)-r(B)$,当然,加个绝对值也是没有问题的:$r(A-B) \ge \lvert r(A)-r(B) \rvert$. 同理,有 $r(A+B) \ge \lvert r(A)+r(B) \rvert$. 证毕.
    \item $V$ 的基 $B_1$ 到 $B_2$ 的过渡矩阵 $P$ 具有下述形式:
    \[P=\begin{pmatrix}\mathrm{Im} & B_1 \\ 0 & B_2\end{pmatrix}\]
    其中 $B_1,B_2$ 分别是域 $\mathbf{F}$ 上 $m\times (n-m),(n-m)\times (n-m)$ 矩阵,
    \[\beta_j=b_{j1}\delta_1+\cdots+b_{jm}\delta_m+b_{j,m+1}\delta_{m+1}+\cdots+b_{jn}a_n\]
    其中 $j=m+1,\cdots,n$. 于是
    \[\beta_j+W=b_{j,m+1}(\alpha_{m+1}+W)+\cdots+b_{jn}(\alpha_n+W)\]
    因此商空间 $V/W$ 的基 $\alpha_{m+1}+W,\cdots,\alpha_n+W$ 到 $\beta_{m+1}+W,\cdots,\beta_n+W$ 的过渡矩阵是 $B_2$.
    \item 设 $\beta_1=\alpha_1+\alpha_2,\cdots,\beta_{n-1}=\alpha_{n-1}+\alpha_n,\beta_n=\alpha_n+\alpha_1$. 由于
    \[(\beta_1,\beta_2,\cdots,\beta_n) = (\alpha_1,\alpha_2,\cdots,\alpha_n)A,\]
    其中
    \[A=\begin{pmatrix}1 & & & & 1 \\ 1 & 1 & & & \\ & 1 & \ddots & & \\ & & \ddots & 1 & \\ & & & 1 & 1\end{pmatrix}.\]
    则有 $\lvert A \rvert = 1 + (-1)^{n+1}=2\neq 0$,所以 $A$ 可逆,可得 $\{\beta_1,\beta_2,\cdots,\beta_n\}$ 和 $\{\alpha_1,\alpha_2,\cdots,\alpha_n\}$ 等价. 这就说明了 $\alpha_1,\alpha_2,\cdots,\alpha_n$ 线性无关的充要条件是 $\beta_1,\beta_2,\cdots,\beta_n$ 线性无关.
    \item 记 $B_1=\{e_{11},e_{12},e_{21},e_{22}\},B_2=\{g_1,g_2,g_3,g_4\}$.\begin{enumerate}
        \item 设 $k_1g_1+k_2g_2+k_3g_3+k_4g_4=O$,可得 $k_1=k_2=k_3=k_4=0$,所以 $g_1,g_2,g_3,g_4$ 线性无关,从而是 $M_2(\mathbf{R})$ 的一组基.
        \item 由 $M_2(\mathbf{R}) \cong \mathbf{R^4}$,所以 $\{e_{11},e_{12},e_{21},e_{22}\}$ 可以表示为 $\mathbf{R^4}$ 中的自然基 $\{e_1,e_2,e_3,e_4\}$,而 $\{g_1,g_2,g_3,g_4\}$ 可表示为 $\{(1,0,0,0)^{\mathbf{T}},(1,1,0,0)^{\mathbf{T}},(1,1,1,0)^{\mathbf{T}},(1,1,1,1)^{\mathbf{T}}\}$.
        
        于是,由 \[\begin{pmatrix}g_1 & g_2 & g_3 & g_4\end{pmatrix}=\begin{pmatrix}e_{11} & e_{12} & e_{21} & e_{22}\end{pmatrix}C\]
        得 \[\begin{pmatrix}e_{11} & e_{12} & e_{21} & e_{22}\end{pmatrix}=\begin{pmatrix}g_1 & g_2 & g_3 & g_4\end{pmatrix}C^{-1}\]
        所以基 $B_2$ 变为 $B_1$ 的变换矩阵为 $C^{-1}=\begin{pmatrix}1 & -1 & 0 & 0 \\ 0 & 1 & -1 & 0 \\ 0 & 0 & 1 & -1 \\ 0 & 0 & 0 & 1\end{pmatrix}$.
        \item 考虑从 $A^2=A$ 中选取较为简单的矩阵,例如由
        \[\begin{pmatrix}a & b \\ 0 & 0\end{pmatrix}^2=\begin{pmatrix}a^2 & ab \\ 0 & 0\end{pmatrix}=\begin{pmatrix}a & b \\ 0 & 0\end{pmatrix}\]
        取 $a=1,b=0$ 或 $1$,得 $A_1=\begin{pmatrix}1 & 0 \\ 0 & 0\end{pmatrix},A_2=\begin{pmatrix}1 & 1 \\ 0 & 0\end{pmatrix}$

        类似地,可取 $A_3=\begin{pmatrix}0 & 0 \\ 0 & 1\end{pmatrix},A_4=\begin{pmatrix}0 & 0 \\ 1 & 1\end{pmatrix}$.

        这就取出了一组满足 $A^2=A$ 的线性无关的 $\{A_1,A_2,A_3,A_4\}$,是 $M_2(\mathbf{R})$ 的一组基 $B_3$.
        \item 先记 $B_2$ 变为 $B_3$ 的变换矩阵为 $D$,即 \[\begin{pmatrix}A_1 & A_2 & A_3 & A_4\end{pmatrix}=\begin{pmatrix}g_1 & g_2 & g_3 & g_4\end{pmatrix}D\]\
        按题 $(2)$ 中所述,此时有 \[\begin{pmatrix}1 & 1 & 0 & 0 \\ 0 & 1 & 0 & 0 \\ 0 & 0 & 0 & 1 \\ 0 & 0 & 1 & 1\end{pmatrix}=\begin{pmatrix}1 & 1 & 1 & 1 \\ 0 & 1 & 1 & 1 \\ 0 & 0 & 1 & 1 \\ 0 & 0 & 0 & 1\end{pmatrix}D\]
        由于上式右端已知矩阵的逆矩阵为上面的 $C^{-1}$,所以在上式两边左乘 $C^{-1}$,可得 \[D = \begin{pmatrix}1 & 0 & 0 & 0 \\ 0 & 1 & 0 & -1 \\ 0 & 0 & -1 & 0 \\ 0 & 0 & 1 & 1\end{pmatrix}\]
        由于矩阵 $A$ 关于 $B_2$ 的坐标为 $(1,1,1,1)^{\mathbf{T}}$,所以 $A$ 关于 $B_3$ 的坐标为
        \[Y=D^{-1}X=\begin{pmatrix}1 \\ 3 \\ -1 \\ 2\end{pmatrix}.\]
    \end{enumerate}
    \item \begin{enumerate}
        \item 初等变换即可.
        \item 同上.
        \item 矩阵 $A$ 秩为 $r$ 可写作 $A=P\begin{pmatrix}E_r & 0 \\ 0 & 0\end{pmatrix}Q = P(E_{11}+E_{22}+\cdots+E_{rr})Q$($E_r$ 是 $r\times r$ 的单位矩阵,$E_{ii}$ 是 $n\times n$ 的只有第 $i$ 行 $i$ 列的这个元素为 $1$,其他元素均为 $0$ 的矩阵). 
        每个 $PE_{ii}Q$ 都是秩为 $1$ 的矩阵,故得证.
        \item 记 $r(A)=r$,把 $A$ 写成 $P\begin{pmatrix}E_r & 0 \\ 0 & 0\end{pmatrix}Q$ 的形式. 构造 $B=Q^{-1}\begin{pmatrix}E_r & 0 \\ 0 & 0\end{pmatrix}P^{-1}$ 可以发现其满足条件,故得证.
    \end{enumerate}
    \item $r(BC)\leq r(B) \leq 1$,得证. 
    
    反之,若 $A$ 是秩为 $1$ 的 $3\times 3$ 矩阵,则存在可逆矩阵 $P,Q$ 使得 $A=P^{-1}E_{11}Q^{-1}$,其中 $E_{11}=\begin{pmatrix}1 & 0 & 0 \\ 0 & 0 & 0 \\ 0 & 0 & 0\end{pmatrix}=\begin{pmatrix}1 \\ 0 \\ 0\end{pmatrix}\begin{pmatrix}1 & 0 & 0\end{pmatrix}$.
    则取 $B=P^{-1}\begin{pmatrix}1 \\ 0 \\ 0\end{pmatrix},C=\begin{pmatrix}1 & 0 & 0\end{pmatrix}Q^{-1}$,有 $A=BC$,证毕.
    \item \begin{enumerate}
        \item $r(\alpha \alpha^{\mathbf{T}})\leq r(\alpha) = 1,r(\beta \beta^{\mathbf{T}})\leq r(\beta) = 1$. 由 $r(A+B) \leq r(A)+r(B)$ 得 $r(A)=r(\alpha \alpha^{\mathbf{T}}+\beta \beta^{\mathbf{T}}) \leq r(\alpha)+r(\beta)=2.$
        \item 若 $\alpha,\beta$ 均为 $\mathbf{0}$ 向量,显然. 否则假设 $\alpha$ 不为 $0$,则由于两向量线性相关,必有确定的 $k$ 使得 $\beta = k\alpha$,把 $\beta$ 用 $\alpha$ 表示之后易证.
    \end{enumerate}
    \item $r(A)=r$ 则 $AX=0$ 的解空间维数 $\mathrm{dim}N(A) = n-r$. 由 $r(A)+r(B)=k$ 得 $r(B)=k-r \leq n-r=\mathrm{dim}N(A)$. 要求 $AB=O$,说明 $B$ 的列向量均为 $AX=0$ 的解,那么只需要选择合适的列向量组拼接成 $B$ 即可(这一定能做到,因为 $B$ 维数不会超过解空间维数).
    \item 由于 $A$ 是 $m\times n$ 矩阵,$r(A)=m$,可知对于矩阵 $A$ 做初等列变换,可使其前 $m$ 列变为单位矩阵,后 $n-m$ 列变为全 $0$ 列.
    因此,存在 $n$ 阶可逆矩阵 $P$ 使得
    \[AP=\begin{pmatrix}E_m & O_{m\times (n-m)}\end{pmatrix}\]
    于是\[AP(AP)^{\mathrm{T}} = \begin{pmatrix}E & O\end{pmatrix} \begin{pmatrix}E \\ O\end{pmatrix}=E_m\]
    所以存在 $B=(PP^{\mathrm{T}}A^{\mathrm{T}})$ 为 $n\times m$ 矩阵,使 $AB=E$.
    \item 利用 $A,B$ 的相抵标准形. 存在 $n$ 阶可逆矩阵 $P_1,Q_1,P_2,Q_2$ 使得
    \[P_1AQ_1=\begin{pmatrix}E_{r_A} & O \\ O & O\end{pmatrix},P_2BQ_2=\begin{pmatrix}O & O \\ O & E_{r_B}\end{pmatrix}\]
    于是 \[AQ_1=P_1^{-1}\begin{pmatrix}E_{r_A} & O \\ O & O\end{pmatrix},P_2B=\begin{pmatrix}O & O \\ O & E_{r_B}\end{pmatrix}Q_2^{-1}\]
    所以 \[AQ_1P_2B=P_1^{-1}\begin{pmatrix}E_{r_A} & O \\ O & O\end{pmatrix}\begin{pmatrix}O & O \\ O & E_{r_B}\end{pmatrix}Q_2^{-1}=O\]
    取 $M=Q_1P_2$ 即可.
    \item \begin{enumerate}
        \item 易证,此处略去.
        \item 注意到 $B$ 的列向量均为 $AX=0$ 的解,设 $AX=0$ 的基础解系为 $\alpha_1,\cdots,\alpha_t(t=n-r)$,则易知
        \[B_{11}=(\alpha_1,0,\cdots,0),B_{12}=(0,\alpha_1,\cdots,0),\cdots,B_{1n}=(0,0,\cdots,\alpha_1),\]
        \[B_{21}=(\alpha_2,0,\cdots,0),B_{22}=(0,\alpha_2,\cdots,0),\cdots,B_{2n}=(0,0,\cdots,\alpha_2),\]
        \[\vdots\]
        \[B_{t1}=(\alpha_t,0,\cdots,0),B_{t2}=(0,\alpha_t,\cdots,0),\cdots,B_{tn}=(0,0,\cdots,\alpha_t)\]
        为 $S(A)$ 的一组基,故 $\mathrm{dim}S(A)=n(n-r)$.
    \end{enumerate}
\end{enumerate}

\centerline{\heiti C组}
\begin{enumerate}
    \item 对 $\begin{pmatrix}E_n & A' \\ A & E_s\end{pmatrix}$ 利用打洞原理有
    \[\begin{pmatrix}E_n-A'A & O \\ O & E_s\end{pmatrix} \leftarrow \begin{pmatrix}E_n & A' \\ A & E_s\end{pmatrix} \rightarrow \begin{pmatrix}E_n & O \\ O & E_s-AA'\end{pmatrix}\]
    所以 $r\begin{pmatrix}E_n-A'A & O \\ O & E_s\end{pmatrix}=r\begin{pmatrix}E_n & O \\ O & E_s-AA'\end{pmatrix}$,即 $s+r(E_n-A'A)=n+r(E_s-AA')$,即
    \[r(E_n-A'A)-r(E_s-AA')=n-s.\]
    \item \begin{enumerate}
        \item 由 \[\begin{pmatrix}A & 0 \\ 0 & B\end{pmatrix}\rightarrow \begin{pmatrix}A & AC \\ 0 & B\end{pmatrix}\rightarrow \begin{pmatrix}A & AC+BD \\ 0 & B\end{pmatrix}=\begin{pmatrix}A & E \\ 0 & B\end{pmatrix}\]
        \[\rightarrow \begin{pmatrix}0 & E \\ -AB & B\end{pmatrix}\rightarrow \begin{pmatrix}0 & E \\ AB & 0\end{pmatrix}\]
        可得.
    \item 用分块矩阵的方法,我们知道 
    \[\begin{pmatrix}A & O \\ O & B\end{pmatrix}\rightarrow \begin{pmatrix}A & O \\ A & B\end{pmatrix}\rightarrow \begin{pmatrix}A & A \\ A & A+B\end{pmatrix}\]
    结合 $AB=BA$,我们知道
    \[\begin{pmatrix}A & A \\ A & A+B\end{pmatrix}\begin{pmatrix}A+B & O \\ -A & E\end{pmatrix}=\begin{pmatrix}AB & A \\ O & A+B\end{pmatrix}\]
    于是
    \[r(A)+r(B)=r\begin{pmatrix}A & O \\ O & B\end{pmatrix}=r\begin{pmatrix}A & A \\ A & A+B\end{pmatrix}\ge \begin{pmatrix}AB & A \\ O & A+B\end{pmatrix}\ge r(AB)+r(A+B)\] 
    \end{enumerate}
    \item 略有超纲,使用贝祖定理,
    \[\exists u(x),v(x),u(x)f_1(x)+v(x)f_2(x)=1\]
    \[r\begin{pmatrix}f_1(A) & O \\ O & f_2(A)\end{pmatrix}=r\begin{pmatrix}f_1(A) & f_1(A)u(A)+f_2(A)v(A) \\ O & f_2(A)\end{pmatrix}=r\begin{pmatrix}f_1(A) & E \\ O & f_2(A)\end{pmatrix}\] 
    \[=r\begin{pmatrix}f_1(A) & E \\ -f_2(A)f_1(A) & O\end{pmatrix}=r\begin{pmatrix}O & E \\ f(A) & O\end{pmatrix}\]
    \item 由于 $A$ 是列满秩矩阵,$B$ 是行满秩矩阵,知存在可逆矩阵 $P_{3\times 3},Q_{2\times 2}$ 使得
    \[A=P\begin{pmatrix}E_2 \\ O\end{pmatrix},B=\begin{pmatrix}E_2 & O\end{pmatrix}Q\]
    于是 \[BA=\begin{pmatrix}E_2 & O\end{pmatrix}QP\begin{pmatrix}E_2 \\ O\end{pmatrix}\]
    由 $(AB)^2=9AB$ 有 \[P\begin{pmatrix}E_2 \\ O\end{pmatrix}\begin{pmatrix}E_2 & O\end{pmatrix}QP\begin{pmatrix}E_2 \\ O\end{pmatrix}\begin{pmatrix}E_2 & O\end{pmatrix}Q=9P\begin{pmatrix}E_2 \\ O\end{pmatrix}\begin{pmatrix}E_2 & O\end{pmatrix}Q\]
    即 \[\begin{pmatrix}E_2 \\ O\end{pmatrix}BA\begin{pmatrix}E_2 & O\end{pmatrix}=9\begin{pmatrix}E_2 \\ O\end{pmatrix}\begin{pmatrix}E_2 & O\end{pmatrix}\]
    也就是 \[\begin{pmatrix}BA & O \\ O & O\end{pmatrix}=\begin{pmatrix}9E_2 & 0 \\ 0 & 0\end{pmatrix}\]
    所以 $BA=9E_2$.
    \item 本题求核空间困难,但只需要求维数,我们考虑求像空间之后求出像空间维数,然后用维数公式求解.
    
    取 $F^{n\times p}$ 的自然基 $\{e_{11},e_{12},\cdots,e_{np}\}$($e_{ij}$ 表示仅有第 $i$ 行第 $j$ 列的元素为 $1$,其他均为 $0$ 的矩阵)

    则 $\mathrm{Im}\ \sigma=\mathrm{span}(\sigma(e_{11}),\cdots,\sigma(e_{np}))$.

    取 $A$ 的列向量,写成 $A=\begin{pmatrix}\alpha_1,\alpha_2,\cdots,\alpha_n\end{pmatrix}$,则 $\sigma(e_{ij})$ 可排列如下:
    \[(\alpha_1,0,\cdots,0),(0,\alpha_1,\cdots,0),\cdots,(0,0,\cdots,\alpha_1)\]
    \[(\alpha_2,0,\cdots,0),(0,\alpha_2,\cdots,0),\cdots,(0,0,\cdots,\alpha_2)\]
    \[\cdots\]
    \[(\alpha_n,0,\cdots,0),(0,\alpha_n,\cdots,0),\cdots,(0,0,\cdots,\alpha_n)\]
    由于 $r(A)=r$,故 $\alpha_1,\alpha_2,\cdots,\alpha_n$ 的极大线性无关组有 $r$ 个向量,不妨设为 $\alpha_1,\alpha_2,\cdots,\alpha_r$. 则下列向量:
    \[(\alpha_{r+1},0,\cdots,0),(0,\alpha_{r+1},\cdots,0),\cdots,(0,0,\cdots,\alpha_{r+1})\]
    \[\cdots\]
    \[(\alpha_n,0,\cdots,0),(0,\alpha_n,\cdots,0),\cdots,(0,0,\cdots,\alpha_n)\]
    均可以被其他向量线性表出. 观察除了上述向量的剩下的向量,可以发现这 $r\times p$ 个向量线性无关,从而 $\mathrm{dim}(\mathrm{Im}\ \sigma) = r\times p$.

    故由维数公式,得 $\mathrm{dim}(\mathrm{Ker}\ \sigma) = \mathrm{dim}F^{n\times p}-\mathrm{dim}(\mathrm{Im} \ \sigma) = (n-r)p$.
\end{enumerate}

\clearpage

\chapter{矩阵运算进阶(II)} \label{chap:矩阵运算进阶(II)}

本讲我们将讨论技巧性更强的一些内容,如特殊矩阵,矩阵可交换、求逆和求幂等. 大部分都是方法为主,理解的内容不多,但对于我们拿到问题有更好的洞察是很重要的.

\section{特殊矩阵}

\subsection{对角矩阵}

我们一般记主对角矩阵为$\diag(d_1,d_2,\dots,d_n)$,准对角矩阵为$\diag(A_1,A_2,\dots,A_n)$.
\begin{theorem}\label{thm:12:对角矩阵的性质}
    设$A$是一个$s \times n$矩阵,把$A$写成列向量与行向量的形式,分别为

    \[ A = \begin{pmatrix}\alpha_1 & \alpha_2 & \cdots & \alpha_n\end{pmatrix} = \begin{pmatrix} \beta_1 \\ \beta_2 \\ \vdots \\ \beta_n \end{pmatrix} \]
    则
    \begin{gather*}
        \begin{pmatrix}\alpha_1 & \alpha_2 & \cdots & \alpha_n\end{pmatrix}
        \begin{pmatrix}
            d_1 &     &        &     \\
                & d_2 &        &     \\
                &     & \ddots &     \\
                &     &        & d_n
        \end{pmatrix} = \begin{pmatrix}d_1\alpha_1 & d_2\alpha_2 & \cdots & d_n\alpha_n\end{pmatrix} \\
        \begin{pmatrix}
            d_1 &     &        &     \\
                & d_2 &        &     \\
                &     & \ddots &     \\
                &     &        & d_n
        \end{pmatrix} \begin{pmatrix} \beta_1 \\ \beta_2 \\ \vdots \\ \beta_n \end{pmatrix} = \begin{pmatrix} d_1\beta_1 \\ d_2\beta_2 \\ \vdots \\ d_n\beta_n \end{pmatrix}
    \end{gather*}

    即$A$右乘对角矩阵$\diag(d_1,d_2,\ldots,d_n)$相当于给$A$的第$i$列元素都乘以$d_i$,$A$左乘对角矩阵$\diag(d_1,d_2,\ldots,d_n)$相当于给$A$的第$i$行元素都乘以$d_i$,其中 $i=1,2,\ldots,n$.
\end{theorem}

\begin{theorem}
    对角矩阵和分块对角矩阵的性质:
    \begin{enumerate}
        \item 对角矩阵$\diag(d_1,d_2,\ldots,d_n)$可逆当且仅当对角线上元素均不为0,且此时逆矩阵为$\diag(d_1^{-1},d_2^{-1},\ldots,d_n^{-1})$.

        \item 分块对角矩阵$\diag(A_1,A_2,\ldots,A_n)$可逆当且仅当每个分块$A_i$可逆,且此时逆矩阵为$\diag(A_1^{-1},A_2^{-1},\ldots,A_n^{-1})$.

        \item 两个对角矩阵$A=\diag(a_1,a_2,\ldots,a_n),\enspace B=\diag(b_1,b_2,\ldots,b_n)$的乘积仍然是对角矩阵,且$AB=\diag(a_1b_1,a_2b_2,\ldots,a_nb_n)$.

              对于乘方运算,有$A^k=\diag(a_1^k,a_2^k,\ldots,a_n^k)$.

        \item 两个准对角矩阵$A=\diag(A_1,A_2,\ldots,A_n),\enspace B=\diag(B_1,B_2,\ldots,B_n)$中$A_i$和$B_i$是同级方阵,则乘积仍然是准对角矩阵,且$AB=\diag(A_1B_1,A_2B_2,\ldots,A_nB_n)$.
    \end{enumerate}
\end{theorem}

这里需要说明的是,本节定理都是通过简单计算即可验证的,因此在此不给出证明.

\subsection{上(下)三角矩阵}

\begin{theorem}
    已知$A,B$都是上三角矩阵,且设$A$的主对角元素分别为$a_{11},\ldots,a_{nn}$,$B$的主对角元素分别为$b_{11},\ldots,b_{nn}$,则
    \begin{enumerate}
        \item $A^{\mathrm{T}}, B^\mathrm{T}$都是下三角矩阵;

        \item $AB$仍然是上三角矩阵,且$AB$的主对角元素为$a_{11}b_{11},\ldots,a_{nn}b_{nn}$;

        \item $A$可逆的充要条件是其主对角元均不为0,且$A$可逆时,$A^{-1}$也是上三角矩阵,并且$A^{-1}$的主对角元素分别为$a_{11}^{-1},\ldots,a_{nn}^{-1}$.
    \end{enumerate}
\end{theorem}

\begin{example}
    已知$A_1,\ldots,A_n$是$n$个对角元都为0的上三角矩阵,证明:$A_1A_2\cdots A_n=O$.
\end{example}

\begin{proof}
    使用数学归纳法. $n=1$时结论显然成立,现在假设命题对$n-1$成立,即$n-1$个对角元都为0的$n-1$阶上三角矩阵的乘积为零矩阵,下面考虑$n$的情况:给定$A_1,A_2,\cdots,A_n$是$n$个对角元都为0的上三角矩阵,记
    \[A=\begin{pmatrix}
        B_i & * \\ O & 0
    \end{pmatrix},\enspace i=1,2,\cdots,n.\]
    其中$B_i$是对角元都为0的$n-1$阶上三角矩阵,由归纳假设可知
    \[B_1B_2\cdots B_{n-1}=O,\]
    于是
    \begin{align*}
        A_1A_2\cdots A_n&=\begin{pmatrix}
            B_1B_2\cdots B_{n-1} & * \\ O & 0
        \end{pmatrix}\begin{pmatrix}
            B_n & * \\ O & 0
        \end{pmatrix} \\
        &=\begin{pmatrix}
            O & * \\ O & 0
        \end{pmatrix}\begin{pmatrix}
            B_n & * \\ O & 0
        \end{pmatrix}=  O.
    \end{align*}
    证毕.
\end{proof}

\subsection{基本矩阵}

只有一个元素为1,其余元素全为0的矩阵称为基本矩阵,第$i$行第$j$列元素为1的基本矩阵记为$E_{ij}$,它们具有如下性质(可以联系左右乘对应行列变换进行记忆):
\begin{theorem}
    基本矩阵计算具有如下性质:
    \begin{enumerate}
        \item $AE_{ij}$的结果就是把$A$的第$i$列移到第$j$列的位置,其余元素都为0的矩阵;

        \item $E_{ij}B$的结果就是把$B$的第$j$行移到第$i$行的位置,其余元素都为0的矩阵;

        \item $E_{ik}E_{lj} = \begin{cases}
                      E_{ij} & k = l    \\
                      O      & k \neq l
                  \end{cases}$.
    \end{enumerate}
\end{theorem}

\subsection{其他矩阵}

其他特殊矩阵如正交矩阵、置换矩阵、幂等矩阵、幂零矩阵等,我们将在后续讲义合适的位置描述它们的性质,那时我们的讨论不局限于本节的运算性质,会有更多的其它性质.

\section{矩阵可交换问题}

首先我们需要强调一点:一般来说在本课程中此类问题直接设可交换矩阵的每一个元素都是未知数即可. 我们来看下面的例子:
\begin{example}\label{ex:12:可交换矩阵1}
    求所有与$A$可交换的矩阵,其中
    \[A=\begin{pmatrix}
            1 & 0 & 0  \\
            0 & 1 & 2  \\
            0 & 1 & -2
        \end{pmatrix}.\]
\end{example}

\begin{solution}
    设$B$为与$A$可交换的矩阵,设
    \[B=\begin{pmatrix}
            a & b & c \\
            d & e & f \\
            g & h & i
        \end{pmatrix},\]
    则
    \[AB=\begin{pmatrix}
            a & b & c \\
            d+2g & e+2h & f+2i \\
            d-2g & e-2h & f-2i
        \end{pmatrix}=BA=\begin{pmatrix}
            a & b+c & 2b-2c \\
            d & e+f & 2e-2f \\
            g & h+i & 2h-2i
        \end{pmatrix}.\]
    由此可得
    \[\begin{cases}
        b=b+c \\
        c=2b-2c \\
        d+2g=d \\
        e+2h=e+f \\
        f+2i=2e-2f \\
        d-2g=g \\
        e-2h=h+i \\
        f-2i=2h-2i
    \end{cases},\]
    很容易解得$b=c=d=g=0$,且$f=2h,\enspace e=3h+i$,因此
    \[B=\begin{pmatrix}
            a & 0 & 0 \\
            0 & 3h+i & 2h \\
            0 & h & i
        \end{pmatrix}.\]
\end{solution}

对于一些矩阵直接设未知数计算比较复杂,这里我们讨论一个基本的技巧,即利用
\[\forall t,\enspace AB=BA \iff (A-tE)B=B(A-tE).\]
这一等式成立是显然的. 运用时难点主要在决定$t$的值,我们要根据矩阵的对角线上元素来决定,原则是使得$B$与$A-tE$相乘的计算过程更为简单(一般是使得0元素更多),这样解方程也会更轻松. 我们看一个简单的例子来体会:
\begin{example}
    求与矩阵$A=\begin{pmatrix}
            3  & 0  & 0 \\
            -1 & 3  & 0 \\
            0  & -1 & 3
        \end{pmatrix}$可交换的矩阵.
\end{example}

\begin{solution}
    由前述分析,取$t=3$可以使得$A-3E$对角线上元素全为0,便于计算,因此我们只需求与$A-3E$可交换的矩阵即可. 与\autoref{ex:12:可交换矩阵1}同样的设未知数法,具体过程省略得到与$A$可交换的矩阵为
    \[B=\begin{pmatrix}
        a & 0 & 0 \\
        b & a & 0 \\
        c & b & a
    \end{pmatrix}.\]
\end{solution}

事实上,我们有如下关于可交换矩阵更一般的结论:
\begin{theorem}
    \begin{enumerate}
        \item 与主对角元两两互异的对角矩阵可交换的方阵只能是对角矩阵;

        \item 准对角矩阵$A$每个对角分块内对角线元素相同,但不同对角块之间不同,则与$A$可交换的矩阵只能是准对角矩阵;

        \item 与所有$n$级可逆矩阵可交换的矩阵为数量矩阵;

        \item 与所有$n$级矩阵可交换的矩阵为数量矩阵.
    \end{enumerate}
\end{theorem}

\begin{proof}
    \begin{enumerate}
        \item 设$B=(b_{ij})_{n\times n}$是与$A$可交换的矩阵,由\autoref{thm:12:对角矩阵的性质}可知,$AB$是$B$的第$i(i=1,2,\cdots,n)$行元素都乘以$\lambda_i$的矩阵,其中$\lambda_i$是$A$的第$i$个对角元素,同理$BA$是$B$的第$j$列元素都乘以$\lambda_j$的矩阵,因此考察$AB=BA$第$i$行$j$列元素可知
        \[\lambda_ib_{ij}=\lambda_jb_{ij}(i,j=1,2,\cdots,n),\]
        由于$i\neq j$时$\lambda_i\neq\lambda_j$,因此$b_{ij}=0$,即$B$是对角矩阵.

        \item 设$A=\diag{\lambda_1E_1,\lambda_2E_2,\cdots,\lambda_sE_s}$,其中$E_i$是$m_i$阶单位矩阵,$m_1+m_2+\cdots+m_k=n$,设$B=(b_{ij})_{n\times n}$是与$A$可交换的矩阵,将$B$做与$A$一样的分块,事实上由于分块矩阵乘法和一般乘法的相似性,我们可以完全套用第一点的证明完成这里的证明,这里不再赘述.

        \item 设$C$与所有$n$级可逆矩阵可交换,由前述1可知$C$至少是对角矩阵,因为$C$起码要与主对角元两两互异的对角矩阵可交换.

        我们将$C$记为$\diag{k_1,k_2,\cdots,k_n}$,进一步地,取可逆矩阵$B=E_{12}+E_{23}+\cdots+E_{n-1,n}+E_{n1}$(回顾第$i$行第$j$列元素为1的基本矩阵记为$E_{ij}$),则$BC=CB$可知
        \[k_1E_{12}+k_2E_{23}+\cdots+k_{n-1}E_{n-1,n}+k_nE_{n1}=k_2E_{12}+k_3E_{23}+\cdots+k_nE_{n-1,n}+k_1E_{n1},\]
        由此可得$k_1=k_2=\cdots=k_n$,即$C=k_1E$是数量矩阵.

        \item 设$C$与所有$n$级矩阵可交换,由前述3可知$C$至少是数量矩阵,因为$C$起码要与所有$n$级可逆矩阵可交换. 事实上我们也不难验证数量矩阵与任意矩阵可交换,因为$AkE=kEA=kA$,其中$k$是任意数,$E$是$n$级单位矩阵,$A$是任意$n$级矩阵. 因此与所有$n$级矩阵可交换的矩阵是数量矩阵.
    \end{enumerate}
\end{proof}

除此之外我们还有一些和特殊的矩阵可交换的结论,我们将在习题中见到它们. 因为技巧性过强正文中不展开叙述,感兴趣的同学可以参考习题C组进行了解.

\section{矩阵的逆进阶求法}

\subsection{给定多项式求逆矩阵}

此类题目出现最为频繁,实际上就是通过一些初中所学的因式分解等基本变换得到需要求逆的矩阵与另一个矩阵相乘可以得到单位矩阵(的一个倍数).
\begin{example}
    设$A$为非零矩阵,且$A^3=O$,证明:$E+A$和$E-A$都可逆.
\end{example}

\begin{proof}
    事实上$E=E+A^3=(E+A)(E-A+A^2)$,因此$E+A$可逆(因为可以写出$(E+A)^{-1}=E-A+A^2$),同理$E=E-A^3=(E-A)(E+A+A^2)$,因此$E-A$可逆(因为可以写出$(E-A)^{-1}=E+A-A^2$).
\end{proof}

\begin{example}
    若$X,Y$是两个列向量,且$X^\mathrm{T}Y=2$,证明:
    \begin{enumerate}
        \item $(XY^\mathrm{T})^k=2^{k-1}(XY^{\mathrm{T}})$;

        \item 如果$A=E+XY^\mathrm{T}$,则$A$可逆,并求其逆矩阵.
    \end{enumerate}
\end{example}

\begin{proof}
    \begin{enumerate}
        \item 事实上$Y^\mathrm{T}X=X^\mathrm{T}Y=2$,因此
              \[(XY^\mathrm{T})^2=X(Y^\mathrm{T}X)Y^\mathrm{T}=2XY^\mathrm{T},\]
              由数学归纳法易证$(XY^\mathrm{T})^k=2^{k-1}(XY^\mathrm{T})$.

        \item 事实上,
        \begin{align*}
            A^2&=(E+XY^\mathrm{T})^2=E+2XY^\mathrm{T}+(XY^\mathrm{T})^2 \\
            &=E+2XY^\mathrm{T}+2XY^\mathrm{T}=E+4XY^\mathrm{T} \\
            &=4E-3A,
        \end{align*}
        因此$A^2-4A+3E=O$,即$A(A-4E)=-3E$,因此$A$可逆,且$A^{-1}=\dfrac{1}{3}(4E-A)$.
    \end{enumerate}
\end{proof}

\subsection{利用分块矩阵初等变换}

在分块矩阵中,我们已经讲解了分块矩阵初等变换打洞法的基础题型,这里再给出一些更一般的例子:
\begin{example}\label{ex:12:打洞法求逆1}
    设$A,B$为$n$阶矩阵,证明:$E\pm AB$可逆$\iff E\pm BA$可逆.
\end{example}

\begin{proof}
    根据分块矩阵初等变换不改变矩阵的秩,我们有
    \[r\begin{pmatrix}
        E\pm BA & O \\ A & E
    \end{pmatrix}=r\begin{pmatrix}
        E & \mp B \\ A & E
    \end{pmatrix}=r\begin{pmatrix}
        E & \mp B \\ O & E\pm AB
    \end{pmatrix},\]
    故$E\pm AB$可逆$\iff E\pm BA$可逆(因为此时上式所有矩阵都可逆).
\end{proof}

\begin{example}
    设$A$为$n$阶矩阵,$B,C$分别为$n \times m$和$m \times n$阶矩阵. 证明:$E_m+CA^{-1}B$可逆$\iff A+BC$可逆.
\end{example}

\begin{proof}
    类似于上例,有
    \[r\begin{pmatrix}
        A+BC & B \\ O & E_m
    \end{pmatrix}=r\begin{pmatrix}
        A & B \\ -C & E_m
    \end{pmatrix}=r\begin{pmatrix}
        A & B \\ O & E_m+CA^{-1}B
    \end{pmatrix},\]
    故$E_m+CA^{-1}B$可逆$\iff A+BC$可逆.
\end{proof}

事实上,总结上述两题的解决方法,都是将待证明的一个矩阵构成分块矩阵的一部分,然后利用初等变换变出另一个矩阵,使得这两个矩阵的可逆性相同,从而得到结论.

\subsection{求逆的分式思想}

虽然矩阵没有除法运算,但是我们如果将$(E-A)^{-1}$写成$\dfrac{E}{E-A}$,再类比泰勒展开
\[\frac{1}{1-x}=1+\sum_{n=1}^\infty x^n \qquad x\in (-1,1)\]
我们可以得到(不严谨!只能用来解题的时候当作初步的思路!)
\[(E-A)^{-1}=\frac{E}{E-A}=E+A+A^2+\cdots\]

\begin{example}
    已知方阵$A$满足$A^k=O$,其中$k$是一个正整数,求$E-A$的逆.
\end{example}

\begin{solution}
    根据我们前面的分析,结合$A^k=O$,我们猜测
    \[(E-A)^{-1}=E+A+A^2+\cdots+A^{k-1}.\]
    事实上我们直接验证
    \[(E-A)(E+A+A^2+\cdots+A^{k-1})=E-A^k=E.\]
    因此$(E-A)^{-1}=E+A+A^2+\cdots+A^{k-1}$.
\end{solution}

\begin{example}
    设$A,B$分别是$n \times m$和$m \times n$的矩阵,且$E_n \pm AB$可逆,则$E_m \pm BA$可逆.
\end{example}
不难发现这一例是\autoref{ex:12:打洞法求逆1} 的推广,因为此处不再限制方阵.

\begin{proof}
    我们猜测
    \begin{align*}
        (E_m-BA)^{-1}&=E_m+(BA)+(BA)^2+\cdots \\
        &=E_m+B(E_n+AB+(AB)^2+\cdots)A \\
        &=E_m+B(E_n-AB)^{-1}A.
    \end{align*}
    事实上经过验证这一结论是正确的(具体过程省略),因此$E_m \pm BA$可逆,且$(E_m-BA)^{-1}=E_m+B(E_n-AB)^{-1}A$.
\end{proof}

\subsection{提逆思想}

这一思想的来源是矩阵逆没有加减相关的运算法则(即没有$(A+B)^{-1}=A^{-1}+B^{-1}$这样的性质),因此我们需要提逆产生一些乘积项来解决问题.
\begin{example}
    设$A$是$n$阶方阵,且$E-A$,$E+A$和$A$都可逆,证明:$(E-A^{-1})^{-1}+(E-A)^{-1}=E$.
\end{example}

\begin{proof}
    由于$(E-A^{-1})^{-1}=(A^{-1}(A-E))^{-1}=(A-E)^{-1}A$,因此
    \[(E-A^{-1})^{-1}+(E-A)^{-1}=(A-E)^{-1}A+(E-A)^{-1}=(E-A)^{-1}(E-A)=E.\]
\end{proof}

\section{矩阵的幂} \label{sec:12:矩阵的幂}

\begin{enumerate}
    \item 找规律

          在矩阵的转置\autoref{ex:9:转置求幂} 中我们已经见识了一种找规律的方式,下面是一种类似的题型:
          \begin{example}
              计算$(PAQ)^k$,其中
              \[P=\begin{pmatrix}2 & 3 \\ 1 & 2\end{pmatrix},\enspace A=\begin{pmatrix}2 & 0 \\ 0 & -1\end{pmatrix},\enspace Q=\begin{pmatrix}2 & -3 \\ -1 & 2\end{pmatrix}\]
          \end{example}
          本质而言此类题目只需要发现中间多次出现的乘积$QP$是很容易处理的矩阵(\autoref{ex:9:转置求幂} 中甚至是一个数)即可解决.

          \begin{solution}
            事实上,$QP=E$($E$是单位矩阵),令$B=PAQ$,则
            $B^2=PA(QP)AQ=PA^2Q$,利用归纳法可得,
            \[B^k=PA^kQ=\begin{pmatrix}
                2^{k+2}+3(-1)^{k+1} & -3\cdot 2^{k+1}+6\cdot (-1)^k \\
                2^{k+1}+2(-1)^{k+1} & -3\cdot 2^k+4\cdot (-1)^k
            \end{pmatrix}.\]
          \end{solution}

          \begin{example}
              设$A=\begin{pmatrix}0 & -1 & 0 \\ 1 & 0 & 0 \\ 0 & 0 & -1 \end{pmatrix},\enspace P^{-1}AP=B$,求$B^{2004}-2A^2$.
          \end{example}
          \begin{solution}
            事实上,$A^2=\begin{pmatrix}
                -1 & 0 & 0 \\ 0 & -1 & 0 \\ 0 & 0 & 1
            \end{pmatrix}$,因此$A^4=E$,于是$A^{2004}=(A^4)^{501}=E$,因此$B^{2004}-2A^2=E-2A^2=\begin{pmatrix}
                3 & 0 & 0 \\ 0 & 3  & 0 \\ 0 & 0 & -1
            \end{pmatrix}$.
          \end{solution}

          还有一种找规律基于幂等矩阵(满足$A^2=A$的矩阵,根据数学归纳法可证明$A$的任意次方都是$A$),显然幂等矩阵的任意次方都与其本身相等是很好的性质,另一种找规律基于对合矩阵,即平方等于单位矩阵的矩阵,我们这里主要与大家分享另一种关于幂零矩阵(矩阵某次幂可以得到零矩阵)的方法,例子如下:
          \begin{example}
              求$A=\begin{pmatrix}a & 1 & 0 & 0 \\ 0 & a & 1 & 0 \\ 0 & 0 & a & 0 \\ 0 & 0 & 0 & a \end{pmatrix}^n$.
          \end{example}
          在上例中,我们采用将矩阵分为$A=tE+B$的方法,会发现矩阵$B$为上三角矩阵且对角线上全为0,这是需要读者记忆的典型的幂零矩阵,未来在幂零矩阵的讨论中我们将严格证明这一点,现在我们只需利用这一性质快速解题.

          \begin{solution}
            设$B=\begin{pmatrix}
                0 & 1 & 0 & 0 \\ 0 & 0 & 1 & 0 \\ 0 & 0 & 0 & 1 \\ 0 & 0 & 0 & 0
            \end{pmatrix}$,故有
            \[B^2=\begin{pmatrix}
                0 & 0 & 1 & 0 \\ 0 & 0 & 0 & 1 \\ 0 & 0 & 0 & 0 \\ 0 & 0 & 0 & 0
            \end{pmatrix},\enspace B^3=\begin{pmatrix}
                0 & 0 & 0 & 1 \\ 0 & 0 & 0 & 0 \\ 0 & 0 & 0 & 0 \\ 0 & 0 & 0 & 0
            \end{pmatrix},\enspace B^4=O,\]
            因此
            \begin{align*}
                A^n&=(aE+B)^n=\sum\limits_{k=0}^nC_n^ka^{n-k}E^kB^k \\
                &=\begin{pmatrix}
                    a^n & C_n^1a^{n-1} & C_n^2a^{n-2} & C_n^3a^{n-3} \\ 0 & a^n & C_n^1a^{n-0} & C_n^2a^{n-2} \\ 0 & 0 & a^n & C_n^1a^{n-1} \\ 0 & 0 & 0 & a^n
                \end{pmatrix}.
            \end{align*}
          \end{solution}

    \item 数学归纳法
          \begin{example}
              求$A=\begin{pmatrix}\cos\alpha & \sin\alpha \\ -\sin\alpha & \cos\alpha\end{pmatrix}^n$.
          \end{example}
          这一问题对应我们常见的旋转变换(所以建议要求读者记忆这一矩阵形式),$n$次方就是旋转$n$次. 当然这是直观而言的结论,严谨说明可以通过数学归纳法证:

          \begin{solution}
            事实上,当$n=1$时结论显然成立,假设$n=k$时结论成立,即
            \[A^k=\begin{pmatrix}\cos k\alpha & \sin k\alpha \\ -\sin k\alpha & \cos k\alpha\end{pmatrix}.\]
            当$n=k+1$时,有
            \begin{align*}
                A^{k+1}&=A^kA=\begin{pmatrix}\cos k\alpha & \sin k\alpha \\ -\sin k\alpha & \cos k\alpha\end{pmatrix}\begin{pmatrix}\cos\alpha & \sin\alpha \\ -\sin\alpha & \cos\alpha\end{pmatrix} \\
                &=\begin{pmatrix}\cos k\alpha\cos\alpha-\sin k\alpha\sin\alpha & \cos k\alpha\sin\alpha+\sin k\alpha\cos\alpha \\ -\sin k\alpha\cos\alpha-\cos k\alpha\sin\alpha & -\sin k\alpha\sin\alpha+\cos k\alpha\cos\alpha\end{pmatrix} \\
                &=\begin{pmatrix}\cos(k+1)\alpha & \sin(k+1)\alpha \\ -\sin(k+1)\alpha & \cos(k+1)\alpha\end{pmatrix}.
            \end{align*}
            因此结论对于$n=k+1$也成立,由数学归纳法可知结论对于任意正整数$n$都成立.
          \end{solution}

          \begin{example}
              证明$\begin{pmatrix}
                      a & c \\ 0 & b
                  \end{pmatrix}^n=\begin{pmatrix}
                      a^n & (a^{n-1}+a^{n-2}b+\dots+b^{n-1})c \\ 0 & b^n
                  \end{pmatrix}$.
          \end{example}
          \begin{proof}
                事实上,当$n=1$时结论显然成立,假设$n=k$时结论成立,即
                \[\begin{pmatrix}
                        a & c \\ 0 & b
                    \end{pmatrix}^k=\begin{pmatrix}
                        a^k & (a^{k-1}+a^{k-2}b+\dots+b^{k-1})c \\ 0 & b^k
                    \end{pmatrix}.\]
                当$n=k+1$时,有
                \begin{align*}
                    \begin{pmatrix}
                        a & c \\ 0 & b
                    \end{pmatrix}^{k+1}&=\begin{pmatrix}
                        a & c \\ 0 & b
                    \end{pmatrix}^k\begin{pmatrix}
                        a & c \\ 0 & b
                    \end{pmatrix} \\
                    &=\begin{pmatrix}
                        a^k & (a^{k-1}+a^{k-2}b+\dots+b^{k-1})c \\ 0 & b^k
                    \end{pmatrix}\begin{pmatrix}
                        a & c \\ 0 & b
                    \end{pmatrix} \\
                    &=\begin{pmatrix}
                        a^{k+1} & (a^k+a^{k-1}b+\dots+b^k)c \\
                        0 & b^{k+1}
                    \end{pmatrix}.
                \end{align*}
                因此结论对于$n=k+1$也成立,由数学归纳法可知结论对于任意正整数$n$都成立.
          \end{proof}

    \item 利用秩为1的矩阵

          这一方法的核心是利用上一讲中\autoref{ex:11:相抵分解} 的结论,我们来看下面的例子进行体会:
          \begin{example}
              已知$M$是秩为 1 的矩阵,记$\tr(M)=b$,讨论$(aE+M)^n$的计算结果.
          \end{example}
          \begin{solution}
            事实上,由\autoref{ex:11:相抵分解}可知,$M^k=b^{k-1}M$,因此
            \begin{enumerate}
                \item 当$b=0$时,$M^k=O(k\geqslant 2)$,因此
                \[(aE+M)^n=a^nE+na^{n-1}M.\]
                \item 当$b\neq 0$时有
                \begin{align*}
                    (aE+M)^n&=\sum\limits_{k=0}^nC_n^ka^{n-k}M^k \\
                    &=\sum\limits_{k=0}^nC_n^ka^{n-k}b^{k-1}M \\
                    &=a^nE+\cfrac{1}{b}\sum\limits_{k=1}^nC_n^ka^{n-k}b^kM \\
                    &=a^nE+\cfrac{1}{b}(\sum\limits_{k=0}^nC_n^ka^{n-k}b^k-a^n)M \\
                    &=a^nE+\cfrac{(a+b)^n-a^n}{b}M \\
                \end{align*}
            \end{enumerate}
          \end{solution}

          \begin{example}
              已知$A$是数域$P$上的一个2阶方阵,且存在正整数$l$使得$A^l=O$,证明:$A^2=O$.
          \end{example}
          事实上,将来我们讨论幂零矩阵的时候将会进一步推广本例的结论.

          \begin{proof}
            \begin{enumerate}
                \item 若$l\leqslant 2$,则$A^2=O$显然成立;
                \item 若$l>2$,则由$A^l=O$可知$A$不可逆,故$r(A)\leqslant 1$,又$A\neq O$,因此$r(A)=1$,故$A^l=(\tr(A))^{l-1}A=O$,因此$\tr(A)=0$,故$A^2=\tr(A)A=O$.
            \end{enumerate}
          \end{proof}

    \item 利用初等矩阵的性质
          \begin{example}
              设$A$为三阶矩阵,$P$为三阶可逆矩阵,$P^{-1}AP=B$,其中
              \[P=\begin{pmatrix}
                      0 & 2 & -1 \\ 1 & 1 & 2 \\ -1 & -1 & -1
                  \end{pmatrix},\enspace B=\begin{pmatrix}
                      0 & 0 & -1 \\ 0 & -1 & 0 \\ -1 & 0 & 0
                  \end{pmatrix},\]
              求$A^{2024}$.
          \end{example}
          \begin{solution}
            事实上$A=PBP^{-1}$,因此$A^{2024}=PB^{2024}P^{-1}$,由于$B^2=E$,因此$B^{2024}=(B^2)^{1012}=E$,因此$A^{2024}=PEP^{-1}=E$.
          \end{solution}

          事实上本题一个关键的洞察在于我们很容易看出$B$这一非常简单的矩阵作为初等矩阵复合的情况(交换1、3行以及每一行都乘以-1),因此其平方为$E$.

    \item 利用对角化和若当标准形:我们将在后续相应章节中讲解.
\end{enumerate}

\vspace{2ex}
\centerline{\heiti \Large 内容总结}

\vspace{2ex}
\centerline{\heiti \Large 习题}

\vspace{2ex}
{\kaishu }
\begin{flushright}
    \kaishu

\end{flushright}

\centerline{\heiti A组}
\begin{enumerate}
    \item 设方阵$A$满足$A^2-A-2E=O$,证明:
          \begin{enumerate}
              \item $A$和$E-A$都是可逆矩阵,并求它们的逆矩阵;

              \item $A+E$和$A-2E$不可能同时可逆.
          \end{enumerate}

    \item 若$A,B$为两个$n$阶矩阵且满足$A+B=AB$,证明:
          \begin{enumerate}
              \item $A-E$和$B-E$均可逆;

              \item $AB=BA$;

              \item $r(A)=r(B)$.
          \end{enumerate}
\end{enumerate}

\centerline{\heiti B组}
\begin{enumerate}
    \item 设$f(x)=1+x+\cdots+x^{m-1}$,$g(x)=1-x$,$A=\begin{pmatrix}
        a & b \\ 0 & a
    \end{pmatrix}$,计算$f(A)g(A)$.

    \item 已知矩阵$A=\begin{pmatrix}
                  1 & 0 & 4 \\ 0 & 1 & 2 \\ 0 & 1 & 2
              \end{pmatrix}$,求证:所有与$A$可交换的矩阵构成$\mathbf{M}_3(\mathbf{R})$的一个子空间,并求子空间的一组基.

    \item 已知矩阵$A=\begin{pmatrix}
                  1 & 0 & 1 \\ 0 & 2 & 0 \\ 1 & 0 & 1
              \end{pmatrix}$,
          \begin{enumerate}
              \item 求所有与$A$可交换的矩阵;

              \item 若$AB+E=A^2+B$,求$B$.
          \end{enumerate}

    \item 设$A \in \mathbf{F}^{n \times n}$,令$C(A)=\{B \in \mathbf{F}^{n \times n} \mid AB=BA\}$.
          \begin{enumerate}
              \item 证明:$C(A)$为$\mathbf{F}^{n \times n}$的一个子空间;

              \item 求$C(E)$;

              \item 当$A$为对角线上元素互不相等的对角阵时,求$C(A)$的维数和一组基.
          \end{enumerate}

      \item 设$A$是$n$阶矩阵,$A^k=O$对某个正整数$k$成立,求证下列方阵可逆,并求它们的逆:
    \begin{enumerate}
        \item $E+A$;
        \item $E-A$;
        \item $E+A+\cfrac{1}{2!}A^2+\cdots+\cfrac{1}{(k-1)!}A^{k-1}$.
    \end{enumerate}
\end{enumerate}

\centerline{\heiti C组}
\begin{enumerate}
    \item 已知数列$\{a_n\},\enspace\{b_n\}$满足$a_0=-1,\enspace b_0=3$,且
    \[\begin{cases}
            a_n=3a_{n-1}+b_{n-1}+2^{n-1} \\
            b_n=2a_{n-1}+4b_{n-1}+2^n
        \end{cases}\]
    求$\{a_n\},\enspace\{b_n\}$的通项公式.

    \item 证明以下两个命题:
          \begin{enumerate}
              \item 与矩阵$I=\begin{pmatrix}
                            0 & 1 &   &        &   \\
                              &   & 1 &        &   \\
                              &   &   & \ddots &   \\
                              &   &   &        & 1 \\
                            1 &   &   &        & 0
                        \end{pmatrix}$可交换的矩阵$A$都可以写成$I$的一个多项式,即$A=a_{11}E+a_{12}I+a_{13}I^2+\cdots+a_{1n}I^{n-1}$;

              \item 与矩阵$J=\begin{pmatrix}
                            0 & 1 &   &        &   \\
                              &   & 1 &        &   \\
                              &   &   & \ddots &   \\
                              &   &   &        & 1 \\
                              &   &   &        & 0
                        \end{pmatrix}$可交换的矩阵$A$都可以写成$J$的一个多项式,即$A=a_{11}E+a_{12}J+a_{13}J^2+\cdots+a_{1n}J^{n-1}$.
          \end{enumerate}
\end{enumerate}

\section*{13 行列式(I)}
\addcontentsline{toc}{section}{13 行列式(I)}

\vspace{2ex}

\centerline{\heiti A组}
\begin{enumerate}
    \item 证明:\begin{enumerate}
        \item (线性性)直接将公理化定义用递归式对第i列展开:
        \begin{align*}
            & D(\alpha_1,\ldots,\lambda\alpha_{i}+\mu\beta_i,\ldots,\alpha_n)\\
            ={}& \sum_{k=1}^{n}(\lambda a_{ki}+\mu b_{ki})A_{ki}\\
            ={}& \lambda \cdot \sum_{k=1}^{n}a_{ki}A_{ki}+\mu \cdot \sum_{k=1}^{n}b_{ki}A_{ki}\\
            ={}& \lambda D(\alpha_1,\ldots,\alpha_{i},\ldots,\alpha_n)+\mu D(\alpha_1,\ldots,\beta_i,\ldots,\alpha_n)
        \end{align*}
        则线性性得证.
        \item (反对称性)使用数归方法证明:\\显然可得,$D(\alpha_1,\alpha_2)=-D(\alpha_2,\alpha_1)$,然后做出归纳假设:对于任意正整数$i,j$,$i,j \in [1,n-1]$且$i \neq j$,有:
        \[ D(\alpha_1,\ldots,\alpha_i,\ldots,\alpha_j,\ldots,\alpha_{n-1})=-D(\alpha_1,\ldots,\alpha_j,\ldots,\alpha_i,\ldots,\alpha_{n-1}) \]
        由此做出递推,对交换前后的行列式的首行做展开:
        \begin{align*}
            D(\alpha_1,\ldots,\alpha_{i},\ldots,\alpha_{j},\ldots,\alpha_n)
            & =\sum_{k=1}^{n}a_{1k}A_{1k}\\
            D(\alpha_1,\ldots,\alpha_{j},\ldots,\alpha_{i},\ldots,\alpha_n)
            & =\sum_{k=1}^{n}a'_{1k}A'_{1k}
        \end{align*}
        其中,除第$i,j$项外,由归纳假设,其余项都满足$a_{1k}=a'_{1k},A_{1k}=-A'_{1k}$,则有$a_{1k}A_{1k}=-a'_{1k}A'_{1k},k\neq i,j$.因此主要考察$a_{1i}A_{1i}+a_{1j}A_{1j}$与$a'_{1i}A'_{1i}+a'_{1j}A'_{1j}$这两项.首先有$a'_{1i}=a_{1j},a'_{1j}=a_{1i}$.然后将$A_{1i}$与$A'_{1j}$两项展开对比:
        \begin{align*}
            A_{1i}&=(-1)^{1+i}(\alpha'_{1},\ldots,\alpha'_{i-1},\alpha'_{i+1},\ldots,\alpha'_{j},\ldots,\alpha'_{n})\\
            A'_{1j}&=(-1)^{1+j}(\alpha'_{1},\ldots,\alpha'_{i-1},\alpha'_{j},\alpha'_{i+1},\ldots,\alpha'_{j-1},\alpha'_{j+1},\ldots,\alpha'_{n})
        \end{align*}
        式中的$\alpha'_k$表示原列向量去掉首行元素后剩余$n-1$个元素组成的新列向量.可以发现,$A'_{1j}$向左交换$j-(i+1)$次后与$A_{1i}$是绝对值一致的.则根据归纳假设,有$(-1)^{j-(i+1)}A'_{1j}=(-1)^{1+j-(1+i)}A_{1i}$,即有$A'_{1j}=-A_{1i}$,所以$a_{1i}A_{1i}+a_{1j}A_{1j}=-(a'_{1j}A'_{1j}+a'_{1i}A'_{1i})$.综上可证:
        \[ D(\alpha_1,\ldots,\alpha_{i},\ldots,\alpha_{j},\ldots,\alpha_n)=D(\alpha_1,\ldots,\alpha_{j},\ldots,\alpha_{i},\ldots,\alpha_n) \]
        \item (规范性)只需使用递归式定义,逐次展开即可.
    \end{enumerate}
    \item 使用倍加列变换的性质,可得$|\alpha_1+\alpha_2+\alpha_3,\alpha_1+3\alpha_2+9\alpha_3,\alpha_1+4\alpha_2+16\alpha_3|=6|\alpha_1,\alpha_2,\alpha_3|=12$
    \item \begin{enumerate}
        \item 首先由反对称矩阵的性质,有$A^T=-A$,其次由矩阵与行列式的性质,可得$|A^T|=|A|$,则可推出$|A|=|-A|=(-1)^n|A|$,又n为奇数,故$|A|=0$,矩阵$A$不可逆.
        \item 只需证明$AB$的秩小于$n$,即$|AB|=0$即可.$B$是奇数阶反对称矩阵,$|B|=0$,则$|AB|=|A||B|=0$得证.
    \end{enumerate}
    \item 根据伴随矩阵的性质,$\begin{pmatrix}
        A & O \\ O & B
    \end{pmatrix}^* = \begin{vmatrix}
        A & O \\ O & B
    \end{vmatrix} \cdot \begin{pmatrix}
        A & O \\ O & B
    \end{pmatrix}^{-1}$.其中,对$\begin{vmatrix}
        A & O \\ O & B
    \end{vmatrix}$做递归式展开,有$\begin{vmatrix}
        A & O \\ O & B
    \end{vmatrix}=\sum_{k=1}^{m}(-1)^{1+k}a_{1k}\begin{vmatrix}
        M_{1k} & O \\ O & B
    \end{vmatrix}$,依此逐次展开可得$\begin{vmatrix}
        A & O \\ O & B
    \end{vmatrix}=|A||B|=ab$.又$ab\begin{pmatrix}
        A & O \\ O & B
    \end{pmatrix}^{-1}=ab\begin{pmatrix}
        A^{-1} & O \\ O & B^{-1}
    \end{pmatrix}=\begin{pmatrix}
        b\cdot aA^{-1} & O \\ O & a\cdot bB^{-1}
    \end{pmatrix}$,最终可得$\begin{pmatrix}
        A & O \\ O & B
    \end{pmatrix}^* = \begin{pmatrix}
        bA^* & O \\ O & aB^*
    \end{pmatrix}$.

    与前一个式子的展开类似,$\begin{pmatrix}
        O & A \\ B & O
    \end{pmatrix}^*=\begin{vmatrix}
        O & A \\ B & O
    \end{vmatrix}\cdot \begin{pmatrix}
        O & A \\ B & O
    \end{pmatrix}^{-1}$,其中对前一步推导做少量修正后可得$\begin{vmatrix}
        O & A \\ B & O
    \end{vmatrix}=(-1)^{mn}|A||B|=(-1)^{mn}ab$,则$(-1)^{mn}ab\begin{pmatrix}
        O & A \\ B & O
    \end{pmatrix}^{-1}=(-1)^{mn}\begin{pmatrix}
        O & a\cdot bB^{-1} \\ b\cdot aA^{-1} & O
    \end{pmatrix}$,最终可得$\begin{pmatrix}
        O & A \\ B & O
    \end{pmatrix}^*=(-1)^{mn}\begin{pmatrix}
        O & aB^* \\ bA^* & O
    \end{pmatrix}$.
    \item 证明;\begin{enumerate}
        \item 对正整数$k$,有$(A^k)^*=(A^*)^k$.从而若$A^m=A$,则$(A^*)^m=(A^m)^*=A^*$.即$A$是幂等矩阵,则$A^*$也是幂等矩阵.同样,若$A^m=0$,则$(A^*)^m=(A^m)^*=0$.即$A$是幂零矩阵,则$A^*$也是幂零矩阵.
        \item $(A^*)^T=(A^T)^*=A^*$.从而若$A$是对称矩阵,则$A^*$也是对称矩阵.$(A^*)^T=(A^T)^*=(-A)^*=(-1)^{n-1}A^*$.从而若$A^*$为偶数阶时也为反对称矩阵,奇数阶时为对称矩阵.
    \end{enumerate}
    \item 对于上三角矩阵,只需看其伴随矩阵的下半部分元素,即$A_{ij},j>i$.对这些代数余子式,要么有一整行或整列为零,要么对角线存在零元素,则这些余子式均为零,伴随矩阵是上三角矩阵.\par 或者使用另法,当$A$是可逆矩阵,即$A$对角线元素不为零,则$A^*=|A|\cdot A^{-1}$,其中$A^{-1}$也是上三角矩阵,因此伴随矩阵$A^*$是上三角矩阵.
    \item $|A|=0$,则$r(A)<n$,故$r(A^*)=\begin{cases}
        1,&r(A)=n-1\\
        0,&r(A)<n-1
    \end{cases}$.当$r(A^*)=0$时,答案显然成立;当$r(A^*)=1$时,由矩阵秩的定义,任意两行(列)必成比例,否则秩大于1,矛盾.综上,原题得证.
    \item $(\alpha_1-\alpha_2-2\alpha_3,2\alpha_1+\alpha_2-\alpha_3,3\alpha_1+\alpha_2+2\alpha_3)=(\alpha_1,\alpha_2,\alpha_3)\begin{pmatrix}
        1 & 2 & 3 \\ -1 & 1 & 1 \\ -2 & -1 & 2
    \end{pmatrix}$,又$\begin{vmatrix}
        1 & 2 & 3 \\ -1 & 1 & 1 \\ -2 & -1 & 2
    \end{vmatrix}=12\neq 0$,因此该矩阵可逆,则$(\alpha_1-\alpha_2-2\alpha_3,2\alpha_1+\alpha_2-\alpha_3,3\alpha_1+\alpha_2+2\alpha_3)$与$(\alpha_1,\alpha_2,\alpha_3)$秩相等,因此这三个向量线性无关.
    \item 只要取$\alpha_3,\alpha_4 \in \mathbf{R}^4$,使得$D(\alpha_1,\alpha_2,\alpha_3,\alpha_4) \neq 0$即可.易得$\begin{vmatrix}
        1 & 1 & 0 & 0 \\
        2 & 4 & 0 & 0 \\
        1 & -1 & 1 & 0 \\
        -1 & 1 & 0 & 1
    \end{vmatrix}=\begin{vmatrix}
        1 & 1 \\
        2 & 4
    \end{vmatrix} \cdot \begin{vmatrix}
        1 & 0 \\
        0 & 1
    \end{vmatrix}=2 \neq 0$.于是可取$\alpha_3=(0,0,1,0)^T,\alpha_4=(0,0,0,1)^T$,能使得${\alpha_1,\alpha_2,\alpha_3,\alpha_4}$为$\mathbf{R}^4$的一组基.
\end{enumerate}

\centerline{\heiti B组}
\begin{enumerate}
    \item \begin{enumerate}
        \item 这三小问实际上都是对原行列式中的某一行进行了替换进行计算。
        \begin{align*}
            A_{21}+A_{22}+A_{23}+A_{24} ={} & \begin{vmatrix}
                3 & 0 & 4 & 1 \\
                1 & 1 & 1 & 1 \\
                0 & -7 & 8 & 3 \\
                5 & 3 & -2 & 2
            \end{vmatrix} = \begin{vmatrix}
                3 & -3 & 1 & -2 \\
                1 & 0 & 0 & 0 \\
                0 & -7 & 8 & 3 \\
                5 & -2 & -7 & -3
            \end{vmatrix} \\
            ={} & (-1)^{2+1} \begin{vmatrix}
                -3 & 1 & -1 \\
                -7 & 8 & 3 \\
                -2 & -7 & -3
            \end{vmatrix} = 148
        \end{align*}
        \item \begin{align*}
            A_{31}+A_{33} ={} & 1A_{31}+0A_{32}+1A_{33}+0A_{34} = \begin{vmatrix}
                3 & 0 & 4 & 1 \\
                2 & 3 & 1 & 4 \\
                1 & 0 & 1 & 0 \\
                5 & 3 & -2 & 2
            \end{vmatrix} = 3 \begin{vmatrix}
                3 & 0 & 4 & 1 \\
                2 & 1 & 1 & 4 \\
                1 & 0 & 1 & 0 \\
                5 & 1 & -2 & 2
            \end{vmatrix} \\
            ={} & 3 \begin{vmatrix}
                3 & 0 & 1 & 1 \\
                2 & 1 & -1 & 4 \\
                1 & 0 & 0 & 0 \\
                5 & 3 & -7 & 2
            \end{vmatrix} = 3 \cdot (-1)^{3+1} \begin{vmatrix}
                0 & 1 & 1 \\
                1 & -1 & 4 \\
                3 & -7 & 2
            \end{vmatrix} = -12
        \end{align*}
        \item \begin{align*}
            M_{41}+M_{42}+M_{43}+M_{44} ={} & -A_{41}+A_{42}-A_{43}+A_{44} = \begin{vmatrix}
                3 & 0 & 4 & 1 \\
                2 & 3 & 1 & 4 \\
                0 & -7 & 8 & 3 \\
                -1 & 1 & -1 & 1
            \end{vmatrix} \\ ={} & \begin{vmatrix}
                3 & 3 & 1 & 4 \\
                2 & 5 & -1 & 6 \\
                0 & -7 & 8 & 3 \\
                -1 & 0 & 0 & 0
            \end{vmatrix} = {-1}^{4+1} \cdot (-1) \begin{vmatrix}
                3 & 1 & 4 \\
                5 & -1 & 6 \\
                -7 & 8 & 3
            \end{vmatrix} = -78
        \end{align*}
    \end{enumerate}
    \item 有问题,之后可能会考虑修改题目,此处略.
    \item \begin{align*}
        \lvert A+B^{-1} \rvert ={} & \lvert B^{-1}BA+B^{-1}E \rvert = \lvert B^{-1} \rvert \cdot \lvert BA+E \rvert \\ ={} & \frac{1}{2} \lvert BA+A^{-1}A \rvert = \frac{1}{2} \lvert B+A^{-1} \rvert \cdot \lvert A \rvert \\ ={} & \frac{3}{2} \lvert A^{-1}+B \rvert = 3
    \end{align*}
    \item 正交矩阵满足 $AA^{\mathrm{T}} = A^{\mathrm{T}}A = E$,所以 $\lvert AA^{T} \rvert = \lvert A \rvert^2 = \lvert E \rvert = 1$. 而 $\lvert A \rvert < 0$,所以 $\lvert A \rvert = -1$.
    \[\lvert E+A \rvert = \lvert AA^{\mathrm{T}}+A \rvert = \lvert A \rvert \cdot \lvert A^{\mathrm{T}}+E \rvert = -\lvert (A+E)^{\mathrm{T}} \rvert = -\lvert E+A \rvert.\]
    故 $\lvert E+A \rvert = 0$.

    以上两道题都多次运用了 $E = AA^{-1} = A^{-1}A$ 的技巧,请大家留意.

    \item \begin{enumerate}
        \item 由于行列式 \[ D = \begin{vmatrix}
            a_{11} & a_{12} & a_{13} & \cdots & a_{1n} \\
            a_{11} & a_{12} & a_{13} & \cdots & a_{1n} \\
            a_{21} & a_{22} & a_{23} & \cdots & a_{2n} \\
            \vdots & \vdots & \vdots &        & \vdots \\
            a_{n-1, 1} & a_{n-1, 2} & a_{n-1, 3} & \cdots & a_{n-1, n}
        \end{vmatrix} = 0\]
        而 $M_1, -M_2, \ldots, (-1)^{n-1}M_n$ 恰是 $D$ 的第一行元素的代数余子式,所以将 $D$ 按第一行展开,可知\[a_{11}M_1+a_{12}(-M_2)+\cdots+a_{1n}(-1)^{n-1}M_n=0.\]
        而 $D$ 的其他行元素与第一行元素的代数余子式乘积之和为 0,于是结论成立.
        \item 因为 $r(A) = n-1$,所以 $M_1, -M_2, \ldots, (-1)^{n-1}M_n$,不全为 0. 且该方程组解空间维数为 1,$M_1, -M_2, \ldots, (-1)^{n-1}M_n$ 正是该方程组的非零解,结论成立.
    \end{enumerate}

    \item $A^* = \lvert A \rvert A^{-1} = 2A^{-1}, B^* = \lvert B \rvert B^{-1} = B^{-1}$,故 \[\lvert 2A^*B^*-A^{-1}B^{-1} \rvert = \lvert 4A^{-1}B^{-1}-A^{-1}B^{-1} \rvert = \lvert 3A^{-1}B^{-1} \rvert = \dfrac{3^n}{\lvert A \rvert \lvert B \rvert} = \dfrac{3^n}{2}.\]
    \item \begin{enumerate}
        \item 设 $A = (a_{ij})$,则 $A^* = (A_{ji})$($A^*$ 的表达式). 而 $A^T = A^*$,有 $a_{ij} = A_{ij}, \forall i, j = 1, 2, \ldots, n$. 而 $A$ 非零,故 $\exists a_{kl} \neq 0$,从而有 \[ \lvert A \rvert = a_{k1}A_{k1}+\cdots+a_{kn}A_{kn} = a_{k1}^2+\cdots+a_{kn}^2 > 0.\]
        \item $\lvert A \rvert > 0$ 有 $A$ 是可逆的,故 $A^* = \lvert A \rvert A^{-1} = A^{\mathrm{T}}$,从而 $A^{\mathrm{T}}A = \lvert A \rvert E$. 两侧取行列式有 $\lvert A \rvert^2 = \lvert A \rvert^n $,结合 $\lvert A \rvert > 0$ 有 $\lvert A \rvert = 1$.
        \item $\lvert A \rvert = 1$,故 $A^{-1} = A^{\mathrm{T}}$,$A$ 是正交矩阵.
        \item \[\lvert E-A \rvert = \lvert AA^{\mathrm{T}}-AE \rvert = \lvert A \rvert \lvert A^{\mathrm{T}}-E\rvert = \lvert A^{\mathrm{T}}-E\rvert = \lvert (A-E)^{\mathrm{T}} \rvert = \lvert A-E \rvert = (-1)^n\lvert E-A \rvert.\]
        $n$ 为奇数,则 $\lvert E-A \rvert = -\lvert E-A \rvert$,即 $\lvert E-A \rvert = 0$.
    \end{enumerate}
    \item $r(A) = n-1$,则 $\lvert A \rvert = 0$ 且 $AX = 0$ 的解空间的维数为 $1$. 而考虑 $AA^* = \lvert A \rvert E = 0$,且 $\exists A_{ij} \neq 0$,所以 $(A_{i1}, A_{i2}, \ldots , A_{in})^{\mathrm{T}}$ 是所求的基础解系.
    \item 设 \[A = \begin{pmatrix}
        a_{11} & a_{12} & \cdots & a_{1n} \\
        a_{21} & a_{22} & \cdots & a_{2n} \\
        \vdots & \vdots &        & \vdots \\
        a_{n1} & a_{n2} & \cdots & a_{nn} \\
    \end{pmatrix}\]
    则 \[A^* = \begin{pmatrix}
        A_{11} & A_{21} & \cdots & A_{n-1, 1} & A_{n1} \\
        A_{12} & A_{22} & \cdots & A_{n-1, 2} & A_{n2} \\
        \vdots & \vdots &        & \vdots     & \vdots \\
        A_{1, n-1} & A_{2, n-1} & \cdots & A_{n-1, n-1} & A_{n, n-1} \\
        A_{1n} & A_{2n} & \cdots & A_{n-1, n} & A_{nn} \\
    \end{pmatrix}\]
    注意到目标行列式是 $A^*$ 中元素 $A_{nn}$ 的代数余子式,也就是 $(A^*)^*$ 中 $(n, n)$ 位置的元素. 由 \ref*{例13.9(4)} $(A^*)^* = \lvert A \rvert^{n-2}A$ 可知结论成立.
    \item 设 $f(x) = c_0+c_1x+c_2x^2+\cdots+c_{n-1}x^{n-1}$,则有 \[\begin{cases}
        c_0+c_1a_1+c_2a_1^2+\cdots+c_{n-1}a_1^{n-1} & = b_1, \\
        c_0+c_1a_2+c_2a_2^2+\cdots+c_{n-1}a_2^{n-1} & = b_2, \\
                                                    & \vdots  \\ %格式修一下
        c_0+c_1a_n+c_2a_n^2+\cdots+c_{n-1}a_n^{n-1} & = b_n. \\
    \end{cases}\] 注意此处我们研究的对象是 $(c_0, c_1, \ldots, c_{n-1})^{\mathrm{T}}$. 因为系数行列式 \[D = \begin{vmatrix}
        1 & a_1 & \cdots & a_1^{n-1} \\
        1 & a_2 & \cdots & a_2^{n-1} \\
        \vdots & \vdots & & \vdots \\
        1 & a_n & \cdots & a_n^{n-1} \\
    \end{vmatrix} = \prod_{1 \leqslant j < i \leqslant n} (a_i-a_j) \neq 0.\] 所以由 Cramer 法则,上述关于 $(c_0, c_1, \ldots, c_{n-1})^{\mathrm{T}}$ 的方程组有唯一解,所以满足条件的多项式函数 $f$ 是唯一存在的.
    \item 令 $A = (\alpha_1, \alpha_2, \ldots, \alpha_n), B = \begin{pmatrix}
        \alpha_1^{\mathrm{T}}\alpha_1 & \alpha_1^{\mathrm{T}}\alpha_2 & \cdots & \alpha_1^{\mathrm{T}}\alpha_n \\
        \alpha_2^{\mathrm{T}}\alpha_1 & \alpha_2^{\mathrm{T}}\alpha_2 & \cdots & \alpha_2^{\mathrm{T}}\alpha_n \\
        \vdots & \vdots & & \vdots \\
        \alpha_n^{\mathrm{T}}\alpha_1 & \alpha_n^{\mathrm{T}}\alpha_2 & \cdots & \alpha_n^{\mathrm{T}}\alpha_n \\
    \end{pmatrix}$,显然 $B = A^{\mathrm{T}}A$,由 \ref*{11.4 秩不等式第 4 个} 有 $r(B) = r(A)$. 进而 $n$ 维向量组 $(\alpha_1, \alpha_2, \ldots, \alpha_n)$ 线性无关等价于 $r(A) = n$,也就等价于 $r(B) = n$,进而等价于 $\lvert B \rvert \neq 0$,命题得证.
    \item $(\Rightarrow)$ 记 $A = (\alpha_1, \alpha_2, \ldots, \alpha_n), \varepsilon _i = (0, \ldots, 1, 0, \ldots, 0)^{\mathrm{T}}$(第 $i$ 个为 1),$E = (\varepsilon _1, \varepsilon _2, \ldots, \varepsilon _n)$. 因为 $\alpha_1, \alpha_2, \ldots, \alpha_n$ 是 $n$ 维线性无关的向量组,故 $\lvert A \rvert \neq 0$,即 $A$ 可逆,$A = EA, AA^{-1} = E$,\[(\alpha_1, \alpha_2, \ldots, \alpha_n) = (\varepsilon _1, \varepsilon _2, \ldots, \varepsilon _n)A, (\varepsilon _1, \varepsilon _2, \ldots, \varepsilon _n) = (\alpha_1, \alpha_2, \ldots, \alpha_n)A^{-1}.\] 即 $\alpha_1, \alpha_2, \ldots, \alpha_n$ 与 $\varepsilon _1, \varepsilon _2, \ldots, \varepsilon _n$ 等价. $\varepsilon _1, \varepsilon _2, \ldots, \varepsilon _n$ 为 $\mathbf{R}^n$ 的一个基,能表出任一 $n$ 维向量,故 $\alpha_1, \alpha_2, \ldots, \alpha_n$ 能表出 $\mathbf{R}^n$ 中任一 $n$ 维向量.  \\
    $(\Leftarrow)$ 若 $\alpha_1, \alpha_2, \ldots, \alpha_n$ 能表出任一 $n$ 维向量,则 $\varepsilon_i = \lambda_{i1}\alpha_1+\cdots+\lambda_{in}\alpha_n, i = 1, 2, \ldots, n$. 即 \[E = (\varepsilon _1, \varepsilon _2, \ldots, \varepsilon _n) = (\alpha_1, \alpha_2, \ldots, \alpha_n)\begin{pmatrix}
        \lambda_{11} & \lambda_{21} & \cdots & \lambda_{n1} \\
        \lambda_{11} & \lambda_{21} & \cdots & \lambda_{n1} \\
        \vdots       & \vdots       &        & \vdots       \\
        \lambda_{11} & \lambda_{21} & \cdots & \lambda_{n1} \\
    \end{pmatrix}.\] 进而 $\lvert \alpha_1, \alpha_2, \ldots, \alpha_n \rvert \begin{vmatrix}
        \lambda_{11} & \lambda_{21} & \cdots & \lambda_{n1} \\
        \lambda_{11} & \lambda_{21} & \cdots & \lambda_{n1} \\
        \vdots       & \vdots       &        & \vdots       \\
        \lambda_{11} & \lambda_{21} & \cdots & \lambda_{n1} \\
    \end{vmatrix} = 1$,所以 $\lvert \alpha_1, \alpha_2, \ldots, \alpha_n \rvert \neq 0$,$\alpha_1, \alpha_2, \ldots, \alpha_n$ 线性无关.
    \item $A$ 的前 $s$ 列组成的 $s$ 阶子式为范德蒙行列式 \[D = \begin{vmatrix}
        1 & a & a^2 & \cdots & a^{n-1} \\
        1 & a^2 & a^4 & \cdots & a^{2(n-1)} \\
        \vdots & \vdots & \vdots & \ddots & \vdots \\
        1 & a^s & a^{2s} & \cdots & a^{s(n-1)} \\
    \end{vmatrix}.\] 由于当 $0 < r < n$ 时,$a^r \neq 1$,因此 $a, a^2, \ldots, a^s$ 两两不同,进而 $D \neq 0$,于是 $r(A) \geqslant s$. 又因为 $A$ 的行数是 $s$,所以 $r(A) \leqslant s$. 从而 $r(A) = s$.
    \item \begin{enumerate}
        \item 线性变换的验证省略.
        \item $(\Rightarrow)$ 若 $\lvert AB \rvert = 0$,则 $\lvert A \rvert = 0$ 或 $\lvert B \rvert = 0$,故 $A$ 或 $B$ 不可逆. 不妨假设 $A$ 不可逆,则存在 $X_0 \neq 0$ 使得 $AX_0 = 0$,$T(X_0) = AX_0B = 0$. 但 $T$ 是可逆的,所以 $T$ 是单射,$T(X) = 0 \Leftrightarrow X = 0$,矛盾. \\
        $(\Leftarrow)$ $\lvert AB \rvert \neq 0$ 则 $A, B$ 可逆,故 $T(X) = AXB$ 的逆映射为 $T^{-1}(X) = A^{-1}XB^{-1}$.
    \end{enumerate}
    \item 因为 $r(A) < n, A_{11} \neq 0$,所以 $r(A) = n-1$,进而由 \ref*{例13.9(6)} 可知 $r(A^*) = 1$,所以有 \[A^* = \begin{pmatrix}
        a_1 \\
        a_2 \\
        \vdots \\
        a_n
    \end{pmatrix} (\lambda_1, \lambda_2, \ldots, \lambda_n).\] 设 $(\lambda_1, \lambda_2, \ldots, \lambda_n) \begin{pmatrix}
        a_1 \\
        a_2 \\
        \vdots \\
        a_n
    \end{pmatrix} = k$,则 \begin{align*}
        (A^*)^2 ={} & \begin{pmatrix}
            a_1 \\
            a_2 \\
            \vdots \\
            a_n
        \end{pmatrix} (\lambda_1, \lambda_2, \ldots, \lambda_n)\begin{pmatrix}
            a_1 \\
            a_2 \\
            \vdots \\
            a_n
        \end{pmatrix} (\lambda_1, \lambda_2, \ldots, \lambda_n) \\ ={} & \begin{pmatrix}
            a_1 \\
            a_2 \\
            \vdots \\
            a_n
        \end{pmatrix} \cdot k \cdot (\lambda_1, \lambda_2, \ldots, \lambda_n) = k \begin{pmatrix}
            a_1 \\
            a_2 \\
            \vdots \\
            a_n
        \end{pmatrix} (\lambda_1, \lambda_2, \ldots, \lambda_n) = kA^*
    \end{align*}
    \item $\forall a_i \in \mathbf{R}, i \in \mathbf{Z}_{+}$ 满足 若 $i \neq j$,则 $a_i \neq a_j$,考虑以范德蒙行列式的形式进行排布. 设 \[ A = \begin{pmatrix}
        1 & 1 & \cdots & 1 & \cdots \\
        a_1 & a_2 & \cdots & a_k & \cdots \\
        a_1^2 & a_2^2 & \cdots & a_k^2 & \cdots \\
        \vdots & \vdots &      & \vdots &       \\
        a_1^{n-1} & a_2^{n-1} & \cdots & a_k^{n-1} & \cdots
    \end{pmatrix}\] 考虑 $A$ 的任意 $n$ 阶主子式 $D_n$,其均构成范德蒙行列式,又 $i \neq j$ 有 $a_i \neq a_j$,所以值均不为 0,也就是说任意 $n$ 个向量都线性无关,其个数也恰好为 $n$,构成 $V$ 的一组基,命题得证.
\end{enumerate}

\centerline{\heiti C组}
\begin{enumerate}
    \item 对原矩阵进行分块初等变换化为上三角块矩阵后进行计算. 因为 $\lvert A \rvert \neq 0$,所以 $A$ 可逆,进而有如下变换:\[\begin{pmatrix}
        E & O \\
        -CA^{-1} & \\ E
    \end{pmatrix} \begin{pmatrix}
        A & B \\
        C & D \\
    \end{pmatrix} = \begin{pmatrix}
        A & B \\
        O & D-CA^{-1}B
    \end{pmatrix},\] 所以 \begin{align*}
        & \begin{vmatrix}
            A & B \\
            C & D \\
        \end{vmatrix} = \begin{vmatrix}
            E & O \\
            -CA^{-1} & E \\
        \end{vmatrix} \begin{vmatrix}
            A & B \\
            C & D \\
        \end{vmatrix} \\ ={} & \begin{vmatrix}
            A & B \\
            O & D-CA^{-1}B
        \end{vmatrix} = \lvert A \rvert \lvert D-CA^{-1}B \rvert = \lvert AD-ACA^{-1}B \rvert.
    \end{align*} 由于 $AC = CA$,所以有 $ACA^{-1} = CAA^{-1} = C$,所以 \[\begin{vmatrix}
        A & B \\
        C & D \\
    \end{vmatrix} = \lvert AD-CB \rvert.\]
    \item 这道题目我们利用了一个分块矩阵作为中间“桥梁”使得其通过分块初等变换之后能分别得到两个方向上的结果. 考虑矩阵 $\begin{pmatrix}
        A & \alpha \\
        -\beta^{\mathrm{T}} & 1
    \end{pmatrix}$,有 \[\begin{pmatrix}
        A & \alpha \\
        -\beta^{\mathrm{T}} & 1
    \end{pmatrix} \begin{pmatrix}
        E & O \\
        \beta^{\mathrm{T}} & E
    \end{pmatrix} = \begin{pmatrix}
        A+\alpha \beta^{\mathrm{T}} & \alpha \\
        O & 1
    \end{pmatrix},\] 所以 \[\begin{vmatrix}
        A & \alpha \\
        -\beta^{\mathrm{T}} & 1
    \end{vmatrix} = \begin{vmatrix}
        A & \alpha \\
        -\beta^{\mathrm{T}} & 1
    \end{vmatrix} \begin{vmatrix}
        E & O \\
        \beta^{\mathrm{T}} & E
    \end{vmatrix} = \begin{vmatrix}
        A+\alpha \beta^{\mathrm{T}} & \alpha \\
        O & 1
    \end{vmatrix} = \lvert A+\alpha \beta^{\mathrm{T}} \rvert.\] 另一方面,\[\begin{pmatrix}
        E & O \\
        \beta^{\mathrm{T}}A^{-1} & 1
    \end{pmatrix} \begin{pmatrix}
        A & \alpha \\
        -\beta^{\mathrm{T}} & 1
    \end{pmatrix} = \begin{pmatrix}
        A & \alpha \\
        O & 1+\beta^{\mathrm{T}}A^{-1}\alpha
    \end{pmatrix}.\] 注意 $\beta^{\mathrm{T}}A^{-1}\alpha$ 的最终结果是一个数. 进而 \[\begin{vmatrix}
        A & \alpha \\
        -\beta^{\mathrm{T}} & 1
    \end{vmatrix} = \begin{vmatrix}
        E & O \\
        \beta^{\mathrm{T}}A^{-1} & 1
    \end{vmatrix} \begin{vmatrix}
        A & \alpha \\
        -\beta^{\mathrm{T}} & 1
    \end{vmatrix} = \begin{vmatrix}
        A & \alpha \\
        O & 1+\beta^{\mathrm{T}}A^{-1}\alpha
    \end{vmatrix} = \lvert A \rvert(1+\beta^{\mathrm{T}}A^{-1}\alpha).\] 所以有 \[\lvert A+\alpha \beta^{\mathrm{T}} \rvert = \lvert A \rvert(1+\beta^{\mathrm{T}}A^{-1}\alpha)\]
    \item 依旧是对分块矩阵做初等分块变换. 考虑到 \[\left(\begin{pmatrix}
        E & E \\
        O & E
    \end{pmatrix} \begin{pmatrix}
        A & B \\
        B & A
    \end{pmatrix} \right)\begin{pmatrix}
        E & -E \\
        O & E
    \end{pmatrix} = \begin{pmatrix}
        A+B & A+B \\
        B & A
    \end{pmatrix} \begin{pmatrix}
        E & -E \\
        O & E
    \end{pmatrix} = \begin{pmatrix}
        A+B & O \\
        B & A-B
    \end{pmatrix},\] 所以有 \begin{align*}
        \begin{vmatrix}
            A & B \\
            B & A
        \end{vmatrix} ={} & \begin{vmatrix}
            E & E \\
            O & E
        \end{vmatrix} \begin{vmatrix}
            A & B \\
            B & A
        \end{vmatrix} \begin{vmatrix}
            E & -E \\
            O & E
        \end{vmatrix} = \begin{vmatrix}
            A+B & O \\
            B & A-B
        \end{vmatrix} \\ ={} & \lvert (A+B)(A-B) \rvert = \lvert A+B \rvert \lvert A-B \rvert
    \end{align*}
    \item 首先有 $\begin{vmatrix}
        \lvert A \rvert & \lvert B \rvert \\
        \lvert C \rvert & \lvert D \rvert
    \end{vmatrix} = \lvert A \rvert \lvert D \rvert-\lvert B \rvert \lvert C \rvert$.
    \begin{enumerate}
        \item 若 $\lvert A \rvert \neq 0$,即 $A$ 可逆. 因为矩阵的初等变换不改变矩阵的秩,所以由 \[\begin{pmatrix}
            E & O \\
            -CA^{-1} & E
        \end{pmatrix} \begin{pmatrix}
            A & B \\
            C & D
        \end{pmatrix} \begin{pmatrix}
            E & -A^{-1}B \\
            O & E \\
        \end{pmatrix} = \begin{pmatrix}
            A & O \\
            O & D-CA^{-1}B
        \end{pmatrix},\] 条件 $r\left(\begin{pmatrix}
            A & B \\
            C & D
        \end{pmatrix}\right) = n$ 以及 $A$ 可逆,可以得到 \[D-CA^{-1}B = O.\] 即若 $A$ 可逆,则 $D = CA^{-1}B$,并且 \[\begin{vmatrix}
            \lvert A \rvert & \lvert B \rvert \\
            \lvert C \rvert & \lvert D \rvert
        \end{vmatrix} = \lvert A \rvert \lvert CA^{-1}B \rvert-\lvert B \rvert \lvert C \rvert = \lvert A \rvert \lvert C \rvert \lvert A^{-1} \rvert \lvert B \rvert-\lvert B \rvert \lvert C \rvert = 0.\]
        \item 若 $\lvert A \rvert = 0$,只需证 $\lvert B \rvert \lvert C \rvert = 0$. 若 $\lvert B \rvert \neq 0$,则由 \[\begin{pmatrix}
            E & O \\
            -DB^{-1} & E
        \end{pmatrix} \begin{pmatrix}
            A & B \\
            C & D
        \end{pmatrix} \begin{pmatrix}
            E & O \\
            -B^{-1}A & E
        \end{pmatrix} = \begin{pmatrix}
            O & B \\
            C-DB^{-1}A & O
        \end{pmatrix},\] 有 $C-DB^{-1}A = O$. 注意到 $\lvert A \rvert = 0$,故 \[\lvert C \rvert = \lvert DB^{-1}A \rvert = \lvert D \rvert \lvert B^{-1} \rvert \lvert A \rvert = 0.\] 同理可证若 $\lvert C \rvert \neq 0$,则 $\lvert B \rvert = 0$.
    \end{enumerate}
    综上,结论成立.
    \item \begin{enumerate}
        \item 采用反证法. 设 $\lvert A \rvert = 0$,则线性方程组 $AX = 0$ 有非零解,设为 $X_0 = (x_1, x_2, \ldots, x_n)^{\mathrm{T}}$,记 \[\lvert x_k \rvert = \mathrm{max} \{\lvert x_1 \rvert, \lvert x_2 \rvert, \ldots, \lvert x_n \rvert\}.\] 由 $X_0 \neq 0$ 可知 $\lvert x_k \rvert > 0$,考虑 $AX = 0$ 的第 $k$ 个方程,有 $\sum_{j=1}^n a_{kj}x_j = 0$,于是 \[\lvert a_{kk} \rvert \lvert x_k \rvert = \lvert -\sum_{j \neq k}a_{kj}x_j \rvert \leqslant \sum_{j \neq k}\lvert a_{kj} \rvert \lvert x_k \rvert.\] 约去 $\lvert x_k \rvert$ 后可得 $\lvert a_{kk} \rvert \leqslant \sum_{j \neq k} \lvert a_{kj} \rvert$,这与条件矛盾. 所以 $\lvert A \rvert \neq 0$.
        \item 构造实函数 \[f(t) = \begin{vmatrix}
            a_{11} & ta_{12} & ta_{13} & \cdots & ta_{1n} \\
            ta_{21} & a_{22} & ta_{23} & \cdots & ta_{2n} \\
            ta_{31} & ta_{32} & a_{33} & \cdots & ta_{3n} \\
            \vdots & \vdots & \vdots &          & \vdots \\
            ta_{n1} & ta_{n2} & ta_{n3} & \cdots & a_{nn} \\
        \end{vmatrix}\] 由于 $A$ 是实矩阵,所以 $f(t)$ 是关于 $t$ 的一个实系数多项式(连续)函数,同时 \[f(0) = a_{11}a_{22}\cdots a_{nn} > 0.\] 当 $t \in [0, 1]$ 时,还有 \[a_{ii} > \sum_{j \neq i} \lvert a_{ij} \rvert \leqslant \sum_{j \neq i} \lvert ta_{ij} \rvert.\] 由 (1) 可知 $f(t)$ 在 $[0, 1]$ 上非零,由连续函数的介值定理可知 $f(1) > 0$,即 $\lvert A \rvert > 0.$
        \item 此为直接推论不再赘述.
    \end{enumerate}
    \item \begin{enumerate}
        \item 设 $\begin{pmatrix}
            A & C \\
            O & B
        \end{pmatrix}$ 的伴随矩阵为 $\begin{pmatrix}
            X & Y \\
            Z & W
        \end{pmatrix}$. 而 $\begin{vmatrix}
            A & C \\
            O & B
        \end{vmatrix} = \lvert A\rvert \lvert B \rvert$,所以有 \[\begin{pmatrix}
            A & C \\
            O & B
        \end{pmatrix} \begin{pmatrix}
            X & Y \\
            Z & W
        \end{pmatrix} = \lvert A\rvert \lvert B \rvert \begin{pmatrix}
            E & O \\
            O & E
        \end{pmatrix}.\] 得到方程组 \[\begin{cases}
            AX+CZ & {}= \lvert A \rvert \lvert B \rvert E \\
            AY+CW & {}= O \\
            BZ & {}= O \\
            BW & {}= \lvert A \rvert \lvert B \rvert E
        \end{cases}\] 考虑一般情况,我们不再单独讨论 $B$ 是否等于 $O$. 所以 $Z = O$, $X = \lvert B \rvert A^*$, $W = \lvert A \rvert B^*$, $Y = -A^*CB^*$. 即 \[\begin{pmatrix}
            A & C \\
            O & B
        \end{pmatrix}^* = \begin{pmatrix}
            \lvert B \rvert A^* & -A^*CB^* \\
            O & \lvert A \rvert B^*
        \end{pmatrix}\]
        \item 若 $A$ 可逆,则可以通过以下的初等分块变换将其化为上三角块矩阵. \[\begin{pmatrix}
            E & O \\
            -CA^{-1} & E
        \end{pmatrix} \begin{pmatrix}
            A & B \\
            C & D
        \end{pmatrix} = \begin{pmatrix}
            A & B \\
            O & D-CA^{-1}B
        \end{pmatrix}.\] 两侧取伴随有 \begin{align*}
            & \left(\begin{pmatrix}
                E & O \\
                -CA^{-1} & E
            \end{pmatrix} \begin{pmatrix}
                A & B \\
                C & D
            \end{pmatrix}\right)^* = \begin{pmatrix}
                A & B \\
                C & D
            \end{pmatrix}^* \begin{pmatrix}
                E & O \\
                -CA^{-1} & E
            \end{pmatrix}^* \\ ={} & \begin{pmatrix}
                A & B \\
                O & D-CA^{-1}B
            \end{pmatrix}^* = \begin{pmatrix}
                \lvert D-CA^{-1}B \rvert A^* & -A^*B(D-CA^{-1}B)^* \\
                O & \lvert A \rvert (D-CA^{-1}B)^*
            \end{pmatrix}
        \end{align*} 而 \[\begin{pmatrix}
            E & O \\
            -CA^{-1} & E
        \end{pmatrix}^* \begin{pmatrix}
            E & O \\
            -CA^{-1} & E
        \end{pmatrix} = \begin{vmatrix}
            E & O \\
            -CA^{-1} & E
        \end{vmatrix} \begin{pmatrix}
            E & O \\
            O & E
        \end{pmatrix} = \begin{pmatrix}
            E & O \\
            O & E
        \end{pmatrix}\] 所以 \begin{align*}
            \begin{pmatrix}
                A & B \\
                C & D
            \end{pmatrix}^* ={} & \begin{pmatrix}
                \lvert D-CA^{-1}B \rvert A^* & -A^*B(D-CA^{-1}B)^* \\
                O & \lvert A \rvert (D-CA^{-1}B)^*
            \end{pmatrix} \begin{pmatrix}
                E & O \\
                -CA^{-1} & E
            \end{pmatrix} \\ ={} & \begin{pmatrix}
                \lvert D-CA^{-1}B \rvert A^*+A^*B(D-CA^{-1}B)^*CA^{-1} & -A^*B(D-CA^{-1}B)^* \\
                -\lvert A \rvert (D-CA^{-1}B)^*CA^{-1} & \lvert A \rvert (D-CA^{-1}B)^*
            \end{pmatrix}&
        \end{align*}
    \end{enumerate}
    \item \begin{enumerate}
        \item 因为 $n=2$ 时 $(A^*)^* = A$,所以 $B = (B^*)^* = A^*$,而 $B$ 的伴随矩阵是唯一的,所以存在唯一的2 阶方阵 $A = B^*$ 使得 $A^* = B$.
        \item $(\Leftarrow)$ 由 \ref*{例13.9(6)} 可得.\\
        $(\Rightarrow)$ \begin{enumerate}
            \item $r(B) = n$ 时,若存在 $A$ 使得 $A^* = B$,则由 $(A^*)^* = \lvert A \rvert^{n-2}A$,有 \[A = \dfrac{1}{\lvert A \rvert^{n-2}}(A^*)^* = \dfrac{1}{\lvert A \rvert^{n-2}}B^* = \dfrac{1}{\lvert A \rvert^{n-2}}\lvert B \rvert B^{-1},\] 而 \[\lvert B \rvert = \lvert A^* \rvert = \lvert A \rvert^{n-1},\] 代入上式可得 \[A = \lvert A \rvert B^{-1} = \sqrt[n-1]{\lvert B \rvert} B^{-1}.\] 从而满足 $A^* = B$ 的矩阵 $A$ 存在,且有 $n-1$ 个.
            \item $r(B) = 1$ 时,存在可逆矩阵 $P, Q$ 使得 \[B = P\begin{pmatrix}
                1 & O \\
                O & O \\
            \end{pmatrix}Q.\] 若存在 $A$ 满足 $A^{*} = B$,则 $r(A) = n-1$,从而存在可逆矩阵 $G, H$ 使得 \[A = G\begin{pmatrix}
                0 & O \\
                O & E_{n-1}
            \end{pmatrix}H,\] 则 \[A^* = H^*\begin{pmatrix}
                0 & O \\
                O & E_{n-1}
            \end{pmatrix}^*G^* = H^*\begin{pmatrix}
                1 & O \\
                O & O \\
            \end{pmatrix}G^* = \lvert HG \rvert H^{-1}\begin{pmatrix}
                1 & O \\
                O & O \\
            \end{pmatrix}G^{-1},\] 由 $A^* = B$ 可得 \[\lvert HG \rvert H^{-1}\begin{pmatrix}
                1 & O \\
                O & O \\
            \end{pmatrix}G^{-1} = P\begin{pmatrix}
                1 & O \\
                O & O \\
            \end{pmatrix}Q,\] 即 \[\lvert HG \rvert\begin{pmatrix}
                1 & O \\
                O & O \\
            \end{pmatrix} = HP\begin{pmatrix}
                1 & O \\
                O & O \\
            \end{pmatrix}QG,\] 记 $C = HP$, $D = QG$,且分块为 $C = \begin{pmatrix}
                C_{11} & C_{12} \\
                C_{21} & C_{22}
            \end{pmatrix}$, $D = \begin{pmatrix}
                D_{11} & D_{12} \\
                D_{21} & D_{22}
            \end{pmatrix}$,其中 $C_{22}, D_{22}$ 是 $n-1$ 阶矩阵,则 \[\lvert HG \rvert\begin{pmatrix}
                1 & O \\
                O & O \\
            \end{pmatrix} = \begin{pmatrix}
                C_{11} & C_{12} \\
                C_{21} & C_{22}
            \end{pmatrix} \begin{pmatrix}
                1 & O \\
                O & O \\
            \end{pmatrix} \begin{pmatrix}
                D_{11} & D_{12} \\
                D_{21} & D_{22}
            \end{pmatrix} = \begin{pmatrix}
                C_{11}D_{11} & C_{11}D_{12} \\
                C_{21}D_{11} & C_{21}D_{12}
            \end{pmatrix},\] 于是 \[\lvert HG \rvert = C_{11}D_{11}, C_{11}D_{12} = O, C_{21}D_{11} = O, C_{21}D_{12} = O.\] 因为 $H, G$ 可逆,所以 $C_{11} \neq 0, D_{11} \neq 0$,于是 $C_{21} = O = D_{12}$. 从而 \begin{align*}
                A ={} & G\begin{pmatrix}
                    0 & O \\
                    O & E_{n-1}
                \end{pmatrix}H = Q^{-1}D\begin{pmatrix}
                    0 & O \\
                    O & E_{n-1}
                \end{pmatrix}CP^{-1} \\ ={} &Q^{-1}\begin{pmatrix}
                    D_{11} & O \\
                    D_{21} & D_{22}
                \end{pmatrix} \begin{pmatrix}
                    0 & O \\
                    O & E_{n-1}
                \end{pmatrix} \begin{pmatrix}
                    C_{11} & C_{12} \\
                    O & C_{22}
                \end{pmatrix} P^{-1} \\ ={} & Q^{-1} \begin{pmatrix}
                    0 & O \\
                    O & D_{22}C_{22}
                \end{pmatrix} P^{-1}.
            \end{align*} 又 \[C_{11}D_{11} = \lvert HG \rvert = \lvert CP^{-1}Q^{-1}D \rvert = \dfrac{1}{\lvert PQ \rvert}\lvert DC \rvert.\] 接下来转为求 $\lvert DC \rvert$. 而 \[DC = \begin{pmatrix}
                D_{11} & O \\
                D_{21} & D_{22}
            \end{pmatrix} \begin{pmatrix}
                C_{11} & C_{12} \\
                O & C_{22}
            \end{pmatrix} = \begin{pmatrix}
                D_{11}C_{11} & D_{11}C_{12} \\
                D_{21}C_{11} & D_{21}C_{12}+D_{22}C_{22}
            \end{pmatrix}.\] 考虑初等分块变换 \begin{align*}
                \begin{pmatrix}
                    1 & O \\
                    -D_{21}D_{11}^{-1} & E_{n-1}
                \end{pmatrix}DC ={} & \begin{pmatrix}
                    1 & O \\
                    -D_{21}D_{11}^{-1} & E_{n-1}
                \end{pmatrix} \begin{pmatrix}
                    D_{11}C_{11} & D_{11}C_{12} \\
                    D_{21}C_{11} & D_{21}C_{12}+D_{22}C_{22}
                \end{pmatrix} \\ ={} & \begin{pmatrix}
                    D_{11}C_{11} & D_{11}C_{12} \\
                    O & D_{22}C_{22}
                \end{pmatrix},
            \end{align*} 故 \[C_{11}D_{11} = \dfrac{1}{\lvert PQ \rvert}\lvert DC \rvert = \dfrac{1}{\lvert PQ \rvert}D_{11}C_{11} \lvert D_{22}C_{22} \rvert,\] 从而 \[\dfrac{1}{\lvert PQ \rvert} \lvert D_{22}C_{22} \rvert = 1,\] 即 \[\lvert D_{22}C_{22} \rvert = \lvert PQ \rvert.\] 命题得证.
            \item $r(B) = 0$ 则是平凡情况,其是所有 $r \leqslant n-2$ 矩阵的伴随矩阵.
        \end{enumerate}
        \item 由 $r(B^*) = r(A) = 1$ 可知 $r(B) = 2$. $B^*B = \lvert B \rvert E = 0$,由此可知 $B$ 的列向量为方程组 $B^*X = 0$ 的解,其基础解系为 \[\alpha_1 = (-1, 1, 0)^{\mathrm{T}}, \alpha_2 = (-1, 0, 1)^{\mathrm{T}}.\] 令 $B = (\alpha_1, \alpha_2, \alpha_3)$,其中 $\alpha_3 = k_1\alpha_1+k_2\alpha_2 = (k_1+k_2, -k_1, -k_2)^{\mathrm{T}}$. 由 $BB^* = 0$ 解得 $k_1 = k_2 = 1$,从而 \[B = \begin{pmatrix}
            -1 & -1 & 2 \\
            1 & 0 & -1 \\
            0 & 1 & -1
        \end{pmatrix}.\]
    \end{enumerate}
\end{enumerate}

\clearpage

\chapter{行列式计算进阶}

\section{化三角形法}

\begin{example}
    计算行列式$D_{n+1}=\begin{vmatrix}
            1      & a_{1}       & a_{2}       & \cdots & a_n       \\
            1      & a_{1}+b_{1} & a_{2}       & \cdots & a_n       \\
            1      & a_{1}       & a_{2}+b_{2} & \cdots & a_n       \\
            \vdots & \vdots      & \vdots      & \ddots & \vdots    \\
            1      & a_{1}       & a_{2}       & \cdots & a_n+b_{n}
        \end{vmatrix}$.
\end{example}

解析:观察行列式的特点,主对角线下方的元素与第 1 行元素对应相同,故用第 1 行的 $-1$ 倍加到下面各行便可使主对角线下方的元素全部变为0,即化为上三角形.

\begin{solution}
    将该行列式第 1 行的 $-1$ 倍分别加到第 $2,3,\ldots,n+1$ 行上去,可得
    \[ D_{n+1}=\begin{vmatrix}
            1 & a_{1} & a_{2} & \ldots & a_n   \\
              & b_{1} &       &        &       \\
              &       & b_{2} &        &       \\
              &       &       & \ddots &       \\
              &       &       &        & b_{n}
        \end{vmatrix}=\prod_{i=1}^n b_i \]
\end{solution}

\section{连加法}

\begin{example}
    计算行列式$D_n=\begin{vmatrix}
            x_1-m  & x_2    & \cdots & x_n    \\
            x_1    & x_2-m  & \cdots & x_n    \\
            \vdots & \vdots & \ddots & \vdots \\
            x_1    & x_2    & \cdots & x_n-m
        \end{vmatrix}$.
\end{example}

\begin{solution}
    \begin{align*}
        D_n & =\begin{vmatrix}
                   \displaystyle\sum_{i=1}^{n} x_i-m & x_2    & \cdots & x_n    \\
                   \displaystyle\sum_{i=1}^{n} x_i-m & x_2-m  & \cdots & x_n    \\
                   \vdots                            & \vdots & \ddots & \vdots \\
                   \displaystyle\sum_{i=1}^{n} x_i-m & x_2    & \cdots & x_n-m
               \end{vmatrix} \\
            & =\left(\sum_{i=1}^{n} x_i-m\right)
        \begin{vmatrix}
            1      & x_2    & \cdots & x_n    \\
            1      & x_2-m  & \cdots & x_n    \\
            \vdots & \vdots & \ddots & \vdots \\
            1      & x_2    & \cdots & x_n-m
        \end{vmatrix}                                   \\
            & =\left(\sum_{i=1}^{n} x_i-m\right)
        \begin{vmatrix}
            1 & x_2 & \cdots & x_n \\
              & -m  &        &     \\
              &     & \ddots &     \\
              &     &        & -m
        \end{vmatrix}                                                \\
            & =(-m)^{n-1}\left(\sum_{i=1}^{n} x_1-m\right)
    \end{align*}
\end{solution}

\section{滚动消去法}

当行列式每两行的值比较接近时,可采用让邻行中的某一行减或者加上另一行的若干倍, 这种方法叫滚动消去法.

\begin{example}
    计算行列式$D_n=\begin{vmatrix}
            1      & 2      & 3      & \cdots & n-1    & n      \\
            2      & 1      & 2      & \cdots & n-2    & n-1    \\
            3      & 2      & 1      & \cdots & n-3    & n-2    \\
            \vdots & \vdots & \vdots & \ddots & \vdots & \vdots \\
            n-1    & n-2    & n-3    & \cdots & 1      & 2      \\
            n      & n-1    & n-2    & \cdots & 2      & 1
        \end{vmatrix},\enspace n \geqslant 2$.
\end{example}

\begin{solution}
    从最后一行开始每行减去上一行
    \begin{align*}
        D_n & =\begin{vmatrix}
                   1      & 2      & 3      & \cdots & n-1    & n      \\
                   1      & -1     & -1     & \cdots & -1     & -1     \\
                   1      & 1      & -1     & \cdots & -1     & -1     \\
                   \vdots & \vdots & \vdots & \ddots & \vdots & \vdots \\
                   1      & 1      & 1      & \cdots & -1     & -1     \\
                   1      & 1      & 1      & \cdots & 1      & -1
               \end{vmatrix}=\begin{vmatrix}
                                 1      & 2      & 3      & \cdots & n-1    & n      \\
                                 2      & 0      & 0      & \cdots & 0      & -2     \\
                                 2      & 2      & 0      & \cdots & 0      & -2     \\
                                 \vdots & \vdots & \vdots & \ddots & \vdots & \vdots \\
                                 2      & 2      & 2      & \cdots & 0      & -2     \\
                                 1      & 1      & 1      & \cdots & 1      & -1
                             \end{vmatrix} \\
            & =\begin{vmatrix}
                   1      & 2      & 3      & \cdots & n-1    & n+1    \\
                   2      & 0      & 0      & \cdots & 0      & 0      \\
                   2      & 2      & 0      & \cdots & 0      & 0      \\
                   \vdots & \vdots & \vdots & \ddots & \vdots & \vdots \\
                   2      & 2      & 2      & \cdots & 0      & 0      \\
                   1      & 1      & 1      & \cdots & 1      & 0
               \end{vmatrix}=2^{n-2}
        \begin{vmatrix}
            1      & 2      & 3      & \cdots & n-1    & n+1    \\
            1      & 0      & 0      & \cdots & 0      & 0      \\
            1      & 1      & 0      & \cdots & 0      & 0      \\
            \vdots & \vdots & \vdots & \ddots & \vdots & \vdots \\
            1      & 1      & 1      & \cdots & 1      & 0
        \end{vmatrix}                      \\
            & =(-1)^{n+1}(n+1) 2^{n-2}
    \end{align*}
\end{solution}

\section{降阶法}

将高阶行列式化为低阶行列式再求解.

\begin{example}
    解行列式$D_n=\begin{vmatrix}
            x     & -1    &        &         &         \\
                  & x     & \ddots &         &         \\
                  &       & \ddots & -1      &         \\
                  &       &        & x       & -1      \\
            a_{0} & a_{1} & \cdots & a_{n-2} & a_{n-1}
        \end{vmatrix}$
\end{example}

\begin{solution}
    按最后一行展开,得
    \begin{align*}
        D_n & =\sum_{i=0}^{n-1}(-1)^{n+i+1}a_i\begin{vmatrix} A & O \\ O & B \end{vmatrix}
        =\sum_{i=0}^{n-1}(-1)^{n+i+1}a_i|A||B|                                                        \\
            & =\sum_{i=0}^{n-1}(-1)^{n+i+1}a_i(x^i)\left((-1)^{n-1-i}\right) =\sum_{i=0}^{n-1}a_i x^i
    \end{align*}

    其中$A\in \mathbf{M}_i(\mathbf{R}),\enspace B\in \mathbf{M}_{n-1-i}(\mathbf{R})$,
    \[ A=\begin{pmatrix}
            x & -1 &        &    &    \\
              & x  & \ddots &    &    \\
              &    & \ddots & -1 &    \\
              &    &        & x  & -1 \\
              &    &        &    & x
        \end{pmatrix},\enspace
        B=\begin{pmatrix}
            -1 &    &        &    &    \\
            x  & -1 &        &    &    \\
               & x  & \ddots &    &    \\
               &    & \ddots & -1 &    \\
               &    &        & x  & -1
        \end{pmatrix} \]
\end{solution}

\begin{example}
    解行列式$D_n=\begin{vmatrix}
            \lambda & a      & a      & a      & \cdots & a      \\
            b       & \gamma & \beta  & \beta  & \cdots & \beta  \\
            b       & \beta  & \gamma & \beta  & \cdots & \beta  \\
            \vdots  & \vdots & \vdots & \vdots & \ddots & \vdots \\
            b       & \beta  & \beta  & \beta  & \cdots & \gamma
        \end{vmatrix}$
\end{example}

\begin{solution}
    从第$n$行到第3行,每行都减去上一行;再从第3列到第$n$列,每列都加到第2列,得

    \begin{align*}
        D_n & =
        \begin{vmatrix}
            \lambda & a            & a            & a            & \cdots & a            \\
            b       & \gamma       & \beta        & \beta        & \cdots & \beta        \\
            0       & \beta-\gamma & \gamma-\beta & 0            & \cdots & 0            \\
            0       & 0            & \beta-\gamma & \gamma-\beta & \cdots & 0            \\
            \vdots  & \vdots       & \vdots       & \vdots       & \ddots & \vdots       \\
            0       & 0            & 0            & 0            & \cdots & \gamma-\beta
        \end{vmatrix}             \\
            & =\begin{vmatrix}
                   \lambda & (n-1)a            & a            & a            & \cdots & a            \\
                   b       & \gamma+(n-2)\beta & \beta        & \beta        & \cdots & \beta        \\
                   0       & 0                 & \gamma-\beta & 0            & \cdots & 0            \\
                   0       & 0                 & \beta-\gamma & \gamma-\beta & \cdots & 0            \\
                   \vdots  & \vdots            & \vdots       & \vdots       & \ddots & \vdots       \\
                   0       & 0                 & 0            & 0            & \cdots & \gamma-\beta
               \end{vmatrix} \\
            & =\begin{vmatrix}
                   \lambda & (n-1)a            \\
                   b       & \gamma+(n-2)\beta
               \end{vmatrix} \cdot \begin{vmatrix}
                                       \gamma-\beta & 0            & \cdots & 0            \\
                                       \beta-\gamma & \gamma-\beta & \cdots & 0            \\
                                       \vdots       & \vdots       & \ddots & \vdots       \\
                                       0            & 0            & \cdots & \gamma-\beta
                                   \end{vmatrix}           \\
            & =(\lambda \gamma+\lambda(n-2)\beta-(n-1)ab)(\gamma-\beta)^{n-2}
    \end{align*}
\end{solution}

\section{升阶法}

升阶法就是把 $n$ 阶行列式增加一行一列变成 $n+1$ 阶行列式,再通过性质化简算出结果,这种计算行列式的方法叫做升阶法或加边法. 升阶法的最大特点就是要找每行或每列相同的因子, 那么升阶之后, 就可以利用行列式的性质把绝大多数元素化为0,这样就达到简化计算的效果.

\begin{example}
    解行列式 $D=\begin{vmatrix}
            0      & 1      & 1      & \cdots & 1      & 1      \\
            1      & 0      & 1      & \cdots & 1      & 1      \\
            1      & 1      & 0      & \cdots & 1      & 1      \\
            \vdots & \vdots & \vdots & \ddots & \vdots & \vdots \\
            1      & 1      & 1      & \cdots & 0      & 1      \\
            1      & 1      & 1      & \cdots & 1      & 0
        \end{vmatrix}$.
\end{example}

\begin{solution}
    使行列式 $D$ 变成 $n+1$ 阶行列式, 即
    \[ D=\begin{vmatrix}
            1      & 1      & 1      & \cdots & 1      & 1      \\
            0      & 0      & 1      & \cdots & 1      & 1      \\
            0      & 1      & 0      & \cdots & 1      & 1      \\
            \vdots & \vdots & \vdots & \ddots & \vdots & \vdots \\
            0      & 1      & 1      & \cdots & 0      & 1      \\
            0      & 1      & 1      & \cdots & 1      & 0
        \end{vmatrix} \]

    再将第一行的 $-1$ 倍加到其他各行, 得:
    \[ D =\begin{vmatrix}
            1      & 1  & \cdots & 1  \\
            -1     & -1 &        &    \\
            \vdots &    & \ddots &    \\
            -1     &    &        & -1
        \end{vmatrix} \]

    从第二列开始, 每列乘以 $-1$ 加到第一列, 得:
    \begin{align*}
        D & =\begin{vmatrix}
                 -(n-1) & 1  & \cdots & 1  \\
                        & -1 &        &    \\
                        &    & \ddots &    \\
                        &    &        & -1
             \end{vmatrix} \\
          & =(-1)^{n+1}(n-1)
    \end{align*}
\end{solution}

\section{数归/递推法}

\begin{example} \label{ex:14:递推法}
    计算行列式$D_n=\begin{vmatrix}
            \cos \beta & 1            &        &              &              \\
            1          & 2 \cos \beta & \ddots &              &              \\
                       & 1            & \ddots & 1            &              \\
                       &              & \ddots & 2 \cos \beta & 1            \\
                       &              &        & 1            & 2 \cos \beta
        \end{vmatrix}$.
\end{example}

\begin{solution}
    \begin{align*}
        D_1 & =\cos\beta                               \\
        D_2 & =\begin{vmatrix}
                   \cos\beta & 1          \\
                   1         & 2\cos\beta
               \end{vmatrix}=2\cos^2\beta-1=\cos2\beta
    \end{align*}

    猜想$D_n=\cos n\beta$. 数学归纳证明:

    假设当 $n=k$ 时,结论成立,即 $D_{k}=\cos k \beta$. 现证当 $n=k+1$ 时,结论也成立. $ n=k+1 $ 时,
    \[ D_{k+1} = \begin{vmatrix}
            \cos \beta & 1            &        &              &              \\
            1          & 2 \cos \beta & \ddots &              &              \\
                       & 1            & \ddots & 1            &              \\
                       &              & \ddots & 2 \cos \beta & 1            \\
                       &              &        & 1            & 2 \cos \beta
        \end{vmatrix} \]

    将 $D_{k+1}$ 按最后一行展开, 得
    \begin{align*}
        D_{k+1}={} & (-1)^{k+1+k+1} \cdot 2 \cos \beta
        \begin{vmatrix}
            \cos \beta & 1            &        &              &              \\
            1          & 2 \cos \beta & \ddots &              &              \\
                       & 1            & \ddots & 1            &              \\
                       &              & \ddots & 2 \cos \beta & 1            \\
                       &              &        & 1            & 2 \cos \beta
        \end{vmatrix} \\
                   & +(-1)^{k+1+k}
        \begin{vmatrix}
            \cos \beta & 1            & 0            & \cdots & 0      \\
            1          & 2 \cos \beta & 1            & \cdots & 0      \\
            0          & 1            & 2 \cos \beta & \cdots & 0      \\
            \vdots     & \vdots       & \vdots       & \ddots & \vdots \\
            0          & 0            & 0            & \cdots & 1
        \end{vmatrix}       \\
        ={}        & 2\cos\beta D_k-D_{k-1}
    \end{align*}

    而$D_{k}=\cos k \beta,\enspace
        D_{k-1}=\cos (k-1)\beta = \cos (k \beta-\beta) = \cos k \beta \cos \beta+\sin k \beta \sin \beta$. 所以有
    \begin{align*}
        D_{k+1} & = 2 \cos \beta D_{k}-D_{k-1}                                               \\
                & =2 \cos \beta \cos k \beta-\cos k \beta \cos \beta-\sin k \beta \sin \beta \\
                & =\cos k \beta \cos \beta-\sin k \beta \sin \beta                           \\
                & =\cos (k+1) \beta
    \end{align*}

    则证得$D_n=\cos n\beta,\enspace n\in \mathbf{N}$.
\end{solution}

下面介绍常系数线性递推数列,为了方便,只介绍二阶情况. 如果$D_n$满足关系式
\[ aD_n+bD_{n-1}+cD_{n-2}=0 \]
解特征方程
\[ ar^2+br+c=0 \]
会有三种根的情况.
\begin{enumerate}
    \item $\Delta>0$, 有两个不等的实根$r_1, r_2$,则有
          \[ D_n=C_1r_1^n+C_2r_2^n \]

    \item $\Delta=0$, 有重实根$r$,则有
          \[ D_n=(C_1+nC_2)r^n \]

    \item $\Delta<0$, 有共轭复根$r=\cos\beta\pm \i\sin\beta$,则有
          \[ D_n=C_1\cos n\beta + C_2\sin n\beta \]
\end{enumerate}
以上式子中的$C_1,C_2$均为任意常数,可以令$n=1,2$获得.

所以其实\autoref{ex:14:递推法} 也可以使用递推式求得答案(留作习题证明略). 不过一般遇到的还是特征根为实数的情况比较多,给出一道练习例题:

\begin{example}
    计算行列式$D_n=
        \begin{vmatrix}
            9 & 5 &        &   &   \\
            4 & 9 & \ddots &   &   \\
              & 4 & \ddots & 5 &   \\
              &   & \ddots & 9 & 5 \\
              &   &        & 4 & 9
        \end{vmatrix}$.
\end{example}

\begin{solution}
    按第一列展开, 得
    \[ D_n=9 D_{n-1}-20 D_{n-2} \]
    即 $ D_n-9 D_{n-1}+20 D_{n-2}=0 $.

    作特征方程
    \[ x^{2}-9 x+20=0 \]
    解得 $ x_1=4,\enspace x_2=5 $. 则
    \[ D_n=A \cdot 4^n+B \cdot 5^n \]

    当 $n=1$ 时, $9=4A+5B$;

    当 $n=2$ 时,$61=16A+25B$.

    解得$A=-4,\enspace B=5$,所以
    \[ D_n=5^{n+1}-4^{n+1} \]
\end{solution}

\section{硬拆法}

\begin{example}
    计算行列式$D_n=\begin{vmatrix}
            1-a_{1} & a_{2}   &        &           &         \\
            -1      & 1-a_{2} & \ddots &           &         \\
                    & -1      & \ddots & a_{n-1}   &         \\
                    &         & \ddots & 1-a_{n-1} & a_{n}   \\
                    &         &        & -1        & 1-a_{n}
        \end{vmatrix}$.
\end{example}

\begin{solution}
    把第一列的元素看成两项的和进行拆列, 得
    \begin{align*}
        D_n= & \begin{vmatrix}
                   1-a_{1} & a_{2}   &        &           &         \\
                   -1      & 1-a_{2} & \ddots &           &         \\
                   0 + 0   & -1      & \ddots & a_{n-1}   &         \\
                   0 + 0   &         & \ddots & 1-a_{n-1} & a_{n}   \\
                   0 + 0   &         &        & -1        & 1-a_{n}
               \end{vmatrix}         \\
        =    & \begin{vmatrix}
                   1  & a_{2}   &        &           &         \\
                   -1 & 1-a_{2} & \ddots &           &         \\
                      & -1      & \ddots & a_{n-1}   &         \\
                      &         & \ddots & 1-a_{n-1} & a_{n}   \\
                      &         &        & -1        & 1-a_{n}
               \end{vmatrix} + \begin{vmatrix}
                                   -a_{1} & a_{2}   &        &           &         \\
                                          & 1-a_{2} & \ddots &           &         \\
                                          & -1      & \ddots & a_{n-1}   &         \\
                                          &         & \ddots & 1-a_{n-1} & a_{n}   \\
                                          &         &        & -1        & 1-a_{n}
                               \end{vmatrix}
    \end{align*}

    上面第一个行列式的值为 1(从第 1 行开始,每一行依次加到下一行),所以
    \[ \begin{aligned}
            D_n & =1-a_{1}\begin{vmatrix}
                              1-a_{2} & a_{3}   &        &           &         \\
                              -1      & 1-a_{3} & \ddots &           &         \\
                                      & -1      & \ddots & a_{n-1}   &         \\
                                      &         & \ddots & 1-a_{n-1} & a_{n}   \\
                                      &         &        & -1        & 1-a_{n}
                          \end{vmatrix} \\
                & =1-a_{1} D_{n-1} \cdot
        \end{aligned} \]

    这个式子对任何 $n \geqslant 2$ 都成立, 因此有
    \begin{align*}
        D_n & =1-a_{1} D_{n-1}                                          \\
            & =1-a_{1}(1-a_{2} D_{n-2})                                 \\
            & =\cdots                                                   \\
            & =1-a_{1}+a_{1} a_{2}+\cdots+(-1)^n a_{1} a_{2} \cdots a_n \\
            & =1+\sum_{i=1}^{n}(-1)^i \prod_{j=1}^i a_j
    \end{align*}
\end{solution}

\section{箭形行列式}

\begin{example}
    计算行列式$\begin{vmatrix}
            a_1    & 1   & 1   & \cdots & 1   \\
            1      & a_2                      \\
            1      &     & a_3                \\
            \vdots &     &     & \ddots       \\
            1      &     &     &        & a_n
        \end{vmatrix}$,其中$a_i\neq 0\enspace(i=1,2,\ldots,n)$.
\end{example}

\begin{solution}
    \[ \text{原式}=\begin{vmatrix}
            a_1-\displaystyle\sum_{i=2}^n\frac{1}{a_i} & 1   & 1   & \cdots & 1   \\
            0                                          & a_2                      \\
            0                                          &     & a_3                \\
            \vdots                                     &     &     & \ddots       \\
            0                                          &     &     &        & a_n
        \end{vmatrix} = \left(\sum_{i=2}^n\frac{1}{a_i}\right) \left(\prod_{j=2}^na_j\right) \]
\end{solution}

\section{Vandermonde 行列式}

\begin{example}
    求行列式 $D_n=\begin{vmatrix}
            1         & 1         & \cdots & 1         \\
            x_1       & x_2       & \cdots & x_n       \\
            x_1^{2}   & x_2^{2}   & \cdots & x_n^{2}   \\
            \vdots    & \vdots    & \ddots & \vdots    \\
            x_1^{n-2} & x_2^{n-2} & \cdots & x_n^{n-2} \\
            x_1^{n}   & x_2^{n}   & \cdots & x_n^{n}
        \end{vmatrix}$.
\end{example}

\begin{solution}
    考虑构造一个$n+1$阶的Vandermonde行列式.
    \[ f(x)=\begin{vmatrix}
            1         & 1         & \cdots & 1         & 1       \\
            x_1       & x_2       & \cdots & x_n       & x       \\
            x_1^{2}   & x_2^{2}   & \cdots & x_n^{2}   & x^{2}   \\
            \vdots    & \vdots    & \ddots & \vdots    & \vdots  \\
            x_1^{n-2} & x_2^{n-2} & \cdots & x_n^{n-2} & x^{n-2} \\
            x_1^{n-1} & x_2^{n-1} & \cdots & x_n^{n-1} & x^{n-1} \\
            x_1^{n}   & x_2^{n}   & \cdots & x_n^{n}   & x^{n}
        \end{vmatrix} \]

    将 $f(x)$ 按第 $n+1$ 列展开, 得
    \[ f(x)=A_{1, n+1}+A_{2, n+1} x+\cdots+A_{n, n+1} x^{n-1}+A_{n+1, n+1} x^{n} \]
    其中 $x^{n-1}$ 的系数为
    \[ A_{n, n+1}=(-1)^{n+(n+1)} D_n=-D_n \]

    又根据 Vandermonde 行列式的结果知
    \[ f(x)=(x-x_1)(x-x_2)\cdots(x-x_n) \prod_{1 \leqslant j<i \leqslant n}(x_i-x_j) \]
    由上式可求得 $x^{n-1}$ 的系数为
    \[ -(x_1+x_2+\cdots+x_n) \prod_{1 \leqslant j<i \leqslant n}(x_i-x_j) \]
    故有
    \[ D_n=(x_1+x_2+\cdots+x_n) \prod_{1 \leqslant j<i \leqslant n}(x_i-x_j) \]
\end{solution}

\section{$^*$利用$|E_m-AB|=|E_n-BA|$} \label{sec:14:利用}

\begin{example}
    求行列式 $\begin{vmatrix}
            0      & 2a_1   & 3a_1   & \cdots & na_1     \\
            a_2    & a_2    & 3a_2   & \cdots & na_2     \\
            a_3    & 2a_3   & 2a_3   & \cdots & na_3     \\
            \vdots & \vdots & \vdots & \ddots & \vdots   \\
            a_n    & 2a_n   & 3a_n   & \cdots & (n-1)a_n \\
        \end{vmatrix}$.
\end{example}

\begin{solution}
    \[ \text{原式}=\prod_{i=1}^na_i \begin{vmatrix}
            0      & 2      & 3      & \cdots & n      \\
            1      & 1      & 3      & \cdots & n      \\
            1      & 2      & 2      & \cdots & n      \\
            \vdots & \vdots & \vdots & \ddots & \vdots \\
            1      & 2      & 3      & \cdots & n-1    \\
        \end{vmatrix} \]

    注意到
    \begin{align*}
        \begin{vmatrix}
            0      & 2      & 3      & \cdots & n      \\
            1      & 1      & 3      & \cdots & n      \\
            1      & 2      & 2      & \cdots & n      \\
            \vdots & \vdots & \vdots & \ddots & \vdots \\
            1      & 2      & 3      & \cdots & n-1
        \end{vmatrix}
         & = \begin{vmatrix}(-1)\left(E_n-\begin{pmatrix}
                1      & 2      & 3      & \cdots & n      \\
                1      & 2      & 3      & \cdots & n      \\
                1      & 2      & 3      & \cdots & n      \\
                \vdots & \vdots & \vdots & \ddots & \vdots \\
                1      & 2      & 3      & \cdots & n
            \end{pmatrix}\right)\end{vmatrix} \\
         & =(-1)^n\begin{vmatrix}E_n-
                      \begin{pmatrix}
                1 \\1\\1\\\vdots\\1
            \end{pmatrix}\begin{pmatrix}1 & 2 & 3 & \cdots & n\end{pmatrix}\end{vmatrix}
    \end{align*}

    而 \[ \begin{vmatrix}E_n-\begin{pmatrix}
                1 \\1\\1\\\vdots\\1
            \end{pmatrix}\begin{pmatrix}1 & 2 & 3 & \cdots & n\end{pmatrix}\end{vmatrix}
        =1-\begin{pmatrix}1 & 2 & 3 & \cdots & n\end{pmatrix}
        \begin{pmatrix}1 \\ 1 \\ 1 \\ \vdots \\ 1\end{pmatrix}
        =-\frac{n^2+n-2}{2} \]

    所以原式$\displaystyle =(-1)^{n+1}\frac{n^2+n-2}{2}\prod_{i=1}^na_i$.
\end{solution}

\section{Laplace定理}

\vspace{2ex}
\centerline{\heiti \Large 内容总结}

希望以上的技巧不需要在考试中用到.

\vspace{2ex}
\centerline{\heiti \Large 习题}

\vspace{2ex}
{\kaishu 我总是尽我的精力和才能来摆脱那种繁重而单调的计算. }
\begin{flushright}
    \kaishu
    —— 约翰·纳皮尔
\end{flushright}

\centerline{\heiti A组}
\begin{enumerate}
    \item $ D_n=\begin{vmatrix}
                  1      & 2      & 3      & 4      & \cdots & n-1    & n      \\
                  x      & 1      & 2      & 3      & \cdots & n-2    & n-1    \\
                  x      & x      & 1      & 2      & \cdots & n-3    & n-2    \\
                  x      & x      & x      & 1      & \cdots & n-4    & n-3    \\
                  \vdots & \vdots & \vdots & \vdots & \ddots & \vdots & \vdots \\
                  x      & x      & x      & x      & \cdots & 1      & 2      \\
                  x      & x      & x      & x      & \cdots & x      & 1      \\
              \end{vmatrix}$(提示:考虑滚动消去法).

    \item $ D_n=\begin{vmatrix}
                  a_1 & b_1 &     &        &         \\
                      & a_2 & b_2 &        &         \\
                      &     & a_3 & \ddots &         \\
                      &     &     & \ddots & b_{n-1} \\
                  b_n &     &     &        & a_n
              \end{vmatrix}$.

    \item $D_n=\begin{vmatrix}
                  a+b & ab  &        &     &     \\
                  1   & a+b & \ddots &     &     \\
                      & 1   & \ddots & ab  &     \\
                      &     & \ddots & a+b & ab  \\
                      &     &        & 1   & a+b \\
              \end{vmatrix}$.

    \item 用递推法解\autoref{ex:14:递推法}.

    \item (P188 第4题)解行列式 $ \begin{vmatrix}
                  a^{2} & (a+1)^{2} & (a+2)^{2} & (a+3)^{2} \\
                  b^{2} & (b+1)^{2} & (b+2)^{2} & (b+3)^{2} \\
                  c^{2} & (c+1)^{2} & (c+2)^{2} & (c+3)^{2} \\
                  d^{2} & (d+1)^{2} & (d+2)^{2} & (d+3)^{2}
              \end{vmatrix} $.

    \item (P189 第6题)设
          \[ D=\begin{vmatrix}
                  1+a_1  & 1      & \cdots & 1      \\
                  1      & 1+a_2  & \cdots & 1      \\
                  \vdots & \vdots & \ddots & \vdots \\
                  1      & 1      & \cdots & 1+a_n
              \end{vmatrix} \]
          \begin{enumerate}
              \item 用递推公式计算行列式$D$;

              \item 硬拆$D$为$2^n$个行列式,计算出结果.
          \end{enumerate}

    \item (P189 第5题(2))解行列式 $\begin{vmatrix}
                  a_{1}+a_{2}         & a_{2}+a_{3}         & \cdots & a_{n-1}+a_n         & a_n+a_{1}         \\
                  a_{1}^{2}+a_{2}^{2} & a_{2}^{2}+a_{3}^{2} & \cdots & a_{n-1}^{2}+a_n^{2} & a_n^{2}+a_{1}^{2} \\
                  \vdots              & \vdots              & \ddots & \vdots              & \vdots            \\
                  a_{1}^{n}+a_{2}^{n} & a_{2}^{n}+a_{3}^{n} & \cdots & a_{n-1}^{n}+a_n^{n} & a_n^{n}+a_{1}^{n}
              \end{vmatrix}$.

    \item (P188 第1题 (5)--(8))解行列式
          \begin{enumerate} \begin{multicols}{2}
                  \item $ D_1=\begin{vmatrix}
                          1 & 2 & 3 & 4 \\2&3&4&1\\3&4&1&2\\4&1&2&3
                      \end{vmatrix} $

                  \item $ D_2=\begin{vmatrix}
                          \lambda+2 & -1        & -1        & -1        \\
                          -1        & \lambda+2 & -1        & -1        \\
                          -1        & -1        & \lambda+2 & -1        \\
                          -1        & -1        & -1        & \lambda+2
                      \end{vmatrix} $

              \end{multicols} \begin{multicols}{2} % dirty hack

                  \item $ D_3=\begin{vmatrix}
                          1^{2} & 2^{2} & 3^{2} & 4^{2} \\
                          2^{2} & 3^{2} & 4^{2} & 5^{2} \\
                          3^{2} & 4^{2} & 5^{2} & 6^{2} \\
                          4^{2} & 5^{2} & 6^{2} & 7^{2}
                      \end{vmatrix} $

                  \item $ D_4=\begin{vmatrix}
                          3 & 2 & 0 & 0 \\
                          1 & 3 & 2 & 0 \\
                          0 & 1 & 3 & 2 \\
                          0 & 0 & 1 & 3
                      \end{vmatrix} $
              \end{multicols} \end{enumerate}

    \item (P190 第9题)解行列式
          \begin{enumerate}
              \item $D=\begin{vmatrix}
                            1      & 2      & \cdots & 2      & 2      \\
                            2      & 2      & \cdots & 2      & 2      \\
                            \vdots & \vdots & \ddots & \vdots & \vdots \\
                            2      & 2      & \cdots & n-1    & 2      \\
                            2      & 2      & \cdots & 2      & n\end{vmatrix}$;

              \item $^*$ $D=\begin{vmatrix}
                            1      & 2      & \cdots & n-1    & n      \\
                            2      & 3      & \cdots & n      & 1      \\
                            3      & 4      & \cdots & 1      & 2      \\
                            \vdots & \vdots & \ddots & \vdots & \vdots \\
                            n      & 1      & \cdots & n-2    & n-1
                        \end{vmatrix}$.
          \end{enumerate}

    \item (P190 第10题(2))证明$\begin{vmatrix}
                  a      & c      & c      & \cdots & c      \\
                  b      & a      & c      & \cdots & c      \\
                  b      & b      & a      & \cdots & c      \\
                  \vdots & \vdots & \vdots & \ddots & \vdots \\
                  b      & b      & b      & \cdots & a
              \end{vmatrix}=\dfrac{b(a-c)^{n}-c(a-b)^{n}}{b-c}$,其中 $b \neq c$, 等式左端是 $n$ 阶行列式.
\end{enumerate}

\centerline{\heiti B组}
\begin{enumerate}
    \item $^*$ 设$A,B,C,D$都是$n$阶方阵,且$AC=CA$,证明$\begin{vmatrix} A&B \\ C&D \end{vmatrix}=|AD-CB|$(为简化,可以只考虑$A$可逆的情况).

    \item $A\in \mathbf{F}^{m\times n}, B\in \mathbf{F}^{n\times m}$,证明$|E_m-AB|=|E_n-BA|$(即\autoref{sec:14:利用} 中的结论).

    \item $A\in \mathbf{F}^{m\times n}, B\in \mathbf{F}^{n\times m}$,证明$|\lambda E_m-AB|=\lambda^{m-n}|\lambda E_n-BA|$(为简化,$\lambda>0,\enspace m>n$).
\end{enumerate}

\centerline{\heiti C组}
\begin{enumerate}
    \item 解行列式
          \begin{multicols}{2} \begin{enumerate}
                  \item $D=\begin{vmatrix}
                                ax+by & ay+bz & az+bx \\
                                ay+bz & az+bx & ax+by \\
                                az+bx & ax+by & ay+bz
                            \end{vmatrix}$

                  \item $D=\begin{vmatrix}
                                x^2+1 & xy    & xz    \\
                                xy    & y^2+1 & yz    \\
                                xz    & yz    & z^2+1
                            \end{vmatrix}$
              \end{enumerate} \end{multicols}

    \item $^*$ 计算行列式$|2E-\alpha_1^\mathrm{T}\beta_1-\alpha_2^\mathrm{T}\beta_2|$,其中$\alpha_1=(a_1,a_2,\ldots,a_n),\enspace \beta_1=(b_1,b_2,\ldots,b_n),\enspace \alpha_2=(c_1,c_2,\ldots,c_n),\enspace \beta_2 = (d_1,d_2,\ldots,d_n)$.(提示:利用$|\lambda E_m-AB|=\lambda^{m-n}|\lambda E_n-BA|$)

    \item 已知$n$阶矩阵$A$满足
          \[ AA^\mathrm{T}=E,\enspace |A|=-1 \]
          求证:$|E+A|=0$.
\end{enumerate}

\chapter{朝花夕拾}

我为这一讲取了一个很有诗意的名字,用以说明这一节我们重在对往日所学知识的回忆. 我们一路走来为了能进入这一讲做了太多准备工作,包括一开始难以理解的抽象空间和映射,以及后面具象但充满技巧性的矩阵与行列式. 但``吹尽狂沙始到金'',这一讲希望读者跟随我们的脚步,回忆起这一路上学习的核心概念和定理,为我们线性方程组一般理论的讨论画下一个完美的句点.

在解的一般理论中,我们将首先讨论有无解以及有解时唯一解和无穷解对应的情况,然后分别讨论齐次与非齐次线性方程组解的结构分别具有什么特征. 除此之外,我们也将利用一般理论讨论一些秩有关的等式和不等式,也将讨论线性方程组中一些特殊的题型. 愿读者在每一个定理的证明和每一个例题的解答中,都能进一步体会之前所学的知识,加深理解,有所感悟.

\section{线性方程组解的一般理论}

\subsection{线性方程组解的一般理论}

\begin{theorem}[线性方程组有解的充要条件] \label{thm:15:有解条件}
    线性方程组有解的充分必要条件是其系数矩阵与增广矩阵有相同的秩.
\end{theorem}
定理的证明非常简单,这里简要介绍思路:将方程组视为$x_1\beta_1+x_2\beta_2+\cdots+x_n\beta_n=\vec{b}$($\beta_i$就是系数矩阵的第$i$列),则有解的条件为$\vec{b}$可以被$\beta_1,\ldots,\beta_n$线性表示,这等价于向量组$(\beta_1,\ldots,\beta_n)$与$(\beta_1,\ldots,\beta_n,\vec{b})$等价,故定理成立.

\begin{theorem} \label{thm:15:方程组解}
    当方程组有解时(注意这个前提),以下定理成立:
    \begin{enumerate}
        \item 如果它的系数矩阵$A$的秩等于未知量的数目$n$,则方程组有唯一解;

        \item 如果$A$的秩小于$n$,则方程组有无穷多个解.
    \end{enumerate}
\end{theorem}
实际上,这一定理就是\autoref{thm:13:Cramer} 结论的一部分,因此我们不再赘述其证明. 实际上,通过上面两个定理我们首先了解了线性方程组有无解的一般准则,然后讨论了有解前提下唯一解、无穷解对应于什么情况. 事实上,有关线性方程组解的情况的讨论至此文意已尽. 无论是理论层面或是解决题目的方面,这两个定理都为我们提供了足量的信息.
\begin{example}
    设$n$阶矩阵$A$的行列式$|A|\neq 0$,记$A$的前$n-1$列形成的矩阵为$A_1$,$A$的第$n$列为$\vec{b}$,问:线性方程组$A_1X=\vec{b}$是否有解?
\end{example}

\begin{solution}

\end{solution}

\subsection{齐次线性方程组解的一般理论}

接下来我们将分别针对齐次和非齐次线性方程组的情况展开关于解的结构性质的讨论. 回顾\autoref{ex:2:常见子空间} 中的讨论,对于齐次线性方程组$AX=0$,我们有:
\begin{theorem}
    齐次线性方程组$AX=\vec{0}$的解空间为$\mathbf{R}^n$的子空间.
\end{theorem}
这一结论告诉我们,齐次线性方程组解构成线性空间,这是一个重要的结构性结论. 在确认其为线性空间后,我们来研究该线性空间的基本性质. 首先是由此引出的关于基础解系的概念. 基础解系即为齐次线性方程组解空间的一组基,且这组基的每一个线性组合都是该方程组的解、然后我们来研究这一空间的维数:
\begin{theorem}\label{thm:15:齐次维数}
    矩阵$A \in \mathbf{M}_{m \times n}(\mathbf{F})$,若$r(A) = r$,则该齐次线性方程组解空间维数为$n - r$.
\end{theorem}
事实上,本定理可以改写为类似于维数公式的形式,即
\begin{equation}\label{eq:15:齐次维数公式}
    r(A) + \dim N(A) = n.
\end{equation}
其中$N(A)$表示$AX=\vec{0}$的解空间,区别在于维数公式中$A$应当替换为线性映射$\sigma$.

我们令$A$是线性映射$\sigma$在出发空间和到达空间基下的矩阵表示,根据矩阵的秩的定义,$r(A)=r(\sigma)$;又根据\autoref{eq:7:方程组与核空间2} 的讨论,$\ker\sigma$和$N(A)$之间是坐标的一一对应关系,因此$\dim N(A)=\dim\ker\sigma$. 因此我们有\autoref{eq:15:齐次维数公式} 也成立,证毕.

我们可以用\autoref{thm:15:齐次维数} 解决很多问题,下面是一个最简单的例子:
\begin{example}
    若$n$元齐次线性方程组$AX = \vec{0}$的解都是$BX = \vec{0}$的解. 证明:$r(B) \leqslant r(A)$.
\end{example}

\begin{proof}

\end{proof}

实际上在前面的讨论中,无论是\autoref{thm:15:有解条件} 的结论,都与列向量组成的线性空间有关,仿佛从未出现过行向量有关的定理. 事实上,我们将在未来讨论了内积空间正交性后展开对行向量空间的讨论,现在囿于概念上的缺乏无法叙述相关定理.

\subsection{非齐次线性方程组解的一般理论}

回顾\autoref{ex:2:常见子空间} 中的讨论我们发现,非齐次线性方程组的解不构成线性空间,但我们可以尝试将其与齐次线性方程组解空间联系起来研究. 对于非齐次线性方程组
\begin{equation} \label{eq:15:非齐次}
    x_1\beta_1+x_2\beta_2+\cdots+x_n\beta_n=\vec{b}
\end{equation}
我们将$n$元齐次线性方程组
\begin{equation} \label{eq:15:齐次}
    x_1\beta_1+x_2\beta_2+\cdots+x_n\beta_n=\vec{0}
\end{equation}
称为其导出组,则我们有:
\begin{theorem}
    如果$n$元非齐次线性方程组有解,则它的解集$U=\{\gamma_0+\eta \mid \eta \in W\}$.
\end{theorem}
其中$\gamma_0$为\autoref{eq:15:非齐次} 的一个解(称为特解),$W$为\autoref{eq:15:齐次} 的解空间(\autoref*{eq:15:齐次} 的解称为通解). 对于通解+特解的结构,如果读者在此之前学习了商空间一节,那么我们就会发现$U$实际上就是$W$的一个仿射子集. 当然如果没有学习相关概念,我们可以想象一个三元非齐次线性方程$ax + by + cz = d$ 和齐次线性方程$ax + by + cz = 0$. 非齐次线性方程的解显然对应一个不过原点的平面,而齐次则过原点. 我们便可以认为是齐次线性方程解平面沿着特解对应的向量平移到非齐次线性方程的解平面,这便是这一结论的几何解释. 同时我们可以得到下述结论:
\begin{enumerate}
    \item $n$元非齐次线性方程组 \ref*{eq:15:非齐次} 的两个解的差是它的导出组 \ref*{eq:15:齐次} 的一个解;

    \item $n$元非齐次线性方程组 \ref*{eq:15:非齐次} 的一个解与它的导出组 \ref*{eq:15:齐次} 的一个解之和仍是非齐次线性方程组 \ref*{eq:15:非齐次} 的一个解.
\end{enumerate}

\begin{proof}
    事实上,\ref*{eq:15:非齐次} 可以写为$AX=\vec{0}$.
    \begin{enumerate}
        \item 设$\gamma_1,\gamma_2$分别是非齐次线性方程组 \ref*{eq:15:非齐次} 的两个解,则
              \[A(\gamma_1-\gamma_2)=\vec{b}-\vec{b}=\vec{0}.\]
        \item 设$\gamma_1$是非齐次线性方程组 \ref*{eq:15:非齐次} 的一个解,$\eta_1$是齐次线性方程组 \ref*{eq:15:齐次} 的一个解,则
              \[A(\gamma_1+\eta_1)=\vec{b}+\vec{0}=\vec{b}.\]
    \end{enumerate}
\end{proof}
实际上根据上述几何描述形象理解这两个结论也不困难. 下面我们将通过一些例子进一步探讨非齐次线性方程组解的结构问题:
\begin{example}
    若$X_0$是$AX=\vec{b}$的一个特解,$X_1,\ldots,X_p$是$AX=\vec{0}$的基础解系,证明:
    \begin{enumerate}
        \item $X_0,X_1,X_2,\ldots,X_p$线性无关;

        \item $X_0,X_0+X_1,X_0+X_2,\ldots,X_0+X_p$线性无关;

        \item $AX=\vec{b}$的任一个解$X$可表示为
              \[X=k_0X_0+k_1(X_0+X_1)+k_2(X_0+X_2)+\cdots+k_p(X_0+X_p),\]
              其中$k_0+k_1+k_2+\cdots+k_p=1$.
    \end{enumerate}
\end{example}

\begin{proof}

\end{proof}

\begin{example}
    设$A$为$s \times n$矩阵,且$r(A)=r$,证明:非齐次线性方程组$AX=\vec{b}$至多存在$n-r+1$个线性无关的解向量.
\end{example}

\begin{proof}

\end{proof}

\section{理论应用}

本节我们将综合线性方程组解的一般理论和之前所学的知识讨论一些秩的等式/不等式问题. 我们首先来看四个最为经典的问题:
\begin{example}
    利用线性方程组解的一般理论,证明以下命题:
    \begin{enumerate}
        \item 设$A,B$分别是$m \times n$和$n \times s$矩阵,且$AB=O$,证明:$r(A)+r(B)\leqslant n$;

        \item 设$A$是$m \times n$实矩阵,证明:$r(A^\mathrm{T}A)=r(A)$;

        \item 设$A,B$分别是$m \times n$和$n \times s$矩阵,则$r(AB)\leqslant\min\{r(A),r(B)\}$;

        \item $A^2=A \iff r(A)+r(E-A)=n$,$A^2=E \iff r(A+E)+r(A-E)=n$.
    \end{enumerate}
\end{example}

\begin{proof}
    \begin{enumerate}
        \item

        \item

        \item

        \item
    \end{enumerate}
\end{proof}

实际上,我们解决此类问题,很多时候等式都需要拆为小于等于和大于等于同时成立进行证明,经常利用维数公式变形的齐次线性方程组解的一般理论,将问题转化为对像与核空间的研究,然后利用包含关系(复杂的题目可能涉及子空间交与和的维数公式)以及已知的简单秩不等式进行证明. 可能部分题目较为困难,但至少请掌握上面例题中的情况.
\begin{example}
    设$A^*$为矩阵$A$的伴随矩阵,证明:
    \[r(A^*)=\begin{cases}
            n & r(A)=n \\ 1 & r(A)=n-1 \\ 0 & r(A) < n-1
        \end{cases}.\]
\end{example}

\begin{proof}

\end{proof}

\section{线性方程组拓展题型}

本节我们将介绍与线性方程组有关的一些题型,可能与高中数学讨论``题型''的学习风格有些类似. 需要注意的是,除了含参问题外,其余问题我们都将分别从齐次和非齐次两个方面进行讨论,给出问题的一般解法. 但实际上这里给出的解法并非能直接套用到所有的题目中,在习题中我们会遇到更多特别的题目. 因此更重要的应当是理解解题思路,而不是死记硬背解题方法.

\subsection{含参数的线性方程组问题}

此类问题一般考察对于含参数的线性方程组,参数取值如何时有解/无解/有唯一解等. 本质而言,\autoref{thm:15:有解条件} 完全可以解决这一问题.

事实上,利用\autoref{thm:15:方程组解} 在有解情况下只需计算行列式判断非常方便,但判断无解需要利用\autoref{thm:15:有解条件},其中线性相关性的判断通常仍然需要我们对系数矩阵进行高斯消元法. 我们来看一个简单的例子:
\begin{example}
    讨论下面方程组的解的情况,并在有解的情况下求解:
    \[\begin{cases}
            x_1+x_2-x_3=1 \\ 2x_1+3x_2+kx_3=3 \\ x_1+kx_2+3x_3=2
        \end{cases}\]
\end{example}
\begin{solution}

\end{solution}

\subsection{线性方程组同解问题}

两个线性方程组同解实际上有两种情况:
\begin{enumerate}
    \item 两线性方程组都无解(注意齐次没有这种情况,因为一定有零解);

    \item 两线性方程组都有解且有相同的解集.
\end{enumerate}

下面的定理给出了两线性方程组同解的充要条件. 实际上,这两个定理的证明很值得作为练习综合运用所学知识:
\begin{theorem}
    $n$元齐次线性方程组 $A_{m_1 \times n}X=\vec{0}$与 $B_{m_2 \times n}X=\vec{0}$同解的充要条件是$r\begin{pmatrix}
            A \\ B
        \end{pmatrix}=r(A)=r(B)$.
\end{theorem}

\begin{proof}

\end{proof}

\begin{theorem}
    $n$元非齐次线性方程组 $A_{m_1 \times n}X=\vec{b}$与 $B_{m_2 \times n}X=\vec{d}$同解的充要条件是
    \begin{enumerate}
        \item $r(A)\neq r(A,\vec{b})$且$r(B)\neq r(B,\vec{d})$;或

        \item $r\begin{pmatrix}
                      A & \vec{b} \\ B & \vec{d}
                  \end{pmatrix}=r\begin{pmatrix}
                      A \\ B
                  \end{pmatrix}=r(A)=r(A,\vec{b})=r(B)=r(B,\vec{d})$.
    \end{enumerate}
\end{theorem}

\begin{proof}

\end{proof}

我们来看一个例子来运用上述定理:
\begin{example}
    已知方程组\begin{gather*}
        \begin{cases}
            x_1+2x_2+3x_3=0  \\
            2x_1+3x_2+5x_3=0 \\
            x_1+x_2+ax_3=0
        \end{cases} \\
        \begin{cases}
            x_1+bx_2+cx_3=0 \\
            2x_1+b^2x_2+(c+1)x_3=0
        \end{cases}
    \end{gather*}
    同解,求$a,b,c$的值.
\end{example}
\begin{solution}

\end{solution}

\subsection{线性方程组公共解问题}

公共解即为两线性方程组解集的交集,我们从齐次和非齐次讨论有公共解的条件:
\begin{theorem}
    对于$n$元齐次线性方程组 (1) $A_{m_1 \times n}X=\vec{0}$与 (2) $B_{m_2 \times n}X=\vec{0}$有
    \begin{enumerate}
        \item (1) 与 (2) 有非零公共解的充要条件是$r\begin{pmatrix} A \\ B \end{pmatrix}<n$;

        \item 设$\eta_1,\eta_2,\ldots,\eta_s\enspace(s=n-r(B))$是 (2) 的基础解系,则(1) 与 (2) 有非零公共解的充要条件是$A\eta_1,A\eta_2,\ldots,A\eta_s$线性相关;

        \item 设$\gamma_1,\gamma_2,\ldots,\gamma_t\enspace(t=n-r(A))$是(1) 的基础解系,$\eta_1,\eta_2,\ldots,\eta_s\enspace(s=n-r(B))$是 (2) 的基础解系,则(1) 与 (2) 有非零公共解的充要条件是$\gamma_1,\gamma_2,\ldots,\gamma_t,\eta_1,\eta_2,\ldots,\eta_s$线性相关.
    \end{enumerate}
\end{theorem}

\begin{proof}

\end{proof}

\begin{theorem}
    对于$n$元非齐次线性方程组(1) $A_{m_1 \times n}X=\vec{b}$与 (2) $B_{m_2 \times n}X=\vec{d}$,若(1) 与 (2) 都有解,则
    \begin{enumerate}
        \item (1) 与 (2) 有公共解的充要条件是$r\begin{pmatrix}
                      A \\ B
                  \end{pmatrix}=r\begin{pmatrix}
                      A & \vec{b} \\ B & \vec{d}
                  \end{pmatrix}$;

        \item 若$r(B)=s$,且$\eta_1,\eta_2,\ldots,\eta_{n-s+1}$是 (2) 的$n-s+1$个线性无关的解,则(1) 与 (2) 有公共解的充要条件是$b$是$A\eta_1,A\eta_2,\ldots,A\eta_{n-s+1}$的凸组合,即存在数$k_1,k_2,\ldots,k_{n-s+1}$使得
              \[\vec{b}=k_1A\eta_1+k_2A\eta_2+\cdots+k_{n-s+1}A\eta_{n-s+1},\]
              其中$k_1+k_2+\cdots+k_{n-s+1}=1$;

        \item 若$r(A)=t$,$r(B)=s$,$\gamma_1,\gamma_2,\ldots,\gamma_{n-t+1}$是(1) 的$n-t+1$个线性无关的解,$\eta_1,\eta_2,\ldots,\eta_{n-s+1}$是 (2) 的$n-s+1$个线性无关的解,则(1) 与 (2) 有公共解的充要条件是存在数$k_1,k_2,\ldots,k_{n-t+1}$和$l_1,l_2,\ldots,l_{n-s+1}$使得
              \[k_1\gamma_1+k_2\gamma_2+\cdots+k_{n-t+1}\gamma_{n-t+1}-l_1\eta_1-l_2\eta_2-\cdots-l_{n-s+1}\eta_{n-s+1}=\vec{0}\]
              其中$k_1+k_2+\cdots+k_{n-t+1}=1$,$l_1+l_2+\cdots+l_{n-s+1}=1$.
    \end{enumerate}
\end{theorem}

\begin{proof}

\end{proof}

这两个定理看起来非常长,实则无需记忆(最多记住二者的第一条),只需要通过证明理解其含义即可. 下面我们看一个简单的例子:
\begin{example}
    设四元齐次线性方程组(1) 为\[\begin{cases}
            2x_1+3x_2-x_3=0 \\ x_1+2x_2+x_3-x_4=0
        \end{cases}\]已知另一个四元齐次线性方程组 (2) 的基础解系为
    \[\alpha_1=(2,-1,a+2,1)^\mathrm{T},\enspace\alpha_2=(-1,2,4,a+8)^\mathrm{T}\]
    \begin{enumerate}
        \item 求方程组 (1) 的一个基础解系;

        \item 当$a$为何值时,方程组 (1) 和 (2) 有非零公共解,并求出非零公共解.
    \end{enumerate}
\end{example}

\begin{solution}

\end{solution}

\subsection{线性方程组反问题}

此类问题即已知方程组的解,要给出原方程组. 我们仍按齐次与非齐次分开的思路讨论此类问题的一般解法. 这里我们之间通过例子来讲解方法:
\begin{example}
    已知$n$维列向量组$\alpha_1,\ldots,\alpha_s$线性无关,求一齐次线性方程组以$\alpha_1,\ldots,\alpha_s$为基础解系.
\end{example}

\begin{solution}

\end{solution}

\begin{example}
    设向量组$\alpha_1,\ldots,\alpha_s$线性无关,求一非齐次线性方程组$AX=\vec{b}$,使其解集以$\alpha_1,\ldots,\alpha_s$为极大线性无关组.
\end{example}

\begin{solution}

\end{solution}

下面我们来看一个更为一般的例子来运用上面介绍的方法:
\begin{example}
    已知$\alpha_1=(1,2,-1,0,4)^\mathrm{T},\enspace\alpha_2=(-1,3,2,4,1)^\mathrm{T},\enspace\alpha_3=(2,9,-1,4,13)^\mathrm{T}$,且有$W=\spa(\alpha_1,\alpha_2,\alpha_3)$.
    \begin{enumerate}
        \item 求以$W$为解空间的一个齐次线性方程组;

        \item 求以$W'=\{\eta+\alpha \mid \alpha\in W\}$为解集的一个非齐次线性方程组,其中$\eta=(1,2,1,2,1)^\mathrm{T}$.
    \end{enumerate}
\end{example}

\begin{solution}

\end{solution}

\vspace{2ex}
\centerline{\heiti \Large 内容总结}

\vspace{2ex}
\centerline{\heiti \Large 习题}

\vspace{2ex}
{\kaishu 即使我说二二得四,三三见九,也没有一字不错。这些既然都错,则绅士口头的二二得七,三三见千等等,自然就不错了。}
\begin{flushright}
    \kaishu
    ——鲁迅,《朝花夕拾》
\end{flushright}

\centerline{\heiti A组}
\begin{enumerate}
    \item 证明以下关于线性方程组解的理论的基本定理:

          第一组(齐次线性方程组解空间的一般理论)
          \begin{enumerate}
              \item 设矩阵 $A \in \mathbf{M}_{m\times n}(\mathbf{F})$,若 $r(A)=r$,则齐次线性方程组 $AX=\vec{0}$ 的解空间 $N(A)$ 是 $\mathbf{F}^n$ 的一个 $n-r$ 维子空间.

              \item 设 $A$ 为 $m \times n$ 矩阵,则
                    \begin{enumerate}
                        \item 齐次线性方程组 $AX=\vec{0}$ 只有零解等价于 $r(A)=n$;

                        \item 齐次线性方程组 $AX=\vec{0}$ 有非零解(无穷解)等价于 $r(A)<n$.
                    \end{enumerate}

              \item 设 $A$ 为 $n$ 阶矩阵,则
                    \begin{enumerate}
                        \item 齐次线性方程组 $AX=\vec{0}$ 只有零解等价于 $|A|\neq 0$;

                        \item 齐次线性方程组 $AX=\vec{0}$ 有非零解(无穷解)等价于 $|A|=0$.
                    \end{enumerate}
          \end{enumerate}

          第二组(非齐次线性方程组解空间的一般理论)
          \begin{enumerate}[resume*]
              \item 对于非齐次线性方程组 $AX=\vec{b}$,下列命题等价:
                    \begin{enumerate}
                        \item $AX=\vec{b}$ 有解;

                        \item $\vec{b} \in R(A)$,即 $\vec{b}$ 可被 $A$ 的列向量组线性表示;

                        \item $r(A,\vec{b})=r(A)$,即增广矩阵的秩等于系数矩阵的秩.
                    \end{enumerate}
          \end{enumerate}

          第三组(线性方程组解的结构的一般理论)
          \begin{enumerate}[resume*]
              \item 设 $X_1,X_2,\ldots,X_s$ 为齐次线性方程组 $AX=\vec{0}$ 的一组解,则 $k_1X_1+k_2X_2+\cdots+k_sX_s$ 也为齐次线性方程组 $AX=\vec{0}$ 的解,其中 $k_1,k_2,\ldots,k_s$ 为任意常数.

              \item 设 $\eta_0$ 为非齐次线性方程组 $AX=\vec{b}$ 的一个解,$X_1,X_2,\ldots,X_s$ 为齐次线性方程组 $AX=\vec{0}$ 的一组解,则 $k_1X_1+k_2X_2+\cdots+k_sX_s+\eta_0$ 也为非齐次线性方程组 $AX=\vec{b}$ 的解.

              \item 设 $\eta_1,\eta_2$ 为非齐次线性方程组 $AX=\vec{b}$ 的两个解,则 $\eta_2-\eta_1$ 为齐次线性方程组 $AX=\vec{0}$ 的解.

              \item 设 $X_1,X_2,\ldots,X_s$ 为非齐次线性方程组 $AX=\vec{b}$ 的一组解,则 $k_1X_1+k_2X_2+\cdots+k_sX_s$ 也为非齐次线性方程组 $AX=\vec{b}$ 的解的充分必要条件是 $k_1+k_2+\cdots+k_s=1$.

              \item 设 $X_1,X_2,\ldots,X_s$ 为非齐次线性方程组 $AX=\vec{b}$ 的一组解,则 $k_1X_1+k_2X_2+\cdots+k_sX_s$ 为齐次线性方程组 $AX=\vec{0}$ 的解的充分必要条件是 $k_1+k_2+\cdots+k_s=0$. 判断以下关于线性方程组解的理论的说法是否正确并说明理由:
          \end{enumerate}

          第四组(一些经典的判断题)
          \begin{enumerate}[resume*]
              \item 方程组 $AX=\vec{b}$ 有唯一解等价于方程组 $AX=\vec{0}$ 只有零解.

              \item 设 $A$ 是 $m \times n$ 矩阵,$B$ 是 $n \times s$ 矩阵,若 $AB=O$,则 $B$ 的列向量为方程组 $AX=\vec{0}$ 的解.

              \item 设 $A$ 是 $n$ 阶非零矩阵,则存在非零矩阵 $B$,使得 $AB=O$ 等价于 $r(A)<n$.

              \item 方程组 $AX=\vec{0}$ 的解为 $BX=\vec{0}$ 的解,则 $r(A) \geqslant r(B)$.

              \item 方程组 $AX=\vec{0}$ 与 $BX=\vec{0}$ 为同解方程组等价于 $r(A)=r(B)$.
          \end{enumerate}

    \item 设$A$为四阶矩阵,$r(A)<4$,且$A_{21}\neq 0$,求方程组$AX=\vec{0}$的通解.

    \item 设$A=(\alpha_1,\alpha_2,\alpha_3,\alpha_4)$为四阶矩阵,方程组$AX=\vec{0}$的通解为$X=k(1,0,-4,0)^\mathrm{T}$,求$A^*X=0$的基础解系.

    \item 设$A$为$n$阶实矩阵,$W=\{\beta\in\mathbf{R}^n \mid \alpha^\mathrm{T}A\beta=0,\enspace \forall \alpha\in\mathbf{R}^n\}$,证明:
          \begin{enumerate}
              \item $\dim W+r(A)=n$;

              \item $W$为$\mathbf{R}^n$的子空间.
          \end{enumerate}

    \item 已知4级方阵$A=(\alpha_1,\alpha_2,\alpha_3,\alpha_4)$的列向量$\alpha_1,\alpha_2,\alpha_4$线性无关,且$\alpha_1=2\alpha_2-\alpha_3$,若$\beta=\alpha_1-\alpha_2+3\alpha_4$,求方程组$AX=\beta$的通解.

    \item 设四元非齐次线性方程组的系数矩阵的秩为3,已知$\eta_1,\eta_2,\eta_3$是它的三个解向量,且$\eta_1=(2,3,4,5)^\mathrm{T},\eta_2+\eta_3=(1,2,3,4)^\mathrm{T}$,求该方程组的通解.

    \item 设$\beta_1,\beta_2,\beta_3$是$n$元非齐次线性方程组$AX=\vec{b}$的三个线性无关的解,且$r(A)=n-2$,求:
          \begin{enumerate}
              \item 导出组$AX=\vec{0}$的一个基础解系;

              \item $AX=2\vec{b}$的一般解.
          \end{enumerate}

    \item 已知$A$是一个$s\times n$矩阵,证明:线性方程组$AX=\vec{b}$对任意列向量$\vec{b}_{s\times 1}$都有解的充要条件是$A$行满秩.

    \item 设$A,B$分别是$m \times n$和$n \times s$矩阵,且$r(B)=n$,证明:若$AB=O$,则$A=O$.

    \item 设$A \in \mathbf{F}^{m \times n},B \in \mathbf{F}^{(n-m) \times n}\enspace(m<n)$,$V_1,V_2$分别为齐次线性方程组$AX=\vec{0}$和$BX=\vec{0}$的解空间,证明:$\mathbf{F}^n=V_1\oplus V_2$的解的充要条件是$\begin{pmatrix} A \\ B \end{pmatrix}X=\vec{0}$只有零解.

    \item 齐次线性方程组\[\begin{cases}
                  x_2+ax_3+bx_4=0  \\
                  -x_1+cx_3+dx_4=0 \\
                  ax_1+cx_2-ex_4=0 \\
                  bx_1+dx_2+ex_3=0
              \end{cases}\]的一般解以$x_3,x_4$作为自由未知量.
          \begin{enumerate}
              \item 求$a,b,c,d,e$满足的的条件;

              \item 求齐次线性方程组的基础解系.
          \end{enumerate}
\end{enumerate}

\centerline{\heiti B组}
\begin{enumerate}
    \item 设$A$为$m \times n$矩阵,$r(A)=m$,$B$是$m$阶可逆矩阵,已知$A$的行空间$R(A^\mathrm{T})$是方程组$CX=\vec{0}$的解空间,证明:$BA$的行向量也是$CX=\vec{0}$的一个基础解系.

    \item 设$A$是$n$阶矩阵,且$A_{11}\neq 0$,证明:方程组$AX=\vec{b}$($\vec{b}$为非零向量)有无穷多解的充要条件为$A^*\vec{b}=\vec{0}$.

    \item 若$n$阶矩阵$A$的各行、各列元素之和都为0,证明:$|A|$的所有元素的代数余子式都相等.

    \item 已知 $\alpha_1,\alpha_2,\ldots,\alpha_s$ 是齐次线性方程组 $AX=\vec{0}$ 的一组基础解系,向量组
          \[\beta_1=t_1\alpha_1+t_2\alpha_2,\ \beta_2=t_1\alpha_2+t_2\alpha_3,\ \ldots,\ \beta_{s-1}=t_1\alpha_{s-1}+t_2\alpha_s\]
          试问当实数 $t_1,t_2$ 满足何条件时,$AX=\vec{0}$ 有基础解系包含向量 $\beta_1,\beta_2,\ldots,\beta_{s-1}$,并写出该基础解系中的其余向量.

    \item (注:本题一般形式在教材第六章补充题1)已知线性方程组
          \[\begin{cases} \begin{aligned}
                      a_{11}x_1+a_{12}x_2+\cdots+a_{1,2n}x_{2n} & =0              \\
                      a_{21}x_1+a_{22}x_2+\cdots+a_{2,2n}x_{2n} & =0              \\
                                                                & \vdotswithin{=} \\
                      a_{n1}x_1+a_{n2}x_2+\cdots+a_{n,2n}x_{2n} & =0              \\
                  \end{aligned}\end{cases}\]
          的一个基础解系为$(b_{11},b_{12},\ldots,b_{1,2n})^\mathrm{T},(b_{21},b_{22},\ldots,b_{2,2n})^\mathrm{T},(b_{n1},b_{n2},\ldots,b_{n,2n})^\mathrm{T}$,求解线性方程组
          \[\begin{cases} \begin{aligned}
                      b_{11}x_1+b_{12}x_2+\cdots+b_{1,2n}x_{2n} & =0              \\
                      b_{21}x_1+b_{22}x_2+\cdots+b_{2,2n}x_{2n} & =0              \\
                                                                & \vdotswithin{=} \\
                      b_{n1}x_1+b_{n2}x_2+\cdots+b_{n,2n}x_{2n} & =0              \\
                  \end{aligned} \end{cases}.\]

    \item 设$A,B\in \mathbf{F}^{n\times n}$,且$r(A)=r,\enspace r(B)=s,\enspace r\begin{pmatrix} A \\ B \end{pmatrix}=k$.
          \begin{enumerate}
              \item 证明:满足$AX=O$的$n$阶方阵$X$全体构成$\mathbf{F}^{n\times n}$的子空间,并求其维数;

              \item 令满足$AX=O$的$n$阶方阵$X$全体构成的子空间为$V_1$,满足$BX=O$的$n$阶方阵$X$全体构成的子空间为$V_2$,求$V_1+V_2$的维数.
          \end{enumerate}

    \item 设$A$是元素全为1的$n$阶方阵.
          \begin{enumerate}
              \item 求行列式$|aE+bA|$,其中$a,b$为实常数;

              \item 已知$1<r(aE+bA)<n$,试确定$a,b$满足的条件,并求下列线性子空间的维数:
                    \[W=\{x \mid (aE+bA)x=0,\enspace x\in\mathbf{R}\}.\]
          \end{enumerate}

    \item 已知线性方程组
          \[\begin{cases} \begin{aligned}
                      a_{11}x_1+a_{12}x_2+\cdots+a_{1n}x_n & =b_1            \\
                      a_{21}x_1+a_{22}x_2+\cdots+a_{2n}x_n & =b_2            \\
                                                           & \vdotswithin{=} \\
                      a_{n1}x_1+a_{n2}x_2+\cdots+a_{nn}x_n & =b_n            \\
                  \end{aligned} \end{cases}\]
          的系数矩阵与
          \[\begin{pmatrix}
                  a_{11} & a_{12} & \cdots & a_{1n} & b_1    \\
                  a_{21} & a_{22} & \cdots & a_{2n} & b_2    \\
                  \vdots & \vdots & \ddots & \vdots & \vdots \\
                  a_{n1} & a_{n2} & \cdots & a_{nn} & b_n    \\
                  b_1    & b_2    & \cdots & b_n    & 0
              \end{pmatrix}\]
          秩相等,求证:上述线性方程组有解.

    \item 设$A=(a_{ij})_{m\times n}$,$\vec{b}$和$X$为$m$元列向量,$Y$为$n$元列向量,证明:
          \begin{enumerate}
              \item 若$AY=\vec{b}$有解,则$A^\mathrm{T}X=\vec{0}$的任一组解都满足$\vec{b}^\mathrm{T}X=\vec{0}$;

              \item 方程组$AY=\vec{b}$有解的充要条件是方程组$\begin{pmatrix}
                            A^\mathrm{T} \\ \vec{b}^\mathrm{T}
                        \end{pmatrix}X=\begin{pmatrix}
                            \vec{0} \\ 1
                        \end{pmatrix}$无解(其中$\vec{0}$是$n$元零向量).
          \end{enumerate}

    \item 判断:设 $A$ 是复数域上 $m \times n$ 阶矩阵,则矩阵秩 $r\left(A^T A\right)=r(A)$.

    \item 证明:对于$m \times n$实矩阵$A$,方程$A^\mathrm{T}AX = A^\mathrm{T}\vec{b}$总是有解,且$A$为方阵时,$A^\mathrm{T}AX = \vec{0}$和$AX=\vec{0}$同解. 当$r(A)=n$时求其解,并证明$A(A^\mathrm{T}A)^{-1}A^\mathrm{T}$是幂等的对称矩阵.

    \item 设$A,B,C$为$n$阶实方阵,且$BAA^\mathrm{T}=CAA^\mathrm{T}$,证明:$BA=CA$.

    \item 设$A$为数域$\mathbf{F}$上的$n$阶方阵,又设线性空间$\mathbf{F^n}$的两个子空间为$W_1=\{X\in\mathbf{F}^n \mid AX=\vec{0}\}$,$W_2=\{X\in\mathbf{F}^n \mid (A-E)X=\vec{0}\}$. 证明:$A^2=A \iff \mathbf{F}^n=W_1\oplus W_2$.

    \item $n$阶方阵$A,B$满足$AB=BA$,证明:$r(AB)+r(A+B) \leqslant r(A)+r(B)$.

    \item 请按序证明以下结论:
          \begin{enumerate}
              \item $A,B$分别是$s \times m,m \times n$矩阵,则$ABX=\vec{0}$与$BX=\vec{0}$同解的充要条件是$r(AB)=r(B)$;

              \item $A,B$分别是$s \times m,m \times n$矩阵,且$r(AB)=r(B)$,则对任意的$n \times t$矩阵都有$r(ABC)=r(BC)$;

              \item 设$A$是$n$阶方阵,则存在正整数$k$使得$r(A^k)=r(A^{k+1})=r(A^{k+2})=\cdots$,且对任意正整数$m$,有$r(A^n)=r(A^{n+m})$.
          \end{enumerate}

    \item 如果齐次线性方程组\[\begin{cases}
                  x_1+x_2+bx_3-x_4+x_5=0   \\
                  2x_1+3x_2+x_3+x_4-2x_5=0 \\
                  x_2+ax_3+3x_4-4x_5=0     \\
                  -3x_1-3x_2-3bx_3+bx_4+(a+2)x_5=0
              \end{cases}\]的解空间是3维的,试求$a,b$的值与解空间的基. 解空间可能为2维吗?
    \item 设$W_1,W_2$分别为$n$元齐次线性方程组$AX=\vec{0}$和$BX=\vec{0}$的解空间,试构造两个$n$元齐次线性方程组,使它们的解空间分别为$W_1 \cap W_2$和$W_1+W_2$.

    \item 已知方程组$\begin{cases}
                  x_1+x_2+ax_3+x_4=1 \\ -x_1+x_2-x_3+bx_4=2 \\ 2x_1+x_2+x_3+x_4=c
              \end{cases}$与$\begin{cases}
                  x_1+x_4=-1 \\ x_2-2x_4=d \\ x_3+x_4=e
              \end{cases}$同解,求$a,b,c,d,e$.
    \item 设有两个非齐次线性方程组 (1) 和 (2),它们的通解分别是$X=\gamma+t_1\eta_1+t_2\eta_2=\delta+k_1\xi_1+k_2\xi_2$. 其中$\gamma=(5,-3,0,0)^\mathrm{T},\eta_1=(-6,5,1,0)^\mathrm{T},\eta_2=(-5,4,0,1)^\mathrm{T},\delta=(-11,3,0,0)^\mathrm{T},\xi_1=(8,-1,1,0)^\mathrm{T},\xi_2=(10,-2,0,1)^\mathrm{T}$,求这两个方程组的公共解.

    \item 若方程组$\begin{cases}
                  x_1+x_2+x_3=0   \\
                  x_1+2x_2+ax_3=0 \\
                  x_1+4x_2+a^2x_3=0
              \end{cases}$与$x_1+2x_2+x_3=a-1$有公共解,求$a$的值及所有公共解.
\end{enumerate}

\centerline{\heiti C组}
\begin{enumerate}
    \item 用方程组的理论证明:一个$n$次多项式不可能有多于$n$个不同的根.

    \item 相容(即有解)的线性方程组$AX=\vec{b}$在怎样的条件下,其解中第$k$个未知量$x_k$都是同一个值?你给的条件是否是充分必要的?

    \item 已知$A$是$n$阶对称矩阵,$\beta$为$n$元非零列向量,$B=\begin{pmatrix}
                  A & \beta \\ \beta^\mathrm{T} & 0
              \end{pmatrix}$,证明:
          \begin{enumerate}
              \item 若$r(A)=n$,则$B$可逆的充要条件是$\beta^\mathrm{T}A^{-1}\beta \neq \vec{0}$;

              \item 若$r(A)=r$,则$r(B)=r$的充要条件是方程组$\begin{cases}
                            AX=\beta \\ \beta^\mathrm{T}X=\vec{0}
                        \end{cases}$有解;

              \item 若$r(A)=n-1$,则$B$可逆的充要条件是$AX=\beta$无解.
          \end{enumerate}

    \item 讨论$b_1,b_2,\ldots,b_n\enspace(n \geqslant 2)$满足什么条件时,下列方程组
          \[\begin{cases} \begin{aligned}
                      x_1+x_2     & =b_1            \\
                      x_2+x_3     & =b_2            \\
                                  & \vdotswithin{=} \\
                      x_{n-1}+x_n & =b_{n-1}        \\
                      x_n+x_1     & =b_n
                  \end{aligned} \end{cases}\]有解,并求解.

    \item 已知$A$是$s \times n$矩阵,$B$是$m \times n$矩阵,$X,\vec{a},\vec{b}$分别是$n,s,m$元列向量,证明:
          \begin{enumerate}
              \item 齐次线性方程组$AX=\vec{0}$和$BX=\vec{0}$同解的充要条件是$A$与$B$行向量等价(列向量不一定);

              \item 齐次线性方程组$AX=\vec{0}$和$BX=\vec{0}$解空间分别为$V_1,V_2$,证明:$V_1 \subseteq V_2$的充要条件是存在$m \times s$矩阵$C$使得$B=CA$;

              \item 线性方程组$AX=\vec{a}$的解都是$BX=\vec{b}$的解的充要条件是增广矩阵$(B,\vec{b})$的每个行向量都可以被$(A,\vec{a})$的行向量线性表示;

              \item 线性方程组$AX=\vec{a}$与$BX=\vec{b}$同解的充要条件是$(A,\vec{a})$与$(B,\vec{b})$行向量等价.
          \end{enumerate}
\end{enumerate}

\chapter{史海拾遗} \label{chap:史海拾遗}

通过前面十余讲的讨论,我们已经将线性代数的一个核心问题——有关线性方程组的解的一般理论完成,其意已尽. ``历史是一面镜子,它照亮现实,也照亮未来. '' 我们很有必要抓住这一时机来完整讨论有关于线性代数的历史,循着历代数学家的脚步重新审视所学的内容,再次感受其中逻辑的自然与顺畅. 很多时候,知道一件事情为什么、怎么来的更为可贵. 另一方面,我们也将在史海中搜寻整个数学大厦中与线性代数紧密相连的部分,开始我们后一阶段更多``未竟之美''的讨论.

提示:本讲中可能出现大量未学习的内容,事实上很多是后续会学习的,也有部分是非常前沿的介绍. 读者可以留个印象,日后再回过头来看,一定会感觉无比亲切.

\section{起点:初等代数}

\subsection{初等代数简介}

代数的英语为 algebra ,源于阿拉伯语单字``al-jabr''(本义为``重聚''),出自《代数学》(阿拉伯语:al-Kitāb al-muḫtaṣar fī ḥisāb al-ğabr wa-l-muqābala)这本书的书名上,意指移项和合并同类项之计算的摘要,其为波斯回教数学家花拉子米于820年所著.

事实上,初见代数一词我们脑海中便会浮现出小学、初中阶段老师反复强调的``用字母表示数''的思想,``一元一次方程''、``合并同类项''、``因式分解''等熟悉的词汇也会出现在眼前. 事实也是如此,初等代数的由来正是用字母表示数后,得到了一系列方程和多项式的有趣问题.

亚历山大港的丢番图(Dióphantos ho Alexandreús,公元200--284),是罗马时代的数学家. 大部分有关丢番图生平的信息来源于5世纪时希腊人梅特罗多勒斯(Metrodorus)在其文集中收录的一篇具有数学谜题性质的《丢番图墓志铭》:
\begin{quote}
    \kaishu
    坟中安葬着丢番图.

    多么令人惊讶,它忠实地记录了所经历的道路.

    上帝给予的童年占六分之一,

    又过十二分之一,两颊长胡,

    再过七分之一,点燃起结婚的蜡烛.

    五年之后天赐贵子,

    可怜迟到的宁馨儿,享年仅及其父之半,便进入冰冷的墓.

    悲伤只有用数论的研究去弥补,

    又过四年,他也走完了人生的旅途.
\end{quote}
据此列一元一次方程可知,丢番图享寿84岁,于33岁时成婚,38岁时生子,80岁时丧子. 丢番图作著的丛书《算术》(Arithmetica)处理求解代数方程组的问题,其中有不少已经遗失,但他的研究在数论中占有重要地位,如丢番图方程、丢番图几何、丢番图逼近等都是数学里的重要领域. 我们简要展开丢番图方程的讨论:
\begin{definition}
    形如
    \[a_1x_1^{b_1}+a_2x_2^{b_2}+\cdots+a_nx_n^{b_n}=c,\enspace a_i,b_i,c\in\mathbf{Z}\enspace(i=1,2,\ldots,n)\]
    的方程称为丢番图方程(或不定方程).
\end{definition}

简而言之,丢番图方程就是未知数只能使用整数的整数系数多项式等式,虽然定义看起来十分简单,学过初等数论的同学应当有些熟悉(事实上在之后的讨论中我们可以看到初等代数与初等数论之间是紧密联系的),但可以说这一方程在数学史上留下了浓墨重彩的一笔. 后来当法国数学家费马研究《算术》一书时,对其中某个方程颇感兴趣并认为其无解,说他对此``已找到一个绝妙的证明'',但却没有记录下来,直到三个世纪后才出现完整的证明. 我们这里简要介绍这一定理,即费马大定理:
\begin{theorem}
    丢番图方程$x^n+y^n-z^n=0$在$n>2$时无正整数解.
\end{theorem}

自费马提出猜想的三百余年以来,无数数学家为证明费马大定理而费尽心血,直至1995年,英国数学家安德鲁·怀尔斯(Andrew Wiles)最终给出了证明. 这一证明使用了代数数论、代数几何等大量现代数学工具. 如果读者有兴趣自行搜索费马大定理证明的历史,我们可以看到对这一看似简单定理的证明的不懈追求推动了现代数学的发展,可谓是意义重大. 或许在数学史中这些中间过程不过是短短的一行文字描述,但这就是人类探索真理的历程的缩影.

或许阅读这一讲义的很多同学都来源于计算机相关专业,我们这里还可以简要介绍丢番图方程与理论计算机的关联. 1900年,希尔伯特提出丢番图问题的可解答性为他在巴黎的国际数学家大会演说中所提出的23个重要数学问题的第十题. 这个问题是对于任意多个未知数的整系数不定方程,要求给出一个可行的算法,使得借助于它通过有限次运算,可以判定该方程有无整数解.

第十问题的解决是众人集体的智慧结晶. 其中美国数学家马丁·戴维斯(Martin Davis)、希拉里·普特南(Hilary Putnam)和朱莉娅·罗宾逊(Julia Robinson)做出了突出的贡献. 而最终的结果,是由俄国数学家尤里·马季亚谢维奇(Yuri Matiyasevich)于1970年所完成的:不可能存在一个算法能够判定任何丢番图方程是否有解. 这一问题涉及到``可计算性''的问题,相关的讨论读者在学习计算理论后将有进一步的了解. 想必读者听闻过罗素悖论或理发师悖论,或者是图灵停机问题,这些都与可计算性密不可分. 实际上,``可计算性''与人类逻辑与知识的边界密切相关——这显然是个异常宏大的主题,留待后人不断探索.

\subsection{西方初等代数发展史简述}

我们回到对于初等代数历史的讨论——刚刚我们显然有些跳脱了,但这些讨论对于了解数学之美也是必要的. 在古代西方,还有几个重要的时间节点值得提及:
\begin{enumerate}
    \item 公元前1800年左右,旧巴比伦斯特拉斯堡泥板书中记述其寻找著二次椭圆方程的解法;

    \item 公元前1600年左右,普林顿322号泥板书中记述了以巴比伦楔形文字写成的勾股数列表;

    \item 公元前800年左右,印度数学家包德哈亚那在其著作《包德哈尔那绳法经》中以代数方法找到了勾股数,给出了一些二次方程的几何解法,且找出了两组丢番图方程组的正整数解;

    \item 公元前300年左右,在《几何原本》的第二卷里,欧几里德给出了有正实数根之二次方程的解法,使用尺规作图的几何方法;

    \item 公元前100年左右,写于古印度的巴赫沙里手稿中使用了以字母和其他符号写成的代数标记法,且包含有三次与四次方程,多达五个未知量的线性方程之代数解,二次方程的一般代数公式,以及不定二次方程与方程组的解法.
\end{enumerate}

此为丢番图之前的初等代数发展重要节点,蕴含着古人朴素的智慧. 丢番图之后,499年,印度数学家阿耶波多在其所著之阿耶波多书里以和现代相同的方法求得了线性方程的自然数解,描述不定线性方程的一般整数解,给出不定线性方程组的整数解,而描述了微分方程;
628年,印度数学家婆罗摩笈多在其所著之梵天斯普塔释哈塔中,介绍了用来解不定二次方程的宇宙方法,且给出了解线性方程和二次方程的规则. 他发现二次方程有两个根,包括负数和无理数根.

此后便迎来一个更为重要的时间节点. 820年,代数(algebra)一词出现,其描述于波斯数学家花拉子米所著之完成和平衡计算法概要中对于线性方程与二次方程系统性的求解方法. 花拉子米常被认为是``代数之父'',其大多数的成果简化后会被收录在书籍之中,且成为现在代数所用的许多方法之一. 990年左右,波斯阿尔卡拉吉在其所著之al-Fakhri中更进一步地以扩展花拉子米的方法论来发展代数,加入了未知数的整数次方及整数开方. 他将代数的几何运算以现代的算术运算代替,定义了单项式并给出了任两个单项式相乘的规则.

此后,初等代数的发展逐步向着现代符号体系与研究方法发展,逐渐演化为了两个方向的问题的讨论:
\begin{enumerate}
    \item 未知数更多的一次方程组的解;

    \item 未知数次数更高的高次方程的解.
\end{enumerate}

前者与我们的主角:线性代数相关,而后者则引发了另一个学科——抽象代数的开端. 初等代数学逐步解决了2、3、4次方程求解问题,这些方程的解都可用系数的四则运算与根式运算给出,即可用根式解这些方程,此时初等代数也因此而达到顶峰. 但当时的数学家们继续探索
5次与5次以上方程的解也试图用根式解出这些方程,经过200余年却无重要进展,直到19世纪抽象代数的发展才有了转机,后续我们也将介绍这其中的故事.

\subsection{中国初等代数发展史简述}

在本节的最后,我们将视角转向东方,总结古代中国人在初等代数学中作出的贡献. 相信读者都十分熟悉这一问题:
\begin{quote}
    \kaishu
    今有雉兔同笼,上有三十五头,下有九十四足,问雉兔各几何?
\end{quote}

这是《孙子算经》(不晚于473年)中提出的著名的鸡兔同笼问题. 在《孙子算经》中还提出了读者在初等数论中就已十分熟悉的``中国剩余定理'',直至现代的密码学研究也无法离开这一重要定理.

实际上,早在《孙子算经》出现前500年左右(公元前100年左右),中国古代数学名著《九章算术》中便处理了代数方程的问题. 其中的``方程章''是世界上最早的系统研究代数方程的专门论著. 它在世界数学历史上最早创立了多元一次方程组的筹式表示方法,以及它的多种求解方法. 《九章算术》把这些线性方程组的解法称为``方程术'',其实质相当于现今的高斯消元法(早于高斯约1900年).

除去线性方程组的贡献,在高次方程方面,中国古代也有相当丰富的成果. 625年左右,中国数学家王孝通在《缉古算经》中找出了三次方程的数值解;1247年,南宋数学家秦九韶在《数书九章》中用秦九韶算法解一元高次方程. 1248年,金朝数学家李冶的《测圆海镜》利用天元术将大量几何问题化为一元多项式方程,是一部几何代数化的代表作. 1300年左右,中国数学家朱世杰处理了多项式代数,发明四元术解答了多达四个未知数的多项式方程组,发明非线性多元方程的消元法,将相关多项式进行乘法、加法和减法运算,逐步消元,将多元非线性方程组化为单个未知数的高次多项式方程;并以数值解出了288个四次、五次、六次、七次、八次、九次、十次、十一次、十二次和十四次多项式方程.

\section{演化:线性代数的产生与发展}

如前所述,初等代数经过数个世纪的发展逐渐演化为了两个大的方向:未知数更多的一次方程组和未知数次数更高的高次方程. 在这两个方向上的发展,使得代数学发展到高等代数的阶段,上面两个方向简而言之就是现在大家熟悉的线性方程组理论(线性代数)和多项式理论(以致后来的抽象代数). 本节我们主要讨论前者,后者我们将在下一节中讨论.

\subsection{行列式与Cramer法则的引入}

在这一部分,我们首先将重点介绍线性方程组理论的开山鼻祖——莱布尼茨. 莱布尼茨(Gottfried Wilhelm Leibniz,1646--1716),德国自然科学家、数学家、物理学家、历史学家和哲学家,和牛顿同为微积分的创建人. 他博览群书,涉猎百科,对丰富人类的科学知识宝库做出了不可磨灭的贡献,行列式与线性方程组理论是他留给人类的财富中很小但很重要的一部分.

莱布尼茨的第一个大的贡献便是引入了新符号. 莱布尼兹首先创立了采用两个记号的双标码记法,他在方程中使用系数
10,11,12;20,21,22;30,31,32,因为两个数字各有所指,所以相当于现代数学中方程系数符号的下标,即相当于$a_{10},a_{11},\ldots$的下标. 莱布尼兹在1693年给洛必达的一封信中给出了一个方程组:
\[\begin{cases}
        10+11x+12y=0 \\
        20+21x+22y=0 \\
        30+31x+32y=0
    \end{cases}\]

事实上,这一方程组有两个未知量和三个方程,当常数项不全为0时,这是一个非齐次线性方程组. 然后莱布尼茨首先对第一行和第二行消去变量$y$,有
\[10\cdot 22+11\cdot 22x-12\cdot 20-12\cdot 21x=0,\]
然后对第一行和第三行消去变量$y$,有
\[10\cdot 32+11\cdot 32x-12\cdot 30-12\cdot 31x=0,\]
对上述两式消去$x$,有
\[10\cdot 21\cdot 32+11\cdot 22\cdot 30+12\cdot 20\cdot 31-12\cdot 21\cdot 30-11\cdot 20\cdot 32-10\cdot 22\cdot 31=0,\]
事实上,这一式与现在我们所熟知的行列式形式
\[\begin{vmatrix}
        10 & 11 & 12 \\
        20 & 21 & 22 \\
        30 & 31 & 32
    \end{vmatrix}=0\]
完全一致. 回顾线性方程组有解的条件,即$(10,20,30)^\mathrm{T}$可以由$(11,21,31)^\mathrm{T},(12,22,32)^\mathrm{T}$线性表示,因此上面行列式等于0是方程组有解的必要条件,即莱布尼茨通过消元法解出了现在线性方程组有解的一个必要条件.

进一步地,莱布尼茨用记号$\overline{1\cdot 2\cdot 3\cdot 4}$表示现在的四阶行列式. 莱布尼兹说式$\overline{1\cdot 2\cdot 3\cdot 4}$由$4!=24$项组成,这些项可以由其中某一项指数的所有置换而得到. 这里为了叙述的完整性,我们先给出置换的相关定义:
\begin{definition}[置换]
    一个集合$S$的\term{置换}\index{zhihuan@置换 (permutation)}是一个从$S$到$S$的双射$p:S\to S$.
\end{definition}

\begin{example}
    设$S=\{1,2,3\}$,定义$p(1)=2,p(2)=3,p(3)=1$是$S$的一个置换,因为它是从$S$到$S$的双射. 我们通常将其记为$p_1=\begin{pmatrix}
            1 & 2 & 3 \\
            2 & 3 & 1
        \end{pmatrix}$,上面一行是按顺序排列的$S$的元素,下一行是按置换后的顺序排列的$S$的元素.

    同理$p_2=\begin{pmatrix}
            1 & 2 & 3 \\
            1 & 3 & 2
        \end{pmatrix}$,$p_3=\begin{pmatrix}
            1 & 2 & 3 \\
            1 & 2 & 3
        \end{pmatrix}$都是$S$的置换.
\end{example}

由此我们发现,对于有限集合$S$而言,其上的置换就是集合中元素多次对换位置,这里所谓的对换就是将两个元素交换. 当一个置换可以表示成连续偶数个对换时,称其为偶置换,否则称其为奇置换. 例如,$p_1$是偶置换,因为我们要先交换1,2得到2,1,3然后交换1,3得到结果. 同理,$p_2$是奇置换,$p_3$是偶置换.

莱布尼茨在文章中表示,由11,22,33,44的第二位数通过偶置换得到的那些项有相同的符号(取正号),其余取相反的符号(负号). 本质上,莱布尼兹知道现代一个行列式的一个组合定义,区别仅在于根据奇偶置换所确定的符号规则被逆序数代替,并且柯西也将这一低阶行列式的情形扩展为了一般的$n$阶情形,我们将在介绍柯西时给出相关的结论,届时我们也将进一步完善上面对于奇偶置换的讨论.

除此之外,基于这一``行列式''的定义此莱布尼茨也给出了最原始的Cramer法则,因此可以称为这一方向理论的鼻祖. 然而,莱布尼茨的很多工作都是后来(1850年左右)才被人们发现的,所以他的方法对后来其他数学家提出的规则几乎没有影响. 事实上,同时代的日本数学家关孝和在其著作《解伏元题法》中也提出了行列式的概念与算法. 而在Cramer法则上,麦克劳林和Cramer的工作更早被人们认识到.

麦克劳林(Maclaurin,1698--1746)是18世纪英国最具有影响的数学家之一. 他自幼聪慧勤奋,11岁便步入大学校门,17岁就以有关引力研究的论文获硕士学位,19岁受聘为阿伯丁马里沙尔学院数学教授,21岁当选为英国皇家学会会员. 麦克劳林最为读者熟知的贡献想必是麦克劳林级数展开式,实际上他还有几何学等方面其他贡献. 线性代数方面,在他1748年的遗著《代数论著》(\emph{A Treatise of Algebra})中,麦克劳林最先开创了用行列式的方法来求解含2个、3个和4个未知量的联立线性方程组. 遗憾的是,麦克劳林没能进一步给出一个明确的法则来确定符号. 虽然,书中的记法不太好,符号变化的规则又比较模糊,但它确实就是我们今天所使用的Cramer法则.

事实上,现在我们所熟知的Cramer法则是由瑞士数学家加布里埃尔·克莱姆(Gabriel Cramer,1704--1752)在1750年的著作《线性代数分析导言》(\emph{Introduction à l'analyse des lignes courbes algébriques})中给出的. 为了确定经过5个点的一般二次曲线的系数,他引入了这一著名的法则,并且因其符号上更为简洁明了的优越性而被人们所接受. 事实上,克莱姆最著名的工作是在1750年发表关于代数曲线方面的权威之作. 他最早证明一个第$n$度的曲线是由$n(n + 3)/2$个点来决定的.

\subsection{线性方程组与行列式的进一步研究}

在前人工作的基础上,关于线性方程组以及行列式的理论有了更快的发展. 裴蜀(E. Bézout,1730--1783),法国数学家. 曾在海军学校和皇家炮兵学校任教,主要从事代数方程理论的研究并取得一系列的成果. 1764年,裴蜀发表论文提出了行列式中项的构成规则和符号的形成规则. 他给出了行列式的一个循环构造规律,同时用不同于莱布尼兹、克莱姆的方法,给出了项的构成规则和符号确定规则. 他所作的成就对后来行列式理论的奠基和发展起着非常重要的作用. 同时,裴蜀在该文中证明了含$n$个未知量的$n$个齐次线性方程组有非零解的条件是其``结式''(系数行列式)等于零,跳出了前人对于求解方程组计算问题的讨论,转向对一般理论的讨论.

在行列式的发展史上,第一个对行列式理论做出连贯的逻辑的阐述,即把行列式理论与线性方程组求解相分离的人,是法国数学家范德蒙(A-T. Vandermonde,1735--1796)——他不仅把行列式应用于解线性方程组,而且对行列式理论本身进行了开创性研究. 范德蒙自幼在父亲的指导下学习音乐,但对数学有浓厚的兴趣,后来终于成为法兰西科学院院士. 他给出了用二阶子式和它们的余子式来展开行列式的法则,这跳出了前人从线性方程组角度研究行列式的范畴,因此就对行列式本身这一点来说,他是这门理论的奠基人. 当然,范德蒙还有一个读者十分熟知的工作,便是计算了范德蒙行列式,这一行列式对于后续的研究有非常重要的地位.

除此之外,范德蒙的工作也得到了进一步的推广. 1772年,拉普拉斯在一篇论文中证明了范德蒙提出的一些规则,并推广了他的展开行列式的方法,便有了大家熟知的按多行(多列)展开的拉普拉斯定理. 1779年,裴蜀(正是前面所介绍的,实际上这里介绍的数学家很多都有工作的交织)发表了一篇《代数方程的一般理论》的文章,这篇论文给出了解决非齐次线性方程组的方法,这个方法是他在克莱姆、范德蒙和拉普拉斯行列式理论基础上的总结. 除此之外,裴蜀在论文中还有其他很多关于行列式理论的发现:他改进了拉普拉斯展开式的另一个改进形式;得出了行列式的两行或两列相同则结果为零的结论;并结合线性方程的消元法得出了著名的``裴蜀定理''等.

接下来对行列式理论做了可谓``大一统''工作的是著名数学家柯西——是的,又是他,一个和欧拉、高斯一样无处不在的数学家. 1812年,柯西率先使用了双下标的方式表示方程组系数(即$a_{11}$这样的有两个数字组成的下标),有趣的是柯西当年还没有使用双竖线的方式表示行列式,而是采用$S(\pm a_{11}a_{22}\cdots a_{nn})$的形式. 现在为人熟知的双竖线的表示形式是后文将要介绍的矩阵论创始人凯莱在1841年率先使用的. 事实上,柯西给出了与现代行列式定义几乎完全一致的版本. 为了展开叙述柯西的理论,我们在此进一步讨论有关于置换的概念. 我们来看一个简单的置换
\[p=\begin{pmatrix}
        1 & 2 & 3 & 4 & 5 \\
        2 & 1 & 4 & 5 & 3
    \end{pmatrix},\]
我们可以将得到上述置换的过程分为两步. 第一步我们对1和2进行置换,得到
\[p_1=\begin{pmatrix}
        1 & 2 & 3 & 4 & 5 \\
        2 & 1 & 3 & 4 & 5
    \end{pmatrix},\]
然后我们对3,4,5进行置换,但保持上一步中已经改变的1和2不变,这一过程可以写为
\[p_2=\begin{pmatrix}
        1 & 2 & 3 & 4 & 5 \\
        1 & 2 & 4 & 5 & 3
    \end{pmatrix}.\]
就可以得到$p$. 事实上这接续的两步和映射的复合运算含义完全一致. 回忆$h=g\circ f$,$h(x)=g(f(x))$是先对$x$作用$f$然后作用$g$,这里实际上也是对$1,2,3,4,5$先做了$p_1$的置换然后做了$p_2$的置换,即$p=p_1\circ p_2$,写成乘法形式即为
\[\begin{pmatrix}
        1 & 2 & 3 & 4 & 5 \\
        2 & 1 & 4 & 5 & 3
    \end{pmatrix}=\begin{pmatrix}
        1 & 2 & 3 & 4 & 5 \\
        1 & 2 & 4 & 5 & 3
    \end{pmatrix}\begin{pmatrix}
        1 & 2 & 3 & 4 & 5 \\
        2 & 1 & 3 & 4 & 5
    \end{pmatrix}.\]
实际上,上面的置换$p_1$相当于将1变为了2,然后2又变回了1,这构成了一个长度为2的循环.$p_2$相当于将3变为了4,4变为了5,然后5又变回了3,因此构成了一个长度为3的循环. 因此上式也可以改写为
\[p=(3,4,5)(1,2)\]
其中$(3,4,5)$就表示从3到4,4到5,然后5到3的循环(事实上$(4,5,3),(5,3,4)$同理表示同一个循环),$(1,2)$也是同理. 事实上,因为$3,4,5$和$1,2$之间没有元素是重复的,因此可以称它们是不相交的. 事实上我们有如下定理,我们不加证明地直接给出:
\begin{theorem}
    $S$上的任意置换$p$都可以表示为长度$\geqslant 2$且不相交的循环的乘积,且这一分解式在不考虑循环顺序(即上面所说的$(3,4,5)$和$(4,5,3)$实则表示一个循环)下是唯一确定的.
\end{theorem}

事实上,我们在前面讲的对换(即两个元素交换顺序)就是长度为2的循环. 关于对换,我们也有一个重要的结论:
\begin{theorem}\label{thm:16:对换乘积}
    $S$上的任意置换$p$都可以表示为对换的乘积.
\end{theorem}

实际上这一结论是很直观的,$1,2,\ldots,n$的任意置换实际上都可以通过两两交换顺序得到. 最愚蠢的方式就是反过来思考如何从任意置换反推到$1,2,\ldots,n$,然后全部顺序调换即可. 反推的方式就是首先找到1的位置,然后一直向左对换到最左端,然后开始找2,对换到第二个位置,以此类推,因此这一结论是显然正确的. 但接下来的证明将会从另一个角度给出更为丰富的结果:

\begin{proof}

\end{proof}

事实上在介绍莱布尼茨的工作时我们就介绍了莱布尼茨利用奇数或偶数次对换作为依据确定行列式定义中每一个排列的乘积前的符号,这里我们给出严谨的关于符号的说明:
\begin{theorem}[置换的符号] \label{thm:16:置换的符号}
    设$p$是$S$上的一个置换,将$p$分解为对换的乘积:
    \[p=p_1\cdots p_k,\]
    则称
    \[\tau(p)=(-1)^k\]
    为$p$的\term{符号}(亦称符号差或奇偶性),它由置换$p$唯一确定且不依赖于对换分解的方法. 此外任取$q,r$也为$S$上的置换,则有
    \[\tau(qr)=\tau(q)\tau(r).\]
\end{theorem}

\begin{proof}

\end{proof}

基于这一定理,我们可以有如下合理的定义:
\begin{definition}
    若$\tau(p)=1$,则称$p$为$S$上的偶置换;若$\tau(p)=-1$,则称$p$为$S$上的奇置换.
\end{definition}

事实上我们在之前也定义了奇置换和偶置换,即从顺序排列的数列经过奇数还是偶数次对换可以得到最终的置换出发进行定义. 两个定义实际上是统一的,统一性由下面这一定理保证:
\begin{theorem}\label{thm:16:置换符号计算公式}
    设$S$上的置换$p$分解为长为$l_1,l_2,\ldots,l_m$的互不相交的循环的乘积,则
    \[\tau(p)=(-1)^{\sum\limits_{k=1}^m(l_k-1)}.\]
\end{theorem}

因为对换是长度为2的循环,事实上每一个$l_i-1$都等于1. 如果上述定理成立,那么奇数次的对换将会使得置换符号为$-1$,因为$-1$的奇数次方,偶置换则会得到符号为$-1$的偶数次方,即为1,因此两个定义统一.

\begin{proof}
    事实上,根据\autoref{thm:16:置换的符号},我们有
    \[\tau(p)=\tau(p_1)\cdots\tau(p_m),\]
    根据\autoref{thm:16:对换乘积} 的证明我们知道,$p_i$可以被写为$l_i-1$个对换的乘积,因此我们有$\tau(p_i)=(-1)^{l_k-1}$,因此有
    \[\tau(p)=(-1)^{\sum\limits_{k=1}^m(l_k-1)}.\]
\end{proof}

接下来我们就要看如何将上述置换的分解与符号的理论用于计算行列式. 柯西从$n$个数$a_1,\ldots,a_n$出发,作乘积$a_1\cdots a_n$,然后类似于范德蒙行列式,作所有不同元之间的差的积$\displaystyle\prod_{1\leqslant i\leqslant j\leqslant n}(a_j-a_i)$,得到乘积
\begin{equation}\label{eq:16:柯西行列式}
    a_1\cdots a_n\prod_{1\leqslant i\leqslant j\leqslant n}(a_j-a_i)
\end{equation}
柯西对这个乘积中各项所含的幂改写成第二个下标,例如把$a_2^3$改写为$a_{23}$,把这样改写后得到的表达式定义为一个行列式,记作$S(\pm a_{11}a_{22}\cdots a_{nn})$.

我们以$n=3$为例展示上面的过程. 根据柯西描述的算法,乘积为
\begin{align*}
               & a_1a_2a_3(a_2-a_1)(a_3-a_1)(a_3-a_2)                                                                               \\
    =          & a_1a_2^2a_3^3-a_1^2a_2a_3^3+a_1^3a_2a_3^2-a_1a_2^3a_3^2+a_1^2a_2^3a_3-a_1^3a_2^2a_3                                \\
    \triangleq & a_{11}a_{22}a_{33}-a_{12}a_{21}a_{33}+a_{13}a_{21}a_{32}-a_{11}a_{23}a_{32}+a_{12}a_{23}a_{31}-a_{13}a_{22}a_{31}.
\end{align*}

事实上这与今天我们熟悉的三阶行列式计算公式完全一致,事实上$n$阶都是一致的. 柯西天才地用一个很简短的抽象公式将前人找到的规律描述了出来,同时也发明了双下标的表示,将行列式可以写成$n\times n$的矩形方阵形式,并且沿用至今.

事实上,由\autoref{eq:16:柯西行列式} 展开得到的式子中的项不难看出都可以写成如下形式:
\[\prod a_{1p(1)}\ldots a_{np(n)}\]
其中$p$是集合$S=\{1,2,\ldots,n\}$上的一个置换,因为乘积前面的$a_1\cdots a_n$会保证每一个数都出现,而后面的乘积由排列组合的知识可知只有$a_1,\ldots,a_n$的次数分别为$0,1,\ldots,n-1$的一个置换才会留下来且前面的系数为1或$-1$. 接下来便有一个问题,即前面的系数究竟是1还是$-1$. 柯西使用的方法与前面介绍的几乎完全一致!他就是通过计算$(-1)^{\sum\limits_{k=1}^m(l_k-1)}$这一方式判断的,其中$l_k$是上述置换$p$进行循环分解后各项的长度. 基于此,我们有行列式定义如下:
\begin{definition}
    $n$阶行列式
    \[\begin{vmatrix}
            a_{11} & a_{12} & \cdots & a_{1n} \\
            a_{21} & a_{22} & \cdots & a_{2n} \\
            \vdots & \vdots & \ddots & \vdots \\
            a_{n1} & a_{n2} & \cdots & a_{nn}
        \end{vmatrix}=\sum_{p\in S_n}\tau(p)\prod_{i=1}^na_{ip(i)},\]
    其中$S_n$是集合$S=\{1,2,\ldots,n\}$上置换的全体,$\tau(p)$是置换$p$的符号.
\end{definition}

在很多教科书上,这一定义也被称为逆序数定义,这是因为置换的符号实际上也可以视为所谓``逆序对''个数的体现. 那么何为逆序对呢?其实一对数就是$(a_i,a_j)$,其中$i<j$,当$a_i>a_j$时,我们称这一对数为逆序对. 例如,对于置换
\[p=\begin{pmatrix}
        1 & 2 & 3 \\
        3 & 1 & 2
    \end{pmatrix},\]
$(3,1)$和$(3,2)$都是逆序对,因为3在1前面但比1大,3在2前面但比2大. 而数对$(1,2)$则是顺序对,因为1在2前面且比2小,这与原先$1,2,3$的排列顺序是统一的.

基于逆序对的定义我们可以有奇置换和偶置换的另一个定义:
\begin{definition}
    我们称逆序数为奇数的置换为奇置换,逆序数为偶数的置换为偶置换.
\end{definition}

显然我们也必须要求这一定义和前面的定义是一致的,事实上我们有如下定理:
\begin{theorem}
    任意置换$p$的逆序数的奇偶性与其可以被分解为对换乘积的个数的奇偶性相同.
\end{theorem}

\begin{proof}

\end{proof}

由这一定理以及\autoref{thm:16:置换符号计算公式} 我们知道逆序数为奇数时,置换符号为$-1$,逆序数为偶数时,置换符号为1,因此这一定义与前面的定义是一致的.

由此我们可以看出,行列式的逆序数定义实际上最开始来源于莱布尼茨、克莱姆等人最朴素的从低阶出发的探索,它们找到了一些规律,这些规律由天才数学家柯西进行抽象总结,得到具有普适性的方法,变成了上面沿用至今的严格定义. 而后人又结合置换、逆序数等理论进行重新叙述,再嵌套一层抽象,最终上面完整的叙述逻辑. 在这里我们看到一个很抽象甚至初看没有什么道理的定义是如何自然演化而来的,实际上是数学理论螺旋式进步以及多个理论(可能对应很多条数学发展支线)在教育家们的手中进行合理排列后所呈现的状态.

更进一步地,1815年,柯西发表了一篇关于行列式理论的基础性文章. 在这篇文章中它不仅用这个名字代替了几个旧的术语,也在文章中给出了系统的一般行列式乘法定理,证明了新组的行列式是原来两个组的行列式的乘积. 在这篇论文中,柯西第一次论述了包括一个给定的矩阵的伴随矩阵的思想,以及通过展开任何行或者列来计算行列式的步骤,完善了范德蒙和拉普拉斯的工作,给出了严谨的证明. 在柯西的行列式的工作中,还涉及到对称矩阵以及相似变换等问题. 在柯西1826年的《微积分在几何中的应用教程》中,讨论了后续学习中将要介绍的一些二次型理论,以及实对称矩阵特征值均为实数(后续会讲解)等重要结论. 除此之外,柯西在相似行列式的研究中,证明了大家熟知的相似变换有相同的特征值的结论. 由此可见,柯西这一数学天才对于后世的影响是无比深远的,从记号层面的革新,到行列式展开、行列式乘法等理论的大一统,以及现在大家熟知的结论的证明,都能看出柯西贡献的突出与伟大.

继柯西之后在行列式理论方面最高产的人就是德国数学家雅可比(C. Jacobi,1804--1851),他引进了函数行列式,即``Jacobi 行列式''(读者学习多元微积分时会十分熟悉这一名词),指出函数行列式在多重积分的变量替换中的作用,给出了函数行列式的导数公式. 雅可比的著名论文《论行列式的形成和性质》标志着行列式系统理论的建成. 事实上,行列式在数学分析、几何学、线性方程组理论、二次型理论等多方面的应用,促使行列式理论自身在19世纪也得到了很大发展. 整个19世纪都有行列式的新结果. 除了一般行列式的大量定理之外,还有许多有关特殊行列式的其他定理都相继得到.

最后笔者要在此说明一点. 我们在讲义中介绍了行列式最常见的三种定义方式. 事实上,按照历史的发展脉络,的确是现在看来最不直观的逆序数定义的思想首先出现的. 《大学数学:代数与几何》中给出的公理化定义有很强的几何直观性,也与列向量组的线性相关性等有很直接的关联,但实际上几何直观源于后来拉格朗日发现行列式和以其列向量构成的四面体的体积之间的关系,是远在莱布尼茨的思想出现之后才讨论的. 而关于行列式展开的定义根据上面的讨论也知道,是范德蒙、拉普拉斯和柯西接力提出并给出严谨证明才得到的.

事实上,笔者在编写讲义的时候就发现,无论从哪个定义出发定义行列式都是显得``毫无道理''的,因为完全缺乏引入,这不像之前研究线性空间那般自然(因为我们发现了方程组行向量间的线性相关性影响了解的唯一性,并且线性空间也有高中学习的平面向量的直观作为基础),行列式的定义直接丢出来会显得非常笨拙而且没有道理,但我们研究其性质会发现它和可逆、矩阵的秩甚至以后的特征值理论有很强的关联,行列式仿佛成为了一个无头但有尾的理论,这可能也是为什么《线性代数应该这样学》完全抛弃了行列式来讲述线性代数——因为这非常难引入且不是必要的. 但笔者还是希望保留行列式这一具有重要历史地位的理论,并且它对于之后的很多研究都有重要意义,所以笔者选择在史海拾遗中从历史的角度提供一种``直观''——它来源于数学家最开始对一些问题的研究. 因此我们详细地描述了莱布尼茨如何从消元法解线性方程组得到类似于现代行列式的定义的过程. 尽管这是低维的情况,但接下来在讲述柯西的工作时,$n$维的情况在逻辑上就像是自然的推广了(我们相信在历史中也是如此,前人对于线性方程组解的形式(如Cramer法则)和行列式的研究都构成了柯西研究的出发点,在此基础上柯西做了更进一步的抽象得到了现在的行列式定义,后人又结合了置换等概念将这一理论进一步形式化),这样我们也勉强算是梳理出了一个有引入的能更让初学者接受的行列式理论:至少我们从线性方程组的解的讨论起步——于是根据朝花夕拾的知识我们知道行列式一定和线性相关、矩阵的秩等概念有很强的关联,得到一些简单的结论,然后不断抽象直至今天呈现在读者面前的令人摸不着头脑的理论,虽然非常冗长,但相信能让读者接受这一概念的引入也是比较自然的.

数学不是魔术,不是从无到有的魔法,其发展历程必然是螺旋式上升的过程,很多不直观的概念可能只是因为历经数百年很多数学家不断改进而使人很难看出当年原本自然的想法源于何处. 希望读者读完本讲后再看教材时,能体会到每一个概念、每一个定理背后的历史厚重感,它们都是历代数学家在前人肩膀上不断总结、创新而来的,这些想法或是沉稳的推进,也可能是天才的智慧. 也许在未来很多年以后的教材上,有一个全新的概念或者结论,它可能只是短短的一行描述,但那也许凝结着现在正阅读着这段文字的你未来很多年研究的心血——这也许就是一种价值的实现.

\subsection{矩阵理论的发展}

随着线性方程组和行列式理论的建立和发展,在行列式基础之上的矩阵理论发展非常迅速. ``矩阵''这个词是由西尔维斯特首先使用的,他是为了将数字的矩形阵列区别于行列式而发明了这个术语. 而实际上,矩阵这个课题在诞生之前就已经发展的很好了. 从行列式的大量工作中明显的表现出来,不管行列式的值是否与问题有关,方阵本身都可以研究和使用,矩阵的许多基本性质也是在行列式的发展中建立起来的. 在逻辑上,矩阵的概念应先于行列式的概念,然而在历史上次序正好相反.

虽然矩阵一词是西尔维斯特率先发明的,但英国数学家凯莱(A. Cayley,1821--1895)一般被公认为是矩阵论的创立者,因为在西尔维斯特创用矩阵术语以前,凯莱对于矩阵的有关概念及其性质就有所研究. 1843年,凯莱即己研究三阶以上的高阶矩阵的行列式理论(\emph{On the theory of
determinants}),L. Gegenbauer、M-Lecat、L. H. Rice等在这个领域又进行了扩展. 1846年,凯莱定义了转置矩阵以及对称矩阵,与现代的定义完全一致. 在1855--1858年间,凯莱在矩阵方面做了许多开创性的工作. 1855年,凯莱注意到在线性方程组中使用矩阵是非常芳便的,因而引进矩阵以简化记号,这就有了现在我们使用的阶梯矩阵等术语以及$AX=\vec{b}$的记号.

1858年,凯莱发表了重要文章《矩阵论的研究报告》(\emph{A memoir on the theory of matrices}). 在该研究报告中,凯莱系统地阐述了矩阵的理论体系,如矩阵概念的引入、相关概念和运算的定义,使得矩阵从零散的知识发展为系统完善的理论体系. 凯莱定义了矩阵加法和数乘运算,并且从变换的复合引入了矩阵乘法的运算法则,也给出了一些特殊矩阵例如零矩阵、单位矩阵等,同时也说明了两个矩阵相乘不符合交换律,但也着重强调了矩阵乘法是可结合的. 除此之外,凯莱也引入了逆矩阵的概念. 凯莱在文章中采用单个的符号表示矩阵,证明了矩阵$A$可逆时,方程$AX=\vec{b}$的解可以写为$X=A^{-1}\vec{b}$,并且也给出了矩阵可逆时
\[A^{-1}=\frac{1}{|A|}A^*.\]
凯莱还利用一般的代数运算和矩阵运算的相似性得出了矩阵的一些结论. 例如当行列式为零时矩阵不可逆,零矩阵不可逆,两个非零矩阵乘积可以为零矩阵等结论. 除此之外,凯莱在文章中采用单个的符号表示矩阵,推出了方阵的特征多项式的形式,并说明了特征多项式的根就是特征值的重要结论. 除此之外,凯莱也证明了我们后面要详细介绍的``哈密顿-凯莱''定理的一部分,这被称为``矩阵理论中最著名的理论之一''.

凯莱第一个把矩阵作为独立的概念提出来,并作为独立的理论加以研究. 可以说,《矩阵论的研究报告》的公开发表,标志着矩阵理论作为一个独立数学分支的诞生. 但我们之前也提到,矩阵一词是西尔维斯特在研究方程的个数与未知量的个数不相同的线性方程组时最先使用的. 因此我们也很有必要接着介绍凯莱的挚友——西尔维斯特在矩阵理论方面的成果.

詹姆斯·约瑟夫·西尔维斯特(James Joseph Sylvester,1814--1897),1829年进入设在利物浦的皇家学会的学校学习,他学习努力,成绩突出,曾因解决了美国抽彩承包人提出的一个排列问题而得到500美元的数学奖金.1846年西尔维斯特进入内殿(Inner Temple)法学协会,并于1850年取得律师资格. 在这期间他和同时进入林肯法律学会的凯莱建立起了深厚的友谊. 他们在从事法律业务的间隙,经常在一起交流数学研究的成果. 西尔维斯特一生致力于纯数学的研究,他和凯莱、哈密顿等人一起开创了自牛顿以来英国纯粹数学的繁荣局面. 西尔维斯特的成就主要在代数方面,在代数方程论、数论等诸领域都有重要的贡献. 西尔维斯特一生创造过许多数学名词,流传至今的如矩阵、判别式等都是他首先使用的.

1850年,西尔维斯特在研究方程的个数与未知量的个数不相同的线性方程组时,由于无法使用行列式(因为行列式要求行列数相同),所以引入了矩阵一词来表由$m$行$n$列元素组成的矩形阵列,西尔维斯特也引入了对角矩阵、数量矩阵等概念.1879年弗罗伯纽斯给出矩阵的秩的概念后,1884年,西尔维斯特给出了零性的概念和零性律:他把矩阵的阶数与秩的差叫做矩阵的零性,并说明了两个(而且可以推广为任意有限数目)矩阵乘积的零性不能比任意因子的零性小,也不能比组成这一乘积的因子的零性之和大. 西尔维斯特的这一零性律现在应当叙述为:
\[r(A)+r(B)-n\leqslant r(AB)\leqslant\min\{r(A),r(B)\},\]
这是矩阵理论中关于矩阵乘积的秩的一个重要定理. 除此之外,西尔维斯特和凯莱也就矩阵方程
\[A_1XB_1+\cdots+A_kXB_k=C\]
和
\[AX-XB=O,\]
这里篇幅有限我们不再展开叙述. 事实上,在之后的二次型理论中,我们还会学习到所谓的西尔维斯特惯性定理,这也是非常核心的理论.

在矩阵论发展史上,弗罗贝尼乌斯(G. Frobenius, 1849--1917)的贡献是不可磨灭的. 1870年左右,群论成为数学研究的主流之一,弗罗贝尼乌斯在柏林时就受到库默尔和克罗内克的影响,对抽象群理论产生兴趣并从事这方面的研究,发表了多篇有价值的论文. 1892年,他重返柏林大学任数学教授. 1893年当选为柏林普鲁士科学院院士. 弗罗贝尼乌斯的主要数学贡献在群论方面,在行列式、矩阵、双线性型以及代数结构方面也有出色的工作. 矩阵论方面,1878年,弗罗贝尼乌斯引进了西尔维斯特$\lambda$矩阵的行列式因子、不变因子和初等因子等概念(在一般的高等代数教材中都会由此引入讨论若当标准形),给出了正交矩阵、相似矩阵、合同矩阵等概念(与现在的定义是完全一致的),指出了各种不同类型的矩阵的关系. 1894年,他又对1878年的不变因子和初等因子理论做了更深入的工作,进一步整理了维尔斯特拉斯不变因子和初等因子的理论. 1879年,弗罗贝尼乌斯引进了矩阵的秩的概念: 矩阵的秩就是矩阵中非零子式的最大阶数. 他也引进了行列式秩的定义:如果一个行列式的所有$r+1$阶子式为0,但至少有一个$r$阶子式不为零,那么就称$r$为行列式的秩.

在上述三位数学家的工作下,矩阵论中的一个核心问题:矩阵约化与分解不断地有了新的突破. 我们将在后续章节用大量的篇幅介绍这一主题——因此这里也许可以算是承上启下的一节. 1854年,约当研究了矩阵化为标准型的问题. 1892年,梅茨勒(H. Metzler)引进了矩阵的超越函数概念并将其写成矩阵的幂级数的形式. 傅立叶、西尔和庞加莱的著作中还讨论了无限阶矩阵问题,这主要是适用方程发展的需要而开始的. 事实上,矩阵由最初作为一种工具经过两个多世纪的发展,现在已成为独立的一门数学分支——矩阵论. 而矩阵论又可分为矩阵方程论、矩阵分解论和广义逆矩阵论等矩阵的现代理论. 矩阵及其理论现已广泛地应用于现代科技的各个领域,相信很多工科读者在将来的学习中将会大量运用这一方面的结果.

\subsection{线性代数的应用:解析几何的发展}

线性代数的发展与解析几何的发展有着密切的联系,应该说二者间在数学发展史上来看是互相促进的关系. 一方面,从希腊时代到1600年几何统治着数学,代数居于附庸的地位. 而解析几何为确立代数在数学界的地位铺平了道路. 1600年以后,代数才从几何统治的桎梏下解放出来,成为一门独立的基础数学科目,占据了它在数学中应有的地位. 另一方面,我们接下来也将会展开很多用线性代数知识解决几何问题的实例.

\subsection{线性空间与线性映射的角度}

前面的讨论我们一直讨论的是线性方程组与行列式关联的历史,本讲义中最重视的线性空间和线性映射理论被搁置了. 这一节我们将重点放在这一部分,供

物理学的发展带动了向量理论及向量空间的发展,而向量理论和向量空间的发展也打开了新的数学前景. 当今数学意义上的向量及向量空间的概念内容丰富,形式多样.

向量空间第一个具体定义的是由皮亚诺(G. Peano,1858--1932)在1888年的《几何计算》中给出,但是影响并不广泛. 直到1918年外尔(Hermann Weyl,1885--1955)的工作,使得人们重新认识到了皮亚诺公理化定义的重要性. 大约在1920年左右,分析学家巴拿赫、哈恩和维纳等对皮亚诺的向量空间做了进一步的研究,并引起了广泛的影响,随之而产生了如赋范向量空间、希尔伯特空间等等.

\section{推广:线性代数之后的线性和代数}

\subsection{泛函分析}

\subsection{抽象代数}

本小节我们接着前面初等代数发展为高等代数的两个方向之二,继续讨论有关于高次方程与多项式理论的历史.

\section{进阶:(线性)代数的进一步发展}

\subsection{抽象代数}

\subsection{泛函分析}

\subsection{抽象代数}

\section{进阶:本世纪的线性代数}

在这一节中,我们就几个专题探讨线性代数中的一些概念在上世纪末到本世纪的过程中的发展,主要探讨矩阵论而非线性空间理论的部分. 这是因为矩阵论的发展相对而言比较接地气,大部分从结果上看非常简捷明快. 本节提及的很多结论,在证明中都有众多数学分支如草蛇灰线穿行其中. 但是,一个初步的展现未必需要完整的表述,省略一些证明也不妨碍读者领略现代矩阵论研究的风采. 至于从线性代数真正意义上``生发开去''的东西,我们将留在下一节讨论.

\subsection{正定性:从数到矩阵,以及本世纪的矩阵论}

二次型在这份讲义当中尚未被置于中心地位. 往早期讲,它的研究缘起于费马大定理,而往晚近讲,二次型的代数理论研究几近一个完全的分支. 我们在此抛开Witt理论之类的东西不谈,暂且只就谈正定性及其与矩阵论的关系,对于二次空间(quadratic space)的研究,及二次型与二次互反律(quadratic reciprocity law)的关系等等主题,读者自可参考志村五郎(Goro Shimura)的书《二次型的算术》(\textit{Arithmetic of Quadratic Forms}).

首先考虑这个问题:如何判断一个多项式是非负的?从初中开始,我们就知道有配方法:如果能把一个多项式拆成一堆非负项的和,那么它当然就是非负的,这个方法在二次型正定性的讨论中也被多次用到. 很自然的,下面这个问题就被提出了:

\begin{quote}
    \kaishu
    是不是任意一个非负多项式都能被拆成一系列多项式的平方和?
\end{quote}

答案是否定的. 一个典型的反例在 Motzkin 1967 年出版的一本关于代数-几何不等式的书中给出:
\[ P(x, y, z) = x^6 + y^4z^2 + y^2z^4 -3x^2y^2z^2 \]

希尔伯特第十七问题就是这个问题的一个推广:
\begin{quote}
    \kaishu
    是不是任意一个非负多项式都能被拆成一系列有理函数的平方和?
\end{quote}

这样的推广终归是成立了,它的证明由 E. Artin 在 1927 年给出,C. N. Delzell 在 1984 年甚至给出了一个算法以构造这样的平方和. 但是,故事到这里还没有结束,把``数''推广到``矩阵'',有趣的事情发生了:

\begin{theorem}[Helton, 2002]\label{thm:16:helton2002}
    对称的矩阵多项式是半正定的,当且仅当它能被拆成一组矩阵多项式的平方和.
\end{theorem}

当然,我们需要澄清一些概念:

\begin{definition}
    一个 $n$ 元的矩阵单项式是一个形如 $am_1m_2 \cdots m_k$ 的连乘积,其中 $a \in \R$,$m_i$ 为矩阵变元 $x_j \enspace(j \leqslant n)$ 或其转置. 一个 $n$ 元的矩阵多项式是有限个 $n$ 元矩阵单项式的和.
\end{definition}

\begin{definition}
    一个 $n$ 元的矩阵多项式 $Q$ 的值域 $\im Q$ 被定义为

    \[ \bigcup_{m = 1}^\infty\left\{Q(A_1, A_2, \ldots, A_n) \mid A_i \in \mathbf{M}_m(\R)\right\} \]

    即在代入任意 $n$ 个 $m$ 阶方阵之后所能得到的结果.
\end{definition}

\begin{definition}
    称一个矩阵多项式 $Q$ 是对称的,如果所有 $\im Q$ 中的矩阵都是对称的;称它是半正定的,如果所有 $\im Q$ 中的矩阵都是半正定的.
\end{definition}

\begin{definition}
    所谓矩阵多项式的平方和,指的是如下形式的有限和:
    \[ P(x_1, x_2, \ldots, x_n) = \sum_{i = 1}^k h_i(x_1, x_2, \ldots, x_n)h_i^\mathrm{T}(x_1, x_2, \ldots, x_n) \]
    其中 $h_i$ 均为 $n$ 元矩阵多项式. 显然,这个平方和一定是半正定的.
\end{definition}

\autoref{thm:16:helton2002} 的证明已经超出了这份讲义所能探讨的范畴. 其原论文足有二十页,证明过程可谓相当详尽,有兴趣的读者可自行查阅.

下面我们再给出一个将数推广到矩阵的相关问题,它在某种意义上也与正定性有关,而且也是本世纪的成果. 最小二乘法在本讲义前面很早的地方就有介绍,同样将其推广到矩阵,就要求我们推广距离的概念:

\begin{definition}
    考虑 $n$ 阶正定矩阵 $A, B$,定义它们之间的迹度量距离(trace metric distance)为:
    \[ \delta(A, B) = \sqrt{\sum_{i = 1}^n \log^2\lambda_i(A^{-1}B)} \]
    其中 $\lambda_i(M)$ 表示矩阵 $M$ 的第 $i$ 个特征值.
\end{definition}

\begin{definition}
    $k$ 个矩阵 $A_1, A_2, \ldots, A_k$ 的 Karcher 均值(Karcher mean)指的是正定矩阵 $X$,使得 $X$ 到所有 $A_i$ 的迹度量距离的平方和最小,将其记作 $\sigma(A_1, A_2, \ldots, A_k)$
\end{definition}

这个概念由 Karcher 在 1973 年先引入到任意度量空间中,这里呈现的是 Moakher 在 2005 年将其引入到矩阵中的版本. 在 2006 年,Bhatia 和 Holbrook 合作的论文呈现了它的一系列性质,并给出了一个猜想. 这个猜想是对下面这个简单事实的推广:

\begin{lemma}
    考虑 $x_1, x_2, \ldots, x_n, y_1, y_2, \ldots, y_n \in \R$,且 $\forall i \leqslant n,\enspace x_i \leqslant y_i$. 记 $x, y$ 分别为使得
    \[ \sum_{i = 1}^n (x - x_i)^2, \quad \sum_{i = 1}^n (y - y_i)^2 \]
    取得极小值的 $x$ 和 $y$,则 $x \leqslant y$.
\end{lemma}

很显然,对吧?他们猜测,Karcher 均值也有这种美好的单调性,不过,首先要在正定矩阵上引进序关系:

\begin{definition}
    如果 $B - A$ 正定,则 $A \leqslant B$.
\end{definition}

\begin{theorem}[Bhatia-Holbrook, 2006, Lawson-Lim, 2011]
    考虑 $A_1, A_2, \ldots, A_n, B_1, B_2, \ldots, B_n$ 为 $m$ 阶正定矩阵,如果 $\forall i \leqslant n,\enspace A_i \leqslant B_i$,则
    \[ \sigma(A_1, A_2, \ldots, A_n) \leqslant \sigma(B_1, B_2, \ldots, B_n) \]
\end{theorem}

等等,为什么是定理?正如我们已经暗示的,这个结果由 Lawson 和 Lim 在 2011 年成功证明了,其证明用到了 Loewner-Heinz 全局非正曲率度量空间(Loewner-Heinz globally nonpositively curved metric spaces)的一些性质,事实上这就是正定矩阵集合全体的复杂性带来的.

提及特征值,我们不妨再看看另外一个不那么幸运的结果,它是一个被证伪的猜想.

\begin{definition}
    称一个方阵 $A$ 的谱半径(spectral radius)为其特征值的绝对值的最大者,记为 $\rho(A)$.
\end{definition}

这个定义是非常有用的,如果我们注意到以下两个定理:

\begin{theorem}
    对于可对角化的矩阵 $U$,以下式子成立:
    \[ \lVert Uv \rVert \leqslant \rho(U) \lVert v \rVert \]
\end{theorem}

这是谱定理的一个直接推论. 注意,如果 $U$ 不是可对角化的,那么这个式子不成立,读者可以自行尝试构造反例,应当不难.

\begin{theorem}
    \[ \rho(A_1A_2\cdots A_k) \leqslant \rho(A_1)\rho(A_2)\cdots \rho(A_k) \]
\end{theorem}

这是 Gelfand 公式的一个直接推论. 也就是说,如果我们限制了所有矩阵 $A_i$ 都是可对角化的(例如在许多物理过程中),可以通过谱半径的乘积来估计乘积的谱半径的上界,这对于很多连续进行线性变换的系统的估计都是非常关键的. 而且,下面这个定义也一样重要:

\begin{definition}[广义谱半径]
    设 $\Sigma$ 为有限个 $m$ 阶方阵的集合,称其广义谱半径(generalized spectral radius)为
    \[ \rho(\Sigma) = \varlimsup_{k \to \infty} \rho_k(\Sigma) \]
    其中
    \[ \rho_k(\Sigma) = \max\{\rho(A_1A_2 \cdots A_k) \mid A_i \in \Sigma,\enspace i \leqslant k\} \]
    $\varlimsup$ 意指上极限,即最大的聚点.
\end{definition}

这个定义的物理意义可以类似地理解:当我们有一大堆可能的线性变换的时候,进行趋近于无穷次的随机选取变换,最终结果的范数最大值很可能落在哪里. 当然,很容易看出来,对于任意的 $k$,都有 $\rho_k(\Sigma) \leqslant \rho(\Sigma)$. 这个倒霉猜想由 Lagarias 和 Wang 在 1995 年提出,被称为有限性猜想:

\begin{quote}
    \kaishu
    对于任意 $\Sigma$,存在 $k$ 使得 $\rho_k(\Sigma) = \rho(\Sigma)$.
\end{quote}

也就是说,广义谱范数可以在有限步内抵达. 但不幸的是,这个猜想在七年后被 Bousch 和 Mairesse 证伪了,证伪的方法事实上是直接构造了一个 $\Sigma$ 的例子. 但是,他们的构造在此如要呈现将花费过多的笔墨,因为他们完全是从迭代函数系统(iterative function system)出发完成构造的,要用到关于 Lyapunov 指数的一些知识. 当然,对于读完这份讲义的读者,这些内容并不会太困难,不妨找来他们的论文一探究竟.

顺便一提,证明对于绝对值最小的特征值来说类似的猜想不成立特别简单,构造的例子如下,读者不妨自行思考一下为什么.
\[
    \begin{bmatrix}
        2 & 0 \\ 0 & 1/2
    \end{bmatrix}, \quad \begin{bmatrix}
        1/3 & 0 \\ 0 & 3
    \end{bmatrix}
\]

\subsection{线性方程组的解:快一点,再快一点!}

这一节讨论的主要是如何求解一个线性方程组
\[ A\vec{x} = \vec{b} \]

当然,读者会说,高斯消元法嘛,第一讲就已经讲过了,如果 $A$ 可逆那就是 $\vec{x} = A^{-1}\vec{b}$ 呗. 也不算错,那如果 $A$ 干脆不是一个方阵呢?确切的解自然不存在,最小二乘解也由 Penrose-Moore 逆给出:

\begin{definition}
    考虑 $m \times n$ 矩阵 $A$,它的 Penrose-Moore 逆为一个 $n \times m$ 矩阵 $A^\dagger$ 满足以下条件:

    \begin{enumerate}
        \item $A A^\dagger A = A$

        \item $A^\dagger A A^\dagger = A^\dagger$

        \item $A^\dagger A$ 和 $AA^\dagger$ 都是对称(或者厄米)的
    \end{enumerate}

    这样的矩阵存在且唯一.
\end{definition}

当然,不难写出近似的精度的定义:

\begin{definition}
    称 $\widetilde{\vec{x}}$ 是 $A\vec{x} = \vec{b}$ 的一个精度为 $\varepsilon$ 的近似解,如果:
    \[ \lVert \widetilde{\vec{x}} - A^\dagger \vec{b} \rVert_A \leqslant \varepsilon \lVert A^\dagger \vec{b} \rVert_A \]
    其中
    \[ \lVert \vec{x} \rVert_A = \sqrt{\vec{x}^\mathrm{T}A\vec{x}} \]
    称为由 $A$ 诱导的范数.
\end{definition}

这是对于一般的矩阵的结果. 通常情况下,我们只要处理实对称矩阵,甚至还能伴随一个更强的条件:

\begin{definition} \index{ruozhuduijiao@弱主对角占优 (weakly diagonally dominant)}
    称矩阵 $(a_{ij})$ 是\term{弱主对角占优}的,如果
    \[ a_{ii} \geqslant \sum_{j \neq i} |a_{ij}| \]
    对于任意 $i$ 成立.
\end{definition}

实对称的主对角占优矩阵对应的线性方程组求解当然可以使用直接求逆的方法,但是由于在处理图论问题以及求解一些椭圆型偏微分方程时,它往往以非常大的规模出现,所以哪怕是非线性时间的算法,也显得比较缓慢了,而到达近线性的突破到现在还不到十年:

\begin{theorem}[Spielman-Teng, 2014]
    存在一个在任意精度 $\varepsilon$ 下求解实对称的主对角占优矩阵 $A$ 对应的方程组 $Ax = b$ 的随机算法,其时间复杂度为
    \[ O(m \log^c n \log(1 / \varepsilon)) \]
    其中 $m$ 为 $A$ 中的非零元个数,$c$ 为一常数.
\end{theorem}

我们无意在此详述这个算法如何运作. 事实上,稍有图论基础的读者应不难看懂原论文. 它的核心在于,使用好的图采样来稀疏化,进而优化求解.

\subsection{随机矩阵:吸引了陶哲轩的未解之谜}

既然提到随机算法,那不妨也介绍一下随机矩阵的相关研究. 这就不像前面几个领域在本世纪内的研究较为松散,反而是如火如荼,哪怕是将近十年的主要结果罗列一番也破费精力. 由于笔者对概率论不甚熟悉,也不曾深究这些结论的证明,在此,我们仅罗列一些结果,有些较新的结果的准确性可能需要明眼的读者自行甄别.

首先,何谓随机矩阵?我们这里只探讨最简单的模型:Bernoulli 矩阵.

\begin{definition}
    称一个矩阵为 Bernoulli 矩阵,如果它的每一个元素都是独立同分布、成功概率 $p = 1/2$ 的 Bernoulli 随机变量.
\end{definition}

第一个问题是,这玩意有多大概率是奇异的?简单观察一下,不难猜想,奇异性的来源也就是行或者列相同,因此,我们给出以下猜想:

\begin{theorem}[Tikhomirov, 2020] \label{thm:16:tik2020}
    一个 $n$ 阶 Bernoulli 矩阵奇异的概率为:
    \[ P_n = \left( \frac{1}{2} + o(1) \right)^n \]
\end{theorem}

这个结果的证明几经波折. 原始的猜想早在上世纪就被正式阐述过,早在 1967 年,Koml\'os 就已经表明,$\displaystyle\lim_{n \to \infty} P_n = 0$. 直到 1995 年,Kahn 等人才给出第一个指数级别的估计 $P_n = O(0.999^n)$. 在 2007 年,陶哲轩和 Van Vu 给出了 $P_n = O(0.958^n)$ 的估计并在第二年获得了如下突破:

\begin{theorem}[Tao-Vu, 2007]
    一个 $n$ 阶 Bernoulli 矩阵奇异的概率为:
    \[ P_n = \left( \frac{3}{4} + o(1) \right)^n \]
\end{theorem}

下一个突破性进展在 2010 年,Bourgain 等人进一步压下了上界:

\begin{theorem}[Bourgain, 2010]
    一个 $n$ 阶 Bernoulli 矩阵奇异的概率为:
    \[ P_n = \left( \frac{1}{\sqrt{2}} + o(1) \right)^n \]
\end{theorem}

最终形成了我们看到的\autoref{thm:16:tik2020}. 事实上,Tikhomirov 证明的结果比这个还要广一点:

\begin{theorem}[Tikhomirov, 2020]
    一个 $n$ 阶,成功概率为 $p \in (0, 1/2]$ 的 Bernoulli 矩阵奇异的概率为:
    \[ P_n = \left( 1 - p + o(1) \right)^n \]
\end{theorem}

Jain 等人在同年也给出了另一个关于任意有限支集的分布代替独立同分布的结果.

另外的问题还有:
\begin{enumerate}
    \item 对于高斯分布的矩阵,结果如何?

    \item 最大特征值,也就是谱半径的分布如何?(Tracy-Widom 规则及其推广)

    \item 所有特征值的分布如何?(Circular law conjecture,由陶哲轩在 2010 年证明)

    \item 范数和逆的范数的分布如何?(Spielman-Teng 猜想,2002 年被形式化提出,Littlewood-Offord 问题的反问题,陶哲轩和 Van Vu 在 2009 年给出了一个初步结果)

    \item 其它非独立同分布的随机模型如何?(关于随机带矩阵(random band matrix),Khorunzhiy 猜想在 2010 年被 Sodin 证明)
\end{enumerate}

\subsection{机器学习!}

当然咯,提到本世纪的线性代数,也不能不提及机器学习的一些发展. 我们仅就几个特定问题,讨论线性代数方法在机器学习中的应用. 这里提及的很多结论会比前面的结果更加轻松,亦可供对此感兴趣的读者仔细参详,甚至自行写出证明.

我们要谈的第一个问题叫做压缩感知(compressed sensing). 给定一个稀疏信号和有限次的测量,测量的次数很可能远少于要复原的变量的个数,这个时候,我们需要多少次测量才能以某个精度复原出原始信号?最初的结果来自于 Donoho 在 2006 年的一篇论文,在 Google 学术上,这篇论文被引次数在笔者写下这一小节时已经达到了惊人的 33115 次,很可能是本世纪引用次数最高的一篇数学论文. 我们下面来介绍一下这篇论文的主要结果,首先需要给出几个定义:

\begin{definition}
    一个向量 $x$ 的 $l_p$-范数定义为:
    \[ \lVert x \rVert_p = \left( \sum_{i}|x_i|^p \right)^{\frac{1}{p}} \]
    其中 $x_i$ 为其分量.
\end{definition}

\begin{definition}
    称一个向量是 $l_p$-稀疏的,如果它满足:
    \[ \lVert x \rVert_p < R \]
    其中 $R$ 为给定正常数,$0 < p \leqslant 1$.
\end{definition}

\section{未来:从线性代数出发能望到多远}

\begin{quote}

    \kaishu
    对于模或者其它具备某种线性操作的数学结构, 例如稍后要介绍的 Abel 范畴中的对象, 复形及其上同调的研究是通称为同调代数的学科的经典内容; 这是本真意义的``线性代数''的一个真子集, 也是本书核心.

    \begin{flushright}
        \kaishu
        ——李文威《代数学方法》卷二:线性代数
    \end{flushright}

\end{quote}

到底什么是线性代数?想必大部分读者都看过那张线性代数九宫格. 当然,手抓饼也可以是线性代数. 但是在此,为了使得我们的讨论不致过于冗长,准李文威老师的思路,我们以线性性作为基准,以此衡量何谓线性代数的发展. 一切事关线性性的探讨都可以被称为线性代数,而我们在这里仅取一瓢之水,希望读者窥一斑可见全豹.

\subsection{模作为线性空间的推广}

当然,我们说线性性,很显然的问题就是,什么样的东西是一个足以保线性性的东西?一个域上的线性空间,那当然可以. 它自身有 Abel 群结构,对于域的加法和数乘作用相容——等等,为什么一定要是一个域呢?域中的除法,以及域的交换性,这些东西看起来对线性性并没有多少意义. 如果你注意到了这一点,那么恭喜你,你已经能够写出模的定义了:

\begin{definition}
    置 $\langle R : +, \cdot \rangle$ 为一环,$\langle M : + \rangle$ 为一 Abel 群. 称 $M$ 为 $R$ 上的左模,如果我们能给它赋予一个映射 $R \times M \to M$ ,以左乘记,满足以下条件:

    \begin{itemize}
        \item $r(m_1 + m_2) = rm_1 + rm_2,\enspace \forall m_1, m_2 \in M,\enspace r \in R$

        \item $(r_1 + r_2)m = r_1m + r_2m,\enspace \forall m \in M,\enspace r_1, r_2 \in R$

        \item $(r_1r_2)m = r_1(r_2m),\enspace \forall m \in M,\enspace r_1, r_2 \in R$

        \item $1_Rm = m,\enspace \forall m \in M$
    \end{itemize}

    对称地,我们也可以定义右模. 事实上,右模无外乎反环 $R^{\mathrm{op}}$ 上的左模.
\end{definition}

对于模理论,深究起来亦破费一番功夫. 这里,我们只提示一个在笔者看来非常有用的性质:

\begin{theorem}[Freyd-Mitchel, 1964]
    任意小的 Abel 范畴都可以满、忠实且正合地嵌入到某个环 $R$ 的左模范畴中去.
\end{theorem}

深入介绍这个定理会让这一节变得相当复杂,或许需要单开一本书来讲. 但是,单从字面意义上讲,读者不难看出,这个意思就是,对 Abel 范畴的研究可以在某个环 $R$ 的左模范畴中完成,而事实上另有一个嵌入定理将另一个更广泛的结构嵌入到 Abel 范畴中:

\begin{theorem}[Quillen-Gabriel, 1972]
    任意一个小的 Quillen 正合范畴都可以被保正合地嵌入到 Abel 范畴中.
\end{theorem}

也就是说,$R$ 的左模范畴可以成为研究一类更广泛的结构的样板,这是其在范畴论中不可或缺的意义之一. 至于模自身的理论,也不可谓不庞大,而且在代数几何等领域的现代发展中发挥着重要的作用. 不过,为了让读者少听点天书,我们还是进入下一个主题吧. 如果读者有兴趣,自可参阅各种现代代数学发展相关的资料.

\subsection{张量积:一个过渡性章节}

张量积的构造其实无外乎线性映射的推广:

\begin{definition}
    置 $V, W, Z$ 为线性空间,称 $V \times W \to Z$ 是双线性的(bilinear),如果它关于 $V$ 和关于 $W$ 都是线性的.
\end{definition}

\begin{definition} \label{thm:16:tensorprod}
    $V$ 和 $W$ 的张量积 $V \otimes W$ 无外乎一种具备泛性质的双线性映射 $\varphi$,即满足对于任意双线性映射 $h$,存在唯一线性映射 $\widetilde h$ 使得下图交换:

    \begin{center}
        \begin{tikzcd}
            V \times W \ar[r, "\varphi"] \ar[rd, "h"] & V \otimes W \ar[d, dashed, "\widetilde h"] \\
            & Z
        \end{tikzcd}
    \end{center}

    也就是说,$\widetilde h \circ \varphi = h$.
\end{definition}

当然,为了确立这个定义的正确性,我们要保证这个空间 $V \otimes W$ 的唯一性. 事实上,它就相当于积空间商掉一些部分:

\begin{align*}
    R = \spa ( & (v_1 + v_2, w) - (v_1, w) - (v_2, w), \\
               & (v, w_1 + w_2) - (v, w_1) - (v, w_2), \\
               & (sv, w) - s(v, w),                    \\
               & (v, sw) - s(v, w))
\end{align*}

于是:
\[ V \otimes W \cong (V \times W) / R \]
不难验证这样给出的 $V \otimes W$ 满足\autoref{thm:16:tensorprod}.

\subsection{从代数到单子:往程序设计范式前进}

\subsection{线性化:群表示的艺术}

我们已经提及,线性性是一个非常好的性质. 对于一般的群,这种性质都是不可能存在的. 但是,如何给一个非线性的结构赋予线性结构呢?我们先看一个事实.

\begin{lemma}
    $\mathcal{L}(V)$ 关于复合操作构成一个群.
\end{lemma}

这个结果的验证应该颇为简单. 随后,我们自然想到,如果研究一个群到这个群的群同态,即那些保群乘法结构的映射,那么我们不久相当于把对群的研究线性化了吗?这样,我们就有了以下定义:

\begin{definition}
    从群 $G$ 到线性空间 $V$ 上的表示(representation)意指一个从 $G$ 到 $\mathcal{L}(V)$ 的群同态.
\end{definition}

通常地,我们会研究实表示和复表示,也就是把 $V$ 取为 $\R^n(\R)$ 和 $\C^n(\C)$. 我们将表示空间的维数称为表示的维数. 对表示论的研究很大程度上是对特征标的研究:

\begin{definition}
    考虑有限维表示 $\rho: G \to \mathcal{L}(V)$,$V$ 是域 $\mathbf{F}$ 上的向量空间,表示 $\rho$ 的特征标被定义为:
    \begin{align*}
        \chi_\rho \colon & G \to F                 \\
                         & g  \mapsto  \tr \rho(g)
    \end{align*}
\end{definition}

当然,对表示论的介绍也足以撑起一本巨著,在此,我们仅提示它在对群论研究中的一些意义,罗列一些用这种方式能够轻松证明,而用通常方法无法证明的结果:
\begin{itemize}
    \item Burnside 定理:所有 $p^aq^b$ 阶的群都是可解群. 这个结果就笔者所知尚未有不使用表示论做出的证明,事实上,表示论给出的证明相当简短.

    \item Feit-Thompson 定理:所有奇数阶群都是可解群. 这个结果的证明非常厚,笔者没看过,但是被其厚度震撼过.

    \item 有限单群分类定理,或称宏伟定理(the Grand theorem). 这是上世纪到本世纪群论最重要的结果之一,其主要部分证明就是应用了表示论引入的特征标理论.

    \item Ore 猜想:有限非 Abel 单群中的任意元素都是换位子. 这个猜想提出于 1951 年,在 2010 年,Liebeck 等人应用李型单群的 Deligne-Lusztig 不可约特征标理论将其完全攻克.
\end{itemize}

此外,表示论在许多物理学领域中发挥着关键性的作用. 一个典型的例子就是现代粒子物理学,其主要工具之一就是李群的表示论. 在量子物理中,哈密顿量的对称群的表示也能够引出多重态(multiplet)的概念,其中不可约的表示部分表征了系统的可能能级. 如果对于这方面的研究感兴趣,笔者(作为一个不怎么懂物理的人)推荐参考 GTM267,B. C. Hall 所著的《写给数学家的量子理论》(\textit{Quantum Theory for Mathematicians})第 17 章的内容.

这里我们另外指明很漂亮的一点,群的表示可以对应地用来建构多面体. 给定三维空间中的一个基向量 $v$,对其应用某有限群中各个元素的矩阵表示,我们可以很自然地得出一个多面体——通过这种方式,我们可以自然得到五种正多面体,它们是多面体点群的三维表示生成的.

如果读者对多面体足够熟悉,那么很自然的一个发现是,我们也可以取三个向量和原点出发构筑一个四面体,然后对这个四面体进行对称操作,即应用矩阵表示得到最终的结果. 事实上,在正多面体的情形下,我们得到的就是一个面的顶点、棱心、面心和原点构成的四面体. 因此,如果将其进行推广,取任意一个它的变形,都可以得到一个具备对应对称性的多面体. 这种方式往往也是计算机中表示一个具备某种对称性的多面体的方法,它的好处是可以利用群表示的方式节省所需的存储空间.

\subsection{拓扑向量空间:从布尔巴基学派的遗产走出}

\subsection{仿射簇,以及代数几何的问题}

\section*{附:本讲义未竟专题概览}

\section*{参考资料}

\begin{enumerate}
    \item \href{https://zh.wikipedia.org/wiki/%E4%BB%A3%E6%95%B0}{维基百科:代数}

    \item \href{https://zhuanlan.zhihu.com/p/574858845}{知乎:代数发展史}

    \item
\end{enumerate}
\vspace{2ex}
\centerline{\heiti \Large 内容总结}

\vspace{2ex}
\centerline{\heiti \Large 习题}

\vspace{2ex}
{\kaishu 如果我们想要预见数学的将来,适当的途径是研究这门科学的历史和现状.}
\begin{flushright}
    \kaishu
    ——庞加莱
\end{flushright}

\centerline{\heiti A组}
\begin{enumerate}
    \item
\end{enumerate}

\centerline{\heiti B组}
\begin{enumerate}
    \item
\end{enumerate}

\centerline{\heiti C组}
\begin{enumerate}
    \item
\end{enumerate}

\phantomsection
\section*{17 多项式}
\addcontentsline{toc}{section}{17 多项式}

\vspace{2ex}

\centerline{\heiti A组}
\begin{enumerate}
    \item
\end{enumerate}

\centerline{\heiti B组}
\begin{enumerate}
    \item
\end{enumerate}

\centerline{\heiti C组}
\begin{enumerate}
    \item
\end{enumerate}

\clearpage

\section*{18 不变子空间}
\addcontentsline{toc}{section}{18 不变子空间}

\vspace{2ex}

\centerline{\heiti A组}
\begin{enumerate}
    \item \begin{enumerate}
        \item $\forall\alpha\in\ker S$,有$S\alpha=0$,则$T(S\alpha)=S(T\alpha)=0$,因此$T\alpha\in\ker S$,即$\ker S$是$T$的不变子空间;
        \item $\forall\alpha\in\im S$,则$\exists\beta\in V, \alpha=S\beta$,则$T\alpha=T(S\beta)=S(T\beta)$,因此$T\alpha\in\im S$,即$\im S$是$T$的不变子空间.
    \end{enumerate}

    \item 由题意有$T(e_1)=2e_1$,$Te_2=e_1+2e_2$.
    \begin{enumerate}
        \item $W_1=\spa(e_1)$,而$T(e_1)=2e_1\in W_1$,故结论成立;
        \item 反证法;设$\mathbf{R}^2=W_1\oplus W_2=\spa(e_1)\oplus W_2$,则$W_2$也是$T$的一维不变子空间,设$W_2=\spa(\alpha)$,由于$\alpha\in V$,可设$\alpha=k_1e_1+k_2e_2,\enspace k_1,k_2\in\mathbf{R}$,且由直和可知$k_2\neq 0$. 设$T\alpha=l\alpha,\enspace l\in\mathbf{R}$,即$T\alpha=T(k_1e_1+k_2e_2)=(2k_1+k_2)e_1+2k_2e_2=l(k_1e_1+k_2e_2)$,比较系数得$k_2=0$,矛盾!
    \end{enumerate}

    \item \begin{enumerate}
        \item 对于$(x,0)\in U$,有$T(x,0)=(0,0)\in U$,故$U$是$T$的不变子空间,且$T\vert_U$是零变换;
        \item 假设存在,则$\dim W=1$,即$W$是一维不变子空间,则$W$中每个非零向量都是$T$的特征向量,但我们很容易求得$T$的特征值只有0,且对应的特征向量都在$U$中,故矛盾(或与上一大题的(2)使用类似方法);
        \item 对于$(x,y)\in\mathbf{F}^2$有$(T/U)((x,y)+U)=T(x,y)+U=(y,0)+U=0+U$,故$T/U$是$\mathbf{F}^2/U$上的零变换.
    \end{enumerate}

    \item 由于$\dim\im T=\dim V-\dim\ker T$,原式等价于
    \[ \dim E(\lambda_1,T)+\cdots+\dim E(\lambda_m,T)+\dim E(0,T)\leqslant\dim V, \]
    根据$\lambda_1,\cdots,\lambda_m$非零互异,以及代数重数大于等于几何重数,可知不等式成立.

    \item 若$k=\dim V$,由于代数重数大于等于几何重数,故$T$最多有$k$个特征值;若$k<\dim V$,则$T$最多有$k$个非零特征值,再加上0,故$T$最多有$k+1$个特征值.

    \item 根据正文中讨论的,$\sigma$的特征值与$A$的一致,因此只需求解特征多项式$f(\lambda)=|\lambda E-A|=\begin{vmatrix}
        \lambda-1 & -2 & -2 \\
        -2 & \lambda-1 & -2 \\
        -2 & -2 & \lambda-1
    \end{vmatrix}=(\lambda+1)^2(\lambda-5)=0$,解得特征值为-1(二重)和5. 解方程$AX=-X$,即$(-E-A)X=0$得基础解系为$\alpha_1=(1,0,-1)^T$,$\alpha_2=(0,1,-1)^T$,这是关于特征值-1的两个线性无关特征向量. 解方程$AX=5X$,即$(5E-A)X=0$得基础解系为$\alpha_3=(1,1,1)^T$,这是关于特征值5的特征向量. 事实上$\alpha_1,\alpha_2,\alpha_3$是$\sigma$的特征向量在基$1,x,x^2$下的坐标,则$\sigma$在特征值-1下的特征子空间为$\spa(1-x^2,x-x^2)$,在特征值5下的特征子空间为$\spa(1+x+x^2)$.

    \item 正文中提到$f(A)$的特征值为$f(\lambda)$,其中$f$为多项式,$\lambda$是$A$的任意特征值,则$B$的特征值为$\mu_i=\lambda_i^3-5\lambda_i^2(i=1,2,3)$,即$\mu_1=-4,\mu_2=-6,\mu_3=-12$,再结合行列式等于特征值之积得到$|B|=\mu_1\mu_2\mu_3=-288$. 而$A+5E$的特征值分别为$\lambda_i+5(i=1,2,3)$,得到$|A+5E|=(1+5)(-1+5)(2+5)=168$. 而$|5E+P^{-1}AP|=|5P^{-1}P+P^{-1}AP|=|P^{-1}||5E+A||P|=|5E+A|=168$.

    \item 根据正文对伴随矩阵特征向量的讨论可知,伴随矩阵的特征向量和原矩阵一致,则$\alpha$也是$A$的特征向量,则由题意有
    \[\begin{pmatrix}
        a & -1 & c \\ 5 & b & 3 \\ 1-c & 0 & -a
    \end{pmatrix}\begin{pmatrix}
        -1 \\ -1 \\ 1
    \end{pmatrix}=\mu\begin{pmatrix}
        -1 \\ -1 \\ 1
    \end{pmatrix},\]
    解线性方程组得$\mu=-1,b=-3,a=c$,则$A^*$对应于$\alpha$的特征值为$\dfrac{|A|}{\mu}=1$. 又由$|A|=\begin{vmatrix}
        a & -1 & a \\ 5 & -3 & 3 \\ 1-a & 0 & -a
    \end{vmatrix}=a-3=-1$可得$a=c=2$.

    \item \label{ex:18:交换基础} 由题意有$AX=\lambda_0X$,则$A(BX)=B(AX)=\lambda_0(BX)$,可见$BX\in V_{\lambda_0}$.
\end{enumerate}

\centerline{\heiti B组}
\begin{enumerate}
    \item \begin{enumerate}
        \item 必要性:显然,因为任何一个向量都是数乘变换的特征向量;
        \item 充分性:由题意,任意向量都必然是$T$的特征向量,由正文中特征向量性质中最后的例题可知$T$是数乘变换.
    \end{enumerate}

    \item \begin{enumerate}
        \item $(\sigma/(\im\sigma))(v+\im\sigma)=\sigma(v)+\im\sigma,\enspace\forall v\in V$,由于$\sigma(v)\in\im\sigma$,则$\sigma(v)+\im\sigma=0+\im\sigma,\enspace\forall v\in V$,则$\sigma/(\im\sigma)$是零映射;
        \item 回忆单射的充要条件是核空间只有出发空间的零元,即$\sigma/(\ker \sigma)(v+U)=\sigma(v)+\ker\sigma=\ker\sigma$当且仅当$v\in\ker\sigma$,即$\sigma(v)\in\ker\sigma$当且仅当$v\in\ker\sigma$. 这一点与$\ker \sigma\cap\im \sigma=\{0\}$等价,因为
        \begin{enumerate}
            \item 必要性:$\forall v\in\ker \sigma\cap\im \sigma=\{0\}$,则$\exists u\in V$使得$v=\sigma(u)$,又$v\in\ker\sigma$,则$\sigma(u)\in\ker\sigma$,由已知$\sigma(v)\in\ker\sigma$当且仅当$v\in\ker\sigma$可得$u\in\ker\sigma$,即$v=\sigma(u)=0$;
            \item 充分性:已知$\ker \sigma\cap\im \sigma=\{0\}$,若$\sigma(v)\in\ker\sigma$,由于$\sigma(v)\in\im\sigma$,则$\sigma(v)\in\ker\sigma\cap\im\sigma=\{0\}$,则$\sigma(v)=0$,即$v\in\ker\sigma$.另一方面,若$v\in\ker\sigma$,则$\sigma(v)=0\in\ker\sigma$显然成立.
        \end{enumerate}
    \end{enumerate}

    \item 设$\lambda$是$T/U$的特征值,则存在非零的$x+U(x\in v,x\notin U)$使得$(T/U)(x+U)=\lambda(x+U)$,即$Tx+U=\lambda x+U$,则$Tx-\lambda x\in U$. 若$\lambda$是$T\vert_U$的特征值,则得证;若不是,则$T\vert_U-\lambda I$可逆(则在有限维线性空间条件下是满射),因此存在$y\in U$使得$(T\vert_U-\lambda I)y=Tx-\lambda x$,即$Ty-\lambda y=Tx-\lambda x$,即$T(x-y)=\lambda(x-y)$,又$x\notin U$,$y\in U$,则$x-y\neq 0$,故$\lambda$是$T$的特征值.

    \item \begin{enumerate}
        \item 一维不变子空间就是特征子空间,由题意可知$T$有$n$个互异特征值,因此每个特征子空间都是一维不变子空间,因此$T$的所有一维不变子空间就是
        \[\spa(e_1),\cdots,\spa(e_n);\]
        \item 事实上,我们很容易发现$\{0\},\spa(e_i),\spa(e_i,e_j),\cdots,\spa(e_1,\cdots,e_n)=V$都是$T$的不变子空间(因为每个$e_i$都是特征向量,所以很容易验证),这里一共有$C_n^0+C_n^1+\cdots+C_n^n=2^n$个不变子空间. 接下来说明$T$的不变子空间只有这$2^n$个. 设$W$是$T$的任一非零不变子空间,且$T$在$W$的基$\beta_1,\cdots,\beta_m$下的矩阵为$A$. 将$W$的基扩充为$V$的基$\beta_1,\cdots,\beta_m,\beta_{m+1},\cdots,\beta_n$,则$T$在$V$的基$\beta_1,\cdots,\beta_n$下的矩阵为
        \[B=\begin{pmatrix}
            A & C \\ O & D
        \end{pmatrix},\]
        则$T$的特征多项式据行列式运算性质可知为$|\lambda E-B|=\begin{vmatrix}
            \lambda E-A & -C \\ O & \lambda E-D
        \end{vmatrix}=|\lambda E-A||\lambda E-D|$,而$T\vert_W$的特征多项式为$|\lambda E-A|$整除$|\lambda E-B|$,由于$T$有$n$个互异特征值,这意味着$T\vert_W$也有$m$个互异特征值. 取其中任一特征值$\mu$,则存在$\beta$使得$T\vert_W\beta=\mu\beta$,即$T\beta=\mu\beta$,则$\mu$也是$T$的一个特征值,即$\mu$等于某个$\lambda_i$,即$T\beta=\lambda_i\beta$,于是$\beta$就是$\spa(e_i)$中的元素,因此某个$ke_i\in W(k\neq 0)$,故因为$W$是子空间,由运算封闭有$\spa(e_i)\subseteq W$. 从而我们也可以知道$T\vert_W$的$m$个互不相同的特征值只能是$\lambda_{i_1},\cdots,\lambda_{i_m}$,且$\spa(e_{i_1},\cdots,e_{i_m})\subseteq W$(因为每个$\spa(e_{i_k})$都在$W$中),由于$\dim W=m$,故$W=\spa(e_{i_1},\cdots,e_{i_m})$,结论得证.
    \end{enumerate}

    \item 本题探讨实数域上的二维线性空间,则不变子空间的维数只能为0,1,2,而维数为0和2对应$\{0\}$和$V$本身,这是无论$a$取何值时都一定有的,而一维不变子空间实际上就是单个特征向量张成的子空间,因此我们可以先求特征值:$|\lambda E-A|=\lambda^2+a-1$,则有如下讨论:
    \begin{enumerate}
        \item $a>1$,实数域上无特征值,因此所有不变子空间就是$\{0\}$和$V$本身;
        \item $a=1$,特征值为0,解得$A$的特征向量为$k(1,0)^T(k\in\mathbb{R})$,则$T$对应的不变子空间为$\spa(\alpha_1)$,当然不要忘记还有$\{0\}$和$V$本身;
        \item $a<1$,分别求解两个互异特征值的特征向量可以得到所有不变子空间为$\{0\}$,$\spa(\alpha_1+\alpha_2\sqrt{1-a})$,$\spa(\alpha_1-\alpha_2\sqrt{1-a})$和$V$本身.
    \end{enumerate}

    \item 由题意显然$A$和$B$不满秩(秩小于$n$),因此$A$和$B$有公共特征值0,且对应特征值0的特征子空间分别为$AX=0$和$BX=0$的解空间,它们的维数分别为$n-r(A)$和$n-r(B)$,二者维数和为$2n-(r(A)+r(B))>n$,因此它们的交集非零,即存在$X\neq 0$使得$AX=BX=0$,即$X$是$A$和$B$的共同的特征向量.

    \item 根据正文中的叙述,$A^2$的特征值就是$A$特征值的平方,此处不再赘述. 而$\displaystyle\sum_{i=1}^{n}\lambda_i^2$就是$A^2$特征值之和,即为$A^2$对角线元素之和,实际上简单验算就知道结果等于$\displaystyle\sum_{j=1}^{n}\displaystyle\sum_{k=1}^{n}a_{jk}a_{kj}$.

    \item 由题意有$(A-kE)X_1=0$,$(A-kE)X_2=lX_1$,$(A-kE)X_3=lX_2$. 设$m_1X_1+m_2X_2+m_3X_3=0$,两边左乘$A-kE$可得$m_2lX_1+m_3lX_2=0$,由于$l\neq 0$可知$m_2X_1+m_3X_2=0$. 两边再次左乘$A-kE$可得$lm_3X_1=0$,可知$m_3=0$,往前代入可知$m_2=m_1=0$,故$X_1,X_2,X_3$线性无关.
\end{enumerate}

\centerline{\heiti C组}
\begin{enumerate}
    \item 设$B$的特征值为$\lambda_1,\cdots,\lambda_n$,则$B$的特征多项式为$f(\lambda)=|\lambda E-B|=(\lambda-\lambda_1)\cdots(\lambda-\lambda_n)$,则$f(A)=(A-\lambda_1E)\cdots(A-\lambda_nE)$. $f(A)$可逆充要条件是$|f(A)|\neq 0$,即$|A-\lambda_iE|\neq 0,\enspace i=1,\cdots,n$,这等价于$A$的特征值不是$\lambda_1,\cdots,\lambda_n$,即$B$的特征值都不是$A$的特征值.

    \item 根据A组\ref*{ex:18:交换基础}题,$BX\in V_{\lambda_0}(A)$,故只需证存在$Z\neq 0$使得$Z\in V_{\lambda_0}(A)$且$BZ=\mu Z$,此时$Z$是$A$和$B$共同的特征向量. 设$X_1,\cdots,X_r$为$V_{\lambda_0}(A)$的基,由于$BX_i\in V_{\lambda_0}(A)$,则$BX_i$可以由$X_1,\cdots,X_r$线性表出,即$BX_i=(X_1,\cdots,X_r)\alpha_i$,其中$\alpha_i=(a_{i1},\cdots,a_{ir})^T$. 因此有
    \[B(X_1,\cdots,X_r)=(X_1,\cdots,X_r)(\alpha_1,\cdots,\alpha_r)=(X_1,\cdots,X_r)P,\]
    其中$r$阶矩阵$P=(\alpha_1,\cdots,\alpha_r)$在复数域上有特征值$\mu$,故存在$Y_0\neq 0$使得$PY_0=\mu Y_0$. 将上式两端右乘$Y_0$,得
    \[B(X_1,\cdots,X_r)Y_0=(X_1,\cdots,X_r)(PY_0)=(X_1,\cdots,X_r)(\mu Y_0)=\mu(X_1,\cdots,X_r)Y_0.\]
    令$Z_0=(X_1,\cdots,X_r)Y_0$,则$Z_0\neq 0$(因为坐标$Y_0\neq 0$)且$BZ_0=\mu Z_0$,即$Z_0$是$B$的特征向量. 又$Z_0=(X_1,\cdots,X_r)Y_0$,即是可由$V_{\lambda_0}$的基线性表示的,故$Z_0\in V_{\lambda_0}$,即$Z_0$是$A$的特征向量. 因此$Z_0$是$A$和$B$共同的特征向量.
\end{enumerate}

\clearpage

\section*{19 相似标准形(I)}
\addcontentsline{toc}{section}{19 相似标准形(I)}

\vspace{2ex}

\centerline{\heiti A组}
\begin{enumerate}
    \item 非常简单,可以举$\begin{pmatrix}
        1 & 0 \\ 0 & 1
    \end{pmatrix}$和$\begin{pmatrix}
        1 & 0 \\ 0 & 2
    \end{pmatrix}$的例子,也可以举一个可对角化另一个不可对角化的例子,因为上述情况矩阵对应的相似标准形不一样,故不相似.

    \item 解$|\lambda E-A|=0$可得特征值为$1,1,-2$. 然后解$(E-A)X=0$得$X=t_1(-1,1,0)^\mathrm{T}+t_2(1,0,1)^\mathrm{T}$;解$(-2E-A)X=0$,得$X=t(-1,-1,1)^\mathrm{T}$. 可见特征值1对应的特征子空间为$\spa((-1,1,0)^\mathrm{T},(1,0,1)^\mathrm{T})$,特征值$-2$对应的特征子空间为 $\spa((-1,-1,1)^\mathrm{T})$. 可知与$A$相似的对角矩阵为 $\mathrm{diag}(1,\;1,\;-2)$.

    \item \begin{enumerate}
        \item $f(\lambda)=|\lambda E-A|=\begin{vmatrix}
            \lambda-a & -b \\ -c & \lambda-d
        \end{vmatrix}=\lambda^2-(a+d)\lambda+(ad-bc)=\lambda^2-(a+d)\lambda+|A|$. 由于$|A|<0$,因此判别式$\Delta=(a+d)^2-4|A|>0$,因此二阶矩阵$A$有两个互异特征值,故可对角化,因此与对角矩阵相似;
        \item 同(1),判别式$\Delta=(a+d)^2-4|A|>0$,因此二阶矩阵$A$有两个不同的特征值,故可对角化.
    \end{enumerate}

    \item \begin{enumerate}
        \item 由Cramer法则,$A$为方阵且$AX=\beta$有解但不唯一,即$|A|\neq 0$,解得$a=-2$或$a=1$,分别代入可知$a=1$时增广矩阵的秩为2,而$r(A)=1$,故无解;$a=-2$时增广矩阵的秩为2,而$r(A)=2$,故有解,故$a=-2$.
        \item 容易求得特征值为$0,-3,3$,求特征向量可知$P=\begin{pmatrix}
            1 & -1 & 1 \\ 1 & 0 & -2 \\ 1 & 1 & 1
        \end{pmatrix}$.
    \end{enumerate}

    \item 由分块矩阵乘法,$A(\alpha_1,\alpha_2,\alpha_3)=(A\alpha_1,A\alpha_2,A\alpha_3)=(\alpha_1,\alpha_1+\alpha_2-2\alpha_3,\alpha_1-2\alpha_2+\alpha_3)=(\alpha_1,\alpha_2,\alpha_3)\begin{pmatrix}
        1 & 1 & 1 \\ 0 & 1 & -2 \\ 0 & -2 & 1
    \end{pmatrix}$. 令$P=(\alpha_1,\alpha_2,\alpha_3)$,则$AP=P\begin{pmatrix}
        1 & 1 & 1 \\ 0 & 1 & -2 \\ 0 & -2 & 1
    \end{pmatrix}=PB$,即$P^{-1}AP=B$,故$A$与$B$相似,故两矩阵的特征值相同,简单求解$B$的特征值即得$A$的特征值为$1,-1,3$.

    \item 每行元素之和为3,则我们知道$\alpha=(1,1,1)^\mathrm{T}$是3对应的特征向量(具体理由只需要验证$A\alpha=3\alpha$即可,这是很常用的性质),而$AX=0$的解是对应特征值0的特征向量,故由题意可知取$P=\begin{pmatrix}
        1 & -1 & 0 \\ 1 & 2 & -1 \\ 1 & -1 & 1
    \end{pmatrix}$,则$P^{-1}AP=\begin{pmatrix}
        3 & 0 & 0 \\ 0 & 0 & 0 \\ 0 & 0 & 0
    \end{pmatrix}$.
\end{enumerate}

\centerline{\heiti B组}
\begin{enumerate}
    \item 回忆例题中$A^2=2A$,即$A(A-2E)=O$的证明,我们知道$A$的特征值只能为$a$或$b$,然后我们可以证明$r(aE-A)+r(bE-A)\leqslant n$和$r(aE-A)+r(bE-A)\geqslant n$,因此$r(aE-A)+r(bE-A)=n$,即$A$可对角化(由于和例题完全类似,这里不再展开具体做法).

    \item 三阶矩阵如果有三个特征值则一定可对角化,故由题意$T$只能有6和7两个特征值. 故$\lambda\neq 6,7$时$T-\lambda I$都是可逆的(回顾正文中的定理),故$T-8I$可逆(故是满射),因此一定存在$(x,y,z)\in\mathbf{C}^3$使得$(T-8I)(x,y,z)=(17,\sqrt{5},2\pi)$,因此存在$(x,y,z)\in\mathbf{C}^3$使得$T(x,y,z)=(17+8x,\sqrt{5}+8y,2\pi+8z)$.

    \item 可对角化矩阵特征值(包括重数)相等则相似标准形(对角矩阵)相等,故一定相似. 不可对角化的举例说明不成立:$\begin{pmatrix}
        0 & 1 & 0 \\ 0 & 0 & 1 \\ 0 & 0 & 0
    \end{pmatrix}$和$\begin{pmatrix}
        0 & 1 & 0 \\ 0 & 0 & 0 \\ 0 & 0 & 0
    \end{pmatrix}$的特征值全为0(重数均为3),但它们对应不同的若当标准形(之后章节会介绍),因此不相似.

    \item 直接计算$|\lambda E-A|=0$,得到$A$的特征值为$-2,6,6$,因为$A$可对角化,则特征值6对应的特征子空间为2维,即$(6E-A)X=0$的解空间为2维,根据齐次线性方程组一般理论,$r(6E-A)=3-2=1$,即$r\begin{pmatrix}
        4 & -2 & 0 \\ -8 & 4 & -a \\ 0 & 0 & 0
    \end{pmatrix}=1$,显然$a=0$. 接下来求解-2和6对应的特征向量即可,得到可行的解$P=\begin{pmatrix}
        -1 & 1 & 0 \\ 2 & 2 & 0 \\ 0 & 0 & 1
    \end{pmatrix}$.

    \item \begin{enumerate}
        \item 上三角矩阵特征值就是所有主对角元$a_{ii}(i=1,2,\cdots,n)$;
        \item 由(1)知主对角元互不相等表示$A$有$n$个互不相等的特征值,故可对角化;
        \item 由(1)知此时$A$有一个$n$重特征值,记为$c$,则所有特征向量都在$(cE-A)X=0$的解空间中,而$r(cE-A)=1$,故解空间维数为$n-1<n$,故$A$不可对角化.
    \end{enumerate}

    \item \begin{enumerate}
        \item $A=\begin{pmatrix}
            2 & -2 & 0 \\ -2 & 1 & -2 \\ 0 & -2 & 0
        \end{pmatrix}$;
        \item $B=\begin{pmatrix}
            0 & -2 & 0 \\ -3 & 1 & -2 \\ 1 & -1 & 2
        \end{pmatrix}$;
        \item 根据讲义本讲开头的定理(线性变换在不同基下的表示),$C_1$实际上就是基$\{\alpha_1,\alpha_2,\alpha_3\}$变为自然基$\{e_1,e_2,e_3\}$的过渡矩阵,简单计算可得$\begin{pmatrix}
            1 & 0 & 0 \\ -1 & 1 & 0 \\ 0 & -1 & 1
        \end{pmatrix}$.
    \end{enumerate}

    \item 不难解得$A$和$B$的特征值均为$-1,0,1$,因此它们都与$\begin{pmatrix}
        -1 & 0 & 0 \\ 0 & 0 & 0 \\ 0 & 0 & 1
    \end{pmatrix}$相似,即$A$与$B$相似. 我们也可以解得$P_1=\begin{pmatrix}
        -1 & 0 & 1 \\ 0 & 1 & 0 \\ 1 & 0 & 1
    \end{pmatrix}$和$P_2=\begin{pmatrix}
        0 & 0 & 1 \\ -1 & -2 & 0 \\ 1 & 1 & 0
    \end{pmatrix}$使得$P_1^{-1}AP_1=P_2^{-1}BP_2=\begin{pmatrix}
        -1 & 0 & 0 \\ 0 & 0 & 0 \\ 0 & 0 & 1
    \end{pmatrix}$,故$(P_1P_2^{-1})^{-1}A(P_1P_2^{-1})=B$,即题目要求的$P=P_1P_2^{-1}=\begin{pmatrix}
        1 & -1 & -2 \\ 0 & -1 & -1 \\ 1 & 1 & 2
    \end{pmatrix}$.

    \item \begin{enumerate}
        \item $A$与$B$相似则特征值相等,我们知道矩阵的对角线元素之和等于特征值之和,行列式等于特征值之积,则它们也一定相等. 根据这一原理非常容易解得$x=0,y=1$.
        \item 事实上$B$就是对角矩阵,实际上这里就是求过渡矩阵$P$使$A$对角化,具体步骤不再赘述,得到$P=\begin{pmatrix}
            1 & 0 & 0 \\ 0 & 1 & 1 \\ 0 & 1 & -1
        \end{pmatrix}$.
    \end{enumerate}

    \item 令$A_1=\begin{pmatrix}
        1 & 2 \\ 4 & 3
    \end{pmatrix}$,$A_2=\begin{pmatrix}
        1 & -3 & 3 \\ 3 & -5 & 3 \\ 6 & -6 & 4
    \end{pmatrix}$,则$A=\begin{pmatrix}
        A_1 & O \\ O & A_2
    \end{pmatrix}$. 首先得到$A_1$和$A_2$的特征值分别为$-1,5$和$-2,-2,4$. 分别求过渡矩阵$P_1$和$P_2$使得$A_1$,$A_2$对角化,容易解得$P_1=\begin{pmatrix}
        -1 & 1 \\ 1 & 2
    \end{pmatrix}$,$P_2=\begin{pmatrix}
        1 & 1 & 1 \\ 1 & 0 & 1 \\ 0 & -1 & 2
    \end{pmatrix}$. 令$P=\begin{pmatrix}
        P_1 & O \\ O & P_2
    \end{pmatrix}$,则
    \[P^{-1}AP=\begin{pmatrix}
        P_1^{-1}A_1P_1 & O \\ O & P_2^{-1}A_2P_2
    \end{pmatrix}=\begin{pmatrix}
        -1 & 0 & 0 & 0 & 0 \\ 0 & 5 & 0 & 0 & 0 \\ 0 & 0 & -2 & 0 & 0 \\ 0 & 0 & 0 & -2 & 0 \\ 0 & 0 & 0 & 0 & 4
    \end{pmatrix},\]
    即$P^{-1}AP=\mathrm{diag}(-1,5,-2,-2,4)$,故$A$与$\diag(-1,5,-2,-2,4)$相似. 总之,$A$的特征值为$-1,5,-2,-2,4$,过渡矩阵$P=\begin{pmatrix}
        P_1 & O \\ O & P_2
    \end{pmatrix}=\begin{pmatrix}
        -1 & 1 & 0 & 0 & 0 \\ 1 & 2 & 0 & 0 & 0 \\ 0 & 0 & 1 & 1 & 1 \\ 0 & 0 & 1 & 0 & 1 \\ 0 & 0 & 0 & -1 & 2
    \end{pmatrix}$.

    (注:通过本题也可以看出分块对角矩阵的特征值是两个对角块矩阵的并集,过渡矩阵就是两个过渡矩阵按同样分块方式排列得到的矩阵.)

    \item 回顾对角化过程可知,令$P=(\xi_1,\xi_2,\xi_3)=\begin{pmatrix}
        1 & 1 & 1 \\ 1 & 2 & 3 \\ 1 & 4 & 9
    \end{pmatrix}$,则$P^{-1}AP=\begin{pmatrix}
        1 & 0 & 0 \\ 0 & 2 & 0 \\ 0 & 0 & 3
    \end{pmatrix}$,即$P^{-1}A^nP=\begin{pmatrix}
        1 & 0 & 0 \\ 0 & 2^n & 0 \\ 0 & 0 & 3^n
    \end{pmatrix}$,故$A^n=P\begin{pmatrix}
        1 & 0 & 0 \\ 0 & 2^n & 0 \\ 0 & 0 & 3^n
    \end{pmatrix}P^{-1}$,最后可以计算得到$A^n\beta=\begin{pmatrix}
        2-2^{n+1}+3^n \\ 2-2^{n+2}+3^{n+1} \\ 2-2^{n+3}+3^{n+2}
    \end{pmatrix}$.

    \item 设$A=\begin{pmatrix}
        A_1 & O \\ O & A_2
    \end{pmatrix}$,其中$A_1=\begin{pmatrix}
        3 & 4 \\ 4 & -3
    \end{pmatrix}$,$A_2=\begin{pmatrix}
        2 & 4 \\ 0 & 2
    \end{pmatrix}$. 回顾矩阵运算进阶中介绍的求幂方法,我们发现$A_2=2E+B$,其中$B=\begin{pmatrix}
        0 & 4 \\ 0 & 0
    \end{pmatrix}$,且$B^2=O$,因此$A_2^n=(2E+B)^n=2^nE+n2^{n-1}B=\begin{pmatrix}
        2^n & n2^{n+1} \\ 0 & 2^n
    \end{pmatrix}$. 再看$A_1$,我们用对角化方法求幂,容易求得$P=\begin{pmatrix}
        2 & -1 \\ 1 & 2
    \end{pmatrix}$使得$P^{-1}A_1P=\begin{pmatrix}
        5 & 0 \\ 0 & -5
    \end{pmatrix}$,故$A_1^n=P\begin{pmatrix}
        5^n & 0 \\ 0 & (-5)^n
    \end{pmatrix}P^{-1}=5^{n-1}\begin{pmatrix}
        4+(-1)^n & 2+2(-1)^{n+1} \\ 2+2(-1)^{n+1} & 1+4(-1)^n
    \end{pmatrix}$,最后结合上面的讨论得到$A^n=\begin{pmatrix}
        A_1^n & O \\ O & A_2^n
    \end{pmatrix}$.

    \item \begin{enumerate}
        \item 由题意,$PB=AP$,利用分块矩阵乘法可知$PB=AP=(AX,A^2X,A^3X)=(AX,A^2X,3AX-2A^2X)$(注意$(AX,A^2X,A^3X)B=(AXB,A^2XB,A^3XB)$是错误的,因为这是$1\times 3$分块和$1\times 1$分块相乘,显然不能这么乘).

        则$PB=(AX,A^2X,3AX-2A^2X)=(X,AX,A^X)\begin{pmatrix}
            0 & 0 & 0 \\ 1 & 0 & 3 \\ 0 & 1 & -2
        \end{pmatrix}$. 等号两边左乘$P^{-1}$有$B=\begin{pmatrix}
            0 & 0 & 0 \\ 1 & 0 & 3 \\ 0 & 1 & -2
        \end{pmatrix}$.
        \item 由于$A$与$B$相似,故有相同的特征值,易求得$B$的特征值为$0,-3,1$,则$A$的特征值也为$0,-3,1$,则$A+E$的特征值根据正文例题可知为$1,-2,2$,则$A+E$的行列式为其特征值之积,即为-4.
    \end{enumerate}

    % \item 思路:将$B$分块为$\begin{pmatrix}
    %     B_1 & O \\ O & B_2
    % \end{pmatrix}$,其中$B_1=\begin{pmatrix}
    %     1 & 0 \\ -1 & 1
    % \end{pmatrix}$,$B_2=\begin{pmatrix}
    %     1 & -1 & -1 \\ -2 & -2 & 2 \\ -3 & -3 & 3
    % \end{pmatrix}$. 我们知道$B^n=\begin{pmatrix}
    %     B_1^n & O \\ O & B_2^n
    % \end{pmatrix}$. $B_1$可以直接用数学归纳法证明(或者这就是一个倍加行变换,幂次是非常显然的),$B_2$使用对角化方法求解幂次,较为复杂,此处省略结果,了解过程即可.
\end{enumerate}

\centerline{\heiti C组}
\begin{enumerate}
    \item 暂略,有需要可以参考\href{https://linearalgebras.com/5c.html}{《线性代数应该这样学》作者给出的答案(见网页第五题)};

    \item \begin{enumerate}
        \item $B^2=\alpha(\alpha^\mathrm{T}\alpha)\alpha^\mathrm{T}=m\alpha\alpha^\mathrm{T}=mB$,其中$m=\alpha^\mathrm{T}\alpha=a_1^2+\cdot+a_n^2$,且易知这就是$\tr(B)$. 用归纳法可知$B^k=m^{k-1}B$,令$t=m^{k-1}$则得证,其中$t=(\tr(B))^{k-1}$.
        \item 这是经典的秩1矩阵,由于$B$的秩为1,则$BX=0$的解空间维数为$n-1$,因此0是$B$的$n-1$重特征值. 而特征值之和等于矩阵的迹。则剩下一个一重特征值为$\tr(B)$(由题意迹不为0).

        然后求特征向量,这里需要一些观察. 特征值0对应的特征向量即为$BX=0$的解,设解为$(x_1,\cdots,x_n)$,不失一般性地,设$a_1\neq 0$,由于$B$的秩为1,因此将$BX=0$展开为线性方程组后,每一行(除去全0的)代表的方程是完全一致的(因为它们成比例),因此我们只需考虑第一行方程
        \[a_1^2x_1+a_1a_2x_2+\cdots+a_1a_nx_n=0,\]
        解得$n-1$个线性无关特征向量为
        \[X_1=(-a_2,a_1,0,\cdots,0),X_2=(-a_3,0,a_1,0,\cdots,0),\cdots,X_{n-1}=(-a_n,0,\cdots,0,a_1).\]
        对于特征值$\tr(B)=\alpha^\mathrm{T}\alpha$,我们可以设特征向量为$X$,则$(\tr(B)E-B)X=(\alpha^\mathrm{T}\alpha E-\alpha\alpha^\mathrm{T})X=0$,即$\alpha^\mathrm{T}X=\alpha\alpha^\mathrm{T}X$,我们可以观察发现$\alpha$就是一个解,则$X_n=\alpha$是其特征向量. 则$P=\begin{pmatrix}
            -a_2 & -a_3 & \cdots & -a_n & a_1 \\ a_1 & 0 & \cdots & 0 & a_2 \\ 0 & a_1 & \cdots & 0 & a_3 \\ \vdots & \vdots & \ddots & \vdots & \vdots \\ 0 & 0 & \cdots & a_1 & a_n
        \end{pmatrix}$,且对角矩阵为$\diag(0,\cdots,0,\alpha^\mathrm{T}\alpha)$.
    \end{enumerate}

    \item \begin{enumerate}
        \item 由题意知$A$的秩为1,因此特征值0对应的特征子空间为$AX=0$的解空间,维数为$n-1$,故特征值0的重数至少为$n-1$. 剩下可能还有一个特征值,我们知道特征值之和等于对角线元素之和,因此最后一个特征值等于$n$,$n$对应的特征子空间维数至少为1,结合0对应特征子空间维数为$n-1$可知$n$对应的特征子空间维数就是1,二者可以张成整个$\mathbf{R}^n$,因此可以对角化.

        下面求过渡矩阵,$n$对应的特征向量只需解方程$(nE-A)X=0$,即得$X_1=(1,1,\cdots,1)^\mathrm{T}$. 0对应的$n-1$个线性无关特征向量就是$AX=0$的基础解系,即$X_2=(-1,1,0,\cdots,0)^\mathrm{T},X_3=(-1,0,1,0,\cdots,0)^\mathrm{T},\cdots,X_n=(-1,0,\cdots,0,1)^\mathrm{T}$,令$P=(X_1,X_2,\cdots,X_n)=\begin{pmatrix}
            1 & -1 & -1 & \cdots & -1 \\ 1 & 1 & 0 & \cdots & 0 \\ 1 & 0 & 1 & \cdots & 0 \\ \vdots & \vdots & \vdots & \ddots & \vdots \\ 1 & 0 & 0 & \cdots & 1
        \end{pmatrix}$,则有$P^{-1}AP=\Lambda=\diag(n,0,\cdots,0)$.

        \item 由上一小问可知$A=P\Lambda P^{-1}$,则$A^k=P\Lambda^kP^{-1}$,故
        \[A^k=P\Lambda^kP^{-1}=P\diag(n^k,0,\cdots,0)P^{-1},\]
        故
        \begin{align*}
            f(A)&=a_mA^m+a_{m-1}A^{m-1}+\cdots+a_1A \\
            &=P(a_m\Lambda^m+a_{m-1}\Lambda^{m-1}+\cdots+a_1\Lambda)P^{-1} \\
            &=P\begin{pmatrix}
                a_mn^m+a_{m-1}n^{m-1}+\cdots+a_1n & 0 & \cdots & 0 \\ 0 & 0 & \cdots & 0 \\ \vdots & \vdots & \ddots & \vdots \\ 0 & 0 & \cdots & 0
            \end{pmatrix}P^{-1} \\
            &=\begin{pmatrix}
                f(n) & 0 & \cdots & 0 \\ 0 & 0 & \cdots & 0 \\ \vdots & \vdots & \ddots & \vdots \\ 0 & 0 & \cdots & 0
            \end{pmatrix}=\dfrac{f(n)}{n}P\begin{pmatrix}
                n & 0 & \cdots & 0 \\ 0 & 0 & \cdots & 0 \\ \vdots & \vdots & \ddots & \vdots \\ 0 & 0 & \cdots & 0
            \end{pmatrix}P^{-1} \\
            &=\dfrac{f(n)}{n}P\Lambda P^{-1}=kA.
        \end{align*}
        其中$k=\dfrac{f(n)}{n}$. 故$f(A)=kA$.

        \item 显然$X=(1,1,x\cdots,1)^\mathrm{T}$是$B$关于特征值$b$的特征向量,由于$b$是特征多项式的单根,则特征子空间维数也为1,故特征向量就是$kX$(其中$k$为非零常数).

        又$f(B)=(B-bE)g(B)=O$,则$Bg(B)=bg(B)$. 令$g(B)$的列向量组为$\alpha_1,\cdots,\alpha_n$,则上式可以写为$B(\alpha_1,\cdots,\alpha_n)=(b\alpha_1,\cdots,b\alpha_n)$,即$B\alpha_i=b\alpha_i$,即$\alpha_i$是$B$关于特征值$b$的特征向量,故$\alpha_i=k_iX=(k_i,k_i,\cdots,k_i)^{\mathrm{T}}$,即$g(B)=\begin{pmatrix}
            k_1 & k_2 & \cdots & k_n \\ k_1 & k_2 & \cdots & k_n \\ \vdots & \vdots & \ddots & \vdots \\ k_1 & k_2 & \cdots & k_n
        \end{pmatrix}$. 由于$B$是实对称矩阵,故$g(B)$也是实对称矩阵(很容易验证实对称矩阵经过幂次、加法运算后仍是实对称矩阵),因此$g(B)=(g(B))^\mathrm{T}$,故$k_1=k_2=\cdots=k_n=k$,即$g(B)=kA$.
    \end{enumerate}

    \item \begin{enumerate}
        \item 设$\alpha$是$B$的特征向量,对应的特征值为$\lambda$,即$B\alpha=\lambda\alpha$,则$A\alpha+B\alpha-AB\alpha=0$,即$A\alpha+\lambda\alpha-A\lambda\alpha=0$,即$(1-\lambda)A\alpha=-\lambda\alpha$. 若$\lambda=1$,则$\alpha=0$,与它是特征向量矛盾,故$\lambda\neq 1$,从而$A\alpha=\dfrac{-\lambda}{1-\lambda}\alpha$,这就说明了$B$的特征向量就是$A$的特征向量.

        由$A+B=AB$可知,$(A-E)(B-E)=E$,故$A-E$可逆,且$(A-E)^{-1}=B-E$,从而$E=(B-E)(A-E)=BA-A-B+E$,故$BA=A+B$,同上理可得$A$的特征向量也是$B$的,得证.

        \item \begin{enumerate}
            \item 必要性:若$A$可对角化,则$\mathbf{R}^n$有一组由$A$的特征向量张成的基,由(1)知$B$与$A$特征向量一致,故$\mathbf{R}^n$也有一组由$B$的特征向量张成的基,即$B$可对角化.
            \item 充分性与必要性类似.
        \end{enumerate}

        \item 由$A+B=AB$可知$A=(A-E)B$且$B=A(B-E)$,由此可得$r(A)\leqslant r(B)\leqslant r(A)$,故$r(A)=r(B)$.
    \end{enumerate}

    \item \begin{enumerate}
        \item 必要性:$A$和$B$在实数域上相似,则存在可逆实矩阵$C$使得$B=C^{-1}AC$,将$A,B,C$视为复数域上的矩阵,则可知$A$和$B$在复数域上也相似.
        \item 充分性:若$A$和$B$在复数域上相似,则存在可逆复矩阵$P=P_1+\i P_2$使得$B=P^{-1}AP$,其中$P_1,P_2$是实矩阵,即$AP=PB$,从而$A(P_1+\i P_2)=(P_1+\i P_2)B$,比较实部虚部可知$AP_1=P_1B$且$AP_2=P_2B$. 构造实系数多项式$f(t)=|P_1+tP_2|$,显然$f(\i)=|P|\neq 0$,则$f(t)$是非零多项式,所以$f(t)$的根只有有限多个,从而存在$t_0\in\mathbf{R}$使得$f(t_0)\neq 0$,则$|P_1+t_0P_2|\neq 0$,从而$P_1+t_0P_2$可逆,从而$A(P_1+t_0P_2)=AP_1+t_0AP_2=P_1B+t_0P_2B=(P_1+t_0P_2)B$,即$(P_1+t_0P_2)^{-1}A(P_1+t_0P_2)=B$,即$A$和$B$在实数域上相似.
    \end{enumerate}

    \item \label{ex:19:交换对角化}
    \begin{enumerate}
        \item \begin{enumerate}
            \item 必要性:$A$有$n$个互不相等的特征值,则$A$一定可对角化,则$A$有$n$个线性无关特征向量,记为$X_1,\cdots,X_n$,则$AX_i=\lambda_iX_i(X_i\neq 0)$,则$A(BX_i)=B(AX_i)=\lambda_i(BX_i)$,即$BX_i$属于$A$的特征子空间$V_{\lambda_i}$,又$\lambda_i$是$A$的单重特征值,对应的特征子空间是一维的,故$V_{\lambda_i}$中任两个向量成比例,即$BX_i=\mu_iX_i$,故$X_i$也是$B$关于特征值$\mu_i$的特征向量.
            \item 充分性:必要性:$A$有$n$个互不相等的特征值,则$A$一定可对角化,则$A$有$n$个线性无关特征向量$X_1,\cdots,X_n$,由于它们也是$B$的特征向量,故$B$可对角化. 设$P=(X_1,\cdots,X_n)$,则$P^{-1}AP=\Lambda_1$,$P^{-1}BP=\Lambda_2$,其中$\Lambda_1$和$\Lambda_2$都是对角矩阵,则它们乘法可交换,即$(P^{-1}AP)(P^{-1}BP)=\Lambda_1\Lambda_2=\Lambda_2\Lambda_1=(P^{-1}BP)(P^{-1}AP)$,上式两边左乘$P$,右乘$P^{-1}$,则$AB=BA$.
        \end{enumerate}
        \item 由于$A$可对角化,所以存在可逆矩阵$P$使得$P^{-1}AP=\diag(\lambda_1E_1,\cdots,\lambda_sE_s)$,其中$\lambda_1,\cdots,\lambda_s$是$A$的所有互异特征值,$E_1,\cdots,E_s$分别是$r_2,\cdots,r_s$阶单位矩阵,且$r_1+\cdots+r_s=n$,由$AB=BA$可知$P^{-1}APP^{-1}BP=P^{-1}BPP^{-1}AP$,根据正文矩阵运算进阶中对矩阵乘法可交换问题的讨论中准对角矩阵的结论,我们有$P^{-1}BP=\diag(B_1,\cdots,B_s)$,其中$B_i$是$r_i$阶矩阵. 由于$B$可对角化,因此$P^{-1}BP$也可对角化(因为它和$B$相似,故它们有相同的相似标准形),从而$B_1,\cdots,B_s$都可对角化,故对于任意的$B_i(i=1,2,\cdots,s)$,存在可逆矩阵$Q_i$使得$Q_i^{-1}B_iQ_i$是对角矩阵. 取$Q=\diag(Q_1,\cdots,Q_s)$,则$Q^{-1}P^{-1}BPQ=\diag(Q_1^{-1}B_1Q_1,\cdots,Q_s^{-1}B_sQ_s)$是对角矩阵,同时有
        \[Q^{-1}P^{-1}APQ=\diag(\lambda_1E_1,\cdots,\lambda_sE_s),\]
        则$Q^{-1}P^{-1}APQ=\diag(\lambda_1Q_1^{-1}Q_1,\cdots,\lambda_sQ_s^{-1}Q_s)=\diag(\lambda_1E_1,\cdots,\lambda_sE_s)$为对角矩阵,因此取$T=PQ$就有$T^{-1}AT$和$T^{-1}BT$都是对角矩阵.
    \end{enumerate}

    \item 此题条件与C组\ref*{ex:19:交换对角化}题相同,因此我们知道存在可逆矩阵$P$使得$P^{-1}AP$和$P^{-1}BP$都是对角矩阵(因为$A$的$n$个线性无关特征向量也是$B$的线性无关特征向量). 设$P^{-1}AP=\diag(\lambda_1,\cdots,\lambda_n)$,$P^{-1}BP=\diag(\mu_1,\cdots,\mu_n)$,并设$f(x)=a_{n-1}x^{n-1}+\cdots+a_1x+a_0$满足$f(A)=B$,则$a_{n-1}A^{n-1}+\cdots+a_1A+a_0E=B$,从而
    \[a_{n-1}(P^{-1}AP)^{n-1}+\cdots+a_1(P^{-1}AP)+a_0E=P^{-1}BP,\]
    即
    \[\begin{cases}
        a_{n-1}\lambda_1^{n-1}+\cdots+a_1\lambda_1+a_0=\mu_1 \\
        a_{n-1}\lambda_2^{n-1}+\cdots+a_1\lambda_2+a_0=\mu_2 \\
        \cdots \\
        a_{n-1}\lambda_n^{n-1}+\cdots+a_1\lambda_n+a_0=\mu_n
    \end{cases},\]
    由于$\lambda_1,\cdots,\lambda_n$两两不同,故上述方程组有唯一解(回忆范德蒙行列式和Cramer法则),从而$a_0,\cdots,a_{n-1}$有唯一解,得证.
\end{enumerate}

\clearpage

\chapter{相似标准形(II)}

虽然对角矩阵十分简洁,但根据我们上一讲中讨论的等价条件可知,可对角化的条件是
较为苛刻的,很多线性变换都不存在如此简洁的矩阵表示.我们考虑更为普遍但也能
保持良好性质的情况,上三角矩阵一定是一个好的突破口.

但上三角矩阵含有的零的数量事实上不够多,因此我们还需进一步讨论分块对角矩阵.
在此过程中我们也将系统性地介绍幂零矩阵,补全我们之前讨论的缺口.

\section{上三角矩阵}
我们从上三角矩阵出发,首先因为之前的讨论中上三角矩阵的一些优良性质大家已经非常熟悉,并且
事实上,复向量空间中的所有线性变换都可以在某组基下得到上三角矩阵的矩阵表示.
我们有如下定理:
\begin{theorem}\label{thm:20:上三角矩阵存在}
    设$V$是有限维复向量空间,$\sigma\in \mathcal{L}(V)$,则
    \begin{enumerate}
        \item $\sigma$关于$V$的某组基有上三角矩阵,记为$A$;

        \item $\sigma$可逆的充要条件是$A$的主对角元均不为0;

        \item $\sigma$的特征值恰为$A$的主对角元.
    \end{enumerate}
\end{theorem}
\begin{proof}
    \begin{enumerate}
        \item 根据相似的定义,这一结论与``任意$n$阶复矩阵一定相似于一个上三角矩阵''是等同的,
        我们可以使用分块矩阵结合数学归纳法的方法进行证明.$n=1$时结论显然,因为任意一阶矩阵
        本身就是上三角矩阵.现假设$n-1$阶复矩阵都可以相似于上三角矩阵,设$A$为$n$阶复矩阵,
        我们任取$A$的一个复特征值$\lambda_1$,设$\alpha_1$为$A$对应于$\lambda_1$的特征向量.
        我们把$\alpha_1$扩充为$\mathbf{C}^n$的一组基,记为$\alpha_1,\alpha_2,\ldots,\alpha_n$,
        记$P_1=(\alpha_1,\alpha_2,\ldots,\alpha_n)$,则$P_1$可逆,且
        \[P_1^{-1}AP_1=\begin{pmatrix}
            \lambda_1 & \alpha' \\ 0 & A_{n-1}
        \end{pmatrix},\]
        我们对$n-1$阶矩阵$A_{n-1}$应用归纳假设,因此存在可逆矩阵$P_2$使得$P_2^{-1}A_{n-1}P_2$为上三角矩阵,
        取$P=P_1\begin{pmatrix}
            1 & 0 \\ 0 & P_2
        \end{pmatrix}$,直接有
        \[P^{-1}AP=\begin{pmatrix}
            \lambda_1 & \alpha'P_1 \\ 0 & P_2^{-1}A_{n-1}P_2
        \end{pmatrix}\]
        为上三角矩阵,因此$n$阶复矩阵一定相似于上三角矩阵.
        \item 这一结论在矩阵计算进阶一讲中已经讨论,此处不再赘述;
        \item 利用线性映射的特征值等于表示矩阵的特征值,且表示矩阵的特征值就是$|\lambda E-A|=0$
        的零点即可得到这一结论.
    \end{enumerate}
\end{proof}

除此之外,在矩阵计算进阶中我们也提到上三角矩阵相乘结果中对角线上元素是原矩阵对角线上
对应元素相乘的结果,其逆的对角线上元素是原矩阵对角线对应元素的逆.以上性质表明,
上三角矩阵是所有线性变换都可以在某组基下得到的且有良好性质的矩阵类型.

事实上,我们在这里介绍这一证明的主要目的在于,这一证明给了我们一个求解线性变换(或矩阵)上三角化的方法.
与对角化类似,线性变换的上三角化依赖于矩阵上三角化,因此我们这里只讨论矩阵的情况.根据上面的证明,
我们只需任意挑选$n$阶矩阵的一个特征值和一个对应的特征向量,然后问题就转化为求解$n-1$阶矩阵的上三角化问题,
那么我们继续求出$n-1$阶矩阵的一个特征值和一个对应的特征向量,依次类推直到一阶的情况.我们给出下面的例子
供读者运用这一方法:
\begin{example}
    设$\sigma\in\mathbf{C}^3$定义为$\sigma(x,y,z)=(2x+y,5y+3z,8z)$,求$\mathbf{C}^3$的一组基使得
    $\sigma$在这组基下的矩阵为上三角矩阵.
\end{example}
\begin{solution}

\end{solution}

事实上,我们研究上三角矩阵(以及后续分块对角矩阵)的思路与研究对角矩阵是类似的,我们会讨论计算方法以及
相关的等价条件.下面我们给出关于对角矩阵的一些等价条件:
\begin{theorem}\label{thm:20:上三角矩阵等价条件}
    设$\sigma\in \mathcal{L}(V)$,且$v_1,v_2,\ldots,v_n$是$V$的基,则以下条件等价:
    \begin{enumerate}
        \item $\sigma$关于$v_1,v_2,\ldots,v_n$的矩阵是上三角的;

        \item 对每个$j=1,\ldots,n$有$\sigma(v_j)\in\spa(v_1,\ldots,v_j)$;

        \item 对每个$j=1,\ldots,n$有$\spa(v_1,\ldots,v_j)$在$\sigma$下不变.
    \end{enumerate}
\end{theorem}
\begin{proof}
    \begin{enumerate}
        \item 
        \item 
        \item 
    \end{enumerate}
\end{proof}

这一定理给出了上三角矩阵的几个充要条件,基于这些充要条件我们可以有如下进一步的理解:
\begin{enumerate}
    \item 我们可以给出\autoref{thm:20:上三角矩阵存在}的另外两种证明,即《线性代数应该这样学》
    5.27给出的两种证明.事实上5.27的证明也使用了数学归纳法,但是之前分块矩阵的归纳是针对矩阵的
    阶数,这里是针对线性变换所在线性空间的维数.我们这给出证明,以便深刻体会数学归纳法的思想:
    
    \begin{proof}
        \begin{enumerate}
            \item 
            \item 
        \end{enumerate}
    \end{proof}
    
    \item 比较对角矩阵的充要条件中要求存在一维不变子空间的分解,这里的等价条件表明上三角矩阵
    要求线性变换存在任意维数的不变子空间即可(对角矩阵的条件实际上蕴含这一条件,因为存在一维不变子空间直和分解,
    我们任意通过直和运算合并这些一维不变子空间就可以得到任意维数的不变子空间,因此对角矩阵的要求
    更强,上三角矩阵不需要这些任意维数的不变子空间能拆解成一维不变子空间的直和).
    
    事实上,因为任意复向量空间上的线性变换都存在一组基使得矩阵表示为上三角矩阵,因此
    任意复向量空间上的线性变换都存在任意维数的不变子空间.因此下面这一例子的结论是显然的:
    \begin{example}
        设$V$是$n$维复向量空间. $\sigma\in \mathcal{L}(V)$,证明:对任意的正整数
        $r\enspace(1\leqslant r\leqslant n)$,$\sigma$有$r$维不变子空间.
    \end{example}
    \begin{proof}
        
    \end{proof}
\end{enumerate}

接下来我们要讨论一个特别的问题,即线性变换/矩阵可交换的性质.我们有如下定理:
\begin{theorem}
    设$V$为$n$维复向量空间,$\sigma,\tau\in \mathcal{L}(V)$,$\sigma\tau=\tau\sigma$,则
    \begin{enumerate}[label=(\arabic*)]
        \item $\sigma$的每个特征子空间都是$\tau$的不变子空间;

        \item $\sigma,\tau$有公共的特征向量.
    \end{enumerate}
\end{theorem}
将这一定理的线性变换改为矩阵实际上是等价的.

\begin{proof}
    \begin{enumerate}[label=(\arabic*)]
        \item 
        \item 
    \end{enumerate}
\end{proof}

接下来我们希望应用这上述定理解决下面的问题:
\begin{example}
    设$V$为$n$维复向量空间,$\sigma,\tau\in \mathcal{L}(V)$,$\sigma\tau=\tau\sigma$,证明:
    \begin{enumerate}[label=(\arabic*)]
        \item 若$\sigma$有$s$个不同的特征值,则$\sigma,\tau$至少有$s$个公共且线性无关的特征向量;

        \item 存在$V$的一组基,使得$\sigma$和$\tau$在这组基下的矩阵均为上三角矩阵.
    \end{enumerate}
\end{example}
\begin{proof}
    \begin{enumerate}[label=(\arabic*)]
        \item 
        \item 
    \end{enumerate}
\end{proof}

这一例子的结论告诉我们:线性变换可交换对应于同时上三角化.例中 2 的结论如果换为矩阵表述应当是:
设$A,B$是复数域上的两个$n$阶矩阵,且$AB=BA$,则存在可逆矩阵$P$使得$P^{-1}AP$和$P^{-1}BP$
同时为上三角矩阵.

\section{核空间的性质 \quad 幂零矩阵}
这一节我们将为后续讨论分块对角矩阵做准备,同时幂零矩阵一节中也将讨论这一特殊矩阵的很多特别的、有趣的性质.

\subsection{核空间的性质}
事实上,根据可对角化的等价条件,我们知道一个线性变换不可对角化实际上是因为它没有足够多的
线性无关的特征向量,也即特征子空间直和后比原空间略小.事实上,我们知道$\sigma$在特征值
$\lambda$下的特征子空间实际上就是$\ker(\lambda I-\sigma)$.我们回顾\autoref{thm:6:核空间性质}:
\begin{theorem}\label{thm:20:核空间性质}
    设$\sigma\in \mathcal{L}(V)$,则有
    \begin{enumerate}
        \item $\{0\}=\ker \sigma^0\subset\ker \sigma^1\subset\cdots\subset
        \ker \sigma^k\subset\ker \sigma^{k+1}\subset\cdots$;

        \item 设$m$是非负整数使得$\ker \sigma^m=\ker \sigma^{m+1}$,则
        \[\ker \sigma^m=\ker \sigma^{m+1}=\ker \sigma^{m+2}=\ker \sigma^{m+3}=\cdots\]

        \item 令$n=\dim V$,则$\ker \sigma^n=\ker \sigma^{n+1}=\ker \sigma^{n+1}=\cdots$.
    \end{enumerate}
\end{theorem}

我们发现,如果我们提高线性变换的幂次,那么我们可以获得更大的核空间,这样扩张后的核空间的
直和是否可以张成整个原空间呢?我们可以提前给出答案:可以,具体的说明我们在讨论完下面的
幂零矩阵之后将会详细展开.

\subsection{幂零矩阵}
基于上面核空间的讨论,并为了方便后面小节的研究,我们将讲解幂零线性变换与幂零矩阵的相关准备知识.
\begin{definition}
    \begin{enumerate}
        \item 一个线性变换称为\keyterm{幂零}[nilpotent]的,如果它的某个幂等于零映射(即将所有向量都
        映射到0的映射);
        \item 一个矩阵$A$称为幂零的,如果存在正整数$m$使得$A^m=O$.
    \end{enumerate}
\end{definition}

根据线性映射矩阵表示很容易知道,幂零线性变换在任意一组基下的矩阵表示都是幂零矩阵.
我们接下来首先讨论幂零线性变换的一些基本性质:
\begin{theorem} \label{thm:20:幂零线性变换性质}
    设线性变换$N\in \mathcal{L}(V)$是幂零的,则
    \begin{enumerate}[label=(\arabic*)]
        \item $N$的所有特征值均为0(等价定义);

        \item $N^{\dim V}$=0;

        \item $V$有一组基使得$N$关于这组基的矩阵对角线和对角线下方元素均为0(等价定义);

        \item $N\pm I$可逆.
    \end{enumerate}
\end{theorem}
\begin{proof}
    \begin{enumerate}[label=(\arabic*)]
        \item 这一结论我们将在下一讲中介绍哈密顿-凯莱定理后给出证明;
        \item 
        \item 
        \item 
    \end{enumerate}
\end{proof}

事实上第1、2、4点都有相应的矩阵的结论,我们将线性变换替换为它的矩阵表示即可,此处不再赘述.
而第三点则解释了我们在求矩阵的幂时将一些矩阵分解为一个矩阵加一个对角线上全为0的矩阵的合理性,
因为后者一定是幂零的.接下来我们通过几个例子进一步讨论、运用幂零矩阵、幂零线性变换的性质:
\begin{example}
    证明:$A$为幂零矩阵$\iff \forall k \in N^+$,\textup{tr}$(A^k)$=\textup{0}.
\end{example}
\begin{proof}

\end{proof}

\begin{example}
	若$A$、$B$为两个$n$阶矩阵且满足$AB-BA=A$,证明:
    \begin{enumerate}[label=(\arabic*)]
        \item $A$不可逆;
        \item $A$是幂零矩阵.
    \end{enumerate}
\end{example}
\begin{proof}
    \begin{enumerate}[label=(\arabic*)]
        \item 
        \item 
    \end{enumerate}
\end{proof}

\section{分块对角矩阵}
\subsection{广义特征子空间与分块对角矩阵}
上一节中我们已经讨论了不可对角化线性变换获得简化矩阵的一般思想,即试图利用核空间增长的性质扩张
特征子空间,使得扩张后的特征子空间(称为广义特征子空间)的直和为原空间.下面我们给出严谨定义:
\begin{definition}
    设$\sigma\in \mathcal{L}(V)$,$\lambda\in\mathbf{F}$是$\sigma$的特征值,若向量$v\neq 0$且存在正整数$j$使得
    $(\sigma-\lambda I)^jv=0$,则称$v$为$\sigma$对应于$\lambda$的\keyterm{广义特征向量}[generalized eigenvector].
    $\sigma$对应于$\lambda$的全体广义特征向量与0向量构成的集合称为$\sigma$相应于$\lambda$的\keyterm{广义特征子空间}[generalized eigenspace],记为$G(\lambda,\sigma)$.
\end{definition}
注意我们不定义广义特征值,因为若$\lambda$原先不是特征值,因此$\sigma-\lambda I$可逆,可逆映射复合仍可逆,故对于任意的$j$,
$(\sigma-\lambda I)^j$仍可逆,即特征值是不会随着线性变换幂次增加而增加的.

实际上,根据\autoref{thm:20:核空间性质},我们有$G(\lambda,\sigma)=\ker (\sigma-\lambda I)^{\dim V}$.
需要补充说明的是,此处引入两个概念称为代数重数(或称重数)和几何重数,其中$\lambda$的代数重数定义为
广义特征子空间的维数,几何重数定义为特征子空间的维数.实际上在不变子空间一讲中我们有类似的定义,我们将在下一讲中
讲解它们的关联.

我们接下来的目标转向我们的主线,即证明任意线性变换的广义特征子空间的和为直和且和为原空间.下面这一定理读者可以回顾
特征值、特征向量的性质以及可对角化的等价条件,我们会发现这些定理具有很大的相似性,因此记忆难度并不大:
\begin{theorem} \label{thm:20:广义特征性质}
    设$V$是有限维的,$\sigma\in \mathcal{L}(V)$.用$\lambda_1,\ldots,\lambda_m$表示$\sigma$的所有互异特征值.
    \begin{enumerate}[label=(\arabic*)]
        \item $\sigma$对应于不同特征值的广义特征向量线性无关;

        \item $\sigma$不同特征值对应的广义特征子空间的和为直和,且$V=G(\lambda_1,\sigma)\oplus\cdots\oplus
        G(\lambda_m,\sigma)$;

        \item $V$有一个由$\sigma$的广义特征向量组成的基;

        \item 每个$G(\lambda_i,\sigma)$在$\sigma$下都是不变的;

        \item 每个$(\sigma-\lambda_j I)\vert_{G(\lambda_j,\sigma)}$都是幂零的.
    \end{enumerate}
\end{theorem}
\begin{proof}
    \begin{enumerate}[label=(\arabic*)]
        \item 
        \item 
        \item 
        \item 
        \item 
    \end{enumerate}
\end{proof}

上述定理更重要的结果在于它我们可以得到任何复向量空间上的线性变换都有如下的分块对角矩阵的标准形:
\begin{theorem}
    设$V$是复向量空间,$\sigma\in \mathcal{L}(V)$.设$\lambda_1,\cdots,\lambda_m$是$\sigma$的所有互不相同的特征值,重数分别为
    $d_1,\cdots,d_m$,则$V$有一组基使得$\sigma$关于这组基的有分块对角矩阵
    \[\begin{pmatrix}
        A_1 &  & O \\  & \ddots &  \\ O &  & A_m
    \end{pmatrix}\]
    其中每个$A_j$都是如下所示的$d_j\times d_j$上三角矩阵
    \[A_j=\begin{pmatrix}
        \lambda_j &  & * \\  & \ddots &  \\ O &  & \lambda_j
    \end{pmatrix}\]
\end{theorem}
\begin{proof}
    
\end{proof}

由此我们得到了一个相比于上三角矩阵更为简单,并且所有线性变换都可以获得的标准形.
在介绍完其存在性后,我们按照惯例需要讨论如何将这一标准形求解出来.事实上,根据上述定理
的证明,我们发现每个对角块都是从一个广义特征子空间得来的,因此我们只需求出各个广义特征子空间的
基,然后写出对应的矩阵即可.如果得到的对角块不是上三角矩阵,我们可以使用在上三角矩阵求法中
讲解的方法进行调整.我们来看一个例子:
\begin{example}
    设$\sigma\in \mathcal{L}(\mathbf{C}^3)$定义为
    \[\sigma(z_1,z_2,z_3)=(6z_1+3z_2+4z_3,6z_2+2z_3,7z_3),\]求一组基使其有分块对角矩阵并写出对应的分块对角矩阵.
\end{example}
\begin{solution}

\end{solution}

事实上,读者会发现虽然整体思路是很简单的,但是中间求解广义特征子空间的过程还是存在一定的困难.因为
当$\dim V$较大时,$G(\lambda,\sigma)=\ker (\sigma-\lambda I)^{\dim V}$的求解需要反复计算幂次,
是很困难的,但事实上根据核空间停止增长的性质可以知道,我们只需要不断提升矩阵的幂次,直到得到的
广义特征子空间不再发生改变就能够停止计算.

\begin{example}
    设$\sigma,\tau\in \mathcal{L}(V)$可逆,证明:$\sigma$和$\tau^{-1}\sigma\tau$有相同的特征值,且重数也相同.
\end{example}
\begin{proof}
    
\end{proof}

\subsection{平方根问题}
在进入下一个话题前,我们先简单介绍线性变换平方根的概念,这一概念在之后内积空间线性变换会进一步说明.
\begin{definition}
    我们称线性变换$\sigma\in \mathcal{L}(V)$的平方根是满足$\tau^2=\sigma$的线性变换$\tau\in \mathcal{L}(V)$.
\end{definition}
在复向量空间中,我们有如下两个结论:
\begin{theorem} \label{thm:20:幂零平方根}
    设$V$是复向量空间.
    \begin{enumerate}[label=(\arabic*)]
        \item 设$N\in \mathcal{L}(V)$幂零,则$(I+N)$有平方根;
        \item 若$\sigma\in \mathcal{L}(V)$可逆,则$\sigma$有平方根.
    \end{enumerate}
\end{theorem}
\begin{proof}
    \begin{enumerate}[label=(\arabic*)]
        \item 
        \item 
    \end{enumerate}
\end{proof}

我们发现,这一定理的证明思路基于$\sqrt{1+x}$的泰勒展开,我们不是第一次看到使用泰勒展开的情况,
在求解矩阵的逆的进阶方法中,求逆的分式思想中也使用了$\cfrac{1}{1-x}$的泰勒展开,足以
体现一些数学直觉对于我们解决一些问题的重要性.
\begin{example}
    定义$N\in \mathcal{L}(\mathbf{F}^5)$为
    \[N(x_1,x_2,x_3,x_4,x_5)=(2x_2,3x_3,-x_4,4x_5,0)\]
    求$(I+N)$的一个平方根.
\end{example}
\begin{solution}

\end{solution}

最后,在开始习题内容前,我们需要讲解一类特殊的题型,即举例或举反例的问题.
一般而言,我们有如下两种思路:
\begin{enumerate}
	\item 考虑几何意义:例如旋转矩阵,特征值的几何意义等
	\begin{example}
		找出有限维实向量空间的一个线性变换$\sigma$,使得$0$是$\sigma$仅有的特征值但$\sigma$不是幂零线性变换.
	\end{example}
	\begin{example}
		找出一个$\sigma\in L(\mathbf{R}^2)$使得$\sigma^4=-I$.
	\end{example}
	\item 考虑简单的情况:例如考虑2阶、3阶的简单线性变换/矩阵
	\begin{example}
		证明或给出反例:$V$上的幂零线性变换的集合是$L(V)$的子空间.
	\end{example}
	很多时候一些反例很难构想就选择记住这一构造思想即可.一些反例可能基于一些简单的结论,但如果未
	思考到位可能很难构造.
\end{enumerate}

\vspace{2ex}
\centerline{\heiti \Large 内容总结}

\vspace{2ex}

\centerline{\heiti \Large 习题}
\vspace{2ex}
{\kaishu }
\begin{flushright}
    \kaishu

\end{flushright}

\centerline{\heiti A组}
\begin{enumerate}
    \item 相抵但不相似.
\end{enumerate}
\centerline{\heiti B组}
\begin{enumerate}
    \item 
\end{enumerate}
\centerline{\heiti C组}
\begin{enumerate}
    \item
\end{enumerate}

\chapter{多项式的进一步讨论}

在前面的讲解中我们讨论了如何利用核空间的性质将线性空间分解为若干个线性变换的不变子空间,得到广义特征子空间和分块对角矩阵的结论. 从本讲起我们希望变换一个角度,从多项式出发推导出这一结论,并由此出发深入讨论多项式与相似标准形的关联.

为了接下来讨论的方便,我们首先介绍一个接下来常用的一类特殊的多项式,它将线性变换(矩阵)代入后多项式值为0. 这样的多项式我们称为线性变换(矩阵)的零化多项式,我们的严谨定义如下:
\begin{definition}[零化多项式] \index{duoxiangshi!linghua@零化 (annihilating polynomial)}
    我们有线性变换和矩阵的零化多项式定义如下:
    \begin{enumerate}
        \item 设$\sigma\in \mathcal{L}(V)$,若$p\in\mathbf{F}[x]$使得$p(\sigma)=0$,则称$p$为$\sigma$的一个\term{零化多项式};

        \item 设$A\in\mathbf{F}^{n\times n}$,若$p\in\mathbf{F}[x]$使得$p(A)=0$,则称$p$为$A$的一个零化多项式.
    \end{enumerate}
\end{definition}

\section{特征多项式 \quad Hamilton-Cayley 定理}

接下来我们首先讨论线性变换的特征多项式,事实上我们在不变子空间一讲中实际上已经提到过矩阵的特征多项式,这里我们将给出线性变换的相关定义并讨论二者关联:
\begin{definition}[特征多项式] \index{duoxiangshi!tezheng@特征 (characteristic polynomial)}
    设$V$是复向量空间,$\sigma\in \mathcal{L}(V)$. 令$\lambda_1,\ldots,\lambda_m$表示$\sigma$的所有互异特征值,重数(即对应的广义特征子空间的维数)分别为$d_1,\ldots,d_m$,则多项式
    \begin{equation}\label{eq:21:线性变换特征多项式}
        p(z)=(z-\lambda_1)^{d_1}\cdots(z-\lambda_m)^{d_m}
    \end{equation}
    称为$\sigma$的\term{特征多项式}.
\end{definition}

根据这一定义,我们有两个直接的结论:
\begin{enumerate}
    \item $\sigma$的特征多项式的次数为$\dim V$,因为\autoref{thm:20:广义特征性质} \ref*{item:20:广义特征性质:2} 保证了$\displaystyle\sum_{i=1}^m d_i=\dim V$;

    \item $\sigma$的特征多项式的零点恰为$\sigma$的全部特征值,这是由上述定义决定的.
\end{enumerate}

\begin{example}
    设$V$是复向量空间,$V_1,\ldots,V_m$都是$V$的非零子空间使得$V=V_1\oplus\cdots\oplus V_m$. 设$\sigma\in \mathcal{L}(V)$,每个$V_j$在$\sigma$下不变. 对每个$j$,令$p_j$表示$\sigma|_{V_j}$的多项式. 证明:$\sigma$的特征多项式为$p_1\cdots p_m$.
\end{example}

\begin{proof}

\end{proof}

设$\sigma\in\mathcal{L}(V)$在任一组基下的表示矩阵为$A$,则$\sigma$和$A$的特征值是完全一样的,因此特征值与其重数都是一致的. 我们回忆不变子空间一讲关于矩阵特征多项式的\autoref{thm:18:特征多项式展开} 并做因式分解:
\begin{equation}\label{eq:21:矩阵特征多项式}
    f(\lambda)=|\lambda I-A|=(\lambda-\lambda_1)^{r_1}\cdots(\lambda-\lambda_m)^{r_m}
\end{equation}
由于\autoref{eq:21:线性变换特征多项式} 和\autoref{eq:21:矩阵特征多项式} 的$k$重根都表示$k$重特征值,且$\sigma$和$A$的特征值及其重数一致,因此我们可以得到$d_i=r_i(i=1,\ldots,m)$. 注意到$d_i$是基于广义特征子空间维数定义的代数重数,$r_i$是基于矩阵特征多项式解的重数定义的代数重数,因此两个代数重数的定义也在此处统一了. 且多项式的$k$重根即为$k$重特征值,特征值的重数也就称为代数重数. 因此此后我们不再区分两种特征多项式和两种代数重数的定义.

在上面的讨论中我们依据广义特征子空间的维数定义特征多项式,接下来我们希望变换思路,利用特征多项式的定义推导广义特征子空间的相关结论,从而将17讲开头提到的三角形中多项式的部分补全. 事实上,接下来要讨论的思路是一般的高等代数教材中惯用的思路,这一正一逆的思路在某种程度上也体现出两个研究体系的等价性.

我们首先要引入 Hamilton-Cayley 定理. 我们的动机是获得零化多项式,使我们向着目标前进. 观察矩阵$A=\begin{pmatrix}
        1 & 2 \\ 0 & -1
    \end{pmatrix}$,我们容易验证$A^2-I=0$,因此$\lambda^2-1$是$A$的一个零化多项式,同时我们发现这是$A$的特征多项式,因此我们可以猜想,是否对于所有的矩阵都有特征多项式是零化多项式呢?事实上,这就是著名的 \term{Hamilton-Cayley 定理}:
\begin{theorem}[Hamilton-Cayley 定理] \label{thm:21:HC} \index{Hamilton@Hamilton-Cayley 定理 (Cayley-Hamilton theorem)}
    设$V$是复向量空间,$\sigma\in \mathcal{L}(V)$. 令$q$表示$\sigma$的特征多项式,则$q(\sigma)=0$.
\end{theorem}
定理的证明我们将从前面讨论的三个相似标准形:对角矩阵,上三角矩阵和分块对角矩阵三个角度给出三个证明. 通过这三个证明我们可以体会到标准形与多项式背后的联系.

\begin{enumerate}
    \item 利用对角矩阵

          为了介绍这一角度的证明,我们将会回顾数学分析或拓扑学中学习的稠密性的定义:
          \begin{definition}

          \end{definition}

          \begin{lemma}

          \end{lemma}

          \begin{proof}

          \end{proof}

    \item 利用上三角矩阵

          \begin{proof}

          \end{proof}

    \item 利用分块对角矩阵

          \begin{proof}

          \end{proof}
\end{enumerate}

接下来我们便可以利用这一定理继续我们逆向推导的过程. 我们的目标同样是找到能在直和后得到原空间的不变子空间的分解方式(即找到广义特征子空间). 实际上,我们可以利用\autoref{thm:17:裴蜀定理} 得到以下关键结论:
\begin{theorem} \label{thm:21:多项式分解与核空间直和}
    设$\sigma\in \mathcal{L}(V)$,且在$\mathbf{F}[x]$中有$p=p_1p_2$,且$p_1,p_2$互素,则有
    \[\ker p(\sigma)=\ker p_1(\sigma)\oplus\ker p_2(\sigma).\]
\end{theorem}

\begin{proof}

\end{proof}

为了得到广义特征子空间的定义,我们还需要将这一定理推广到因式更多的情况,证明只需要依照\autoref{thm:21:多项式分解与核空间直和} 然后进行数学归纳法即可,此处不再赘述:
\begin{theorem} \label{thm:21:多项式分解与核空间直和2}
    设$\sigma\in \mathcal{L}(V)$,且在$\mathbf{F}[x]$中有$p=p_1p_2\cdots p_s$,且$p_1,p_2,\ldots,p_s$两两互素,则有\[\ker p(\sigma)=\ker p_1(\sigma)\oplus\ker p_2(\sigma)\oplus\cdots\oplus\ker p_s(\sigma).\]
\end{theorem}

这一定理表明,将多项式分解为互素的多项式乘积,原多项式作用于线性变换的核空间等于分解后各个互素因式作用于线性变换的核空间的直和. 我们结合 Hamilton-Cayley 定理,如果$p$是$\sigma$的特征多项式,故$p(\sigma)=0$,则$\ker p(\sigma)$就是全空间$V$. 接下来我们将特征多项式分解为互素因式乘积,有
\[p(\lambda)=(\lambda-\lambda_1)^{r_1}(\lambda-\lambda_2)^{r_2}\cdots(\lambda-\lambda_m)^{r_m},\]
其中$\lambda_1,\ldots,\lambda_m$为$\sigma$的所有互异特征值,$r_1,\ldots,r_m$为特征值的重数. 然后由于分解的因式显然是两两互素的,因此根据\autoref{thm:21:多项式分解与核空间直和2},我们有
\[\ker p(\sigma)=V=\ker (\sigma-\lambda_1I)^{r_1}\oplus\cdots\oplus\ker (\sigma-\lambda_mI)^{r_m},\]
这或许就是一种巧合,我们从多项式的角度也推导出了和广义特征子空间相近的结论. 我们回顾\autoref{thm:20:广义特征性质} (2):
\[V=G(\lambda_1,\sigma)\oplus\cdots\oplus G(\lambda_m,\sigma),\]
其中$G(\lambda_i,\sigma)=\ker (\sigma-\lambda_iI)^{\dim V}$,这与上式的形式是类似的,但这里将广义特征子空间定义中$(\sigma-\lambda I)$由于核空间扩张所需的幂次降低了. 除此之外,$\ker (\sigma-\lambda_iI)^{r_i}\enspace(i=1,2,\ldots,m)$也是$\sigma$的不变子空间,因此我们也能得到分块对角矩阵的标准形,并且\autoref{thm:20:广义特征性质} 的其他结论也可以基于此得到,此处不再赘述.
\begin{example}
    设$\sigma\in \mathcal{L}(V)$,$p(z)=a_nx^n+\cdots+a_1x\in\mathbf{F}[x]$是$\sigma$的一个零化多项式,其中$a_1\neq 0$,证明:
    \[V=\ker \sigma\oplus\im\sigma.\]
\end{example}

\begin{proof}

\end{proof}

下面我们希望将广义特征子空间定义中$(\sigma-\lambda I)$的幂次进一步降低到下界(即找到最低的幂次使得核空间停止增长),这需要引入极小多项式的概念.

\section{极小多项式及其性质}

\begin{definition}
    我们有如下线性变换和矩阵的极小多项式定义:
    \begin{enumerate}
        \item 设$\sigma\in \mathcal{L}(V)$,则$\sigma$的极小多项式是唯一一个使得$p(\sigma)=0$的次数最小的首一多项式;

        \item 设$A\in\mathbf{F}^{n\times n}$,则$A$的极小多项式是唯一一个使得$p(A)=0$的次数最小的首一多项式.
    \end{enumerate}
\end{definition}
这一定义的合理性需要下述定理保证,我们只证明线性变换的角度,矩阵实际上只需要将定理和证明中的线性变换替换为矩阵即可:
\begin{theorem}\label{thm:21:极小多项式存在}
    设$\sigma\in \mathcal{L}(V)$,则存在唯一一个次数最小的首一多项式$p$使得$p(\sigma)=0$.
\end{theorem}

\begin{proof}

\end{proof}

这一定理的证明表明$V$上每个线性变换的极小多项式的次数最多为$(\dim V)^2$,而若$V$为复向量空间时,由 Hamilton-Cayley 定理我们知道极小多项式的次数最多为$\dim V$,事实上实空间也有这样的结论,我们将在实空间上的线性变换一讲中讨论.

如果需要计算极小多项式,我们可以给出一个算法化的描述. 对于$m=1,2,\ldots$,我们相继考虑线性方程组
\[a_0M(I)+a_1M(\sigma)+\cdots+a_{m-1}M(\sigma^{m-1})+M(\sigma^m)=0,\]
直到这一方程组有一个解$a_0,a_1,\ldots,a_{m-1}$,此时$a_0,a_1,\ldots,a_{m-1},1$即为极小多项式的次数.
\begin{example} \label{ex:21:最小多项式}
    求矩阵$A=\begin{pmatrix}
            0 & 0 & 0 \\ 1 & 0 & 2 \\ 2 & 1 & -1
        \end{pmatrix}$和$B=\begin{pmatrix}
            2 & 2 & 1 \\ 0 & 2 & -1 \\ 0 & 0 & -3
        \end{pmatrix}$的最小多项式.
\end{example}

\begin{solution}
    \begin{enumerate}
        \item

        \item
    \end{enumerate}
\end{solution}

下面我们给出一些简单线性变换/矩阵的极小多项式:
\begin{enumerate}
    \item 幂零线性变换:$N\in \mathcal{L}(V)$且$N^l=0$,但$N^{l-1}\neq 0$($l$称为幂零指数),极小多项式为$\lambda^l$;

    \item 幂等线性变换:$\sigma\in \mathcal{L}(V)$且$\sigma^2=\sigma$,极小多项式为$\lambda^2-\lambda$或$\lambda$或$\lambda-1$;

    \item 对合线性变换:$\sigma\in \mathcal{L}(V)$且$\sigma^2=I$,极小多项式为$\lambda^2-1$或$\lambda+1$或$\lambda-1$;

    \item 引入\term{若当块}\index{ruodangkuai@若当块 (Jordan block)}. 若域$\mathbf{F}$上的一个$r$级矩阵形如\[\begin{pmatrix}
                  a & 1 &        &   \\
                    & a & \ddots &   \\
                    &   & \ddots & 1 \\
                    &   &        & a
              \end{pmatrix}\]
          则称其为一个$r$级若当块(1级显然就是1阶矩阵),记作$J_r(a)$,其中$a$是对角线上元素. 不难得到其极小多项式等于特征多项式$(\lambda-a)^r$.
\end{enumerate}

我们利用多项式的带余除法以及 Hamilton-Cayley 定理可以得到下述简单的结论:
\begin{theorem}
    设$\sigma\in \mathcal{L}(V)$.
    \begin{enumerate}
        \item $q\in\mathbf{F}[x]$,则$q(\sigma)=0$当且仅当$q$是$\sigma$的极小多项式的多项式倍;

        \item 设$\mathbf{F}=\mathbf{C}$,则$\sigma$的特征多项式是$\sigma$的极小多项式的多项式倍.
    \end{enumerate}
\end{theorem}

\begin{proof}
    \begin{enumerate}
        \item

        \item
    \end{enumerate}
\end{proof}

在\autoref{ex:21:最小多项式} 中我们不难发现,两个矩阵的极小多项式和特征多项式根一致,实际上这是对任意线性变换(或矩阵)都成立的结论:
\begin{theorem} \label{thm:21:极小多项式与特征多项式相同根}
    设$\sigma\in \mathcal{L}(V)$,则$\sigma$的极小多项式的零点恰好是$\sigma$的特征值,即极小多项式与特征多项式在$\mathbf{F}$中有相同的根(重数可以不同).
\end{theorem}

\begin{proof}

\end{proof}

这一定理是非常重要的,它关系到下一小节关于多项式和标准形关系的讨论,且\autoref{ex:21:最小多项式} 也可以基于此有更快的解法:

\begin{solution}
    \begin{enumerate}
        \item

        \item
    \end{enumerate}
\end{solution}

除此之外,我们还可以得到一个推论:
\begin{corollary}
    相似的矩阵有相同的极小多项式.
\end{corollary}

\begin{proof}

\end{proof}

\section{多项式与标准形的应用}

在最后一小节我们尝试将两种描述线性变换的角度(标准形和多项式)联系起来,主要的桥梁就是上一小节中讨论的极小多项式. 在前文讨论特征多项式诱导的不变子空间分解时,我们将广义特征子空间定义中需要求核空间的线性变换幂次降低,而依据\autoref{thm:21:极小多项式与特征多项式相同根} 以及特征多项式是极小多项式的倍式可知,这一幂次还可以进一步降低:
\begin{theorem} \label{thm:21:极小多项式与分解}
    设$\sigma\in \mathcal{L}(V)$,$\sigma$的极小多项式为$p=(\lambda-\lambda_1)^{s_1}\cdots(\lambda-\lambda_m)^{s_m}$,则有
    \[\ker p(\sigma)=V=\ker (\sigma-\lambda_1I)^{s_1}\oplus\cdots\oplus\ker (\sigma-\lambda_mI)^{s_m}.\]
\end{theorem}

\begin{proof}

\end{proof}

通过\autoref{thm:21:极小多项式存在} 我们知道,极小多项式的因式次数无法继续降低,否则不为零化多项式,因此它也给出了广义特征子空间定义中需要求核空间的线性变换的幂次为何值时,核空间会停止增长,并且这是一个下界,基于此我们更进一步地理解了极小多项式因式次数的含义.

实际上我们也可以逆向思考,如果我们已知空间的不变子空间分解,我们应当如何求解极小多项式. 实际上这一结论是很直观的,答案是各个不变子空间的极小多项式的最小公倍式,严谨叙述如下:
\begin{theorem}
    设$\sigma\in\mathcal{L}(V)$,如果$V$能分解成$\sigma$的一些非平凡不变子空间的直和:
    \[V=U_1\oplus\cdots\oplus U_m,\]
    且$\sigma\vert_{U_i}$的极小多项式为$p_i$,则$\sigma$的极小多项式为
    \[p=\lcm(p_1,\ldots,p_m).\]
    其中$\lcm(p_1,\ldots,p_m)$表示$p_1,\ldots,p_m$的最小公倍式.
\end{theorem}

\begin{proof}

\end{proof}

这一结论的应用或许并不直接,但如果我们考虑线性变换在不变子空间直和分解下的分块对角矩阵,那么这一分块对角矩阵的极小多项式实际上就等于各个分块的极小多项式的最小公倍式.
\begin{example} \label{thm:21:若当形矩阵极小多项式}
    我们在此继续引入\term{若当形矩阵}\index{ruodangxingjuzhen@若当形矩阵 (Jordan matrix)},即由若干个若当块组成的分块对角矩阵. 设$A$为若当形矩阵,$A=\diag(J_{r_1}(a),\ldots,J_{r_s}(a),J_{t_1}(b),\ldots,J_{t_m}(b))$,其中$r_1\leqslant\cdots\leqslant r_s$,$t_1\leqslant\cdots\leqslant t_m$,则$A$的极小多项式$p$为$\lcm((\lambda-a)^{r_1},\ldots,(\lambda-a)^{r_s},(\lambda-b)^{t_1},(\lambda-b)^{t_m})$,即为$(\lambda-a)^{r_s}(\lambda-b)^{t_m}$. 实际上,这一结论还可以进一步推广,但描述较为繁杂,读者只需从此例理解基本思想即可.
\end{example}
在\autoref{thm:21:极小多项式与分解} 中我们了解了极小多项式中因子幂次与广义特征子空间的关联. 加入极小多项式的各个因式的次数均为1,这与可对角化线性变换的不变子空间分解是一致的!因此我们可以得到下面的结论:
\begin{theorem}
    设$\sigma\in \mathcal{L}(V)$,$\sigma$可对角化当且仅当$\sigma$的极小多项式能分解成不同的一次因式的乘积.
\end{theorem}

\begin{proof}

\end{proof}

这给出了线性变换可对角化的另一等价条件,基于此,不变子空间一讲中给出矩阵多项式判断可对角化的习题都可以``秒杀'',例如幂等矩阵、对合矩阵可对角化,但幂零矩阵除非自身为0否则一定不可对角化,高于1阶的若当块矩阵一定不可对角化,包含高于1阶的若当块矩阵的若当形矩阵也一定不可对角化.

我们也可从矩阵的角度来理解. 例如幂等矩阵$A$满足$A^2=A$,根据多项式诱导的不变子空间分解,我们很容易得到$A$为幂等矩阵的充要条件为$r(A)+r(A-E)=n$,其它对合矩阵等情况各位同学也可以自己写出等价条件,虽然形式上可以千变万化,但实质就是多项式诱导的不变子空间分解.

除此之外,联系多项式互素分解与不变子空间分解的对应关系,这也表明线性变换可对角化当且仅当其各个广义特征子空间就是其特征子空间,即满足代数重数等于几何重数. 或者说$\sigma$的每个广义特征向量都是其特征向量.
\begin{example}
    证明:设$\sigma\in \mathcal{L}(V)$,若$\sigma$可对角化,则对于$\sigma$的任意非平凡不变子空间$U$,都有$\sigma\vert_U$可对角化.
\end{example}

\begin{proof}

\end{proof}

\begin{example}
    已知某个实对称矩阵$A$的特征多项式为$\lambda^5+3\lambda^4-6\lambda^3-10\lambda^2+21\lambda-9$,求$A$的极小多项式.
\end{example}

\begin{solution}

\end{solution}

\begin{example}
    设$V$为$n$阶方阵构成的线性空间,$\sigma\in \mathcal{L}(V),\enspace \forall A\in V,\enspace \sigma(A)=2A-3A^{\mathrm{T}}$.
    \begin{enumerate}
        \item 求$\sigma$的特征值;

        \item 证明:$\sigma$可对角化.
    \end{enumerate}
\end{example}

\begin{solution}
    \begin{enumerate}
        \item

        \item
    \end{enumerate}
\end{solution}

我们需要补充说明一点,虽然矩阵相似不随数域改变而改变,但可对角化与数域有关. 例如实矩阵$A$的极小多项式为$\lambda^3-1$,在它在实数域上无法分解为互素一次因式的乘积,复数域上则可以,这表明$A$在实数域上不可对角化,但在复数域上可以.

\vspace{2ex}
\centerline{\heiti \Large 内容总结}

\vspace{2ex}
\centerline{\heiti \Large 习题}

\vspace{2ex}
{\kaishu }
\begin{flushright}
    \kaishu

\end{flushright}

\centerline{\heiti A组}
\begin{enumerate}
    \item
\end{enumerate}

\centerline{\heiti B组}
\begin{enumerate}
    \item
\end{enumerate}

\centerline{\heiti C组}
\begin{enumerate}
    \item
\end{enumerate}

\section*{22 若当标准形}
\addcontentsline{toc}{section}{22 若当标准形}

\vspace{2ex}

\centerline{\heiti A组}
\begin{enumerate}
    \item
\end{enumerate}

\centerline{\heiti B组}
\begin{enumerate}
    \item
\end{enumerate}

\centerline{\heiti C组}
\begin{enumerate}
    \item
\end{enumerate}

\clearpage

\input{./专题/23 内积空间.tex}
\chapter{内积空间上的算子(I)}

\section{自伴算子和正规算子}

由前面一章,我们成功的给线性空间加上了度量,使其升格成了内积空间,认识了一些新朋友(投影映射),
或是更了解了一些老朋友(线性泛函). 但之前学的那些线性映射似乎还没搭上边,那么本章我们就要研究一下
它们与内积有关的性质. 

\subsection{伴随}

\begin{definition}
    \keyterm{伴随} 设 $ T \in \mathcal{L}(V, W) $, $ T $ 的伴随满足如下条件 $ T^*: W \rightarrow V $
    : $ \forall v \in V, w \in W, \langle Tv, w \rangle = \langle v, T^*w \rangle$ 
\end{definition}

这样的一个东西定义出来,在本书的视角下一般先考虑以下问题:是个映射吗?良定义吗?线性吗?好消息是,对伴随而言,
这三个问题都是肯定的. 这里就良定义做个解释,线性的验证留给读者. 

我们考虑如下的线性泛函 $ \varphi : V \rightarrow F, \varphi (v) = \langle Tv, w \rangle $,那么利用一下刚学到的
\ref{Riesz 表示定理},存在唯一的 $ u \in W $,使得 $ \varphi (v) = \langle v, u \rangle $,再结合一下伴随的定义,
只需要定义 $ T^*w = u $ 即可. % TODO 引用

讲完了定义就轮到了性质,伴随有如下的运算性质. 
\begin{enumerate}
    \item $ \forall S, T \in \mathcal{L}(V, W), (S + T)^* = S^* + T^* $;

    \item $  \forall \lambda \in \mathbf{F}, T \in \mathcal{L}(V, W), (\lambda T)^* = \overline{\lambda} T^* $;

    \item $ \forall T \in \mathcal{L}(V, W), (T^*)^* = T $;
    
    \item 对 $ V $ 上的恒等算子 $ I $ 有 $ I^* = I $;
    
    \item $ \forall T \in \mathcal{L}(V, W), S \in \mathcal{L}(W, U), (ST)^* = T^*S^* $. 
\end{enumerate}

接着再研究一下它的核空间和像空间. 

设 $ T \in \mathcal{L}(V, W) $. 则

\begin{enumerate}
    \item $ \mathrm{null} T^* = (\mathrm{range} T)^{\perp } $;
    
    \item $ \mathrm{range} T^* = (\mathrm{null} T)^{\perp } $;
    
    \item $ \mathrm{null} T = (\mathrm{range} T^*)^{\perp } $;
    
    \item $ \mathrm{range} T = (\mathrm{null} T^*)^{\perp } $. 
\end{enumerate}

以上性质均不难证明,大家可以自己试试,顺带回顾一下内积的运算方法和证明线性空间相等的方法. 

而关于特征值、不变子空间等等的性质,就先简单看两道例题,有一个最基本的了解. 

\begin{example}
    设 $ T \in \mathcal{L}(V), \lambda \in \mathbf{F} $. 证明:
    $ \lambda $ 是 $ T $ 的特征值当且仅当 $ \overline{\lambda} $
    是 $ T^* $ 的特征值. 
\end{example}

\begin{example}
    设 $ T \in \mathcal{L}(V) $ 且 $ U $ 是 $ V $ 的子空间. 证明:
    $ U $ 在 $ T $ 下不变当且仅当 $ U^{\perp} $ 在 $ T^* $ 下不变. 
\end{example}

映射本身研究的差不多了,我们就该看看对应的矩阵有些什么性质了. 不过在此之前,我们要讨论一种
新的对矩阵的操作. 

\begin{definition} \keyterm{共轭转置}
    $ m \times n $ 矩阵的共轭转置是先互换行和列,然后对每个元素取复共轭得到的 $ n \times m $ 矩阵. 

    即矩阵 $ A = (a_{ij})_{m \times n} $,则 $ A $ 的共轭转置阵 $ \overline{A}^{\mathrm{T}} = (\overline{a_{ji}})_{n \times m} $
\end{definition}


有了这重铺垫,我们就可以好好讨论一下伴随映射对应的矩阵了. 

确定一个映射的矩阵都是要取定基的,而在内积空间上,我们取基的时候更喜欢用标准正交基,所以注意,
下面这个定理只对标准正交基成立. 

\begin{theorem}
    设 $ T \in \mathcal{L}(V, W) $,$ (e_1, \ldots , e_n) $ 是 $ V $ 的一组标准正交基,
    $ (f_1, \ldots , f_m) $ 是 $ W $ 的一组标准正交基,有 $ T(e_1, \ldots , e_n) = (f_1, \ldots , f_m)A $,$ A = (a_{ij})_{m \times n} $,
    $ T^*(f_1, \ldots , f_m) =(e_1, \ldots , e_n)B $,$ B = (b_{ij})_{n \times m} $,则 $ B $ 是 $ A $ 的共轭转置. 
\end{theorem}

\begin{proof}
    首先确定矩阵 $ A $ 的元素. 因为 $ (f_1, \ldots , f_m) $ 是 $ W $ 的一组标准正交基,
    所以有 
    \[ 
    Te_j = \langle Te_j, f_1 \rangle f_1 + \cdots + \langle Te_j, f_m \rangle f_m, \forall j = 1, \ldots , n.  
    \]
    也就是说,$ a_{ij} = \langle Te_j, f_i \rangle $. 
    那么同理,对于矩阵 $ B $ 而言,$ b_{ij} = \langle T^*f_j, e_i \rangle $. 
    所以有 
    \[ 
        a_{ij} = \langle Te_j, f_i \rangle = \langle e_j, T^*f_i \rangle
        = \overline{\langle T^*f_i, e_j \rangle} = \overline{b_{ji}}.
    \]
    所以,矩阵 $ B $ 是 $ A $ 的共轭转置. 
\end{proof}

对于一般线性映射的伴随就介绍上面的这些了. 接下来还是看些限制更多、性质更好的线性映射,
比如算子. 

\subsection{自伴算子}

限制成算子的话,原本的算子与其伴随就被限制在同一块内积空间上了. 很自然的,我们就会
开始思考一件事情,如果一个算子和它的伴随相等,那么会发生什么?

\begin{definition} \keyterm{自伴算子}
    若算子 $ T \in \mathcal{L}(V) $ 满足 $ T = T^* $,则其被称为自伴算子.  
\end{definition}

写成内积的语言就是 $ \forall v, w \in V, \langle Tv, w \rangle = \langle v, Tw \rangle $.  

容易验证自伴算子对加法和数乘都是封闭的. 而根据上面对伴随的阐述,我们可以做一个类比:
伴随在 $ \mathcal{L}(V) $ 上的作用如同复共轭在 $ \mathbf{C} $ 上的作用. 所以自伴算子
可以类比为实数. 关于这方面的类比在实内积空间上的算子一章会进行更深入的阐述. 

那么,自伴算子有这么好的定义,自然也少不了几条优美的性质. 

\begin{theorem}
    自伴算子的特征值都是实数. 
\end{theorem}

证明只需要结合特征值和自伴算子的定义就行了. 这条性质的几何意义就是自伴算子对
特征向量方向上的向量仅仅是拉伸的作用,而不产生旋转或对称的作用. 

以下的两条定理建立在复数域上,也是对复内积空间的结构进行一个初步的了解,以及加深一下
算子和数的类比. 

\begin{theorem}
    设 $ V $ 是复内积空间,$ T \in \mathcal{L}(V) $. 若 $ \forall v \in V $,
    $ \langle Tv, v \rangle = 0 $,则 $ T = 0 $. 
\end{theorem}

证明利用的正是之前的 \ref{例22.2} 的 (4) 式. 

\begin{theorem}
    设 $ V $ 是复内积空间,$ T \in \mathcal{L}(V) $. 则 $ T $ 是自伴的
    当且仅当 $ \forall v \in V, \langle Tv, v \rangle \in \mathbf{R} $
\end{theorem}

证明利用的则是实数减去其共轭等于 0 推出的一系列等价变形. 这一定理也进一步地显示出
自伴算子与实数的相似性. 

下面这个定理是 \ref{定理 23.3} 的一般情况. 

\begin{theorem}
    若 $ T $ 是 $ V $ 上的自伴算子,$ \forall v \in V $,
    $ \langle Tv, v \rangle = 0 $,则 $ T = 0 $. 
\end{theorem}

复内积空间上已经处理过了,实内积空间上利用 \ref{例22.2} 的 (2) 式与内积的对称性即可证明.  
事实上,在实内积空间上能做到 $ \forall v \in V , \langle Tv, v \rangle = 0 $ 
的算子绝大部分都是非自伴的,下面这道例题给出了其满足的性质. 

\begin{example}
    设 $ V $ 是实内积空间, $ T \in \mathcal{L}(V) $,
    $ \forall v \in V, \langle Tv, v \rangle = 0 $.
    证明:$ T^* = -T $
\end{example}
   
满足这样性质的算子在实内积空间上叫做反对称算子,如果我们故意将虚轴定义错误(即将 0 包括进去)的话,
反对称算子就可以类比为“虚轴”上的数. 自伴算子和反对称算子的交集是 0 算子,就如同实轴与“虚轴”的交点是原点. 

\subsection{正规算子}

自伴算子的讨论就先阐述这么多. 之前将算子和数进行了类比,着重关注了他们相似的地方,
现在来看看它们的不同之处,而最大的不同应该就是算子对乘法并没有交换律. 所以,如果
一个算子与其伴随的乘法是可交换的,它又会有些什么特殊之处呢?

\begin{definition} \keyterm{正规算子}
    若算子 $ T \in \mathcal{L}(V) $ 满足 $ TT^* = T^*T $,则其被称为正规算子. 
\end{definition}

很显然,自伴算子其实也是正规算子. 

和自伴算子一样,我们来简单研究一下正规算子的性质. 首先是正规算子的一个等价条件.

\begin{theorem}
    算子 $ T \in \mathcal{L}(V) $ 是正规的当且仅当 $ \forall v \in V,
    \lVert Tv \rVert = \lVert T^*v \rVert $. 
\end{theorem}

这也表明,对于任意一个正规算子 $ T $ ,其核空间和其伴随映射的核空间相等. 

接下来的两条性质则是着重关注正规算子的特征向量. 

\begin{theorem}
    设 $ T \in \mathcal{L}(V) $ 是正规的,且 $ v \in V $ 是 $ T $ 相应于
    特征值 $ \lambda $ 的特征向量,则 $ v $ 也是 $ T^* $ 相应于特征值
    $ \overline{\lambda} $ 的特征向量. 
\end{theorem}

这是 \ref{例 23.1} 在正规算子条件下的加强,它不仅反映了算子与其伴随的特征值
在数值上的关系,也反映出了特征空间的关系. 从这里出发,你可以先思考一下正规算子
的不变子空间是怎样的,如果有困难的话不妨结合一下 \ref{例 23.2}.

在学特征值时我们就学过,同一映射的属于不同特征值的特征向量是线性无关的. 在正规算子条件下,
这一结论也得到了加强,从原先的线性无关变为互相正交. 

\begin{theorem}
    设 $ T \in \mathcal{L}(V) $ 是正规的,则 $ T $ 的相应于不同特征值
    的特征向量是正交的. 
\end{theorem}

\begin{proof}
    设 $ \alpha, \beta $,是 $ T $ 的不同特征值,$ u, v $ 分别是相应的特征向量,
    则 $ Tu = \alpha u, Tv = \beta v $. 由 \ref{定理 23.7} 有 $ T^*v = \overline{\beta} v $. 
    从而 
    \begin{align*}
        (\alpha - \beta)\langle u, v \rangle
        & = \langle \alpha u, v \rangle - \langle u, \overline{\beta}v \rangle \\
        & = \langle Tu, v \rangle - \langle u, T^*v \rangle \\
        & = 0. 
    \end{align*}
    而 $ \alpha \neq \beta $,所以 $ \langle u, v \rangle = 0 $,即 $ u, v $ 正交.          
\end{proof}

这个定理很有意思,因为它既涉及了可对角化条件中的特征向量,也涉及了内积空间上的正交. 
而这两条正是我们寻求在内积空间上算子对应矩阵简化表示的重要条件,将在下一节进行着重阐述. 

\section{谱定理}



可能很多同学对于行秩、列秩相等以及转置的几何意义很感兴趣.实际上我们有两种获得转置矩阵的
方式,第一种来源于我们之前讨论的对偶空间上的线性映射对应的矩阵,这种方式可能不够直观.
另一种获得的方法基于伴随算子.接下来我们将说明这些定义的统一性,深刻理解转置的内涵.

我们可以研究矩阵及其转置的关系,我们可以用一个图形来表示:

\begin{figure}[H]
    \centering
    \small
    \begin{tikzpicture}
        \tikzset{->-/.style={decoration={
            markings,
            mark=at position .6 with {\arrow{stealth}}},postaction={decorate}}}

        \draw[rotate=45] (0, 6) rectangle (-3, 3) rectangle (-5, 0)
            (-3, 3) rectangle(-3.35, 3.35)
            coordinate (xr) at (-2, 4)
            coordinate (xn) at (-4, 2)
            coordinate (x) at (-2, 2)
            coordinate (0n) at (-3, 3)
            node at (-1, 5) {行空间}
            node at (-4, 1) {$A$的核空间}
            node at (-1.5, 6.5) {$\dim r$}
            node at (-4, 4) {$\mathbf{R}^n$}
            node at (-6, 3) {$\dim n-r$};

        \draw[rotate=30] (6, 2) rectangle (3.5, -2) rectangle (0, -4)
            (3.5, -2) rectangle (3.85, -2.35)
            coordinate (b) at (4.5, 0.5)
            coordinate (0m) at (3.5, -2)
            node at (5, 1.5) {列空间}
            node at (2, -3) {$A^{\mathrm{T}}$的核空间}
            node at (7, 0) {$\dim r$}
            node at (5, -3) {$\mathbf{R}^m$}
            node at (4, -4.5) {$\dim m-r$};

        \foreach \point in {xr, x, xn, 0n, b, 0m}
            \fill[black] (\point) circle (1pt);

        \node [left] at (xr) {$x_r$};
        \node [below right] at (x) {$x=x_r+x_n$};
        \node [left] at (xn) {$x_n$};
        \node [right] at (0n) {0};
        \node [right] at (b) {$b$};

        \draw[->-,very thick] (xr) -- node[above,sloped] {$Ax_r = b$} (b);
        \draw[->-,very thick] (x) -- node[below,sloped] {$Ax = b$} (b);
        \draw[->-,very thick] (xn) -- node[below,sloped] {$Ax_n = 0$} (0m);

        \draw[dashed,thick] (xr) -- (x) -- (xn);

    \end{tikzpicture}
\end{figure}

我们观察到以下几点:
\begin{enumerate}
    \item 矩阵的行空间与解空间(零空间)互为正交补(直观理解两个空间就是互相垂直且互为补空间),这一点应当是在正交的内容中有所提及的;
    \item 矩阵的列空间与其转置矩阵的零空间互为正交补,这一点实际与上一条等价.
\end{enumerate}

接下来我们来看行秩(列秩比较显然,此处不再详细展开).我们首先得到解空间($N(A)$)的维数,这可以直接
根据维数公式得到:$\dim N(A) =n-r(A)$,根据正交补的性质,我们的可以得到行秩即为
$n-(n-r(A))=r(A)$.于是我们得到了一个基于正交补的行秩解释.

\vspace{2ex}
\centerline{\heiti \Large 内容总结}

\vspace{2ex}

\centerline{\heiti \Large 习题}
\vspace{2ex}
{\kaishu }
\begin{flushright}
    \kaishu

\end{flushright}
\centerline{\heiti A组}
\begin{enumerate}
    \item
\end{enumerate}
\centerline{\heiti B组}
\begin{enumerate}
    \item
\end{enumerate}
\centerline{\heiti C组}
\begin{enumerate}
    \item
\end{enumerate}

\chapter{内积空间上的算子(II)}

\section{正交矩阵和酉矩阵}

前面我们对正规算子和自伴算子做了相当充分的工作,从这章开始我们准备对一般的算子做些工作. 

本节我们将唤醒一些沉睡的记忆,如果你已经忘了过渡矩阵或矩阵的相似,可以移步到前面的章节再回顾一下. 如果你还在这的话,那么坐稳,我们马上开始. 

\vspace{2ex}

\subsection{定义}

为了更好地引进正交矩阵和酉矩阵,我们有必要把共轭转置说的更清楚些. 
共轭转置有着以下的运算性质,虽然都是看起来很显然的事情,此处还是稍稍赘述一下:

设有矩阵 $ A $, $ B $ 和数 $ \lambda \in \mathbf{C}$,则

\begin{enumerate}    
    \item $ (\overline{A + B})^{\mathrm{T}} = \overline{A}^{\mathrm{T}} + \overline{B}^{\mathrm{T}} $;
    
    \item $ \overline{(AB)}^{\mathrm{T}} = \overline{B}^{\mathrm{T}} \overline{A}^{\mathrm{T}} $;
    
    \item $ (\overline{\lambda A})^{\mathrm{T}} = \overline{\lambda} \enspace \overline{A}^{\mathrm{T}} $;
    
    \item $ \overline{\overline{A}^{\mathrm{T}}}^{\mathrm{T}} = A $. 
\end{enumerate}

共轭转置说清楚后,便可以由此定义正交矩阵和酉矩阵. 

\begin{definition} \keyterm{酉矩阵} \keyterm{正交矩阵} 
    在复数域(实数域)上,矩阵 $ A $ 满足 $ \overline{A}^{\mathrm{T}} A = E $( $ {A}^{\mathrm{T}} A = E $ ),
    则矩阵 $ A $ 被称为酉矩阵(正交矩阵) 
\end{definition}

而如何刻画正交矩阵和酉矩阵的性质呢?下面的一个定理揭示了其与标准正交基的关系,可以从中窥得一些性质.  

\begin{theorem}
    设 $ (e_1, e_2, \ldots , e_n) $ 是复(实)内积空间 $ V $ 上的标准正交基,$ (f_1, f_2, \ldots , f_n) $ 是 $ V $ 上的一组基,
    从 $ (e_1, e_2, \ldots , e_n) $ 到 $ (f_1, f_2, \ldots , f_n) $ 的过渡矩阵为 $ A $. 则 $ (f_1, f_2, \ldots , f_n) $ 是标准正交基的
    充要条件是 $ A $ 为酉矩阵(正交矩阵). 
\end{theorem}

以下仅针对复内积空间的情况进行证明. 

\begin{proof}
    由过渡矩阵的定义,$ (f_1, f_2, \ldots , f_n) $ = $ (e_1, e_2, \ldots , e_n)A $,$ A = (a_{ij})_{n \times n} $. 

    由矩阵乘法的运算,可以得到
    \[ f_i = \sum_{j = 1}^{n} a_{ji}e_j , \enspace f_k = \sum_{j = 1}^{n} a_{jk}e_j. \]

    对两者做内积,有
    \[
    \langle f_i, f_k \rangle = \left\langle \sum_{j = 1}^{n} a_{ji}e_j, \sum_{j = 1}^{n} a_{jk}e_j \right\rangle
    = \sum_{j = 1}^{n} a_{ji}\overline{a_{jk}} 
    \]

    注意到 $ a_{ji}, j = 1, \ldots , n $ 是 $ A^{\mathrm{T}} $ 的第 $ i $ 行的元素,
    $ \overline{a_{jk}}, j = 1, \ldots , n $ 是 $ \overline{A} $ 的第 $ k $ 列的元素.
   
    定义 $ B = A^{\mathrm{T}}\overline{A} = (b_{ik})_{n \times n} $,则 $ \langle f_i, f_k \rangle = b_{ik} $. 

    必要性:如果 $ (f_1, f_2, \ldots , f_n) $ 是一组标准正交基,则
    \[
        b_{ik} = \langle f_i, f_k \rangle = 
        \begin{cases}
            1, & i = k \\
            0, & i \neq k 
        \end{cases}    
    \]

    由此可知 $ B = E $, $ \overline{B} = \overline{A}^{\mathrm{T}} A = \overline{E} = E $,即 $ A $ 是酉矩阵. 
    
    充分性:将必要性证明推理过程倒写即可. 

\end{proof}

如果这条定理中的 $ (e_1, e_2, \ldots , e_n) $ 取为该空间的自然基,就会有 $ (f_1, f_2, \ldots , f_n) = A $,我们便可以不太严谨地
得到如下的这个结论

\begin{theorem}
    矩阵 $ A $ 是酉矩阵(正交矩阵)等价于其列向量构成标准正交基. 
\end{theorem}

证明是平凡的,就交给你自己验证了. 

那提到了过渡矩阵,我们也就不得不提与之息息相关的一个等价关系——相似了. 
相信你已经回忆起来,相似实际上是同一个算子在不同基下的矩阵表示之间的关系,
实现这个变化正是依赖于两组基之间的过渡矩阵. 而我们的主线正是依靠基变换实现的,
只不过我们现在用的都是标准正交基,在基变换上也要有所升级. 所以,
让我们先定义两个特殊一点的相似关系:

\begin{definition}
    \begin{enumerate}

        \item \keyterm{酉相似}:复内积空间上,若 $ B = P^{-1}AP = \overline{P}^{\mathrm{T}}AP $,
        则称矩阵 $ A $ 与矩阵 $ B $ 酉相似. 

        \item \keyterm{正交相似}:实内积空间上,若 $ B = P^{-1}AP = {P}^{\mathrm{T}}AP $,
        则称矩阵 $ A $ 与矩阵 $ B $ 正交相似. 
    \end{enumerate}
\end{definition}

它俩的特殊之处你可能一下子没看出来,不过没关系,我们可以先回到它们对应的算子上去看看. 

\subsection{等距同构}

由之前一章,我们知道,算子与其伴随在同一组标准正交基下的矩阵表示是互为共轭对称的,所以设对应的算子是 $ S $,
则其应该满足 $ S^*S = I $. 那么这个性质能将我们导向何处呢?

考虑两侧同时作用向量 $ u $,再与 向量 $ v $ 做内积,那么我们得到了如下的式子:
\[ \langle S^*Su, v \rangle = \langle u, v \rangle. \]
再结合伴随的定义,稍微做个变换,就有了下面这个美妙的结果:
\[ \langle Su, Sv \rangle = \langle u, v \rangle. \]
也就是说,这个算子 $ S $ 同时作用在两个向量上的话不改变它们的内积. 
更进一步的话,如果取 $ v = u $,我们就能得到最终的结果:
\[ \lVert Su \rVert = \lVert u \rVert \]
算子 $ S $ 保持范数. 

\begin{definition}
    \keyterm{等距同构} 算子 $ S \in \mathcal{L}(V) $ 称为等距同构,如果 $ \forall v \in V $,
    都有 $ \lVert Su \rVert = \lVert u \rVert $.  
\end{definition}

注意我们这里虽然使用了共轭对称,但是从伴随的角度上来说复内积空间和实内积空间其实是一样的,
也就是说等距同构的概念在这两类空间上是一致的,只不过刻画上会有所差距,之后会有所介绍. 此外,
也常称实内积空间上的等距同构为正交算子,复内积空间上的等距同构称为酉算子. 

让我们看道简单的例题加深一下对等距同构的印象. 
\begin{example}
    设 $ \lambda_1, \ldots , \lambda_n $ 都是模为 1 的标量,
    $ e_1, \ldots , e_n $ 是 $ V $ 的标准正交基,$ S \in \mathcal{L}(V) $ 
    满足 $ Se_j = \lambda_je_j $,证明 $ S $ 是等距同构. 
\end{example}

\section{正定矩阵}

\vspace{2ex}
\centerline{\heiti \Large 内容总结}

\vspace{2ex}

\centerline{\heiti \Large 习题}
\vspace{2ex}
{\kaishu }
\begin{flushright}
    \kaishu

\end{flushright}
\centerline{\heiti A组}
\begin{enumerate}
    \item
\end{enumerate}
\centerline{\heiti B组}
\begin{enumerate}
    \item
\end{enumerate}
\centerline{\heiti C组}
\begin{enumerate}
    \item
\end{enumerate}

\input{./专题/26 二次型.tex}
\phantomsection
\section*{27 极分解与奇异值分解}
\addcontentsline{toc}{section}{27 极分解与奇异值分解}

\vspace{2ex}

\centerline{\heiti A组}
\begin{enumerate}
    \item
\end{enumerate}

\centerline{\heiti B组}
\begin{enumerate}
    \item
\end{enumerate}

\centerline{\heiti C组}
\begin{enumerate}
    \item
\end{enumerate}

\clearpage

\phantomsection
\section*{28 实空间上的算子}
\addcontentsline{toc}{section}{28 实空间上的算子}

\vspace{2ex}

\centerline{\heiti A组}
\begin{enumerate}
    \item
\end{enumerate}

\centerline{\heiti B组}
\begin{enumerate}
    \item
\end{enumerate}

\centerline{\heiti C组}
\begin{enumerate}
    \item
\end{enumerate}

\clearpage

\section*{29 行列式(II)}
\addcontentsline{toc}{section}{29 行列式(II)}

\vspace{2ex}

\centerline{\heiti A组}
\begin{enumerate}
    \item
\end{enumerate}

\centerline{\heiti B组}
\begin{enumerate}
    \item
\end{enumerate}

\centerline{\heiti C组}
\begin{enumerate}
    \item
\end{enumerate}

\clearpage

\input{./专题/30 线性代数与解析几何基础.tex}
\chapter{线性代数与多元微积分}

\section{向量函数的导数}

\section{行列式的导数}

\section{Jacobi 行列式}

\vspace{2ex}
\centerline{\heiti \Large 内容总结}

\vspace{2ex}
\centerline{\heiti \Large 习题}

\vspace{2ex}
{\kaishu }
\begin{flushright}
    \kaishu

\end{flushright}

\centerline{\heiti A组}
\begin{enumerate}
    \item
\end{enumerate}

\centerline{\heiti B组}
\begin{enumerate}
    \item
\end{enumerate}

\centerline{\heiti C组}
\begin{enumerate}
    \item
\end{enumerate}

\section*{32 线性代数与统计学}
\addcontentsline{toc}{section}{32 线性代数与统计学}

\vspace{2ex}

\centerline{\heiti A组}
\begin{enumerate}
    \item
\end{enumerate}

\centerline{\heiti B组}
\begin{enumerate}
    \item
\end{enumerate}

\centerline{\heiti C组}
\begin{enumerate}
    \item
\end{enumerate}

\clearpage


\backmatter
{\small
\printindex
\printindex[sym]
}

\end{document}
