\chapter*{序}

\section*{一些初衷}

我为这本讲义起了一个大胆的标题,它来源于浙江大学竺可桢学院线性代数II(H)课程选用的教材《线性代数应该这样学》(英文原版名:《Linear Algebra Done Right》)。我们带着半娱乐性质地将最后两个单词像矩阵求逆一样(见封面设计)进行了颠倒,得到了本书的英文名:《Linear Algebra Left Undone》。

接下来我们遇到了一个问题:中文名应该是什么呢?郑俊达同学提供了一个可行解:《线性代数:未竟之美》。转念一想,这一标题不能更契合我们的编写初衷。事实上,我们认为现行的大部分线性代数或高等代数教材具有如下问题,它们也困扰了笔者和许多读者的学习,我们也给出了解决的方案:
\begin{enumerate}
    \item 从线性代数的角度来看,它们的讲解顺序不够自然,大部分教材都从行列式起步,缺乏引入地给出各种概念,使得读者无法理解线性代数的本质。可以说这些教材应当更名``行列式与矩阵计算'',因为线性代数着重研究的线性空间和线性映射反而成为了边缘内容。因此我们采取了更好的讲解思路,更能体现线性代数的美感而非延续高中填鸭式的数学教育——事实上那根本称不上数学,那样的讲授思路根本不够``数学'',失去了数学本身的自然之美,而且使得读者误解数学、厌恶数学;

    \item 浙江大学竺可桢学院两学期线性代数课程选择的《大学数学:代数与几何》和《线性代数应该这样学》教材采用了从抽象空间引入的方式,更贴近本质。但实践过程中许多同学会对``为什么要一开始就学习这些抽象内容''缺乏概念,特别是《线性代数应该这样学》对于工科同学而言``数学味道太浓'',因此最后可能学习效果还不如填鸭式地灌输解题方法。因此我们在讲义中相当于为教材做了很多的注脚,并且优化了整体设计,提供了大量例题习题,都是为了能更自然地引入抽象内容,让读者知道我们为何要学习这些内容,这些内容当年在数学家眼中最自然的状态是什么,这样才能使得抽象的概念易于被初学者接受;

    \item 我们的例题和习题编排也是精心设计过的,不会出现大部分教材使用过程中``上课讲的、作业做的和考试考的脱节''的情况,这一问题不只是很多数学基础课教学的问题,也是国内各个专业都存在的教学问题,笔者也深受其害,所以编写例题习题特别注重对概念和定理的理解、对方法的掌握,不会出现教材中说什么知识很重要但没有例子体会很重要的这种抽象情况,并且大量的习题贴近所学知识也贴近考试,让读者通过习题更好地掌握知识而非反而迷惑不知道自己学了什么,才能更好地体会线性代数的美感而非感受到题海的压迫;

    \item 除了自然的美感外,更重要的是还有``未竟''的美感。线性代数是一个古老而年轻的学科。它发轫于早先对线性方程组的研究,经历了漫长的几何和代数的交错作用,最后又在近世代数的发展过程中被严格化。直到现在,一些相关的内容,例如线性代数群的研究尚且方兴未艾,在现代数学的种种支线当中也有着重要的应用。另外,它的方法论,尤其是其对代数结构的研究在现代数学中也具备着代表性。因此,我们希望呈现一个更广阔的线性代数观,从线性代数出发,对它的现代发展和它在现代数学的各个分支的应用进行一些导论性的介绍,这一方面是为了使得平淡的叙述更加有趣,另一方面也是为了回答一个疑问:线性代数到底有什么用?我们相信,这是许多初学者都有的一个问题,回答这个问题既需要对线性代数的深入学习,也需要有一个现代数学的全局观,这也就带来了本书的另一个部分,未竟专题,也是我们这本书的标题来源。
\end{enumerate}

古人有三不朽:立德、立功、立言,著书立说即为立言。虽说我完全不可能因为编写了一本基础课的讲义而有如此崇高的地位。但在我心里,我已经通过这本讲义将我的热情、我的想法传达给了不少的读者,这样也无愧于我在浙江大学的本科四年。未来或许这本讲义会淹没在历史的风尘中,但我想只要它的某行文字曾经给予读者一丝丝的启发,或更实际地帮助了读者得到了心仪的分数,我想它就是有价值的,我本人的价值也得到了一定的实现。

\section*{本书的面世}

自2021年秋参与浙江大学竺可桢学院学研部(现竺可桢学院学业指导中心)组织的朋辈辅学活动以来,我即将第三次参与线性代数荣誉课的辅学活动。犹记讲义最初的简陋版本,那时为了辅学的期中、期末准备的简单复习提要,里面因为本人时间有限甚至缺少了特征值与特征向量的内容。那时的讲义基本都是知识点的罗列,缺少了许多重要的例题和证明,犹记第一次拿起这个讲义站上讲台的时候,我深刻体会到了这一讲义的不足,因此那次的授课整体而言较为玄学,比较干瘪。因此在2022年再次参加辅学授课时,我借着疫情放开考试延迟的机会分了六个大专题写出了一本相对完整的适合于《大学数学:代数与几何》的复习讲义。里面的讲解比较全面,习题也十分丰富,可以说在复习资料中已经能算过得去的一版了。

但我并不满足于此,我希望这本讲义能成为一本真正的完整的讲义,能兼具配套学习、考试复习的功能,并且在保证体系严谨完整的前提下有更优化的讲解逻辑。因此在2023年的暑假,我基于原先的复习版本进行扩展重排。在这一版本中,我将原先复习资料中的粗略描述都换为了严谨的完整叙述,并且反复打磨讲解顺序,从而更自然地将另一本教材《线性代数应该这样学》的内容自然融合,并且添加了大量的remark更适合于初学者学习。更重要的是,我们中间添加了许多文字叙述,一方面自然引入我们要讲解的内容,这对于初学者而言是很重要的insight,另一方面反复强调我们的行文逻辑,对推进逻辑做适当总结,使得读者能更快地形成体系,同时也补充了很多拓展内容,一些是为了方便读者更自然地理解抽象内容,有一些是契合``未竟之美''的标题,让读者能体会到数学的美感,体会到学习线性代数后我们知识的边界可以推广到多远。

\section*{参考文献}

本讲义作为浙江大学竺可桢学院线性代数荣誉课的辅学讲义,因此核心思路来源于我们选择的教材《大学数学:代数与几何(第二版)》(居余马,李海中)、《大学数学:代数与几何学习辅导》(林翠琴,居余马)、《线性代数应该这样学(第三版)》([美]Sheldon Axler)。

在编写与修订的过程中,我也参考了其他非常多优质的教材或辅导资料,如《高等代数(第二版)》(丘维声)、《高等代数:学习指导书》(丘维声)、《高等代数学(第四版)》(谢启鸿,姚慕生,吴泉水)、《高等代数学(第四版)配套学习用书》(谢启鸿,姚慕生)、《线性代数辅导讲义》(汤家凤)、《高等代数强化讲义》(李扬)以及《高等代数考研:高频真题分类精解300例》等,在复数域的引入部分我也简单参考了王晓光老师的《复变函数讲义(2023版)》,在矩阵计算等专题则部分参考了《数值分析》(Timothy Sauer)、《矩阵分析》(Roger A.Horn,Charles R.Johnson)等计算数学著作。

最后如果读者学完本讲义后对代数学有浓厚的兴趣,非常推荐读者学习后续的抽象代数课程。这里推荐与我同级的图灵班同学编写的\href{https://frightenedfoxcn.github.io/notes/series/alg-for-cs/}{《写给计算机系学生的代数》}作进一步的了解,我们许多高级专题都对这一讲义有引用。

\section*{致谢}

我或许首先需要感谢2022年疫情放开之下的寒冬,没有这学期线性代数考试的延期,我也不会有如此充裕的时间整理出本讲义较为完整的底稿,也就没有这一完整讲义的面世。

我还需要感谢同级的王和钧同学,感谢他当年push我写出了最初版本的复习提要。我要感谢比我低一级的郭苗苗同学,感谢她当年反复邀请我走上讲台实现梦想,虽然可能第一次授课效果一般,但这对于后来我不断打磨授课方式,打磨讲义有非常重要的意义。我也应当感谢竺可桢学院学研部(现竺可桢学院学业指导中心)给我提供了一个辅学的平台,让我通过讲义将我的热情能传达给更多的读者。

我想我也应该特别感谢数学科学学院的吴志祥、谈之奕和刘康生老师,他们在我线性代数入门过程中做了重要的引路人的工作,讲义中许多讲解思路也来源于他们精彩的授课。我也应该特别感谢数学科学学院的王晓光老师,他在复变函数课程以及讲义上的热情以及倾注的心血启发我也应该将我的学习思路和经验通过讲义传达给更多人,并且启发我思考如何从更高的观点、更自然的角度引导读者学习新知识,享受追求真理的过程。

感谢每一位读者和支持者,没有你们的支持,我也不会有如此的热情坚持编写这本讲义,正是有了大家的支持才有了接下来越来越好的版本的面世。特别感谢林前旭同学,他将这本讲义推广到了\href{https://mp.weixin.qq.com/s/nOQ0xzJ0mX2_8JclcKiWdA}{浙江大学微信公众号},这对我而言是极大地鼓舞。

最后,我需要感谢在编写的过程中对本讲义提供了直接的支持的同学:
\begin{itemize}
    \item 梅敏炫同学主编了讲义内积部分;
    \item 高天健同学对线性空间、矩阵、行列式、特征值以及内积等部分提供了宝贵的新讲解思路;
    \item 王普同学主编了线性代数与微积分部分;
    \item 郑俊达同学主编了解析几何部分;
    \item 周健均同学编写了行列式计算进阶以及奇异值分解的应用部分;
    \item 朱熙哲、谢集、郑俊达、郑涵文、李英琦同学负责了答案的初次编写,金政羽、张晋恺、江舜尧、任朱明、赵嘉瑞同学负责了历年卷答案以及正文答案的二次编写;
    \item 王鹤翔同学设计了本讲义的封面;
    \item 李英琦同学全权负责了本讲义的格式设计以及插图;
    \item 刘泓健同学负责了讲义的未竟专题部分。
\end{itemize}
除此之外,还有为我们的\href{https://github.com/yhwu-is/Linear-Algebra-Left-Undone}{GitHub仓库}提出 issue 以及提交 PR 的其他同学。没有他们的支持,这本讲义不会有今天的完整程度。

\begin{flushright}
    \kaishu
    吴一航 \\
    浙江大学计算机科学与技术学院 \\
    \verb|yhwu_is@zju.edu.cn| \\
    2023 年 8 月
\end{flushright}
