\phantomsection
\section*{10 矩阵运算进阶(I)}
\addcontentsline{toc}{section}{10 矩阵运算进阶(I)}

\vspace{2ex}

\centerline{\heiti A组}
\begin{enumerate}
    \item 由题意有 $P_2AP_1 = E$,从而有 $A=P_2^{-1}P_1^{-1}=P_2P_1^{-1}$.

    \item 由题意知 $B = E_{ij}A$,所以 $BA^{-1}=E_{ij}$ 从而 $B$ 可逆,同时可得 $AB^{-1}=A(A^{-1}E_{ij}^{-1})=E_{ij}$.

    \item $Q = (\alpha_1+\alpha_2,\alpha_2,\alpha_3)=(\alpha_1,\alpha_2,\alpha_3)\begin{pmatrix}1 & 0 & 0 \\ 1 & 1 & 0 \\ 0 & 0 & 1\end{pmatrix}=P\begin{pmatrix}1 & 0 & 0 \\ 1 & 1 & 0 \\ 0 & 0 & 1\end{pmatrix}$.

          记 $E_{12}(1)=\begin{pmatrix}1 & 0 & 0 \\ 1 & 1 & 0 \\ 0 & 0 & 1\end{pmatrix}$,有 $Q^{-1}AQ=E_{12}(1)^{-1}P^{-1}APE_{12}(1)=\begin{pmatrix}1 & 0 & 0 \\ 0 & 1 & 0 \\ 0 & 0 & 2\end{pmatrix}$.

    \item 我们只求解第一个矩阵的逆,第二个方法类似. 事实上正文例题中也有类似的,本题甚至更简单,因为左下角已经是零矩阵,所以不再需要``打这个洞''. 设$A$和$D$分别是$m$、$n$阶矩阵,我们首先将第二个分块行左乘$-BD^{-1}$加到第一个分块行,目的很明确,就是把右上角的洞打出来:
          \[\left(\begin{array}{cc:cc}
                      A & B & E_m & O \\ O & D & O & E_n
                  \end{array}\right)\rightarrow\left(\begin{array}{cc:cc}
                      A & O & E_m & -BD^{-1} \\ O & D & O & E_n
                  \end{array}\right),\]
          接下来再用$A^{-1}$和$D^{-1}$分别左乘第一分块行和第二分块行,得到
          \[\left(\begin{array}{cc:cc}
                      E_m & O & A^{-1} & -A^{-1}BD^{-1} \\ O & E_n & O & D^{-1}
                  \end{array}\right),\]
          由此可得原矩阵的逆就是上述虚线右侧的$\begin{pmatrix}
                  A^{-1} & -A^{-1}BD^{-1} \\ O & D^{-1}
              \end{pmatrix}$.
\end{enumerate}

\centerline{\heiti B组}
\begin{enumerate}
    \item 此处仅给出答案,具体过程略.
          \begin{enumerate}
              \item $\begin{pmatrix}a_{21} & a_{22} & a_{23} \\ a_{11} & a_{12} & a_{13} \\ a_{31} & a_{32} & a_{33}\end{pmatrix}$.

              \item $\begin{pmatrix}-a_{11} & -a_{12} & -a_{13} \\ a_{21} & a_{22} & a_{23} \\ a_{31} & a_{32} & a_{33}\end{pmatrix}$.

              \item $\begin{pmatrix}-a_{13} & -a_{12} & a_{11} \\ a_{23} & a_{22} & -a_{21} \\ a_{33} & a_{32} & -a_{31}\end{pmatrix}$.

              \item $\begin{pmatrix}-a_{11}-a_{12} & -a_{12}-a_{13} & -a_{13}-a_{11} \\ a_{21}+a_{22} & a_{22}+a_{23} & a_{23}+a_{21} \\ a_{31}+a_{32} & a_{32}+a_{33} & a_{33}+a_{31}\end{pmatrix}$.
          \end{enumerate}

    \item \begin{enumerate}
              \item 略.

              \item 设 $A=(a_{ij})_{3\times 2}$,$e_{ij}$ 为 $\mathbf{R}^{3\times 2}$ 的自然基. 因为 $PAQ = \begin{pmatrix}a_{12}+a_{22} & 0 \\ a_{22} & 0 \\ 0 & 0\end{pmatrix}$,所以 $\sigma(e_{12}) = e_{11},\sigma(e_{22}) = e_{11}+e_{21},\sigma(e_{11})=\sigma(e_{21})=\sigma(e_{31})=\sigma(e_{32})=0$.

                    于是 $\ker\sigma = \spa (e_{11},e_{21},e_{31},e_{32}),\im\sigma = \spa (e_{11},e_{11}+e_{21})$.

              \item 令 $B_1=\{e_{12},e_{22},e_{11},e_{21},e_{31},e_{32}\},B_2=\{e_{11},e_{11}+e_{21},e_{12},e_{22},e_{31},e_{32}\}$,则均为 $\mathbf{R^{3\times 2}}$ 的基,且 $\sigma(\varepsilon)=(\eta)\begin{pmatrix}E_2 & 0 \\ 0 & 0\end{pmatrix}$.
          \end{enumerate}

    \item 见教材 P147/例 5
\end{enumerate}

\centerline{\heiti C组}
\begin{enumerate}
    \item 使用数学归纳法. 当 $n=1$ 时,$A=\begin{pmatrix}a\end{pmatrix}(a\neq 0)$,$B$ 取任意一阶矩阵均成立;假设 $n-1$ 阶成立,$A = \begin{pmatrix}A_1 & \alpha \\ \beta & a_{nn}\end{pmatrix}$,其中 $A_1$ 为 $n-1$ 阶矩阵且存在 $n-1$ 阶下三角矩阵 $B_1$ 使得 $B_1A_1$ 为上三角矩阵,则有
          \[\begin{pmatrix}B_1 & O \\ O & 1\end{pmatrix}\begin{pmatrix}A_1 & \alpha \\ O & a_{nn}-\beta A^{-1}\alpha\end{pmatrix} = \begin{pmatrix}B_1A_1 & B_1\alpha \\ O & a_{nn}-\beta A_1^{-1}\alpha\end{pmatrix}\]为上三角矩阵.
          而 \[\begin{pmatrix}E_{n-1} & O \\ -\beta A_1^{-1} & 1\end{pmatrix}\begin{pmatrix}A_1 & \alpha \\ \beta & a_{nn}\end{pmatrix}=\begin{pmatrix}A_1 & \alpha \\ O & a_{nn}-\beta A_1^{-1} \alpha\end{pmatrix}\]
          故 $B=\begin{pmatrix}B_1 & O \\ O & 1\end{pmatrix}\begin{pmatrix}E_{n-1} & O \\ -\beta A_1^{-1} & 1\end{pmatrix}$ 符合条件($B_1$ 为下三角矩阵,故 $B$ 也是).

    \item 证明:$ \forall \alpha \in W,\enspace A_{12} \alpha = \vec{0} $.
          \begin{align*}
              \begin{pmatrix} O_{k \times 1} \\ \alpha \end{pmatrix}
               & = A^{-1}A \begin{pmatrix} O_{k \times 1} \\ \alpha \end{pmatrix} = A^{-1} \begin{pmatrix} A_{11} & A_{12} \\ A_{21} & A_{22} \end{pmatrix} \begin{pmatrix} O_{k \times 1} \\ \alpha \end{pmatrix} = A^{-1} \begin{pmatrix} O_{l \times 1} \\ A_{22} \alpha \end{pmatrix} \\
               & = \begin{pmatrix} B_{11} & B_{12} \\ B_{21} & B_{22} \end{pmatrix} \begin{pmatrix} O_{l \times 1} \\ A_{22} \alpha \end{pmatrix} = \begin{pmatrix} B_{12} A_{22} \alpha \\ B_{22} A_{22} \alpha \end{pmatrix}
          \end{align*}
          故 $ B_{12} A_{22} \alpha = \vec{0} $,故我们可以推测如下定义:$ \sigma \in \mathcal{L}(W,U),\enspace \sigma(\alpha) = A_{22} \alpha $.
          只需证明 $ \sigma $ 是单射且满射即可.

          单射:$ \sigma(\alpha) = \sigma(\beta) \implies A_{22}(\alpha - \beta) = \vec{0} $. 又有 $ A_{12} \alpha = A_{12} \beta = \vec{0} \implies A_{12}( \alpha - \beta) = \vec{0} $. 故 $ \begin{pmatrix} A_{12} \\ A_{22} \end{pmatrix} (\alpha - \beta) = \vec{0} $. 由于 $ \begin{pmatrix} A_{12} \\ A_{22} \end{pmatrix} $ 列满秩($ A $ 可逆),故 $ \alpha = \beta $.

          满射:$ \forall \gamma \in U,\enspace B_{12} \gamma = \vec{0} $.
          \begin{align*}
              \begin{pmatrix} O_{l \times 1} \\ \gamma \end{pmatrix}
               & = A A^{-1} \begin{pmatrix} O_{l \times 1} \\ \gamma \end{pmatrix} = A \begin{pmatrix} B_{11} & B_{12} \\ B_{21} & B_{22} \end{pmatrix} \begin{pmatrix} O_{l \times 1} \\ \gamma \end{pmatrix}      \\
               & = A \begin{pmatrix} O_{k \times 1} \\ B_{22} \gamma \end{pmatrix} = \begin{pmatrix} A_{11} & A_{12} \\ A_{21} & A_{22} \end{pmatrix} \begin{pmatrix} O_{k \times 1} \\ B_{22} \gamma \end{pmatrix} \\
               & = \begin{pmatrix} A_{12} B_{22} \gamma \\ A_{22} B_{22} \gamma \end{pmatrix}
          \end{align*}
          故 $ \exists B_{22} \gamma \in W,\enspace A_{22} B_{22} \gamma = \gamma \in U $.
\end{enumerate}

\clearpage
