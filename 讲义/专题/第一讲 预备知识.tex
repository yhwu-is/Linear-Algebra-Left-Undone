\chapter{预备知识}

\indent 线性代数作为大学的第一门数学课,预修要求并不高。我们默认读者具有基本的
高中数学知识,因此关于集合、映射以及向量的基本知识我们不在此赘述。这一讲
我们将介绍本书中常见的概念——等价类,最常用的算法之一——高斯消元法以及为
引入线性空间做铺垫的基本代数结构的内容。

\section{等价类}


\section{高斯消元法}


\section{基本代数结构}


\vspace{2ex} 
\centerline{\heiti \Large 内容总结}
\vspace{2ex} 

在本讲中我们介绍了

\centerline{\heiti \Large 习题}
\vspace{2ex} 
{\kaishu 我这门课很简单,只有简单的加减乘除四则运算,甚至除法都不太需要。}
\begin{flushright}
    \kaishu
	——浙江大学数学科学学院教授吴志祥
\end{flushright}
\centerline{\heiti A组}
\begin{enumerate}
	\item 
\end{enumerate}
\centerline{\heiti B组}
\begin{enumerate}
	\item 
\end{enumerate}
\centerline{\heiti C组}
\begin{enumerate}
	\item 
\end{enumerate}