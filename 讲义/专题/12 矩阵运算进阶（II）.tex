\chapter{矩阵运算进阶(II)} \label{chap:矩阵运算进阶(II)}

本讲我们将讨论技巧性更强的一些内容,如特殊矩阵,矩阵可交换、求逆和求幂等. 大部分都是方法为主,理解的内容不多,但对于我们拿到问题有更好的洞察是很重要的.

\section{特殊矩阵}

\subsection{对角矩阵}

我们一般记主对角矩阵为$\diag(d_1,d_2,\dots,d_n)$,准对角矩阵为$\diag(A_1,A_2,\dots,A_n)$.
\begin{theorem}
    设$A$是一个$s \times n$矩阵,把$A$写成列向量与行向量的形式,分别为

    \[ A = \begin{pmatrix}\alpha_1 & \alpha_2 & \cdots & \alpha_n\end{pmatrix} = \begin{pmatrix} \beta_1 \\ \beta_2 \\ \vdots \\ \beta_n \end{pmatrix} \]
    则
    \begin{gather*}
        \begin{pmatrix}\alpha_1 & \alpha_2 & \cdots & \alpha_n\end{pmatrix}
        \begin{pmatrix}
            d_1 &     &        &     \\
                & d_2 &        &     \\
                &     & \ddots &     \\
                &     &        & d_n
        \end{pmatrix} = \begin{pmatrix}d_1\alpha_1 & d_2\alpha_2 & \cdots & d_n\alpha_n\end{pmatrix} \\
        \begin{pmatrix}
            d_1 &     &        &     \\
                & d_2 &        &     \\
                &     & \ddots &     \\
                &     &        & d_n
        \end{pmatrix} \begin{pmatrix} \beta_1 \\ \beta_2 \\ \vdots \\ \beta_n \end{pmatrix} = \begin{pmatrix} d_1\beta_1 \\ d_2\beta_2 \\ \vdots \\ d_n\beta_n \end{pmatrix}
    \end{gather*}

    即$A$右乘对角矩阵$\diag(d_1,d_2,\ldots,d_n)$相当于给$A$的第$i$列元素都乘以$d_i$,$A$左乘对角矩阵$\diag(d_1,d_2,\ldots,d_n)$相当于给$A$的第$i$行元素都乘以$d_i$,其中 $i=1,2,\ldots,n$.
\end{theorem}

\begin{theorem}
    对角矩阵和分块对角矩阵的性质:
    \begin{enumerate}
        \item 对角矩阵$\diag(d_1,d_2,\ldots,d_n)$可逆当且仅当对角线上元素均不为0,且此时逆矩阵为$\diag(d_1^{-1},d_2^{-1},\ldots,d_n^{-1})$.

        \item 分块对角矩阵$\diag(A_1,A_2,\ldots,A_n)$可逆当且仅当每个分块$A_i$可逆,且此时逆矩阵为$\diag(A_1^{-1},A_2^{-1},\ldots,A_n^{-1})$.

        \item 两个对角矩阵$A=\diag(a_1,a_2,\ldots,a_n),\enspace B=\diag(b_1,b_2,\ldots,b_n)$的乘积仍然是对角矩阵,且$AB=\diag(a_1b_1,a_2b_2,\ldots,a_nb_n)$.

              对于乘方运算,有$A^k=\diag(a_1^k,a_2^k,\ldots,a_n^k)$.

        \item 两个准对角矩阵$A=\diag(A_1,A_2,\ldots,A_n),\enspace B=\diag(B_1,B_2,\ldots,B_n)$中$A_i$和$B_i$是同级方阵,则乘积仍然是准对角矩阵,且$AB=\diag(A_1B_1,A_2B_2,\ldots,A_nB_n)$.
    \end{enumerate}
\end{theorem}

这里需要说明的是,本节定理都是通过简单计算即可验证的,因此在此不给出证明.

\subsection{上(下)三角矩阵}

\begin{theorem}
    已知$A,B$都是上三角矩阵,且设$A$的主对角元素分别为$a_{11},\ldots,a_{nn}$,$B$的主对角元素分别为$b_{11},\ldots,b_{nn}$,则
    \begin{enumerate}
        \item $A^{\mathrm{T}}, B^\mathrm{T}$都是下三角矩阵;

        \item $AB$仍然是上三角矩阵,且$AB$的主对角元素为$a_{11}b_{11},\ldots,a_{nn}b_{nn}$;

        \item $A$可逆的充要条件是其主对角元均不为0,且$A$可逆时,$A^{-1}$也是上三角矩阵,并且$A^{-1}$的主对角元素分别为$a_{11}^{-1},\ldots,a_{nn}^{-1}$.
    \end{enumerate}
\end{theorem}

\begin{example}
    已知$A_1,\ldots,A_n$是$n$个对角元都为0的上三角矩阵,证明:$A_1A_2\cdots A_n=O$.
\end{example}

\begin{proof}

\end{proof}

\subsection{基本矩阵}

只有一个元素为1,其余元素全为0的矩阵称为基本矩阵,第$i$行第$j$列元素为1的基本矩阵记为$E_{ij}$,它们具有如下性质(可以联系左右乘对应行列变换进行记忆):
\begin{theorem}
    基本矩阵计算具有如下性质:
    \begin{enumerate}
        \item $AE_{ij}$的结果就是把$A$的第$i$列移到第$j$列的位置,其余元素都为0的矩阵;

        \item $E_{ij}B$的结果就是把$B$的第$j$行移到第$i$行的位置,其余元素都为0的矩阵;

        \item $E_{ik}E_{lj} = \begin{cases}
                      E_{ij} & k = l    \\
                      O      & k \neq l
                  \end{cases}$.
    \end{enumerate}
\end{theorem}

\subsection{其他矩阵}

其他特殊矩阵如正交矩阵、置换矩阵、幂等矩阵、幂零矩阵等,我们将在后续讲义合适的位置描述它们的性质,那时我们的讨论不局限于本节的运算性质,会有更多的其它性质.

\section{矩阵可交换问题}

首先我们需要强调一点:一般来说在本课程中此类问题直接设可交换矩阵的每一个元素都是未知数即可. 我们来看下面的例子:
\begin{example}
    求所有与$A$可交换的矩阵,其中
    \[A=\begin{pmatrix}
            1 & 0 & 0  \\
            0 & 1 & 2  \\
            0 & 1 & -2
        \end{pmatrix}.\]
\end{example}

对于一些矩阵直接设未知数计算比较复杂,这里我们讨论一个基本的技巧,即利用
\[\forall t,\enspace AB=BA \iff (A-tE)B=B(A-tE).\]
这一等式成立是显然的. 运用时难点主要在决定$t$的值,我们要根据矩阵的对角线上元素来决定,原则是使得$B$与$A-tE$相乘的计算过程更为简单(一般是使得0元素更多),这样解方程也会更轻松. 我们看一个简单的例子来体会:
\begin{example}
    求与矩阵$A=\begin{pmatrix}
            3  & 0  & 0 \\
            -1 & 3  & 0 \\
            0  & -1 & 3
        \end{pmatrix}$可交换的矩阵.
\end{example}

\begin{solution}

\end{solution}

\begin{theorem}
    我们有如下关于可交换矩阵更一般的结论:
    \begin{enumerate}
        \item 与主对角元两两互异的对角矩阵可交换的方阵只能是对角矩阵;

        \item 准对角矩阵$A$每个对角分块内对角线元素相同,但不同对角块之间不同,则与$A$可交换的矩阵只能是准对角矩阵;

        \item 与所有$n$级可逆矩阵可交换的矩阵为数量矩阵;

        \item 与所有$n$级矩阵可交换的矩阵为数量矩阵.
    \end{enumerate}
\end{theorem}

\begin{proof}

\end{proof}

除此之外我们还有一些和特殊的矩阵可交换的结论,我们将在习题中见到它们. 因为技巧性过强正文中不展开叙述,感兴趣的同学可以参考习题C组进行了解.

\section{矩阵的逆进阶求法}

\subsection{给定多项式求逆矩阵}

此类题目出现最为频繁,实际上就是通过一些初中所学的因式分解等基本变换得到需要求逆的矩阵与另一个矩阵相乘可以得到单位矩阵(的一个倍数).
\begin{example}
    设$A$为非零矩阵,且$A^3=O$,证明:$E+A$和$E-A$都可逆.
\end{example}

\begin{proof}

\end{proof}

\begin{example}
    若$X,Y$是两个列向量,且$X^\mathrm{T}Y=2$,证明:
    \begin{enumerate}
        \item $(XY^\mathrm{T})^k=2^{k-1}(XY^{\mathrm{T}})$;

        \item 如果$A=E+XY^\mathrm{T}$,则$A$可逆,并求其逆矩阵.
    \end{enumerate}
\end{example}

\begin{proof}

\end{proof}

\subsection{利用分块矩阵初等变换}

在分块矩阵中,我们已经讲解了分块矩阵初等变换打洞法的基础题型,这里再给出一些更一般的例子:
\begin{example}\label{ex:12:打洞法求逆1}
    设$A,B$为$n$阶矩阵,证明:若$E\pm AB$可逆,则$E\pm BA$可逆.
\end{example}

\begin{proof}

\end{proof}

\begin{example}
    设$A$为$n$阶矩阵,$B,C$分别为$n \times m$和$m \times n$阶矩阵. 证明:$E_m+CA^{-1}B$可逆$\iff A+BC$可逆.
\end{example}

\begin{proof}

\end{proof}

\subsection{求逆的分式思想}

虽然矩阵没有除法运算,但是我们如果将$(E-A)^{-1}$写成$\dfrac{E}{E-A}$,再类比泰勒展开
\[\frac{1}{1-x}=1+\sum_{n=1}^\infty x^n \qquad x\in (-1,1)\]
我们可以得到(不严谨!只能用来解题的时候当作初步的思路!)
\[(E-A)^{-1}=\frac{E}{E-A}=E+A+A^2+\cdots\]

\begin{example}
    已知方阵$A$满足$A^k=O$,其中$k$是一个正整数,求$E-A$的逆.
\end{example}

\begin{solution}

\end{solution}

\begin{example}
    设$A,B$分别是$n \times m$和$m \times n$的矩阵,且$E_n \pm AB$可逆,则$E_m \pm BA$可逆.
\end{example}
不难发现这一例是\autoref{ex:12:打洞法求逆1} 的推广,因为此处不再限制方阵.

\begin{proof}

\end{proof}

\subsection{提逆思想}

这一思想的来源是矩阵逆没有加减相关的运算法则(即没有$(A+B)^{-1}=A^{-1}+B^{-1}$这样的性质),因此我们需要提逆产生一些乘积项来解决问题.
\begin{example}
    设$A$是$n$阶方阵,且$E-A$,$E+A$和$A$都可逆,证明:$(E-A^{-1})^{-1}+(E-A)^{-1}=E$.
\end{example}

\begin{proof}

\end{proof}

\section{矩阵的幂} \label{sec:12:矩阵的幂}

\begin{enumerate}
    \item 找规律

          在矩阵的转置\autoref{ex:9:转置求幂} 中我们已经见识了一种找规律的方式,下面是一种类似的题型:
          \begin{example}
              计算$(PAQ)^k$,其中
              \[P=\begin{pmatrix}2 & 3 \\ 1 & 2\end{pmatrix},\enspace A=\begin{pmatrix}2 & 0 \\ 0 & -1\end{pmatrix},\enspace Q=\begin{pmatrix}2 & -3 \\ -1 & 2\end{pmatrix}\]
          \end{example}
          本质而言此类题目只需要发现中间多次出现的乘积$QP$是很容易处理的矩阵(\autoref{ex:9:转置求幂} 中甚至是一个数)即可解决.

          \begin{solution}

          \end{solution}

          \begin{example}
              设$A=\begin{pmatrix}0 & -1 & 0 \\ 1 & 0 & 0 \\ 0 & 0 & -1 \end{pmatrix},\enspace P^{-1}AP=B$,求$B^{2004}-2A^2$.
          \end{example}
          \begin{solution}

          \end{solution}

          还有一种找规律基于幂等矩阵(满足$A^2=A$的矩阵,根据数学归纳法可证明$A$的任意次方都是$A$),显然幂等矩阵的任意次方都与其本身相等是很好的性质,另一种找规律基于对合矩阵,即平方等于单位矩阵的矩阵,我们这里主要与大家分享另一种关于幂零矩阵(矩阵某次幂可以得到零矩阵)的方法,例子如下:
          \begin{example}
              求$A=\begin{pmatrix}a & 1 & 0 & 0 \\ 0 & a & 1 & 0 \\ 0 & 0 & a & 0 \\ 0 & 0 & 0 & a \end{pmatrix}^n$.
          \end{example}
          在上例中,我们采用将矩阵分为$A=tE+B$的方法,会发现矩阵$B$为上三角矩阵且对角线上全为0,这是需要读者记忆的典型的幂零矩阵,未来在幂零矩阵的讨论中我们将严格证明这一点,现在我们只需利用这一性质快速解题.

          \begin{solution}

          \end{solution}

    \item 数学归纳法
          \begin{example}
              求$A=\begin{pmatrix}\cos\alpha & \sin\alpha \\ -\sin\alpha & \cos\alpha\end{pmatrix}^n$.
          \end{example}
          这一问题对应我们常见的旋转变换(所以建议要求读者记忆这一矩阵形式),$n$次方就是旋转$n$次. 当然这是直观而言的结论,严谨说明可以通过数学归纳法证:

          \begin{solution}

          \end{solution}

          \begin{example}
              证明$\begin{pmatrix}
                      a & c \\ 0 & b
                  \end{pmatrix}^n=\begin{pmatrix}
                      a^n & (a^{n-1}+a^{n-2}b+\dots+b^{n-1})c \\ 0 & b^n
                  \end{pmatrix}$.
          \end{example}
          \begin{proof}

          \end{proof}

    \item 利用秩为1的矩阵

          这一方法的核心是利用上一讲中\autoref{ex:11:相抵分解} 的结论,我们来看下面的例子进行体会:
          \begin{example}
              已知$M$是秩为 1 的矩阵,记$\tr(M)=b$,讨论$(aE+M)^n$的计算结果.
          \end{example}
          \begin{solution}

          \end{solution}

          \begin{example}
              已知$A$是数域$P$上的一个2阶方阵,且存在正整数$l$使得$A^l=O$,证明:$A^2=O$.
          \end{example}
          事实上,将来我们讨论幂零矩阵的时候将会进一步推广本例的结论.

          \begin{proof}

          \end{proof}

          \begin{example}
              已知数列$\{a_n\},\enspace\{b_n\}$满足$a_0=-1,\enspace b_0=3$,且
              \[\begin{cases}
                      a_n=3a_{n-1}+b_{n-1}+2^{n-1} \\
                      b_n=2a_{n-1}+4b_{n-1}+2^n
                  \end{cases}\]
              求$\{a_n\},\enspace\{b_n\}$的通项公式.
          \end{example}
          \begin{solution}

          \end{solution}

    \item 利用初等矩阵的性质
          \begin{example}
              设$A$为三阶矩阵,$P$为三阶可逆矩阵,$P^{-1}AP=B$,其中
              \[P=\begin{pmatrix}
                      0 & 2 & -1 \\ 1 & 1 & 2 \\ -1 & -1 & -1
                  \end{pmatrix},\enspace B=\begin{pmatrix}
                      0 & 0 & -1 \\ 0 & -1 & 0 \\ -1 & 0 & 0
                  \end{pmatrix},\]
              求$A^{2022}$.
          \end{example}
          \begin{solution}

          \end{solution}

    \item 利用对角化和若当标准形:我们将在后续相应章节中讲解.
\end{enumerate}

\vspace{2ex}
\centerline{\heiti \Large 内容总结}

\vspace{2ex}
\centerline{\heiti \Large 习题}

\vspace{2ex}
{\kaishu }
\begin{flushright}
    \kaishu

\end{flushright}

\centerline{\heiti A组}
\begin{enumerate}
    \item 设方阵$A$满足$A^2-A-2E=O$,证明:
          \begin{enumerate}
              \item $A$和$E-A$都是可逆矩阵,并求它们的逆矩阵;

              \item $A+E$和$A-2E$不可能同时可逆.
          \end{enumerate}

    \item 若$A,B$为两个$n$阶矩阵且满足$A+B=AB$,证明:
          \begin{enumerate}
              \item $A-E$和$B-E$均可逆;

              \item $AB=BA$;

              \item $r(A)=r(B)$.
          \end{enumerate}
\end{enumerate}

\centerline{\heiti B组}
\begin{enumerate}
    \item 已知矩阵$A=\begin{pmatrix}
                  1 & 0 & 4 \\ 0 & 1 & 2 \\ 0 & 1 & 2
              \end{pmatrix}$,求证:所有与$A$可交换的矩阵构成$\mathbf{M}_3(\mathbf{R})$的一个子空间,并求子空间的一组基.

    \item 已知矩阵$A=\begin{pmatrix}
                  1 & 0 & 1 \\ 0 & 2 & 0 \\ 1 & 0 & 1
              \end{pmatrix}$,
          \begin{enumerate}
              \item 求所有与$A$可交换的矩阵;

              \item 若$AB+E=A^2+B$,求$B$.
          \end{enumerate}

    \item 设$A \in \mathbf{F}^{n \times n}$,令$C(A)=\{B \in \mathbf{F}^{n \times n} \mid AB=BA\}$.
          \begin{enumerate}
              \item 证明:$C(A)$为$\mathbf{F}^{n \times n}$的一个子空间;

              \item 求$C(E)$;

              \item 当$A$为对角线上元素互不相等的对角阵时,求$C(A)$的维数和一组基.
          \end{enumerate}
\end{enumerate}

\centerline{\heiti C组}
\begin{enumerate}
    \item 证明以下两个命题:
          \begin{enumerate}
              \item 与矩阵$I=\begin{pmatrix}
                            0 & 1 &   &        &   \\
                              &   & 1 &        &   \\
                              &   &   & \ddots &   \\
                              &   &   &        & 1 \\
                            1 &   &   &        & 0
                        \end{pmatrix}$可交换的矩阵$A$都可以写成$I$的一个多项式,即$A=a_{11}E+a_{12}I+a_{13}I^2+\cdots+a_{1n}I^{n-1}$;

              \item 与矩阵$J=\begin{pmatrix}
                            0 & 1 &   &        &   \\
                              &   & 1 &        &   \\
                              &   &   & \ddots &   \\
                              &   &   &        & 1 \\
                              &   &   &        & 0
                        \end{pmatrix}$可交换的矩阵$A$都可以写成$J$的一个多项式,即$A=a_{11}E+a_{12}J+a_{13}J^2+\cdots+a_{1n}J^{n-1}$.
          \end{enumerate}
\end{enumerate}
