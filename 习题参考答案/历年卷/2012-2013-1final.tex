\section*{2012-2013学年线性代数I(H)期末}
\addcontentsline{toc}{section}{2012-2013学年线性代数I(H)期末}

\begin{center}
    任课老师:统一命卷\hspace{4em} 考试时长:120分钟
\end{center}

\begin{enumerate}
    \item [一、](10分)求实线性方程组 $\begin{cases}x_1-x_2+2x_3 = 1 \\ x_1-2x_2-x_3=2 \\ 3x_1-x_2+5x_3=3 \\ 2x_1-2x_2-3x_3 = 4\end{cases}$ 的解集.
    \item [二、](10分)设 $A$ 是域 $\mathbf{F}$ 上的 $m\times n$ 矩阵,$A$ 的秩 $r(A)=1.$
    \begin{enumerate}[label=(\arabic*)]
        \item 证明存在(列向量)$X\in \mathbf{F^m}$ 和 $Y\in \mathbf{F^n}$ 使得 $A=XY^\mathbf{T}$,其中 $Y^\mathbf{T}$ 是 $Y$ 的转置.
        \item $X$ 和 $Y$ 是否唯一?
    \end{enumerate}
    \item [三、](10分)定义线性映射 $T:\mathbf{R^n} \to \mathbf{R^n}$ 如下:对任意的 $(x_1,x_2,\cdots,x_n) \in \mathbf{R^n}$
    \[T(x_1,x_2,\cdots,x_n)=(x_1+x_2,x_2+x_3,\cdots,x_n+x_1).\]
    试给出 $T$ 的核 $\mathrm{Ker} \ (T)$ 和 $T$ 的像 $\mathrm{Im} \ (T)$ 的维数.
    \item [四、](10分)设 $V$ 是域 $\mathbf{F}$ 上有限维线性空间,$T:V\to V$ 是线性映射. 证明 $V$ 的非零向量都是 $T$ 的特征向量当且仅当存在 $\alpha \in \mathbf{F}$,使 $T(v)=\alpha v$ 对于任何 $v \in V$ 成立.
    \item [五、](10分)设 $A=\begin{pmatrix}a & b \\ c & d\end{pmatrix}$ 是可逆矩阵,其中 $b\neq 0$;$\lambda$ 是 $A$ 的特征值.
    \begin{enumerate}[label=(\arabic*)]
        \item 证明 $\lambda \neq 0.$
        \item 证明 $(b,\lambda-a)^\mathbf{T}$ 是属于 $\lambda$ 的特征向量.
        \item 若 $A$ 有两个不同的特征值 $\lambda_1$ 和 $\lambda_2$,求可逆矩阵 $P$ 使 $P^{-1}AP$ 是对角矩阵.
    \end{enumerate}
    \item [六、](10分)实对称矩阵 $A=\begin{pmatrix}0 & -1 & 1 \\ -1 & 0 & 1 \\ 1 & 1 & 0\end{pmatrix}.$ 求正交矩阵 $Q$ 使 $Q^\mathbf{T}AQ$ 是对角矩阵.
    \item [七、](10分)设 $V$ 是欧式空间,$T:V\to V$ 是线性映射,$\lambda \in \mathbf{R}$, $u$ 是 $V$ 的非零向量.
    证明:$\lambda$ 是 $T$ 的特征值且 $u$ 是属于 $\lambda$ 的特征向量当且仅当对于任何 $v\in V$ 成立 $(T(u),v) = \lambda(u,v).$
    \item [八、](10分)设 $n$ 阶实对称阵 $A$ 满足方程 $A^2-6A+5I_n=0$,其中 $I_n$ 是 $n$ 阶单位矩阵.
    \begin{enumerate}[label=(\arabic*)]
        \item 证明 $A$ 是正定的.
        \item 若 $n=2$,试给出全部有可能与 $A$ 相似(注意,不是相合!)的对角矩阵.
    \end{enumerate}
    \item [九、](20分)判断下面命题的真伪.若它是真命题,给出一个简单证明;若它是伪命题,举一个具体的反例将它否定.
    \begin{enumerate}[label=(\arabic*)]
        \item 若有限维线性空间 $V$ 的线性映射 $T:V \to V$ 是可对角化的,则 $T$ 是同构.
        \item 若 $A,B$ 是对称矩阵,则 $AB$ 也是对称矩阵.
        \item 若 $n$ 阶方阵 $A,B$ 中的 $A$ 是可逆的,则 $AB$ 与 $BA$ 是相似的.
        \item 若 $n(n>1)$ 阶方阵 $A$ 的特征多项式是 $f(\lambda)=\lambda^n$,则 $A$ 是零矩阵.
    \end{enumerate}
\end{enumerate}

\clearpage
