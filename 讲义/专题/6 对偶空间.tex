\chapter{对偶空间} \label{chap:对偶空间}

本讲我们将介绍一个略有些抽象的概念,即线性空间的对偶. 如果你是第一次学习线性代数,可以选择先略过本讲以及之后涉及本讲的一些讨论.

面对对偶这样一个全新且抽象的概念,我们选择从直观入手. 当我们在谈对偶的时候,我们在谈一种普遍性的结构:如果我们说 $A$ 和 $B$ 对偶,那么所有关于 $A$ 的叙述都能被翻译成关于 $B$ 的叙述,而且保持其正确性. 这种观点最经典的代表就在几何当中. 考虑$\mathbf{R}^2$空间,我们可以有如下两条陈述:
\begin{enumerate}
    \item 两点确定一条直线;
    \item 两条线一定交于一点(这需要由给平行直线补上无穷远处的交点来实现).
\end{enumerate}
在如上两条陈述中,点和线就构成了一组对偶的对象:我们把两点翻译成两条线,一条直线翻译成一点,确定翻译成交于——这与诗词中的对偶手法似乎也有些类似. 再考虑三维情形,三个点确定一个平面,三个平面确定一个点(同样补上无穷远处的点),这就构成了点与平面的对偶. 以此类推,一个点作为一个零维的对象总是与一个比总空间维数少 $1$(我们一般直接将其称作“余一维”)的对象对偶,这个对象被我们称作\term{超平面}. 在 $\R^n$ 中,它被如下的方程所唯一确定:
\[a_1 x_1 + a_2 x_2 + \cdots + a_n x_n = 1\]
最后的值是 $1$ 是因为,如果这个等式的结果为 $b$,那我们(在它非零时,等于零的情况我们将来会发现十分特殊)总是可以把它除到左边去,通过改变系数的手法让它能够化为这样的一个方程. 这也已经暗示了这是一个 $n - 1$ 维的对象,因为我们通过一个 $n$ 个未知量的方程降低了一个``自由度''. 那么,接下来我们要考虑的是对于线性空间来说又会如何.

% 最后我们用一张图来总结商空间这一节的基本思路:

% \begin{figure}[H]
%     \centering
%     \large
%     \begin{tikzpicture}
%         \draw node at (3, 3.5) {$\pi$}
%             node (set) at (2, 2) {$\{a,b,c,d,\ldots\}$}
%             node [below of=set] {\normalsize$(a - d \in U)$}
%             node (Ka) at (4.5, 3.5) {$K_a$}
%             node (Kb) at (5, 1) {$K_b$}
%             node (Kc) at (6.5, 0.5) {$K_c$}
%             coordinate (a) at (1.1, 2.2)
%             coordinate (b) at (1.5, 1.8)
%             coordinate (c) at (1.9, 1.8)
%             coordinate (d) at (2.3, 2.3);

%         % \foreach \x in {1, 1.5, ..., 9}
%         %     \foreach \y in {0, 0.5, ..., 4}
%         %         \fill[black] (\x, \y) circle (1pt);

%         % \foreach \pt in {a, b, c, d}
%         %     \fill[red] (\pt) circle (1pt);

%         \draw[thick] (4, 0) rectangle (9, 4)
%             (4, 2) -- (6, 4)
%             (5, 3) -- (6, 0)
%             (5.5, 1.5) -- (8, 0);

%         \draw[thick,violet,-latex] (a) to[out=45, in=180] (4, 3);
%         \draw[thick,violet,-latex] (b) to[out=-30, in=180] (4, 1.5);
%         \draw[thick,violet,-latex] (c) to[out=-15, in=90] (Kc);
%         \draw[thick,violet,-latex] (d) to[out=10, in=180] (4, 2.7);
%     \end{tikzpicture}

%     \normalsize
%     等价类:仿射子集(如何进一步描述其结构)

%     商集:仿射子集构成的集合;商空间:商集上定义加法和数乘

%     商映射:自然映射

%     商空间的基与维数:两种方法
% \end{figure}

\section{对偶空间与对偶映射}

很直观地,我们发现,上面所给的方程无非是一个从 $\R^n$ 到 $\R$ 的线性映射
\[f(x_1, x_2, \ldots, x_n) = a_1 x_1 + a_2 x_2 + \cdots + a_n x_n\]
在取值$1$下的完全原像$\{x: f(x) = 1\}$(记$x=(x_1,x_2,\ldots,x_n)$). 因此,我们类似地考虑从 $V$ 到基域 $\R$的线性映射,并给它一个新名字:

\begin{definition}{}{}
    称 $\mathcal{L}(V, \R)$ 上的元素为 $V$ 上的一个\term{线性泛函}.
\end{definition}
有时我们也将线性泛函称为线性函数,实际上它们都表示将$V$中向量映射到数域上的线性映射. 需要强调的一点是,这里$V$是$\R$上的线性空间,所以线性泛函的到达空间为$\R(\R)$(可以翻回线性映射的定义,我们要求两个线性空间定义的数域是一致的). 我们来看几个线性泛函的例子:
\begin{enumerate}
    \item 定义$\sigma:\mathbf{F}^n\to\mathbf{F}$为$\sigma(x_1,\ldots,x_n)=c_1x_1+\cdots+c_nx_n$,其中$c_1,\ldots,c_n\in\mathbf{F}$,则$\sigma$是线性泛函;

    \item 定义$\sigma:\mathbf{R}[x]_n\to\mathbf{R}$为$\sigma(p(x))=\displaystyle\int_0^1p(x)\mathrm{d}x$,则$\sigma$是线性泛函.
\end{enumerate}

对于复数域上的线性空间,这个定义以及以下的叙述也都是类似的,只是将实数域变为复数域,就不再赘述了. 考虑线性泛函 $f\colon V \to \R$,根据我们前面的讨论,集合 $\{x: f(x) = 1\}$ 就是一个超平面,并且事实上这一超平面反过来也唯一确定了$f$:

\begin{lemma}{}{cong_functional_hyperplane}
    如果 $\{x: f(x) = 1\}$ 和 $\{x: g(x) = 1\}$ 表示同一个超平面,则线性泛函 $f$ 和线性泛函 $g$ 在任意点取值都相同.
\end{lemma}

\begin{proof}
    考虑 $f$ 在向量空间中的任取的一点 $v \neq 0$ 上的取值 $f(v)$. 如果 $f(v)$ 不为 $0$,此时,根据线性性我们就有
    \[f(\dfrac{1}{f(v)}v)=\dfrac{1}{f(v)}f(v)=1,\]
    因此 $(\dfrac{1}{f(v)}v \in \{x: f(x) = 1\}$,也就是说它在 $f$ 确定的超平面上. 又$\{x: f(x) = 1\}$ 和 $\{x: g(x) = 1\}$ 表示同一个超平面,因此我们也有 $g(\dfrac{1}{f(v)} v) = 1$,从而再次由线性性得出 $g(v) = f(v)$.

    如果 $f(v)=0$,取$v'$使得$f(v')\neq 0$,由之前的讨论可知$f(v')=g(v')$,同理由线性性可知$0\neq f(v+v')=g(v+v')$,因此$g(v)=g(v+v')-g(v')=f(v+v')-f(v')=f(v)$.
\end{proof}

但需要注意的是,上述证明中我们实际上漏了一种可能,它就是那些``在全空间上都为$0$的泛函''——在我们的证明中,形式化的对 $f(v)$ 为零情形的证明总是依赖于存在一个向量使它取值非零,但我们并不能保证这一点. 但是,这样全空间上都为$0$的泛函显然是取值一致的. 唯一的问题是,它对应的超平面是什么?实际上,它对应的超平面是一个无穷远点处的超平面,我们后面会大致说明这一点,但这个东西的形式化留待后面射影几何的未竟专题讨论,在此我们只要确立它的唯一性即可.

由此我们发现,线性泛函实际上与超平面是一一对应的. 因此线性泛函的几何意义也就是所谓超平面,因此线性空间中向量可以视为一个点,点是一维的对象,线性泛函则可以被视为与点对应的余一维的超平面. 接下来我们给全体线性泛函构成的集合一个名字,事实上,根据\autoref{thm:线性映射全体构成线性空间},我们知道这是一个线性空间,我们称其为对偶空间:
\begin{definition}{}{}
    称 $\mathcal{L}(V, \R)$ 为 $V$ 的\term{对偶空间},记作 $V^*$.
\end{definition}

到目前为止,引言中的大部分直观已经被形式化地表出了:点被我们形式化为线性空间$V$中的一个向量,超平面则对应于对偶空间中的一个线性泛函. 接下来的问题是,点和超平面可以有直观上的一一对应,向量和线性泛函是否也是这样的呢?也就是说,我们要证明 $V$ 和 $V^*$ 同构,这样两个空间中的元素就有一一对应的关系:

% 证明的一个非常简单的思想来自于超平面方程,它有 $a_1, \ldots, a_n$ 这么多的变元来确定,这些变元明显可以视作是超平面空间的一组基. 那么,让我们用线性泛函这套术语来将这种直观表述出来:

\begin{theorem}{}{对偶空间同构}
    $V \cong V^*$.
\end{theorem}

\begin{proof}
    取定 $V$ 的一组基 $e_1, e_2, \ldots, e_n$,然后考虑 $f_1, f_2, \ldots, f_n$ 为线性泛函且满足
    \[f_i(e_j) = \begin{cases}
            1, & i = j,    \\
            0, & i \neq j.
        \end{cases},\]
    如果我们使用 Kronecker 记号 $\delta_{ij}=\begin{cases}
            1, & i = j,    \\
            0, & i \neq j.
        \end{cases}$,这个条件可以写成
    \[f_i(e_j) = \delta_{ij}, \forall i, j = 1, 2, \ldots, n\]
    这就是对前面所谓的超平面方程中的系数的形式化. 接下来,我们先验证这确实是 $V^*$ 的一组基,考虑固定 $f \in V^*$,任取 $V$ 的一个向量 $v = v_1 e_1 + v_2 e_2 + \ldots v_n e_n$,我们就有
    \[f(v) = f(v_1 e_1) + f(v_2 e_2) + \cdots + f(v_n e_n)\]
    注意到
    \[f_i(v) = f_i (v_1 e_1 + v_2 e_2 + \cdots + v_n e_n) = v_i\]
    我们就可以将上式写成
    \begin{align*}
        f(v) & = f(f_1(v) e_1) + f(f_2(v) e_2) + \cdots + f(f_n(v) e_n) \\
             & = f_1(v) f(e_1) + f_2(v) f(e_2) + \cdots + f_n(v) f(e_n) \\
             & = (f(e_1) f_1 + f(e_2) f_2 + \cdots + f(e_n) f_n)(v)
    \end{align*}
    因此可以将 $f$ 表出成 $f_i$ 的线性组合. 然后我们知道$\mathcal{L}(V, \R)$的维数等于$V$的维数,所以这组向量的长度也是合理的,因此构成一组基(当然读者也可以直接证明线性无关来说明这一点). 然后,$e_i \mapsto f_i$ 自然地生成了一个线性映射
    \begin{equation*}
        \begin{aligned}
            \varphi: V & \to V^*     \\
            e_i        & \mapsto f_i
        \end{aligned},
    \end{equation*}
    根据\autoref{thm:线性映射构造},我们知道这可以唯一确定一个线性映射,定义方式就是$\sum_{i=1}^nc_ie_i \mapsto\sum_{i=1}^nc_if_i$,并且由于这一映射显然是满射(像包含了$V^*$的所有基),且出发空间与到达空间维数相同,因此根据\autoref{thm:双射等价条件} 可知这就是我们希望得到的两个线性空间之间的同构映射.
\end{proof}

读者很容易怀疑为什么我们明明可以直接使用$\mathcal{L}(V, \R)$的维数直接地证明这一结论,却使用找基和构造同构映射的方式证明. 这是因为上述构造出的基非常重要,我们称其为原空间的基的\term{对偶基},其中的元素 $f_i$ 也记作 $e^*_i$,这是非常合理的,因为它与原来的基有直接的对应关系.

\begin{example}{}{}
    设$V=\mathbf{R}[x]_3$,对于$g(x)\in V$,定义:
    \[f_1(g(x))=\displaystyle\int_0^1g(x)\,\mathrm{d}x,\enspace f_2(g(x))=\int_0^2g(x)\,\mathrm{d}x,\enspace f_3(g(x))=\int_0^{-1}g(x)\,\mathrm{d}x,\]
    \begin{enumerate}
        \item 证明:$f_1,f_2,f_3$是$V^*$的一组基;

        \item 求出$V$的一组基$g_1(x),g_2(x),g_3(x)$,使得$f_1,f_2,f_3$是$g_1,g_2,g_3$的对偶基.
    \end{enumerate}
\end{example}

\begin{solution}

\end{solution}

除此之外需要强调的是,这样一组基一定依赖于原来的基,上面构造出的同构也是如此——如果我们一开始给$V$取的是另一组基,这里的对偶基一定是另一组基,同构映射也会发生变化. 像这样随着基的取法不同而发生变化的同构映射根据我们在\nameref{subsec:自然同构}中的描述应当称其为不自然的. 事实上,不存在从 $V$ 到 $V^*$ 的自然同构,这个命题在 Mac Lane 提出范畴论的原始论文中被称为是可以由范畴论所形式化的,而范畴论是数学中更为高深的分支,在此我们就不再深入探讨,待本书的最后一个未竟专题展开讨论.

但是,另一个范畴论的观点还是可以讨论的. 当我们说对偶的时候,我们提到了关于对象的叙述的翻译的问题,如开头中的例子点与线之间的相互翻译对应——这一点我们已经通过线性空间与其对偶空间的同构形式化. 但我们似乎还忽略了一个翻译,就是``确定''和``交于''的之间的翻译. 很自然地,这两个词汇实际上就是线性空间元素之间的对应,即线性映射,于是我们接下来讲探讨线性映射之间的翻译. 在进一步讲解之前,我们首先需要介绍交换图的概念:
\begin{definition}{交换图}{交换图}
    我们首先定义什么是一个\term{图(diagram)}. 一个图可以视为以一个代数结构(例如线性空间)为顶点,以某种映射(例如线性映射)为边的图,例如:
    \begin{center}
        \begin{tikzcd}
            V_1 \arrow[r, "\sigma_1"] \arrow[rd, "\sigma_3"'] & V_2 \arrow[d, "\sigma_2"] \\
            & V_3
        \end{tikzcd}
    \end{center}
    如果我们称一个图是\term{交换的(commutative)},则意味着对于图中的任意两个顶点间的任意两条有向路径,路径上映射的复合结果相同. 例如上图中,我们考察从$V_1$到$V_3$的两条路径,则如果$\sigma_2\circ\sigma_1=\sigma_3$,这个图是交换的.
\end{definition}

作为一个简单的练习我们可以画出如下交换图,从图中我们可以读出 $\sigma\circ(\rho\circ\psi) = (\sigma\circ\rho)\circ\psi$,即映射复合的结合律.
\[
    \tikzcdset{arrow style=tikz, diagrams={>=stealth}}
    \begin{tikzcd}
        V_1 \arrow[r, "\psi"] \ar[rd, "\rho\circ\psi"'] & V_2 \arrow[d, "\rho"] \ar[rd, "\sigma\circ\rho"] \\ & V_3 \ar[r, "\sigma"'] & V_4
    \end{tikzcd}
\]

在定义了交换图后我们可以考虑以下交换图:
\begin{center}
    \begin{tikzcd}
        V \rar["f"] \dar[swap, "\text{dual}"] & W \dar["\text{dual}"] \\
        V^* & W^* \lar["f^*"]
    \end{tikzcd}
\end{center}
% 这里的翻译真的对得上吗
其中 $\text{dual}$ 表示如\autoref{thm:对偶空间同构} 定义的同构,图中的 $f^*$ 就是我们翻译过去的东西,它被称为\term{对偶映射}. 实际上这里有些复杂,因为我们定义的$f^*$需要将一个 $W \to \R$ 的函数 $\varphi$ 映到一个 $V \to \R$ 的函数,是``映射之间的映射''. 我们需要一些耐心,我们首先可以检查我们需要的是什么,以及我们现在有什么来做出定义. 一个简化的重要步骤是对$f^*$进行``取值计算'',研究$f^*(\varphi)$是什么——实际上是一个$V \to \R$ 的函数,$f^*$接受一个$W \to \R$ 的函数 $\varphi$,而且我们恰好只有一个已知的 $f$ 是 $V\to W$ 中的映射,于是我们不妨将$\varphi$与$f$复合,这样正好可以凑出一个$V \to \R$ 的函数,即我们有如下非常自然的定义:
\begin{definition}{}{对偶映射}
    给定 $f\colon V \to W$,则定义对偶映射 $f^*: W^* \to V^*$ 为
    \[f^*(\varphi) = \varphi \circ f\]
\end{definition}
实际上类似的处理手法在各种地方都会出现,事实上这里我们用$\varphi$与$f$复合,很直观的看法就是用$\varphi$把$f$拉回,这在几何中非常常见. 事实上有了上面的解释后,我相信读者不必苦于记忆这一看似抽象的定义,而是知道这一定义最自然的构造就是我们有两个映射然后复合一下即可.

为什么说这是范畴论的观点呢?因为我们所谓的翻译实际上也就翻译了线性空间和线性变换,这二者就是一个范畴的要素,这样的翻译(在保映射复合的情况下)就被称为一个\term{函子}. 而这个函子是\term{反变}的,也就是说,它会把原来的箭头翻转过来(如上图中的情况). 下面我们要验证的就因此被称为函子性:

\begin{lemma}{}{}
    考虑线性映射 $f\colon U \to V, g: V \to W$,则
    \[
        (f \circ g)^* = g^* \circ f^*
    \]
\end{lemma}

\begin{proof}
    对于类似形式(在几何中称为推出和拉回)的映射,我们证明等式的思路都是直接代入定义展开证明:
    \[
        (f \circ g)^* (\varphi) = \varphi \circ (f \circ g) = (\varphi \circ f) \circ g = (f^*(\varphi)) \circ g = g^* (f^*(\varphi)) = (g^* \circ f^*)(\varphi)
    \]
\end{proof}

我想读者读到这里心里会有一个自然的疑问,便是为什么我们定义的映射是$V\to W$的,但对偶映射却是$W^*\to V^*$的?事实上,我们肯定是能够构造出$V^*\to W^*$的映射的,因为我们有$V\to V^*$的同构,它们的元素之间是一一对应的,同理$W$和$W^*$也是如此. 然后我们可以仿照$f$的定义来定义$f^*$,如果我们设$\alpha_i(i=1,2,\ldots,n)$为$V$的一组基,$\beta_i(i=1,2,\ldots,m)$为$W$的一组基,且
\[f(\alpha_i)=c_{i1}\beta_1+c_{i2}\beta_2+\cdots+c_{im}\beta_m,\]
那么我们可以定义$f^*(\alpha_i^*)=\sum\limits_{i=1}^m c_{ij}\beta_i^*$,这样的定义是非常直接的继承. 但是,根据我们之前的讨论,这样的定义是不自然的,因为它依赖于基的选取. 但我们看\autoref{def:对偶映射} 的定义则不依赖于基的选取. 当然,一定有读者还保有一丝希望,希望我们能够找到一个$V^*\to W^*$的自然映射,但我们前面已经分析过,我们在定义时只有一个已知的 $f$ 是 $V\to W$ 中的映射,如果我们接收一个$V\to \R$的函数,这两个映射是无法直接复合的,更不用说得到一个$W\to \R$的函数了.

我们已经认识到\autoref{def:对偶映射} 的定义非常自然,并且具有反变的性质,接下来我们来看这一映射是否也具有我们希望拥有的其它性质. 在线性代数中,显然我们希望线性映射的对偶映射也是线性的,这一点我们可以直接验证:
\begin{lemma}{}{}
    给定线性映射 $f\colon V \to W$,则 $f^*: W^* \to V^*$ 是线性映射.
\end{lemma}
\begin{proof}
    我们直接按照定义展开验证即可:
    \begin{gather*}
        \sigma^*(f_1+f_2)=(f_1+f_2)\circ\sigma=f_1\circ\sigma+f_2\circ\sigma=\sigma^*(f_1)+\sigma^*(f_2) \\
        \sigma^*(\lambda f)=(\lambda f)\circ\sigma=\lambda(f\circ\sigma)=\lambda\sigma^*(f)
    \end{gather*}
\end{proof}

当然我们也期望所有的对偶映射和线性映射一样也构成一个线性空间$\mathcal{L}(W^*,V^*)$,这一点我们可以直接验证:
\begin{lemma}{}{}
    给定 $f$ 和 $g$ 为 $V \to W$ 的线性映射,则
    \begin{enumerate}
        \item $(f + g)^* = f^* + g^*$;
        \item $(\lambda f)^* = \lambda f^*$.
    \end{enumerate}
\end{lemma}

这个引理的验证交由读者自行处理,也是展开即可. 这样子看下来,我们就已经建立了对偶空间的整个框架了. 时刻记住,它并没有第一眼看上去那般繁杂与抽象,它无非是一种超平面的代数表达. 但是,反变函子还留给我们最后的一个问题:既然对偶空间对应的函子是反过来的,那再对偶一次不就回去了吗?因此,我们就有了对双对偶空间的研究,最显然的结论就是:

\[
    V \cong V^{**}
\]

这当然看起来是一句废话. 从超平面的观点看,一次对偶把点映到了超平面,两次对偶就把它映了回去——且慢,这样的操作中,我们把点映回了点,于是我们会有一种期望,即这种同构是不是就变自然了呢?很幸运,答案是肯定的.

\begin{theorem}{}{}
    存在一个自然同构 $\psi: V \to V^{**}$.
\end{theorem}

\begin{proof}
    又一次我们遇到了映射套映射的情况,那当然类似于之前对偶映射的讨论,我们的处理手段就是对需要定义的映射进行``取值计算''. 于是我们取$\psi$在$v\in V$上的值,得到$\psi(v)$,它是一个将线性泛函映到实数的线性泛函. 这仍然是映射套映射的情况,于是我们不妨更进一步,取在$f\colon V^*\to\R$上的值,我们有$(\psi(v))(f)$是一个实数,这时我们要素齐全,直接令
    \[(\psi(v))(f) = f(v),\enspace\forall f\in V^*,\enspace v\in V.\]
    类似于对偶映射,这是我们最自然的选择,并且这一映射完全不依赖于基的选择,因此是自然的. 剩下的工作就是验证这个映射具有我们希望的性质,我们放在下面这一例题中证明.
\end{proof}

\begin{example}{}{}
    设$V$为有限维线性空间. 我们定义$\psi:V\to V^{**}$为
    \[(\psi(v))(f) = f(v),\enspace\forall f\in V^*,\enspace v\in V.\]
    \begin{enumerate}
        \item 证明:$\psi$是线性映射;

        \item 证明:若$\sigma\in\mathcal{L}(V,V)$,则$\sigma^{**}\circ\psi=\psi\circ\sigma$,这里$\sigma^{**}$是$\sigma^*$的对偶映射;

        \item 证明:$\psi$是$V$到$V^{**}$的同构映射.
    \end{enumerate}
\end{example}

\begin{proof}
    \begin{enumerate}
        \item
        \item
        \item
    \end{enumerate}
\end{proof}

需要注意的是,在证明这是一个线性同构的过程中,从技术上我们依然要依赖于对 $V$ 取任意一组基,因此它实际上依赖于 $V$ 的有限性. 如果 $V$ 并非有限,那么实际上问题会变得相当复杂,在此不做讨论. 虽然无限维情形的对偶空间依然可以用 $\mathcal{L}(V, \R)$ 来定义,超平面的直观则就不再适用了.

\section{零化子}

在引理 \ref{lem:cong_functional_hyperplane} 的证明中,我们探讨了一个使得线性泛函取值非零的向量,并且我们实际上也已经发现了,这样的向量在构建子空间的时候发挥了重要的作用. 接下来,我们要问的是,那些为零的点呢?既然线性泛函是一个从 $V$ 到 $\R$ 的函数,对于这个函数的零点的探讨也是理所应当的. 记一个线性泛函 $f$ 的零点集为 $\ker f$,则它当然也是一个线性空间. 而为了表现对偶空间的用处,我们就要研究那些零点包含某些东西的线性泛函构成的集合作为对偶空间的子集的结构——这种想法实际上也是所谓的泛函分析的起点:对一个函数空间中的结构的研究.

于是,第一个定义就理所应当了:

\begin{definition}{}{}
    设 $U$ 为 $V$ 的子空间,则称
    \[
        U^0 = \{\varphi \in V^*: \forall u \in U, \varphi(u) = 0\}
    \]
    为 $U$ 的\term{零化子}.
\end{definition}

在零化子中的线性泛函,是那些零点集包含 $U$ 中的线性泛函. 为什么不定义零点集恰好是 $U$ 的线性泛函构成的集合呢?稍微想一想就会发现那样定义的集合结构非常难看:它对于线性泛函的加法和数乘都不封闭. 这就引出了零化子的第一个性质:

\begin{theorem}{}{}
    零化子构成一个对偶空间的子空间.
\end{theorem}

\begin{proof}
    首先,任何空间的零化子都包含 $0$ 泛函,它将所有向量都映到 $0$,因此零化子必定非空;

    其次,如果 $\alpha, \beta \in U^0$,则不难发现 $\alpha + \beta \in U^0$,因为对于任意 $u \in U$,有

    \[
        (\alpha + \beta) (u) = \alpha (u) + \beta (u) = 0
    \]

    同理,数乘封闭性也能这样证明.
\end{proof}

那么,定义了一个线性空间之后,我们的第一个问题就是,它的维数是多少. 为了解决这一问题,我们首先给出一个非常有对称性的定理,零化子的维数是这一定理自然的推论:

\begin{theorem}{}{对偶空间的维数引理}
    设 $V = U_1 \oplus U_2$,则 $V^* = U_1^0 \oplus U_2^0$,而且 $U_1^0 = U_2^*, U_2^0 = U_1^*$.
\end{theorem}

% 在证明这个定理之前,读者请先停下来观察一下这里的对称性. 需要关注的是,零化子是一个相对于某个大空间来说的东西,于是,这个定理说的就是,我们可以用一个相对于大空间来说的东西描述其中的小空间的对偶空间,这也就是为什么在讨论了对偶空间之后,我们立刻就引入了零化子的概念.

\begin{proof}
    为了证明直和,我们首先要证 $U_1^0 \cap U_2^0 = 0$. 考虑 $U_1^0 \cap U_2^0$ 中的线性泛函 $\phi$,则不难发现对于任意的 $U_1 \oplus U_2$ 中的元素 $u_1 + u_2, u_1 \in U_1, u_2 \in U_2$,都有 $\phi(u_1 + u_2) = \phi(u_1) + \phi(u_2) = 0$. 因此,$\phi$ 是零泛函.

    接下来,为了验证同构,我们考虑定义线性泛函 $\phi \in U_2^*$ 的扩张为一个 $V^*$ 中的线性泛函 $\tilde{\phi}$,它限制在 $U_2$ 上的取值均与 $\phi$ 相同,而限制在 $U_1$ 上为零泛函,中间部分由这两部分线性张成. 按照定义,不难发现 $\tilde \phi \in U_1^0$. 如此就构造了 $U_2^* \to U_1^0$ 的线性映射,其逆映射为 $U_1^0$ 中的泛函限制到 $U_2^*$ 上. 在此构造的基础上,请读者自行填充剩下的验证细节.
\end{proof}

于是,我们就有了一个显然的推论:

\begin{theorem}{}{零化子维数}
    $\dim U^0 = \dim V - \dim U$
\end{theorem}
\begin{proof}
    设$V=U\oplus W$,则由\autoref{thm:对偶空间的维数引理} 有$V^*=U^0\oplus W^0$,且$W^0=U^*$,于是$\dim U^0=\dim V^*-\dim W^0=\dim V-\dim U^*=\dim V-\dim U$.
\end{proof}
看到减号,我们不难发现,$U^{00}$,即两次取零化子的过程保持维数不变. 继续参照上面的定理的精神,我们不难发现,$U = U^{00}$,这个结论的证明方法有多种,留作习题请读者自行完成. 一个提示是,这个结论同样只对有限维线性空间成立.

这个时候,让我们再回头来看看怎么形象的理解这个玩意儿. 为此,首先要做的是建立零点集和零化子之间的关系. 以下几个性质都是显然的:

\begin{lemma}{}{}
    \begin{enumerate}
        \item $f \in (\ker f)^0$
        \item $\forall f \in U^0, \ker f \subset U$
    \end{enumerate}
\end{lemma}

我们还要进行一个定义,这实际上是零点集的推广,但是我们要换一个记号:

\begin{definition}{}{}
    设 $U$ 为 $V^*$ 的子空间,则定义
    \[
        Z(U) = \{v \in V: \forall \varphi \in U, \varphi(v) = 0\}
    \]
\end{definition}

也就是说,$Z(U)$ 就是那些 $U$ 中的线性泛函的公共零点的集合,即:
\[
    Z(U) = \bigcap_{\varphi \in U} \ker \varphi
\]
它显然也是一个线性空间. 这样,上面引理的第一、第二条就有了一个看上去更加对称的推论:

\begin{lemma}{}{NU性质}
    \begin{enumerate}
        \item 如果 $U$ 是 $V^*$ 的子空间,则 $U = Z(U)^0$
        \item 如果 $U$ 是 $V$ 的子空间,则 $U = Z(U^0)$
    \end{enumerate}
\end{lemma}

\begin{proof}
    我们只证明第一条. 考虑 $\phi \in U$,则 $Z(U)$ 中的所有元素一定包含于 $\ker \phi$ 中,因此,$\phi$ 一定把其中的所有元素映到 $0$,即 $\phi \in Z(U)^0, U \subset Z(U)^0$. 然后,我们需要考虑维数问题. 考虑 $U$ 中的一组基 $e_1, e_2,\ldots, e_m$,其对偶基 $e_1^*, e_2^*, ..., e_m^*$ 可以扩张成 $V$ 的一组基 $e_1^*, e_2^*, \cdots e_m^*, e_{m + 1},\ldots, e_n$,则根据基的线性无关性,不难得出 $Z(U)$ 的基就是 $e_{m + 1},\ldots, e_n$. 也就是说,$\dim Z(U) = \dim V - \dim U$,于是 $\dim U = \dim Z(U)^0$,我们得到 $U = Z(U)^0$.
\end{proof}

于是,我们有了一个更加舒适的视角来理解零化子:它无非就是对偶空间中的零点集. 因此,我们应当很容易理解对于单个线性映射,以下性质无非是这条结论的推论:

\begin{theorem}{}{对偶映射像和核的性质}
    设 $V$ 和 $W$ 都是有限维线性空间,$\sigma\in\mathcal{L}(V,W)$,则
    \begin{enumerate}
        \item $\ker\sigma^*=(\im\sigma)^0$;
        \item $\dim\ker\sigma^*=\dim\ker\sigma+\dim W-\dim V$;
        \item $\dim\im\sigma^*=\dim\im\sigma$;
        \item $\im\sigma^*=(\ker\sigma)^0$.
    \end{enumerate}
\end{theorem}

\begin{corollary}{}{对偶映射单满射}
    设$V$和$W$都是有限维线性空间,$\sigma\in\mathcal{L}(V,W)$,则
    \begin{enumerate}
        \item $\sigma$是单射当且仅当$\sigma^*$是满射;

        \item $\sigma$是满射当且仅当$\sigma^*$是单射.
    \end{enumerate}
\end{corollary}

\section{再论商空间}

我们已经探讨了很多直和的式子,也讨论了零点之类的问题. 现在,我们想问的是,这种零点是否依然还有别的方式来理解?为此,我们需要仔细观察一下零空间的结构. 首先,我们考虑单个线性泛函 $\varphi$ 的情形.

我们就已经知道,$\ker \varphi$ 是一个线性空间,而这个东西无非就是 $\varphi^{-1} (0)$. 那么,我们就自然想要知道 $\varphi^{-1} (\lambda)$ 到底长什么样,这些东西一般被称为 $\varphi$ 的\term{纤维}. 下面我们考虑一个引理:

\begin{lemma}{}{}
    设 $r \in \R$,则对于任意 $v' \in \varphi^{-1}(r), v \in \ker \varphi$,都有 $v + v' \in \varphi^{-1} (r)$.
\end{lemma}

这个引理粗看有点线性性的意味,它的证明也是如此,在此就不赘述了. 而在发现这个引理之后,也就不难联想到以下结果,它完整地描述了纤维的结构:

\begin{theorem}{}{纤维的结构}
    $\varphi^{-1} (r) = v' + \ker \varphi$,其中 $v'$ 为 $\varphi^{-1} (r)$ 中的任意一个向量.
\end{theorem}

\begin{proof}
    我们仅证明左侧包含于右侧,另一个包含由上面的引理保证. 考虑 $v'' \in \phi^{-1} (r)$,则 $\varphi (v') = \varphi (v'') = r$,因此 $\varphi(v' - v'') = 0, v' - v'' \in \ker \varphi$.
\end{proof}

这事实上就是一个的仿射子集,我们在商空间中已经介绍过. 根据语境的不同,我们也会称其为线性簇甚至线性流形. 于是我们从线性泛函的纤维再次推导出了仿射子集的结构,这相比于之前基于等价关系的推导更为直观. 很显然,我们定义的超平面 $a^\mathrm{T}x = 1$ 就是一个仿射子集,而且,一个泛函所对应的超平面,按照\autoref{lem:cong_functional_hyperplane} 所述,就是它在 $1$ 处的纤维. 于是我们也再次理解了仿射子集的几何意义,它所表达的就是那些``近似线性对象'',比如线、面,等等.

按照泛函的纤维的线性性的直观,我们也应该有以下结论:

\begin{lemma}{}{}
    设 $\lambda, \mu$ 为两个实数,$v_\lambda \in \varphi^{-1}(\lambda), v_\mu \in \varphi^{-1} (\mu)$ 则

    \begin{enumerate}
        \item $\varphi^{-1} (\lambda + \mu) = v_\lambda + v_\mu + \ker \varphi$
        \item $\varphi^{-1} (\lambda \mu) = \mu v_\lambda + \ker \varphi = \lambda v_\mu + \ker \varphi = \mu \varphi^{-1} (\lambda) = \lambda \varphi^{-1} (\mu)$
    \end{enumerate}
\end{lemma}

其证明如上显然. 然后,我们发现,按照这样的方式,所有纤维所构成的集合就定义了加法和数乘运算,而且不难验证它构成了一个线性空间;其维数则无非是压到了 $\R$ 的维数,也就是 $1$.

按照这样的观点,则不难理解如何将这样的结论推广到一般的线性映射上去. 一个线性映射 $f\colon V \to W$ 无非就是将一个线性空间线性地压缩到了 $\im f$ 上去,其在原像集上的体现无非是那一族纤维. 这个性质由以下定理描述:

\begin{theorem}{}{}
    设 $f\colon V \to W$ 是一个线性映射,则 $f^{-1} (w) = v_0 + \ker f$,其中 $w \in W, v_0 \in f^{-1} (w)$.
\end{theorem}

回忆所有这样的纤维(仿射子集)构成的集合就是商空间 $V/\ker f$. 因此我们可以重述上面的结论,就是 $V/\ker f$ 是一个线性空间,而且 $V/\ker f \simeq \im f$. 更进一步的,实际上我们有一个天然的漂亮的方法,来给出这样的一个核:

\begin{lemma}{}{}
    若 $W$ 是 $V$ 的一个子空间,则存在线性映射 $V \to V$,使得其在 $W$ 上的点取值均为 $0$,而在 $W$ 以外的点取值均不为 $0$.
\end{lemma}

\begin{proof}
    考虑 $V$ 的直和分解 $V = W \oplus W'$. 构造线性映射 $f$ 使得其在 $W$ 上为零映射,在 $W'$ 上为恒同映射即可.
\end{proof}

因此,任意子空间都是线性映射的一个核,我们就给出了以下定理:

\begin{theorem}{}{}
    设 $W$ 为 $V$ 的子空间,则 $V/W$ 是一个线性空间,而且 $\dim V/W = \dim V - \dim W$.
\end{theorem}

这是在上面的构造中,$V/W \simeq W'$. 到此为止,本节最核心的结果就已经建成了. 最后的一个结果将本节的三个核心概念:对偶空间、零化子和商空间串联起来:

\begin{theorem}{}{}
    $(V/W)^* = W^0$.
\end{theorem}

\begin{proof}
    我们已经表明,$V/W \simeq W'$ 并给出了一个同构. 考虑 $W \oplus W' = V$,则 $V^* = W^* \oplus (W')^*$,其中 $(W')^* = W^0$.
\end{proof}

\begin{corollary}{}{}
    设$U$是有限维线性空间$V$的子空间,则
    \[\dim V/U=\dim V-\dim U.\]
\end{corollary}

第二点是,商空间具备泛性质. 我们在很久以前就提到过泛性质的问题,而在这里,商空间实际上可以被看成一个推出,也就是说,如果以下图表中的实线箭头交换,则:

\begin{center}
    \begin{tikzcd}
        W \rar[hook, "i"] \dar[twoheadrightarrow, "p"] & V \dar[twoheadrightarrow, "p_*"] \arrow[rdd, bend left, "\varphi"] & \\
        1 \rar[hook, "i_*"] \arrow[rrd, bend right, "\tilde{i}"] & V/W \arrow[rd, dashed, "\varphi_*"] & \\
        & & Q
    \end{tikzcd}
\end{center}

则存在唯一的虚线箭头使得图表交换. 下面我们给出证明.

\begin{proof}
    上图中的 $i, p, i_*, p_*$ 和 $\tilde{i}$ 都是显然看出的,所以,我们只要按照正常的方式,从 $\varphi$ 构造出 $\varphi_*$ 即可. 左下方三角的交换性是毋庸置疑的,因此,只要有右上的三角的交换性即可. 考虑 $V/W$ 中的元素 $v + W$,我们定义
    \[ \varphi_*(v + W) = \varphi(v) \]
    这个映射是良定且唯一的,这由 $\varphi$ 对商的相容性保证,请读者自行在习题中完成证明.
\end{proof}

这样的构造也被称为余纤维积. 更深入地探讨这样的构造不是我们在这里会去做的事情,但是我们会把更多关于推出的问题留在习题中,以供有兴趣的同学参考.

最后的一个反思是,对偶空间的意义到底在哪?在最开始,我们已经提到过,线性泛函就是超平面,超平面与点具备对偶关系. 而这样的对偶关系延伸开去,就是两种对线性方程组的视角,其中,超平面的交的视角我们将在线性方程组一般理论的讨论中展开,这里我们给出点的视角. 如果将 $(a_{i1}, a_{i2},\ldots, a_{in})$ 视作一个点,那么我们的求解任务就是找一个超平面过 $m$ 个点. 而我们在中学阶段就已经知道,要证明一组点共面是一个比较麻烦的事情,但是求一组面是否有交点往往来说是比较轻松的. 因此,我们通过对偶空间进行了一个翻译工作,将求一组点所共的那些面转化成了一些面所交的那些点,进而对问题完成了简化.

\begin{summary}

\end{summary}

\begin{exercise}
    \exquote[塞尔日·兰(Serge Lang)]{如果一本教科书只能让学生按部就班地阅读,那该是多么无趣的一件事啊!}

    \begin{exgroup}
        \item 证明以下两个命题:
        \begin{enumerate}
            \item 设$\varphi\in \mathcal{L}(V,\mathbf{F}),\enspace\varphi\neq 0$. 证明:$\dim V/(\ker\varphi)=1$;

            \item 设$U$是$V$的子空间且$\dim V/U=1$,则存在$\varphi\in \mathcal{L}(V,\mathbf{F})$使得$\ker\varphi=U$.
        \end{enumerate}
    \end{exgroup}

    \begin{exgroup}
        \item 设$U$和$W$是线性空间$V$的子空间. 构造同构映射证明:若$V=U\oplus W$,则$U$和$V/W$同构.

        \item 设$U$是$V$的子空间,$\Gamma:\mathcal{L}(V/U,W)\to \mathcal{L}(V,W)$定义为$\Gamma(S)=S\circ\pi$. 证明:
        \begin{enumerate}
            \item $\Gamma$是线性映射;

            \item $\Gamma$是单的;

            \item $\im\Gamma=\{T\in \mathcal{L}(V,W) \mid \forall u\in U,\enspace Tu=0\}$.
        \end{enumerate}

        \item 实际上,零化子有一个更广泛的版本,考虑 $S$ 为 $V$ 的一个子集,也能如上构造 $S^0$,尝试证明,这样的构造满足以下性质:
        \begin{enumerate}
            \item $S^{00} = \spa S$
            \item $S \subset T \iff T^0 \subset S^0$
            \item 若 $U_1, U_2$ 为 $V$ 的两个子空间,证明 \begin{enumerate}
                      \item $(U_1 + U_2)^0 = U_1^0 \cap U_2^0$
                      \item $(U_1 \cap U_2)^0 = U_1^0 + U_2^0$
                  \end{enumerate}(提示:反复利用前面两个性质)
            \item 进而得出对于零点集的相应结论
        \end{enumerate}
    \end{exgroup}

    \begin{exgroup}
        \item
    \end{exgroup}
\end{exercise}
