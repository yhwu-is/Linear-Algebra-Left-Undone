\chapter{行列式计算进阶}

\section{化三角形法}
\begin{example}
    计算行列式 
    $\mathrm{D}_{n+1}=\left|\begin{array}{ccccc}
    1 & a_{1} & a_{2} & \cdots & a_{n} \\
    1 & a_{1}+b_{1} & a_{2} & \cdots & a_{n} \\
    1 & a_{1} & a_{2}+b_{2} & \cdots & a_{n} \\
    \vdots & \vdots & \vdots & \ddots & \vdots \\
    1 & a_{1} & a_{2} & \cdots & a_{n}+b_{n}
    \end{array}\right|$.
\end{example}

解析:观察行列式的特点,主对角线下方的元素与第一行元素对应相同,故用第一行的 $(-1)$ 倍加到下面各行便可使主对角线下方的元素全部变为零。即化为上三角形。

\begin{solution}
    将该行列式第一行的 $(-1)$ 倍分别加到第 $2,3 \cdots(n+1)$ 行上去,可得
    $$
    \mathrm{D}_{n+1}=\left|\begin{array}{ccccc}
    1 & a_{1} & a_{2} & \ldots & a_{n} \\
    0 & b_{1} & 0 & 0 & 0 \\
    0 & 0 & b_{2} & 0 & 0 \\
    \vdots & \vdots & \vdots & \ddots & \vdots \\
    0 & 0 & 0 & \ldots & b_{n}
    \end{array}\right|=\prod_{i=1}^nb_{i}
    $$
\end{solution}

\section{连加法}

\begin{example}
    计算行列式 
$D_{n}=\begin{vmatrix}
x_{1}-m & x_{2} & \cdots & x_{n} \\
x_{1} & x_{2}-m & \cdots & x_{n} \\
\vdots & \vdots & \ddots & \vdots \\
x_{1} & x_{2} & \cdots & x_{n}-m
\end{vmatrix}$
\end{example}

\begin{solution}
$$
\begin{aligned}
    \mathrm{D}_{\mathrm{n}}&=
    \left|\begin{array}{cccc}
    \displaystyle\sum_{i=1}^{n} x_{i}-m & x_{2} & \cdots & x_{n} \\
    \displaystyle\sum_{i=1}^{n} x_{i}-m & x_{2}-m & \cdots & x_{n} \\
    \vdots & \vdots & \ddots & \vdots \\
    \displaystyle\sum_{i=1}^{n} x_{i}-m & x_{2} & \cdots & x_{n}-m
    \end{array}\right|\\
    &=\left(\sum_{i=1}^{n} x_{i}-m\right)
    \left|\begin{array}{cccc}
    1 & x_{2} & \cdots & x_{n} \\
    1 & x_{2}-m & \cdots & x_{n} \\
    \vdots & \vdots & \ddots & \vdots \\
    1 & x_{2} & \cdots & x_{n}-m
    \end{array}\right|\\
    &=\left(\sum_{i=1}^{n} x_{i}-m\right)
    \left|
    \begin{array}{cccc}
    1 & x_{2} & \cdots & x_{n} \\
    0 & -m & \cdots & 0 \\ \vdots & \vdots & \ddots & \vdots \\
    0 & 0 & \cdots & -m
    \end{array}\right|\\
    &=(-m)^{n-1}\left(\sum_{i=1}^{n} x_{1}-m\right)
\end{aligned}
$$
\end{solution}

\section{滚动消去法}
当行列式每两行的值比较接近时,可采用让邻行中的
某一行减或者加上另一行的若干倍, 这种方法叫滚动消去法。

\begin{example}
    计算行列式 
$\mathrm{D}_{n}=\left|\begin{array}{cccccc}
1 & 2 & 3 & \cdots & n-1 & n \\
2 & 1 & 2 & \cdots & n-2 & n-1 \\
3 & 2 & 1 & \cdots & n-3 & n-2 \\
\vdots & \vdots & \vdots & \ddots & \vdots & \vdots \\
n-1 & n-2 & n-3 & \cdots & 1 & 2\\
n & n-1 & n-2 & \cdots & 2 & 1
\end{array}\right|(n \geq 2)$
\end{example}

\begin{solution}
    从最后一行开始每行减去上一行
$$
\begin{aligned}
D_{n}=&\left|\begin{array}{cccccc}
1 & 2 & 3 & \cdots & n-1 & n \\
1 & -1 & -1 & \cdots & -1 & -1 \\
1 & 1 & -1 & \cdots & -1 & -1 \\
\vdots & \vdots & \vdots & \ddots & \vdots & \vdots \\
1 & 1 & 1 & \cdots & -1 & -1\\
1 & 1 & 1 & \cdots & 1 & -1
\end{array}\right|=\left|\begin{array}{cccccc}
1 & 2 & 3 & \cdots & n-1 & n \\
2 & 0 & 0 & \cdots & 0 & -2 \\
2 & 2 & 0 & \cdots & 0 & -2 \\
\vdots & \vdots & \vdots & \ddots & \vdots & \vdots \\
2 & 2 & 2 & \cdots & 0 & -2 \\
1 & 1 & 1 & \cdots & 1 & -1
\end{array}\right| \\
=&\left|\begin{array}{cccccc}
1 & 2 & 3 & \cdots & n-1 & n+1 \\
2 & 0 & 0 & \cdots & 0 & 0 \\
2 & 2 & 0 & \cdots & 0 & 0 \\
\vdots & \vdots & \vdots & \ddots & \vdots & \vdots \\
2 & 2 & 2 & \cdots & 0 & 0 \\
1 & 1 & 1 & \cdots & 1 & 0
\end{array}\right|=2^{n-2}
\left|\begin{array}{cccccc}
1 & 2 & 3 & \cdots & n-1 & n+1 \\
1 & 0 & 0 & \cdots & 0 & 0 \\
1 & 1 & 0 & \cdots & 0 & 0 \\
\vdots & \vdots & \vdots & \ddots & \vdots & \vdots \\
1 & 1 & 1 & \cdots & 1 & 0
\end{array}\right|\\
=&(-1)^{n+1}(n+1) 2^{n-2} .
\end{aligned}
$$
\end{solution}

\section{降阶法}

将高阶行列式化为低阶行列式再求解. 

\begin{example}
    解行列式
$D_n=\left|\begin{array}{cccccc}
x & -1 & 0 & \ldots & 0 & 0 \\
0 & x & -1 & \ldots & 0 & 0 \\
0 & 0 & x & \ddots & 0 & 0 \\
\vdots & \vdots & \vdots & \ddots & \vdots & \vdots \\
0 & 0 & 0 & \ldots & x & -1 \\
a_{0} & a_{1} & a_{2} &\ldots &a_{n-2}&a_{n-1}
\end{array}\right|$
\end{example}

\begin{solution}
    按最后一行展开,得
$$
\begin{aligned}
D_{n}&=\sum_{i=0}^{n-1}(-1)^{n+i+1}a_i\left|\begin{array}{cccccc}
A&O\\
O&B
\end{array}\right|
=\sum_{i=0}^{n-1}(-1)^{n+i+1}a_i|A|\cdot|B|\\
&=\sum_{i=0}^{n-1}(-1)^{n+i+1}a_i\left(x^i\right)\left[(-1)^{n-1-i}\right]
=\sum_{i=0}^{n-1}a_ix^i
\end{aligned}
$$

其中$A\in \mathbf{R}^{i\times i}, B\in \mathbf{R}^{(n-1-i)\times (n-1-i)}$
$$
A=\begin{pmatrix}
x & -1 & 0 & \ldots & 0 &0\\
0 & x & -1 & \ldots & 0&0\\
0 & 0 & x & \ddots & 0&0\\
\vdots & \vdots & \vdots & \ddots & \vdots & \vdots\\
0 & 0 & 0 & \ldots &x & -1\\
0 & 0 & 0 & \ldots &0 & x
\end{pmatrix},
B=\begin{pmatrix}
-1 & 0 & 0 & \ldots & 0 &0\\
x & -1 & 0 & \ldots & 0 &0\\
0 & x & -1 & \ldots & 0 &0\\
\vdots & \vdots & \vdots & \ddots & \vdots\\
0 & 0 & 0 & \ddots &-1 & 0\\
0 & 0 & 0 & \ldots &x & -1
\end{pmatrix}
$$
\end{solution}

\begin{example}
    解行列式
$\mathrm{D}_{n}=\left|\begin{array}{cccccc}
\lambda & a & a & a & \cdots & a \\
b & \gamma & \beta & \beta & \cdots & \beta \\
b & \beta & \gamma & \beta & \cdots & \beta \\
\vdots & \vdots & \vdots & \vdots & \ddots & \vdots \\
b & \beta & \beta & \beta & \cdots & \gamma
\end{array}\right|$
\end{example}

\begin{solution}
    从第$n$行到第3行,每行都减去上一行;再从第3列到第$n$列,
每列都加到第二列,得

$$
\begin{aligned}
\mathrm{D}_{n}&=
\left|\begin{array}{cccccc}
\lambda & a & a & a & \cdots & a \\
b & \gamma & \beta & \beta & \cdots & \beta \\
0 & \beta-\gamma & \gamma-\beta & 0 & \cdots & 0 \\
0 & 0 & \beta-\gamma & \gamma-\beta & \cdots & 0 \\
\vdots & \vdots & \vdots & \vdots & \ddots & \vdots \\
0 & 0 & 0 & 0 & \cdots & \gamma-\beta
\end{array}\right|\\
&=\left|\begin{array}{cccccc}
\lambda & (n-1)a & a & a & \cdots & a \\
b & \gamma+(n-2)\beta & \beta & \beta & \cdots & \beta \\
0 & 0 & \gamma-\beta & 0 & \cdots & 0 \\
0 & 0 & \beta-\gamma & \gamma-\beta & \cdots & 0 \\
\vdots & \vdots & \vdots & \vdots & \ddots & \vdots \\
0 & 0 & 0 & 0 & \cdots & \gamma-\beta
\end{array}\right|\\
&=\left|\begin{array}{cc}
\lambda & (n-1)a \\
b & \gamma+(n-2)\beta
\end{array}\right| \cdot
\left|\begin{array}{cccc}
\gamma-\beta & 0 & \cdots & 0 \\
\vdots & \vdots & \ddots & \vdots \\
0 & 0 & \cdots & \gamma-\beta
\end{array}\right|\\
&=(\lambda \gamma+\lambda(n-2)\beta-(n-1)ab)(\gamma-\beta)^{n-2}  
\end{aligned}
$$
\end{solution}

\section{升阶法}

升阶法就是把 $\mathrm{n}$ 阶行列式增加一行一列变成 $\mathrm{n}+1$ 阶行列式,再通过性质化简算出 结果,这种计算行列式的方法叫做升阶法或加边法. 升阶法的最大特点就是要 找每行或每列相同的因子, 那么升阶之后, 就可以利用行列式的性质把绝大多数 元素化为 0 , 这样就达到简化计算的效果.

\begin{example}
    解行列式 $D=\left|\begin{array}{cccccc}
        0 & 1 & 1 & \cdots & 1 & 1 \\
        1 & 0 & 1 & \cdots & 1 & 1 \\
        1 & 1 & 0 & \cdots & 1 & 1 \\
        \vdots & \vdots & \vdots & \ddots & \vdots & \vdots \\
        1 & 1 & 1 & \cdots & 0 & 1 \\
        1 & 1 & 1 & \cdots & 1 & 0
        \end{array}\right|$
\end{example}

\begin{solution}
    使行列式 D 变成 $n+1$ 阶行列式, 即
$$
\mathrm{D}=\left|\begin{array}{cccccc}
1 & 1 & 1 & \cdots & 1 & 1 \\
0 & 0 & 1 & \cdots & 1 & 1 \\
0 & 1 & 0 & \cdots & 1 & 1 \\
\vdots & \vdots & \vdots & \ddots & \vdots & \vdots \\
0 & 1 & 1 & \cdots & 0 & 1 \\
0 & 1 & 1 & \cdots & 1 & 0
\end{array}\right| .
$$

再将第一行的 $(-1)$ 倍加到其他各行, 得:
$$
D=\left|\begin{array}{cccccc}
1 & 1 & 1 & \cdots & 1 & 1 \\
-1 & -1 & 0 & \cdots & 0 & 0 \\
-1 & 0 & -1 & \cdots & 0 & 0 \\
\vdots & \vdots & \vdots & \ddots & \vdots & \vdots \\
-1 & 0 & 0 & \cdots & -1 & 0 \\
-1 & 0 & 0 & \cdots & 0 & -1
\end{array}\right| .
$$

从第二列开始, 每列乘以 $(-1)$ 加到第一列, 得:
$$
\begin{aligned}
D &=\left|\begin{array}{cccccc}
-(n-1) & 1 & 1 & \cdots & 1 & 1 \\
0 & -1 & 0 & \cdots & 0 & 0 \\
0 & 0 & -1 & \cdots & 0 & 0 \\
\vdots & \vdots & \vdots & \ddots & \vdots & \vdots \\
0 & 0 & 0 & \cdots & -1 & 0 \\
0 & 0 & 0 & \cdots & 0 & -1
\end{array}\right| \\
&=(-1)^{n+1}(n-1) .
\end{aligned}
$$
\end{solution}


\section{数归/递推法}

\begin{example}
    计算行列式 
$D_{n}=\left|\begin{array}{cccccc}
\cos \beta & 1 & 0 & \cdots & 0 & 0 \\
1 & 2 \cos \beta & 1 & \cdots & 0 & 0 \\
0 & 1 & 2 \cos \beta & \cdots & 0 & 0 \\
\vdots & \vdots & \vdots & \ddots & \vdots & \vdots \\
0 & 0 & 0 & \cdots & 2 \cos \beta & 1 \\
0 & 0 & 0 & \cdots & 1 & 2 \cos \beta
\end{array}\right|$.
\end{example}

\begin{solution}
    $n=1,D_1=\cos\beta$

$n=2,D_2=\begin{vmatrix}
\cos\beta&1\\
1&2\cos\beta
\end{vmatrix}=2\cos^2\beta-1=\cos2\beta$

猜想$D_n=\cos n\beta,$数学归纳证明:

假设当 $n=k$ 时,结论成立.即: $D_{k}=\cos k \beta$. 现证当 $n=k+1$ 时,结论也成立.
$$
\text { 当 } n=k+1 \text { 时, } D_{k+1}=\left|\begin{array}{cccccc}
\cos \beta & 1 & 0 & \cdots & 0 & 0 \\
1 & 2 \cos \beta & 1 & \cdots & 0 & 0 \\
0 & 1 & 2 \cos \beta & \cdots & 0 & 0 \\
\vdots & \vdots & \vdots & \ddots & \vdots & \vdots \\
0 & 0 & 0 & \cdots & 2 \cos \beta & 1 \\
0 & 0 & 0 & \cdots & 1 & 2 \cos \beta
\end{array}\right| \text {. }
$$

将 $D_{k+1}$ 按最后一行展开, 得
$$
\begin{aligned}
D_{k+1}=&(-1)^{k+1+k+1} \cdot 2 \cos \beta\left|\begin{array}{ccccc}
\cos \beta & 1 & 0 & \cdots & 0 \\
1 & 2 \cos \beta & 1 & \cdots & 0 \\
0 & 1 & 2 \cos \beta & \cdots & 0 \\
\vdots & \vdots & \vdots & \ddots & \vdots \\
0 & 0 & 0 & \cdots & 2 \cos \beta
\end{array}\right|\\
&+(-1)^{k+1+k}\left|
\begin{array}{ccccc}
\cos \beta & 1 & 0 & \cdots & 0 \\
1 & 2 \cos \beta & 1 & \cdots & 0 \\
0 & 1 & 2 \cos \beta & \cdots & 0 \\
\vdots & \vdots & \vdots & \ddots & \vdots \\
0 & 0 & 0 & \cdots & 1
\end{array}\right|\\
&=2\cos\beta D_k-D_{k-1}
\end{aligned}
$$

而$D_{k}=\cos k \beta$, 
$D_{k-1}=\cos (k-1)\beta
=\cos (k \beta-\beta)
=\cos k \beta \cos \beta+\sin k \beta \sin \beta$

所以有
$$
\begin{aligned}
D_{k+1}&= 2 \cos \beta D_{k}-D_{k-1} \\
&=2 \cos \beta \cos k \beta-\cos k \beta \cos \beta-\sin k \beta \sin \beta \\
&=\cos k \beta \cos \beta-\sin k \beta \sin \beta \\
&=\cos (k+1) \beta .
\end{aligned}
$$

则证得$D_n=\cos n\beta$, $n\in \mathbf{N}$。
\end{solution}

下面介绍常系数线性递推数列,为了方便,只介绍二阶情况。
如果$D_n$满足关系式
$$
aD_n+bD_{n-1}+cD_{n-2}=0
$$

解特征方程
$$
ar^2+br+c=0
$$

会有三种根的情况。

(1)$\Delta>0$, 有两个不等的实根$r_1, r_2$,则有
$$
D_n=C_1r_1^n+C_2r_2^n
$$

(2)$\Delta=0$, 有重实根$r$,则有
$$
D_n=(C_1+nC_2)r^n
$$

(3)$\Delta<0$, 有共轭复根$r=\cos\beta\pm i\sin\beta$,则有
$$
D_n=C_1\cos n\beta + C_2\sin n\beta
$$

以上式子中的$C_1,C_2$均为任意常数,可以令$n=1,2$获得。

所以其实例7也可以使用递推式求得答案(留作习题证明略)。
不过一般遇到的还是特征根为实数的情况比较多,给出一道练习例题:

\begin{example}
    计算行列式 
$\mathrm{D}_{\mathrm{n}}=
\left|\begin{array}{cccccccc}
9 & 5 & 0 & 0 & \cdots & 0 & 0 & 0 \\
4 & 9 & 5 & 0 & \cdots & 0 & 0 & 0 \\
0 & 4 & 9 & 5 & \cdots & 0 & 0 & 0 \\
\vdots & \vdots & \vdots & \vdots & \ddots & \vdots & \vdots & \vdots \\
0 & 0 & 0 & 0 & \cdots & 4 & 9 & 5 \\
0 & 0 & 0 & 0 & \cdots & 0 & 4 & 9
\end{array}\right|$
\end{example}

\begin{solution}
    按第一列展开, 得
$$
D_{n}=9 D_{n-1}-20 D_{n-2} .
$$

即
$$
D_{n}-9 D_{n-1}+20 D_{n-2}=0
$$

作特征方程
$$
x^{2}-9 x+20=0
$$

解得
$$
x_{1}=4, x_{2}=5 \text {. }
$$

则
$$
D_{n}=A \cdot 4^n+B \cdot 5^n
$$

当 $n=1$ 时, $9=4A+5B$;

当 $n=2$ 时, $61=16A+25B$.

解得
$$
A=-4, B=5,
$$

所以
$$
D_n=5^{n+1}-4^{n+1}
$$
\end{solution}


\section{硬拆法}

\begin{example}
    计算行列式 $\mathrm{D}_{n}=\left|\begin{array}{cccccc}1-a_{1} & a_{2} & 0 & \cdots & 0 & 0 \\ -1 & 1-a_{2} & a_{3} & \cdots & 0 & 0 \\ 0 & -1 & 1-a_{3} & \cdots & 0 & 0 \\ \vdots & \vdots & \vdots & \ddots & \vdots & \vdots \\ 0 & 0 & 0 & \cdots & 1-a_{n-1} & a_{n} \\ 0 & 0 & 0 & \cdots & -1 & 1-a_{n}\end{array}\right|$
\end{example}

\begin{solution}
    把第一列的元素看成两项的和进行拆列, 得
$$
\begin{aligned}
\mathrm{D}_{n}=&\left|\begin{array}{cccccc}
1-a_{1} & a_{2} & 0 & \cdots & 0 & 0 \\
-1+0 & 1-a_{2} & a_{3} & \cdots & 0 & 0 \\
0+0 & -1 & 1-a_{3} & \cdots & 0 & 0 \\
\vdots & \vdots & \vdots & \ddots & \vdots & \vdots \\
0+0 & 0 & 0 & \cdots & 1-a_{n-1} & a_{n} \\
0+0 & 0 & 0 & \cdots & -1 & 1-a_{n}
\end{array}\right|\\
=&\left|\begin{array}{cccccc}
1 & a_{2} & 0 & \cdots & 0 & 0 \\
-1 & 1-a_{2} & a_{3} & \cdots & 0 & 0 \\
0 & -1 & 1-a_{3} & \cdots & 0 & 0 \\
\vdots & \vdots & \vdots & \ddots & \vdots & \vdots \\
0 & 0 & 0 & \cdots & 1-a_{n-1} & a_{n} \\
0 & 0 & 0 & \cdots & -1 & 1-a_{n}
\end{array}\right| \\
&+\left|\begin{array}{cccccc}
-a_{1} & a_{2} & 0 & \cdots & 0 & 0 \\
0 & 1-a_{2} & a_{3} & \cdots & 0 & 0 \\
0 & -1 & 1-a_{3} & \cdots & 0 & 0 \\
\vdots & \vdots & \vdots & \ddots & \vdots & \vdots \\
0 & 0 & 0 & \cdots & 1-a_{n-1} & a_{n} \\
0 & 0 & 0 & \cdots & -1 & 1-a_{n}
\end{array}\right|
\end{aligned}
$$

上面第一个行列式的值为 1 (从第1行开始,每一行依次加到下一行), 所以
$$
\begin{aligned}
D_{n} &=1-a_{1}\left|\begin{array}{ccccc}
1-a_{2} & a_{3} & \cdots & 0 & 0 \\
-1 & 1-a_{3} & \cdots & 0 & 0 \\
\vdots & \vdots & \ddots & \vdots & \vdots \\
0 & 0 & \cdots & 1-a_{n-1} & a_{n} \\
0 & 0 & \cdots & -1 & 1-a_{n}
\end{array}\right| \\
&=1-a_{1} D_{n-1} \cdot
\end{aligned}
$$

这个式子在对于任何 $n(n \geq 2)$ 都成立, 因此有
$$
\begin{aligned}
D_{n} &=1-a_{1} D_{n-1} \\
&=1-a_{1}\left(1-a_{2} D_{n-2}\right)=\cdots=1-a_{1}+a_{1} a_{2}+\cdots+(-1)^n a_{1} a_{2} \cdots a_{n} \\
&=1+\sum_{i=1}^{n}(-1)^i \prod_{j=1}^{i} a_{j} .
\end{aligned}
$$
\end{solution}

\section{箭形行列式}

\begin{example}
    计算行列式$\begin{vmatrix}
        a_1 &1&1&\ldots&1\\
        1&a_2\\
        1&&a_3\\
        \vdots&&&\ddots\\
        1&&&&a_n
        \end{vmatrix}$,其中$a_i\neq 0,i=1,2,\cdots n$        
\end{example}

\begin{solution}
    $$
\text{原式}=\begin{vmatrix}
a_1-\displaystyle\sum_{i=2}^n\frac{1}{a_i} &1&1&\ldots&1\\
0&a_2\\
0&&a_3\\
\vdots&&&\ddots\\
0&&&&a_n
\end{vmatrix}
=\left(\sum_{i=2}^n\frac{1}{a_i}\right)
\left(\prod_{j=2}^na_j\right)
$$
\end{solution}

\section{Vandermonde行列式}

\begin{example}
    求行列式 $D_{n}=\left|\begin{array}{cccc}
        1 & 1 & \cdots & 1 \\
        x_{1} & x_{2} & \cdots & x_{n} \\
        x_{1}^{2} & x_{2}^{2} & \cdots & x_{n}^{2} \\
        \vdots & \vdots & \vdots & \vdots \\
        x_{1}^{n-2} & x_{2}^{n-2} & \cdots & x_{n}^{n-2} \\
        x_{1}^{n} & x_{2}^{n} & \cdots & x_{n}^{n}
        \end{array}\right|$
\end{example}

\begin{solution}
    考虑构造一个$n+1$阶的Vandermonde行列式。

$$
f(x)=\left|\begin{array}{ccccc}
1 & 1 & \cdots & 1 & 1 \\
x_{1} & x_{2} & \cdots & x_{n} & x \\
x_{1}^{2} & x_{2}^{2} & \cdots & x_{n}^{2} & x^{2} \\
\vdots & \vdots & \ddots & \vdots & \vdots \\
x_{1}^{n-2} & x_{2}^{n-2} & \cdots & x_{n}^{n-2} & x^{n-2} \\
x_{1}^{n-1} & x_{2}^{n-1} & \cdots & x_{n}^{n-1} & x^{n-1} \\
x_{1}^{n} & x_{2}^{n} & \cdots & x_{n}^{n} & x^{n}
\end{array}\right| .
$$

将 $f(x)$ 按第 $n+1$ 列展开, 得
$$
f(x)=A_{1, n+1}+A_{2, n+1} x+\cdots+A_{n, n+1} x^{n-1}+A_{n+1, n+1} x^{n},
$$

其中, $x^{n-1}$ 的系数为
$$
A_{n, n+1}=(-1)^{n+(n+1)} D_{n}=-D_{n} \text {. }
$$

又根据范德蒙德行列式的结果知
$$
f(x)=\left(x-x_{1}\right)\left(x-x_{2}\right) \cdots\left(x-x_{n}\right) \prod_{1 \leq j<i \leq n}\left(x_{i}-x_{j}\right) .
$$

由上式可求得 $x^{n-1}$ 的系数为
$$
-\left(x_{1}+x_{2} \cdots x_{n}\right) \prod_{1 \leq j<i \leq n}\left(x_{i}-x_{j}\right),
$$

故有
$$
D_{n}=\left(x_{1}+x_{2}+\cdots+x_{n}\right) \prod_{1 \leq j<i \leq n}\left(x_{i}-x_{j}\right)
$$
\end{solution}

\section{$^*$利用$|E_m-AB|=|E_n-BA|$}

\begin{example}
    求行列式$\begin{vmatrix}
        0&2a_1&3a_1&\ldots&na_1\\
        a_2&a_2&3a_2&\ldots&na_2\\
        a_3&2a_3&2a_3&\ldots&na_3\\
        \vdots&\vdots&\vdots&\ddots&\ldots\\
        a_n&2a_n&3a_n&\ldots&(n-1)a_n\\
        \end{vmatrix}$
\end{example}

\begin{solution}
    $$
\text{原式}=\prod_{i=1}^na_i
\begin{vmatrix}
0&2&3&\ldots&n\\
1&1&3&\ldots&n\\
1&2&2&\ldots&n\\
\vdots&\vdots&\vdots&\ddots&\ldots\\
1&2&3&\ldots&n-1\\
\end{vmatrix}
$$

注意到
$$
\begin{aligned}
\begin{vmatrix}
0&2&3&\ldots&n\\
1&1&3&\ldots&n\\
1&2&2&\ldots&n\\
\vdots&\vdots&\vdots&\ddots&\ldots\\
1&2&3&\ldots&n-1\\
\end{vmatrix}&=\begin{vmatrix}(-1)\left(E_n-
\begin{pmatrix}
1&2&3&\ldots&n\\
1&2&3&\ldots&n\\
1&2&3&\ldots&n\\
\vdots&\vdots&\vdots&\ddots&\ldots\\
1&2&3&\ldots&n\\
\end{pmatrix}\right)\end{vmatrix}\\
&=(-1)^n\begin{vmatrix}E_n-
\begin{pmatrix}
1\\1\\1\\\vdots\\1
\end{pmatrix}(1, 2, 3, \cdots, n)\end{vmatrix}
\end{aligned}
$$

而
$$
\begin{vmatrix}E_n-
\begin{pmatrix}
1\\1\\1\\\vdots\\1
\end{pmatrix}(1, 2, 3, \cdots, n)\end{vmatrix}
=\begin{vmatrix}1-
(1, 2, 3, \cdots, n)\begin{pmatrix}
1\\1\\1\\\vdots\\1
\end{pmatrix}\end{vmatrix}
=-\frac{n^2+n-2}{2}
$$

$\therefore$原式$\displaystyle=(-1)^{n+1}\frac{n^2+n-2}{2}\prod_{i=1}^na_i$
\end{solution}

\centerline{\heiti \Large 内容总结}

希望以上的技巧不需要在考试中用到。

\centerline{\heiti \Large 习题}
\vspace{2ex}
{\kaishu }
\begin{flushright}
    \kaishu

\end{flushright}
\centerline{\heiti A组}
\begin{enumerate}
    \item $
    \begin{vmatrix}
    1&2&3&4&\ldots&n-1&n\\
    x&1&2&3&\ldots&n-2&n-1\\
    x&x&1&2&\ldots&n-3&n-2\\
    x&x&x&1&\ldots&n-4&n-3\\
    \vdots&\vdots&\vdots&\vdots&\ddots&\vdots&\vdots\\
    x&x&x&x&\ldots&1&2\\
    x&x&x&x&\ldots&x&1\\
    \end{vmatrix}$(提示:考虑滚动消去法)
    \item $
    D_n=\begin{vmatrix}
    a_1&b_1&0&\ldots&0&0\\
    0&a_2&b_2&\ldots&0&0\\
    0&0&a_3&\ddots&0&0\\
    \vdots&\vdots&\vdots&\ddots&\vdots&\vdots\\
    0&0&0&\ldots&a_{n-1}&b_{n-1}\\
    b_n&0&0&\ldots&0&a_n
    \end{vmatrix}$
    \item $D_n=\begin{vmatrix}
    a+b&ab&0&0&0&\ldots&0&0\\
    1&a+b&ab&0&0&\ldots&0&0\\
    0&1&a+b&ab&0&\ldots&0&0\\
    0&0&0&0&0&\ldots&a+b&ab\\
    0&0&0&0&0&\ldots&1&a+b\\
    \end{vmatrix}$
    \item 用递推法解例7。
    \item (P188\ T4)解行列式
    $
    \left|\begin{array}{llll}
    a^{2} & (a+1)^{2} & (a+2)^{2} & (a+3)^{2} \\
    b^{2} & (b+1)^{2} & (b+2)^{2} & (b+3)^{2} \\
    c^{2} & (c+1)^{2} & (c+2)^{2} & (c+3)^{2} \\
    d^{2} & (d+1)^{2} & (d+2)^{2} & (d+3)^{2}
    \end{array}\right|
    $
    
    \item (P189\ T6)设
    $$
    D=\begin{vmatrix}
    1+a_1&1&\ldots&1\\
    1&1+a_2&\ldots&1\\
    \vdots&\vdots&\ddots&\vdots\\
    1&1&\ldots&1+a_n
    \end{vmatrix}
    $$
    
    (1)用递推公式计算行列式D
    
    (2)硬拆D为$2^n$个行列式,计算出结果
    
    \item (P189\ T5(2))解行列式
    $\begin{vmatrix}
    a_{1}+a_{2} & a_{2}+a_{3} & \cdots & a_{n-1}+a_{n} & a_{n}+a_{1} \\
    a_{1}^{2}+a_{2}^{2} & a_{2}^{2}+a_{3}^{2} & \cdots & a_{n-1}^{2}+a_{n}^{2} & a_{n}^{2}+a_{1}^{2} \\
    \vdots & & & & \vdots \\
    a_{1}^{n}+a_{2}^{n} & a_{2}^{n}+a_{3}^{n} & \cdots & a_{n-1}^{n}+a_{n}^{n} & a_{n}^{n}+a_{1}^{n}
    \end{vmatrix}$
    
    \item (P188\ T1(5)-(8))解行列式
    $$\begin{aligned}
    &(1)D_1=\begin{vmatrix}
    1&2&3&4\\2&3&4&1\\3&4&1&2\\4&1&2&3
    \end{vmatrix}
    &(2)D_2=\begin{vmatrix}
    \lambda+2&-1&-1&-1\\
    -1&\lambda+2&-1&-1\\
    -1&-1&\lambda+2&-1\\
    -1&-1&-1&\lambda+2
    \end{vmatrix}\\
    &(3)D_3=\left|\begin{array}{cccc}
    1^{2} & 2^{2} & 3^{2} & 4^{2} \\
    2^{2} & 3^{2} & 4^{2} & 5^{2} \\
    3^{2} & 4^{2} & 5^{2} & 6^{2} \\
    4^{2} & 5^{2} & 6^{2} & 7^{2} 
    \end{array}\right|
    &(4)D_4=\begin{vmatrix}
    3&2&0&0\\
    1&3&2&0\\
    0&1&3&2\\
    0&0&1&3
    \end{vmatrix}
    \end{aligned}$$
    
    \item (P190\ T9)解行列式
    
    (1) $D=\left|\begin{array}{ccccc}
    1 & 2 & \cdots & 2 & 2 \\
    2 & 2 & \cdots & 2 & 2 \\
    \vdots & & & & \vdots \\
    2 & 2 & \cdots & n-1 & 2 \\
    2 & 2 & \cdots & 2 & n\end{array}\right|$ ;
    (2) $*D=\left|\begin{array}{ccccc}
    1 & 2 & \cdots & n-1 & n \\
    2 & 3 & \cdots & n & 1 \\
    3 & 4 & \cdots & 1 & 2 \\
    \vdots & & & & \vdots \\
    n & 1 & \cdots & n-2 & n-1
    \end{array}\right|$
    
    \item (P190\ T10(2))证明
    $\left|\begin{array}{ccccc}
    a & c & c & \cdots & c \\
    b & a & c & \cdots & c \\
    b & b & a & \cdots & c \\
    \vdots & & & & \vdots \\
    b & b & b & \cdots & a
    \end{array}\right|=\dfrac{b(a-c)^{n}-c(a-b)^{n}}{b-c}$
    
    其中 $b \neq c$, 等式左端是 $n$ 阶行列式.
\end{enumerate}
\centerline{\heiti B组}
\begin{enumerate}
    \item $^*$设$A,B,C,D$都是$n$阶方阵,且$AC=CA$,求证$\begin{vmatrix}
    A&B\\
    C&D
    \end{vmatrix}=|AD-CB|$
    
    (为简化,可以只考虑$A$可逆的情况)
    
    \item $A\in \mathbf{F}^{m\times n}, B\in \mathbf{F}^{n\times m}$,求证$|E_m-AB|=|E_n-BA|$(即“进阶性质9”)
    
    \item $A\in \mathbf{F}^{m\times n}, B\in \mathbf{F}^{n\times m}$,求证$|\lambda E_m-AB|=\lambda^{m-n}|\lambda E_n-BA|(\text{为简化,}\lambda>0, m>n)$
\end{enumerate}
\centerline{\heiti C组}
\begin{enumerate}
    \item 解行列式
    
    (1)$D=\begin{vmatrix}
    ax+by&ay+bz&az+bx\\
    ay+bz&az+bx&ax+by\\
    az+bx&ax+by&ay+bz
    \end{vmatrix}$;
    (2)$D=\begin{vmatrix}
    x^2+1&xy&xz\\
    xy&y^2+1&yz\\
    xz&yz&z^2+1
    \end{vmatrix}$
    \item $*$计算行列式$|2E-\alpha_1^T\beta_1-\alpha_2^T\beta_2|$
    
    其中$\alpha_1=(a_1, a_2, \cdots, a_n), \beta_1=(b_1,b_2,\cdots,b_n), 
    \alpha_2=(c_1, c_2, \cdots, c_n), \beta_2 = (d_1, d_2, \cdots, d_n)$
    
    (提示:利用$|\lambda E_m-AB|=\lambda^{m-n}|\lambda E_n-BA|$)
    
    \item 已知$n$阶矩阵$A$满足
    $$
    AA^T=E,|A|=-1
    $$
    
    求证:$|E+A|=0$
\end{enumerate}
