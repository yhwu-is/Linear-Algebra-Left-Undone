\chapter{预备知识}

线性代数作为大学的第一门数学课,预修要求并不高. 我们默认读者具有高中数学基础,因此关于集合、映射、向量的基本常识我们不在此赘述. 接下来我们将介绍基本代数结构,以便后续线性空间的引入,然后我们将介绍本书中常见的概念——等价类和最常用的算法之一——高斯-若当消元法.

\section{基本代数结构}

我们选择从基本代数结构谈起,因为在以往的实践中我们深切地体会到直接引入线性空间的跳跃. 因此我们希望从更具象的例子开始,首先引入``代数结构''这一基本概念,然后在下一节中自然地引出线性空间的定义. 为了接下来定义的方便,我们首先介绍集合的笛卡尔积运算:

\begin{definition}{笛卡尔积}{笛卡尔积} \index{dikaerji@笛卡尔积 (Cartesian product)}
    设$A$和$B$是两个非空集合,我们把集合
    \[A \times B = \{(a,b) \mid a \in A, b \in B\}\]
    称为集合 $A$ 和 $B$ 的\term{笛卡尔积}. 如果 $A = B$,我们也可以简记为 $A^2$.
\end{definition}

笛卡尔积的定义是非常直观的,它实际上就是两个集合中的元素两两组合构成的有序对全体. 我们来看一些简单的例子:

\begin{example}{}{}
    \begin{enumerate}
        \item 若 $A = \{1,2\}$,$B = \{3,4\}$,则 $A \times B=\{(1,3),(1,4),(2,3),(2,4)\}$;
        \item 若 $A = B = \mathbf{R}$,则根据定义,$A \times B$ 的元素是形如 $(a,b)$ 的有序对,其中 $a, b \in \mathbf{R}$. 从几何上不难看出,我们可以将 $\mathbf{R} \times \mathbf{R}$ 视为二维平面上的点集(可以简记为 $\mathbf{R}^2$),其中的元素 $(a,b)$ 对应于平面上的一个点,这一点的横坐标为 $a$,纵坐标为 $b$.
    \end{enumerate}
\end{example}

在介绍完笛卡尔积的定义后,我们开始考察一个简单的例子:实数集$\mathbf{R}$,它是一个集合. 在初中我们便知道,在$\mathbf{R}$上我们可以定义加法和乘法两种运算. 本质而言,运算是一种映射(或者更通俗而言,函数):

\begin{center}
    \begin{tabular}{rrcl}
        $+\enspace\colon$      & $\mathbf{R}\times\mathbf{R}$ & $\to$     & $\mathbf{R}$ \\
                               & $(a,b)$                      & $\mapsto$ & $a+b$        \\
        $\times\enspace\colon$ & $\mathbf{R}\times\mathbf{R}$ & $\to$     & $\mathbf{R}$ \\
                               & $(a,b)$                      & $\mapsto$ & $a\times b$
    \end{tabular}
\end{center}

因此 $+$ 和 $\times$ 都以实数的有序对作为函数的自变量,函数值也是一个实数. 或许读者看到这里还是对运算的定义有些许迷茫,但如果我们回忆映射的基本定义 $f \colon A \to B$ 表示给 $A$ 中的任意元素 $a$ 指派一个 $B$ 中的元素 $f(a)$,并将加法乘法写成 $+(2,3) = 5$,$\times(2,3) = 6$,想必就会恍然大悟:$+$ 和 $\times$ 实际上就是函数名,函数做的事情就是输入两个自变量然后进行加法/乘法运算得到结果,并把这个结果指派给自变量作为函数值.

在上述讨论中,我们所做的事情很简单,就是给定一个集合,然后在这一集合的元素之间定义运算. 实际上这就是代数系统的定义:
\begin{definition}{代数系统}{} \index{daishuxitong@代数系统 (algebraic system)}
    一般地,我们把一个非空集合$X$和与$X$相关的若干代数运算$f_1,\ldots,f_k$组成的系统称为\term{代数系统}(简称代数系),记作$\langle X \colon f_1,\ldots,f_k\rangle$.
\end{definition}

特别注意的是,代数系统上定义的运算必须保证封闭性,也就是运算后的结果必须仍然在集合$X$中. 这事实上早已由映射的方式对运算的定义保证了.

不难理解,代数系统其中蕴含的性质与其中定义的运算具有的性质是关联很大的. 例如从经典分析学的角度实数实际上是是这样一些东西的组合 $\langle\mathbf{R}\colon 0, 1, +, *, \leqslant\rangle$,只是我们在没有歧义时将其简记为了 $\mathbf{R}$. 接下来我们以实数域为例,介绍在代数学中关心的几个运算性质. 我们首先讨论实数域上的加法运算,以下性质对于任意$a,b,c\in\mathbf{R}$都成立:

\begin{enumerate}
    \item 结合律:$(a+b)+c=a+(b+c)$;

    \item 单位元:存在一个元素$0$,使得$a+0=0+a=a$;

    \item 逆元:对于任意$a$,存在一个元素$-a$,使得$a+(-a)=(-a)+a=0$(0为单位元);

    \item 交换律:$a+b=b+a$.
\end{enumerate}

对于乘法运算(可记为$\cdot$或$\times$),单位元一般记为$1$(更一般的可以记为$e$),逆元记为$a^{-1}$. 事实上,我们可以给出更多的例子:
\begin{example}{}{Abel 群}
    \begin{enumerate}
        \item 代数系统$\langle \mathbf{R}\setminus\{0\}\colon\circ\rangle$定义的一般乘法运算

        \item 代数系统$\langle \mathbf{R}^2\colon+\rangle$定义的平面向量的加法
    \end{enumerate}
    均满足上述四条运算性质.
\end{example}

事实上,我们可以对上面的定义做进一步的抽象. 我们可以忽略集合中元素的意义差异(元素可以表示实数,也可以在上述例子中表示平面向量等几何对象),同时也可以忽略运算定义的差异,只关心运算作用于集合元素的性质. 对于一般的代数系统$\langle G\colon\circ\rangle$,我们有如下定义:
\begin{definition}{群}{群}\index{qun@群 (group)}
    若运算$\circ$满足结合律,则称代数系统$\langle G\colon\circ\rangle$为\term{半群}\index{qun!banqun@半群 (semigroup)};若在半群基础上存在单位元,则称之为\term{含幺半群}\index{qun!hanyaobanqun@含幺半群 (monoid)};若在含幺半群基础上每个元素存在逆元,则称之为\term{群};若在群的基础上运算还满足交换律,则称之为\term{Abel 群},也称\term{交换群}\index{qun!abel@Abel 群 (Abelian group), 交换群 (commutative group)}.
\end{definition}

\autoref{def:群} 给出了我们本节第一个要讨论的代数结构——群的定义. 简而言之,代数结构就是在集合上定义具有某些特定性质的运算后得到的一类代数系统. 事实上,教材中42--44页给出了大量抽象的例子有助于同学们理解上述一系列群的定义,并且我们在后续学习矩阵的时候也会遇到一些群结构,相信这些实例能使读者体会到``在集合上定义运算''的方式的多样与抽象.

为方便书写,对于\autoref{def:群} 定义的群$\langle G\colon\circ\rangle$,在不引起混淆的情况下我们可以简写为群$G$. 除此之外,我们还需要指出以下两点:
\begin{theorem}{}{群的单位元逆元唯一}
    \begin{enumerate}
        \item 群的单位元唯一;

        \item 群中每个元素的逆元唯一.
    \end{enumerate}
\end{theorem}

\begin{proof}
    \begin{enumerate}
        \item 设$e_1$和$e_2$都是群$G$的单位元,则
              \[e_1=e_1\circ e_2=e_2.\]

        \item 设$b$和$c$都是$a$的逆元,则
              \[b=b\circ e=b\circ(a\circ c)=(b\circ a)\circ c=e\circ c=c.\]
    \end{enumerate}
\end{proof}

其中第一点的证明直接使用了单位元的性质,第二点的证明则使用了结合律和逆元的性质. 这里关于唯一性的证明是非常重要的:我们只需假设要证明唯一的东西有两个,然后说明这两个必然相等即可. 这一思想在之后证明矩阵的逆唯一等问题时也会用到,因此此处特别给出证明强调.

事实上,在很多集合上我们不仅可以定义一种运算,也可以定义两种甚至更多运算,在代数结构中我们仅讨论最多两种运算的情况. 事实上,我们最开始的实数集合定义加法和乘法的例子便可以引入一个新的代数结构——域:
\begin{definition}{域}{} \index{yu@域 (field)}
    我们称代数系统 $\langle F\colon+,\circ\rangle$ 为一个\term{域},如果
    \begin{enumerate}
        \item $\langle F\colon+\rangle$ 是交换群,其单位元记作0;

        \item $\langle F\setminus\{0\}\colon\circ\rangle$ 是交换群;

        \item 运算 $\circ$ 对 $+$ 满足左、右分配律,即
              \begin{gather*}
                  a \circ (b + c) = a \circ b + a \circ c \\
                  (b + c) \circ a = b \circ a + c \circ a
              \end{gather*}
    \end{enumerate}
\end{definition}

显然,实数域 $\mathbf{R}$ 上定义一般的实数加法和乘法后构成一个域. 实际上我们熟悉的例如有理数、实数等集合关于一般的加法和乘法运算都构成域,因此我们会经常使用``有理数域''、``实数域''等说法. 我们称数集对数的加法和乘法构成的域为数域,注意此处运算的定义必须是数学分析中定义的数的加法和乘法,不能是自定义的运算.

\begin{theorem}{}{}
    关于数域,我们有如下两个结论:
    \begin{enumerate}
        \item 数集 $F$ 对数的加法和乘法构成数域的充要条件为:$F$ 包含 $0,1$ 且对数的加、减、乘、除(除数不为 $0$)运算封闭;

        \item 任何数域都包含有理数域 $\mathbf{Q}$,即 $\mathbf{Q}$ 是最小的数域.
    \end{enumerate}
\end{theorem}

\begin{proof}
\begin{enumerate}
    \item 从充分性和必要性两个角度证明:
    \begin{enumerate}
        \item 充分性:只需证明 $\langle F\colon+\rangle$,$\langle F\setminus\{0\}\colon\circ\rangle$ 是交换群,并且乘法对加法满足左、右分配律. 交换群的第一条是运算封闭性,这在充分性的出发点中已经包含. 下面我们证明其它性质:
        \begin{enumerate}
            \item $\langle F\colon+\rangle$ 是交换群:
            \begin{enumerate}
                \item 加法结合律和交换律是小学数学基础知识,故必然满足;
                \item 加法单位元:因为我们要求 $F$ 包含 $0$,在普通数的运算中就是自然的加法单位元;
                \item 加法逆元:因为我们要求减法封闭,对于任意的 $a \in F$,$0 - a = -a \in F$,而 $a + (-a) = 0$,故 $F$ 中的每个元素都有加法逆元.
            \end{enumerate}
            因此,$F$ 在加法下构成一个交换群.
            \item $\langle F\setminus\{0\}\colon\circ\rangle$ 是交换群:
            \begin{enumerate}
                \item 乘法结合律和交换律是小学数学基础知识,故必然满足;
                \item 乘法单位元:因为我们要求 $F$ 包含 $1$,在普通数的运算中就是自然的乘法单位元;
                \item 乘法逆元:因为我们要求除法封闭,对于任意的 $a \in F \backslash \{0\}$,$1/a \in F$,而 $a \cdot (1/a) = 1$,故 $F$ 中的每个非零元素都有乘法逆元.
            \end{enumerate}
            因此,$F$ 中的非零元素在乘法下构成一个交换群.
            \item 乘法对加法满足左、右分配律,这也是小学数学基础知识,故必然满足.
        \end{enumerate}
        因此,$F$ 对数的加法和乘法构成数域.

        \item 必要性:因为 $F$ 对加法构成阿贝尔群,所以 $F$ 对数的加减封闭,且包含单位元 $0$;因为 $F \backslash \{0\}$ 对乘法构成阿贝尔群,所以 $F$ 对数的乘除封闭,且包含单位元 $1$.
    \end{enumerate}

    \item 设 $\mathbf{F}$ 是任意的一个数域,我们将证明 $\mathbf{Q} \subseteq \mathbf{F}$.
    \begin{enumerate}
        \item 由于 $\mathbf{F}$ 是一个数域,它至少包含乘法单位元 $1 \neq 0$.通过交换律和结合律,我们可以反复使用加法构造正整数:
        \[
        1, \enspace 1 + 1 = 2, \enspace 1 + 1 + 1 = 3, \ldots
        \]
        因此,所有正整数都属于 $\mathbf{F}$.对于负整数,由于加法逆元的存在性,对于每个正整数 $n$,$-n \in \mathbf{F}$,即负整数也在数域 $\mathbf{F}$ 中.因此,所有整数 $\mathbf{Z}$ 都在 $\mathbf{F}$ 中.
        \item 由于 $\mathbf{F}$ 是一个数域,所以它对乘法和除法(除数不为 $0$)封闭,故 $\forall p,q \in \mathbf{Z}, p,q \in \mathbf{F}, \dfrac{p}{q} \in \mathbf{F}$,故 $\mathbf{Q} \subseteq \mathbf{F}$.
    \end{enumerate}
\end{enumerate}
\end{proof}

事实上,如果加法和乘法的定义不是数的加法和乘法,我们可以定义除了数域之外的域,我们将在本讲介绍完等价类的概念后给出这样的例子.

除此之外,有一个细节需要强调. 根据 $\langle F\setminus\{0\}\colon\circ\rangle$ 是交换群,我们知道 $\circ$ 满足交换律,故不难发现上述定义的左、右分配律实际上是可以互相推导的,这里将它们都列举出来是因为接下来我们要引入的代数结构——环——对 $\circ$ 运算的要求有所降低,不一定能保证交换律成立,但也有广泛的应用:

\begin{definition}{环}{环的定义} \index{huan@环 (ring)}
    我们称代数系统 $\langle R\colon+,\circ\rangle$ 为一个\term{环},如果
    \begin{enumerate}
        \item $\langle R\colon+\rangle$ 是交换群,其单位元记作 $0$;

        \item $\langle R\colon\circ\rangle$ 是幺半群;

        \item 运算 $\circ$ 对 $+$ 满足左、右分配律,即
              \begin{gather*}
                  a \circ (b + c) = a \circ b + a \circ c \\
                  (b + c) \circ a = b \circ a + c \circ a
              \end{gather*}
    \end{enumerate}

    若进一步每个非 $0$($+$ 运算单位元)元素关于 $\circ$ 都有逆元,则称之为\term{除环}\index{huan!chu@除环 (division ring)}. 另外,若上述定义中 $\circ$ 运算满足交换律,则称为\term{交换环}\index{huan!jiaohuan@交换环 (commutative ring)},结合上述除环和交换环两个定义,我们可以发现,交换除环即为域.
\end{definition}

\begin{example}{}{}
    利用定义验证下述关于代数系统的结论:
    \begin{enumerate}
        \item 整数集 $\mathbf{Z}$ 对整数的加法和乘法构成一个交换环,但不是域;

        \item 设 $C[a,b]$ 是闭区间 $[a,b]$ 上的连续函数的集合;它对函数的加法和乘法构成一个环;

        \item 设 $\mathbf{Q}(\sqrt{2})=\{a + b\sqrt{2} \mid a, b \in \mathbf{Q}\}$,则 $\mathbf{Q}(\sqrt{2})$ 是一个数域.
    \end{enumerate}
\end{example}

\begin{proof}
\begin{enumerate}
    \item 证明整数集 $\mathbf{Z}$ 对整数的加法和乘法构成一个交换环:
    \begin{enumerate}
        \item 加法封闭性:对任意 $a, b \in \mathbf{Z}$,有
        $a + b \in \mathbf{Z}$;

        \item 加法结合律:小学数学,不再赘述;

        \item 加法单位元:取 $0 \in \mathbf{Z}$ 作为加法单位元即可,因为对任意 $a \in \mathbf{Z}$,有$a + 0 = 0 + a = a$;

        \item 加法逆元:取相反数 $-a \in \mathbf{Z}$ 即可,因为对任意 $a \in \mathbf{Z}$,有 $a + (-a) = 0$;

        \item 加法交换律:小学数学,不再赘述;

        \item 乘法封闭性:对任意 $a, b \in \mathbf{Z}$,有$a \cdot b \in \mathbf{Z}$;

        \item 乘法结合律:小学数学,不再赘述;

        \item 乘法单位元:取 $1 \in \mathbf{Z}$ 作为乘法单位元即可,因为对任意 $a \in \mathbf{Z}$,有 $1 \cdot a = a \cdot 1 = a$;

        \item 分配律:小学数学,不再赘述.
        因此,整数集 $\mathbf{Z}$ 对整数的加法和乘法构成一个交换环.
    \end{enumerate}
    但是 $\mathbf{Z}$ 不是域,因为 $\mathbf{Z}$ 中的非零元素没有乘法逆元. 例如 $2$ 的逆元 $\dfrac{1}{2} \notin \mathbf{Z}$.
    \item 此题关于环的证明和上题类似,留给读者自行证明.
    \item 证明 $\mathbf{Q}(\sqrt{2})$ 是一个数域:
    \begin{enumerate}
        \item 加法封闭性:对任意 $x, y \in \mathbf{Q}(\sqrt{2})$,有$x + y = (a + b\sqrt{2}) + (c + d\sqrt{2}) = (a + c) + (b + d)\sqrt{2} \in \mathbf{Q}(\sqrt{2})$,故加法封闭;
        \item 乘法封闭性:对任意 $x, y \in \mathbf{Q}(\sqrt{2})$,有$x \cdot y = (a + b\sqrt{2}) \cdot (c + d\sqrt{2}) = (ac + 2bd) + (ad + bc)\sqrt{2} \in \mathbf{Q}(\sqrt{2})$,故乘法封闭;
        \item 交换律和结合律:$\mathbf{Q}$ 中的加法和乘法本身满足交换律和结合律,而 $\sqrt{2}$ 是无理数,与有理数的运算也满足这些运算律. 因此,在 $\mathbf{Q}(\sqrt{2})$ 中,加法和乘法同样满足交换律和结合律;
        \item 加法单位元:取 $0 + 0\sqrt{2} \in \mathbf{Q}(\sqrt{2})$ 即可;
        \item 乘法单位元:取 $1 + 0\sqrt{2} \in \mathbf{Q}(\sqrt{2})$ 即可;
        \item 加法逆元,对于任意 $x = a + b\sqrt{2} \in \mathbf{Q}(\sqrt{2})$,它的加法逆元是 $-a - b\sqrt{2} \in \mathbf{Q}(\sqrt{2})$;
        \item 乘法逆元,除 $0$ 以外,对于任意 $x = a + b\sqrt{2} \in \mathbf{Q}(\sqrt{2})$,它的乘法逆元是 $y = \dfrac{1}{a + b\sqrt{2}} \in \mathbf{Q}(\sqrt{2})$,我们将 $y$ 继续化简:
        \[ y = \frac{1}{a + b\sqrt{2}} = \frac{a - b\sqrt{2}}{(a + b\sqrt{2})(a - b\sqrt{2})} = \frac{a - b\sqrt{2}}{a^2 - 2b^2}\]
        $a,b \in Q$,故 $a^2 - 2b^2 \neq 0$,分母是有理数.
        \item 分配律:过于冗长且平凡的计算,不再赘述. 当然根据前面的讨论,读者自行证明时只需要验证其中一个分配律即可.
    \end{enumerate}
    综上,$\mathbf{Q}(\sqrt{2})$ 是一个数域.
\end{enumerate}
\end{proof}

我想大部分读者都会对抽象出代数结构的原因表示不解,如果这个问题无法解答,我想在下一章直接引入抽象的线性空间更会引发同学们对于``学了这个有什么用''的怀疑. 我们可以举一些不那么贴切但具象的例子来说明这其中的意义. 读者高中阶段想必大都经受过解析几何的摧残,大家在拿到题目时总会首先观察到题目属于``定点''、``定值''或是``极值''等问题,大家将自动与自己做题的经验或技巧匹配用于解答这几类问题. 同理,在研究一个特定的代数系统(例如定义了加法和乘法的实数域)的性质时,我们可以首先将其归类为群、环或是域等,然后我们只需要利用群环域各自的性质来研究这个代数系统的性质,而不需要再去研究这个代数系统的具体定义. 在这一过程中我们找到了一个模型,即将一个孤立的问题转化为了对一个更广泛的问题的研究,正如将解决上千道解析几何问题转化为研究几种作为模型的题目的解法. 这一``寻找模型''的思想在将来的学习生活中我们将经常遇见,在实际中例如投资股票时我们可以将投资转化为提高投资组合的期望收益而尽力降低方差(风险)的求取极值的问题,在理论中,例如在计算理论的学习中我们会将各种各样不同的计算机架构抽象成图灵机模型,这在可计算性的研究中是最基础的模型. 对于这类抽象问题感兴趣的同学不妨可以选择数学科学学院的抽象代数等课程,或是阅读本讲义的``后继''教程\href{https://frightenedfoxcn.github.io/notes/series/alg-for-cs/}{《写给计算机系学生的代数》}作进一步的了解. 事实上,对于对理论感兴趣的同学,抽象代数将是必不可少的基础课程,它将是密码学、量子计算、计算理论以及编程语言理论等诸多领域的必要基础.

当然,这段描述因为涉及的知识容量较大,大概无法说服每一个读者. 但我们会在学习线性空间、线性映射的过程中不断重复这些思想,直到读者具备的知识容量足够时,一定能领会其中的奥妙.

\section{复数域的引入}

本书前半段讨论的框架是实数域、复数域都适用的,当然为了简化,我们的例子大都来源于实数. 从多项式一讲开始,我们便会开始强调实数域和复数域结论的不同,因此我们有必要在此引入复数域,复数域有很多种等价的定义方法,此处我们选择使用二维向量的运算定义.

直观来看,实数位于数轴上,复数则分布在二维平面上,因此我们可以先考虑平面点集$\mathbf{R}^2$,并在其上定义加法和乘法运算使其成为一个域. 我们回顾高中学习的平面向量知识,我们记$\vec{e}_1=(1,0)$,$\vec{e}_2=(0,1)$,则$\mathbf{R}^2$上的任一向量$\vec{u}=(x,y)$可写为$x\vec{e}_1+y\vec{e}_2$. 此外,我们仍沿袭高中对向量长度的定义,即$\lvert\vec{u}\rvert=\sqrt{x^2+y^2}$.

在\autoref{ex:Abel 群} 中我们已经验证了$\mathbf{R}^2$上的向量加法满足Abel群的条件,因此我们只需要定义$\mathbf{R}^2$上的乘法使得代数系统$\langle\mathbf{R}^2\setminus\{(0,0)\}\colon\circ\rangle$也为Abel群. 这一乘法的构造需要满足一些自然的条件,同时也能实现构成Abel群的要求. 事实上,我们有如下定理:
\begin{theorem}{}{复数乘法构造}
    平面点集$\mathbf{R}^2$上存在唯一的乘法$\circ$,满足
    \begin{enumerate}
        \item (单位元) $\vec{u}\circ\vec{e}_1=\vec{e}_1\circ\vec{u}=\vec{u},\enspace\forall\vec{u}\in\mathbf{R}^2$;

        \item (长度可乘性) $\lvert\vec{u}\circ\vec{v}\rvert=\lvert\vec{u}\rvert\lvert\vec{v}\rvert$.
    \end{enumerate}
    此乘法满足交换律,且使得$\langle\mathbf{R}^2\colon+,\circ\rangle$成为域.
\end{theorem}

上述定理中第一个条件是非常自然的,因为在二维平面上,$\{(x,0) \mid x\in\mathbf{R}\}$实际上就是实数轴,因此$\vec{e}_1=(1,0)$相当于实数1,因此作为乘法单位元是非常自然的. 第二条长度可乘则看起来没那么自然,但在接下来的证明中我们将会了解到其意义.

\begin{proof}
    对任意向量$\vec{u}=(a,b)=a\vec{e}_1+b\vec{e}_2,\enspace \vec{v}=(c,d)=c\vec{e}_1+d\vec{e}_2$,我们利用乘法的第一条性质有
    \[\vec{u}\circ\vec{v}=ac\vec{e}_1+(ad+bc)\vec{e}_2+bd\vec{e}_2\circ\vec{e}_2.\]
    由此可见$\vec{u}\circ\vec{v}=\vec{v}\circ\vec{u}$,因此乘法满足交换律. 同时可知,要定义乘法,关键是定义$\vec{e}_2\circ\vec{e}_2$的值.

    记$\vec{e}_2\circ\vec{e}_2=(x,y)$,由长度可乘性知$x^2+y^2=1$,另一方面
    \[(\vec{e}_1+\vec{e}_2)\circ(\vec{e}_1-\vec{e}_2)=\vec{e}_1-\vec{e}_2\circ\vec{e}_2=(1-x,-y).\]
    由$|\vec{e}_1+\vec{e}_2|=|\vec{e}_1-\vec{e}_2|=\sqrt{2}$以及长度可乘性可得
    \[4=|(\vec{e}_1+\vec{e}_2)\circ(\vec{e}_1-\vec{e}_2)|^2=(1-x)^2+y^2.\]
    结合 $x^2+y^2=1$ 求出$x=-1,\enspace y=0$. 这说明
    \[\vec{e}_2\circ\vec{e}_2=-\vec{e}_1.\]
    由此得乘法的定义$\vec{u}\circ\vec{v}=(ac-bd)\vec{e}_1+(ad+bc)\vec{e}_2$,即
    \[(a,b)\circ(c,d)=(ac-bd,ad+bc).\]
    可验证,此乘法以$\vec{e}_1$为单位元,等式$(ac-bd)^2+(ad+bc)^2=(a^2+b^2)(c^2+d^2)$表明乘法满足长度可乘性. 上述证明亦表明乘法唯一(只能这么构造$\vec{e}_2\circ\vec{e}_2$).

    接下来我们很容易验证$\langle\mathbf{R}^2\colon+,\circ\rangle$满足域的定义,我们留作习题供读者自行验证.
\end{proof}

在\autoref{thm:复数乘法构造} 赋予的乘法下,$\langle\mathbf{R}^2\colon+,\circ\rangle$称为复数域$\mathbf{C}$. 我们自然地将$\vec{e}_1$合理简记为1,同时$\vec{e}_2$简记为$\i$,因为此时$(a,b)$即为$a+b\i$,并且利用$\vec{e}_2^2=-\vec{e}_1$可知$\i^2=-1$,这与我们熟知的虚数单位的定义是统一的. 这一代数表示引入的相关概念,如实部、虚部、纯虚数,以及复数四则运算法则在高中阶段大家都已熟知,在此不再赘述.

非零复数$z=x+y\i$也可写为极坐标的形式,即$z=|z|(\cos\theta+\i\sin\theta)$,其中$|z|=\sqrt{x^2+y^2}$为复数的平面表示的模长,$\theta\in\mathbf{R}$为连接原点与$z$的有向线段与$x$轴正方向的夹角(在相差$2\pi$整数倍的意义下唯一). 我们称$\theta$为复数$z$的辐角. 关于复数的模长我们有经典的三角不等式:

\begin{theorem}{}{}
    设 $z, w \in \mathbf{C}$,则有 $|z + w| \leqslant |z| + |w|$.
\end{theorem}

这一定理的几何意义是非常显然的,我们将 $z$ 和 $w$ 放在平面直角坐标系中观察就可以明白这就是经典三角不等式的复数版本,此处证明略去. 等号成立的条件也显而易见,即 $z$ 和 $w$ 要么至少一个为 $0$,要么都非零且 $z$ 和 $w$ 位于从原点出发的同一条射线上. 除此之外还有一些大家在高中时就已经熟知的结果,例如:

\begin{theorem}{}{}
    设$z = (a, b) \in \mathbf{C}$,称 $\overline{z} = (a, -b)$ 为 $z$ 的共轭复数,则 $\overline{z} z = |z|^2$
\end{theorem}

直接计算即可证明这一结论. 总而言之,如上定义的复数域的所有性质都和高中所学是完全一致的,因为无论代数计算还是几何表示都完全一致,因此此后我们默认读者具有足够的基础知识,不再赘述一些基本常识.

\section{等价关系}

我们时常需要讨论集合中元素之间的关系. 例如直线间的平行、垂直、相交,或是数之间的大于、等于、小于关系. ``关系''在我们的讲义中将会多次出现,因此我们很有必要在此形式化定义这一概念,并强调其中一类特定的关系——等价关系.

我们首先从(二元)关系入手. 实际上,这里的二元关系和日常生活中的关系是紧密相连的,例如在父子关系中如果将全人类作为谈论的背景集合,那么$(\text{小头爸爸}, \text{大头儿子})$这一有序二元组是符合这一关系的,但$(\text{章鱼哥}, \text{海绵宝宝})$显然不符合. 因此我们可以将父子关系看作笛卡尔积集合$\text{人类}\times\text{人类}$的子集. 更一般化的,集合$A$中的关系可以由$A\times A$的子集
\[\{(a,b) \mid a,b\in A, \enspace a\,R\,b\}\]
来刻画,其中$R$是这个关系本身(实质上是两个元素之间的某种性质),例如之前讨论的父子关系,或是数学中的大于、小于或同余等. 事实上,反过来,由$A\times A$的子集可以确定一个关系,例如我把全世界所有的父子组合放在这个集合中,那么这个集合就定义了人类中的父子关系.
\begin{example}{}{}
    以下是一些关系的例子:
    \begin{enumerate}
        \item 设$A=\mathbf{R}$,则$A\times A$的子集
              \[\{(a,b)\in A\times A \mid a^2+b^2=1\}\]
              定义了一个关系$R$,即
              \[a\,R\,b \iff a^2+b^2=1.\]

        \item 设$A=\{1,2,3\}$,则$A\times A$的子集
              \[\{(1,1),(1,2),(1,3),(2,2),(2,3),(3,3)\}\]
              定义了一个关系$R$,即
              \[a\,R\,b \iff a\leqslant b.\]

        \item 设$A$为任意数集,$A\to A$ 的函数$f$也是一种关系,集合$A\times A$的子集
              \[B=\{(a,b)\in A\times A \mid b=f(a)\}\]
              刻画了这一关系. 换言之,函数是一种特殊的关系,它要求$\forall a\in A$有且仅有一个元素$b\in A$使得$(a,b)\in B$,其中$B$为上述定义的$A\times A$的子集,这个子集就是我们熟知的函数图像.

        \item 设$A=\mathbf{Z}$,关系$R$满足$a\,R\,b\iff a\equiv b \bmod n$,即模$n$同余,则$A\times A$的子集
              \[\{(a,b)\in A\times A \mid a\equiv b \bmod n\}\]
              可以刻画这一关系.
    \end{enumerate}
\end{example}

接下来我们要讨论一种特别的关系,即等价关系. 它对关系$R$有一定的规定:
\begin{definition}{}{等价关系}
    集合$A$中关系若满足以下条件:
    \begin{itemize}
        \item (自反性) $\forall a\in A, \enspace a\,R\,a$;

        \item (对称性) 若$a\,R\,b$,则$b\,R\,a$;

        \item (传递性) 若$a\,R\,b$,$b\,R\,c$,则$a\,R\,c$,
    \end{itemize}
    则称$R$为$A$的一个等价关系. 进一步地,若$R$是集合$A$的一个等价关系且$a,b\in A$,若$a\,R\,b$,则称$a$,$b$关于$R$是等价的,并把$A$中所有与$a$等价的元素集合
    \[\overline{a}=\{b\in A \mid b\,R\,a\}\]
    称为$a$所在的等价类,$a$称为这个等价类的代表元素,并记$\{\overline{a}\}$为所有等价类为元素构成的集族.
\end{definition}

我们可能需要一个例子来理解这些概念. 我们不难证明,初等数论中的同余关系是一种等价关系,以模3同余为例,我们取整体集合为正整数集合,对于3,它的等价类就是所有和3模3同余的元素集合,即所有3的倍数. 同理,对于1,它所在的等价类就是模3余1的全体正整数,2所在的等价类是全体模3余2的正整数. 除此之外,我们还发现一个特点,即这三个等价类将原集合分成了三个无交集的子集
\begin{gather*}
    \overline{0}=\{3k\mid k\in\mathbf{Z}\} \\
    \overline{1}=\{3k+1\mid k\in\mathbf{Z}\} \\
    \overline{2}=\{3k+2\mid k\in\mathbf{Z}\}
\end{gather*}
且它们的并集就是原集合,即这三个等价类构成了原集合的一个\term{分划}\index{fenhua@分划 (partition)}(即分为并为原集合且互不相交的子集). 这一结论对所有等价类都成立,是很直观的结论:
\begin{theorem}{}{等价类的性质}
    设$R$是集合$A$的等价关系,则由所有不同的等价类构成的子集族$\{\overline{a}\}$是$A$的分划. 反之,我们也可以基于分划在$A$中定义等价关系.
\end{theorem}

为了证明这一定理,我们首先需要一个引理:
\begin{lemma}{}{等价引理}
    设$R$是集合$A$的等价关系,$a,b\in A$,则$\overline{a}=\overline{b}\iff a\,R\,b$.
\end{lemma}
这一引理说明 $a$ 和 $b$ 等价当且仅当它们所在的等价类相同,或者说在同一个等价类中,相信根据等价类的定义这是很显然的结论.

\begin{proof}
\begin{enumerate}
    \item 充分性:$\overline{a}=\{ x \in A \mid x \,R\, a \}$,$\overline{b}=\{ x \in A \mid x \,R\, b \}$,因为 $a \in \overline{a}$,并且 $\overline{a} = \overline{b}$,所以 $a \in \overline{b}$,故 $a \,R\, b$.
    \item 必要性:$\forall c \in \overline{a}$,我们有 $c \,R\, a$,若 $a \,R\, b$,根据传递性可得 $c \,R\, b$,即 $c \in \overline{b}$,故 $\overline{a} \subseteq \overline{b}$,同理可得 $\overline{b} \subseteq \overline{a}$,故 $\overline{a} = \overline{b}$.
\end{enumerate}
\end{proof}

根据这一引理,我们可以迅速得到一个重要的推论,即等价类要么相等要么不相交:
\begin{corollary}{}{等价类的性质}
    设$R$是集合$A$的等价关系,$a,b\in A$,则下面二者必成立其一:
    \begin{enumerate}
        \item $\overline{a}\cap\overline{b}=\varnothing$;

        \item $\overline{a}=\overline{b}$.
    \end{enumerate}
\end{corollary}

\begin{proof}
    要证明二者成立其一,经过简单的逻辑变换就知道,我们只需要证明一条不成立时另一条必然成立即可. 我们假设第一条不成立,即 $\overline{a}\cap\overline{b}\neq\varnothing$,则存在 $x \in \overline{a}$ 且 $x \in \overline{b}$. 因此有 $x \,R\, a$ 且 $x \,R\, b$,由等价关系的对称性可得 $a \,R\, x$,再由传递性可得 $a \,R\, b$,则由\autoref{lem:等价引理}可得 $\overline{a}=\overline{b}$.
\end{proof}

由这一结论我们便可以很容易地完成\autoref{thm:等价类的性质}的证明:

\begin{proof}
\begin{enumerate}
    \item 要证明所有不同的等价类构成的子集族是 $A$ 的一个分划,只需证每个元素 $a$ 都至少属于一个子集,且任意两个不同的子集交为空集. 前者显然,后者由\autoref{cor:等价类的性质}易得.
    \item 设 $ \mathcal{P} $ 是集合 $ A $ 的一个分划,我们可以构造等价关系 $ R $:对于任意 $ a, b \in A $,定义 $ a \,R\, b $ 当且仅当存在一个子集 $ C \in \mathcal{P} $ 使得 $ a \in C $ 且 $ b \in C $.

    接下来验证 $R$ 是等价关系:
    \begin{enumerate}
        \item 自反性:对于任意 $ a \in A $,$ a $ 属于某个子集 $ C \in \mathcal{P} $,因此 $ a \,R\, a $.
        \item 对称性:如果 $ a \,R\, b $,则存在 $ C \in \mathcal{P} $ 使得 $ a, b \in C $. 因此,$ b \,R\, a $.
        \item 传递性:如果 $ a \,R\, b $ 且 $ b \,R\, c $,则存在 $ C_1, C_2 \in \mathcal{P} $ 使得 $ a, b \in C_1 $ 且 $ b, c \in C_2 $. 由于 $ \mathcal{P} $ 是分划,$ C_1 $ 和 $ C_2 $ 要么相等,要么没有交集. 由于 $ b \in C_1 \cap C_2 $,必须 $ C_1 = C_2 $,因此 $ a, c \in C_1 $,即 $ a \,R\, c $.
    \end{enumerate}
\end{enumerate}
\end{proof}

进一步此我们可以定义商集的概念:
\begin{definition}{商集}{} \index{shangji@商集 (quotient set)}
    设$R$是集合$A$的等价关系,以关于$R$的等价类为元素的集合(实际上是集合构成的集合,又称集族)$\{\overline{a}\}$称为$A$对$R$的\term{商集},记为$A/R$. 由
    \[\pi(a) = \overline{a}, \enspace \forall a\in A\]
    定义的$A$到$A/R$上的映射$\pi$称为$A$到$A/R$上的自然映射.
\end{definition}
我们可以看到,自然映射$\pi$将$A$中的元素$a$映到自己所在的等价类$\overline{a}$. 基于上述定义,我们可以完成在基本代数结构一节中遗留的一个问题:我们能否定义非数域的域?答案是肯定的,如果同学们对密码学感兴趣的应当听闻过有限域这一概念,接下来我们将通过简单的例子来说明这一概念.

\begin{example}{}{有限域}
    设$\mathbf{Z}_n$是$\mathbf{Z}$关于模$n$同余关系$R$的商集(也称$\mathbf{Z}$的模$n$剩余类),即
    \[\mathbf{Z}_n=\mathbf{Z}/R=\{\overline{0},\overline{1},\ldots,\overline{n-1}\}.\]
    即$\mathbf{Z}_n$中的元素是$n$个集合,其中第$i$个集合是全体模$n$余$i-1\enspace(i=1,2,\ldots,n)$的整数构成的集合.

    在$\mathbf{Z}_n$上我们仍然希望它可以继承$\mathbf{Z}$上的加法和乘法运算,因为这是整数作为环结构最重要的两种运算(减法就是加上相反数,除法无法在整数集合定义,因为会带来非整数的有理数). 我们首先定义对加法运算的继承:
    \[\overline{a}\oplus\overline{b}=\overline{a+b}.\]
    它的含义是$a$对应的等价类与$b$对应的等价类的和就是$a$和$b$的和对应的等价类,因此这样的定义是自然地``继承''了$\mathbf{Z}$上的加法运算. 举个例子,当$n=3$时,$\overline{1}+\overline{2}=\overline{1+2}=\overline{3}=\overline{0}$.

    接下来我们需要定义乘法$\circ$,同样是一个自然的定义,即$\overline{a}\circ\overline{b}=\overline{ab}$. 我们很容易验证$\forall n\in\mathbf{Z}$且$n\geqslant 2$,$\langle \mathbf{Z}_n\colon\oplus,\circ\rangle$构成一个含幺交换环(因为较为显然此处从略). 我们要讨论的是何时$\langle \mathbf{Z}_n\colon\oplus,\circ\rangle$构成域,由此我们便构造了一个非数域的域(因为集合的元素不是数,是等价类),并且元素个数是有限的.

    我们这里可以给出结论:$\langle \mathbf{Z}_n\colon\oplus,\circ\rangle$是域当且仅当$n$是素数. 这一结论的证明需要一些数论的知识,我们放在习题中供感兴趣的同学证明. 实际上,如果一个域中只有有限个元素,这样的 $\mathbf{Z}_n$ 称为有限域. 可以证明对于任何一个域,要么自然数集是其一个子集,此时我们称域的\term{特征}为 $0$,例如之前介绍的数域(有理数域、实数域、复数域等),要么它包含了某个 $\mathbf{Z}_p$,这时我们称域的特征为 $p$. 本书中没有特别说明时均假设域的特征为 $0$,即我们总是可以找到任意多有限个的不同整数.
\end{example}

上面的定义看起来非常简单而美好,但我们遗漏了一个重要的细节没有讨论:上面定义的加法和乘法,真的是``合理''的吗?. 我们都知道 $n = 3$ 时,$\overline{1} = \overline{124}$,$\overline{2} = \overline{365}$,然而在定义加法和乘法的时候,我们并没有检验
\begin{gather} \label{eq:模加法和乘法相容的例子}
    \overline{1} \oplus \overline{2} = \overline{124} \oplus \overline{365} \\
    \overline{1} \circ \overline{2} = \overline{124} \circ \overline{365}
\end{gather}
是否成立. 更一般地,令 $R$ 表示模 $n$ 同余的等价关系,若 $a\,R\,b$,$c\,R\,d$,是否有
\begin{gather*}
    \overline{a} \oplus \overline{c} = \overline{b} \oplus \overline{d} \\
    \overline{a} \circ \overline{c} = \overline{b} \circ \overline{d}
\end{gather*}
成立?我们先不讨论在模 $n$ 同余的等价关系中上述等式是否成立(我们后面会证明的确是成立的),而是先来看一下违反这一性质会带来什么后果,从而理解这一要求的含义是什么. 事实上,违反这一性质会使得 $\mathbf{Z}_n$ 上定义的加法和乘法不再是一个本讲开头所说的映射. 我们以加法为例展开讨论(乘法同理). 因为 $a\,R\,b$ 表明 $\overline{a} = \overline{b}$,$c\,R\,d$ 表明 $\overline{c} = \overline{d}$,因此如果 $\mathbf{Z}_n$ 上的加法是一个映射 $\oplus(x,y) \colon \mathbf{Z}_n \to \mathbf{Z}_n$,那么必须有 $\oplus(\overline{a},\overline{c}) = \oplus(\overline{b},\overline{d})$,否则两个完全一致的等价类的加法会得到不同的等价类,也就是说加法可以把同一组元素映到多个值,那么显然不再是一个映射,乘法也是同理. 当然,很幸运的是,上面的定义保证了这一``合理性'',例如前面给出的\autoref{eq:模加法和乘法相容的例子},$\overline{1} \oplus \overline{2} = \overline{0} = \overline{124} \oplus \overline{365}$,这里加法仍然满足映射的条件. 当然我们可以给出更严谨的一般性证明,在此之前我们先给这一``合理''的性质一个更规范的定义:
\begin{definition}{}{}
    设$A$是一个集合,$R$是$A$上的一个等价关系,若$A$上有二元运算$+$,且我们在$A/R$上以如下自然的方式继承了$A$上的$\oplus$运算:
    \[\overline{a}\oplus\overline{b}=\overline{a+b},\enspace\forall a,b\in A,\]
    则我们称$\oplus$与$R$\term{相容(compatible)}当且仅当$a\,R\,b$,$c\,R\,d$时有$\overline{a}\oplus\overline{c}=\overline{b}\oplus\overline{d}$,或者说等价的,$(a+b)R(c+d)$.
\end{definition}

在这种情况下,我们也称$\oplus$在$A/R$上是\term{良定义(well-defined)}的,但良定义这一词用途较广,一般指这一定义在当前语境下是否可接受,不一定用在商集相关的场合.

这一定义有两个要点,其一是把原集合上的运算自然地继承到商集上,其二是这一继承还能使得运算是一个映射. 事实上,相容性(良定义性)是非常重要的,所谓良定义,即这个东西就其自身不产生矛盾,这在形而上学家的论述中最常出现的例子是``圆的方''这个语汇. 良定义的问题在商集中特别典型,因为这是这个系列中最容易出现从某个集合``诱导''(即自然地投射到商结构)定义出新东西的情形,它也有相当大的可能性(尽管本讲义中显然避免了这种情况)是不良定义的. 在给出规范定义后,我们可以证明如下结论:

\begin{theorem}{}{}
    模$n$同余关系$R$与加法$\oplus$和乘法$\circ$相容.
\end{theorem}
\begin{proof}
    我们只证明加法的相容性,乘法的相容性留作练习. 设$a\,R\,b$,$c\,R\,d$,则可以设$b=a+kn$,$d=c+ln$,其中$k,l\in\mathbf{Z}$,则$\overline{b}\oplus\overline{d}=\overline{b+d}=\overline{a+c+(k+l)n}=\overline{a+c}=\overline{a}\oplus\overline{c}$,故$\oplus$与$R$相容.
\end{proof}

\section{高斯-若当消元法}

线性代数这一学科起源于对线性方程组解的研究,而高斯-若当消元法是最基础的求解线性方程组的方法,是之后章节的重要的基础.

那么什么是线性方程组呢?在小学阶段我们便已经知道:含有未知数的等式称为方程. 那时我们只接触了一元一次方程,即形如 $ax = b$ 的方程,其中 $a, b$ 是已知的常数,$x$ 是未知数. 现在我们将未知数增多,将形如:
\[a_1x_1 + a_2x_2 + \cdots + a_nx_n = b\]
的方程称为线性方程,其中 $a_1, a_2, \cdots, a_n, b$ 是已知的常数,$x_1, x_2, \cdots, x_n$ 是未知数,未知量的个数 $n$ 称为方程的``元数''.

实际上,所谓``线性''实际上就是各个未知数都是一次的并且未知量之间只有加减运算,例如方程
\[2x_1 + 3x_2 + x_3 +5x_4 = 5\]
是线性方程,但
\[2x_1 + 3x_2^2 + x_3 + 5x_4 = 5\]
和
\[2x_1 + 3\sqrt{x_2x_3} + 5x_4 = 5\]
都不是线性方程,因为 $x_2^2$ 不是一次式,$\sqrt{x_2x_3}$ 中 $x_2$ 和 $x_3$ 之间是乘法运算. 实际上,这里的``线性''是非常符合直觉的,例如线性方程 $ax + by = c$ 代表平面上的直线,而非线性方程 $x_1x_2 = 1$ 代表双曲线(是曲线);线性方程 $ax + by + cz = d$ 是平面方程,而 $ax^2 + by^2 +cz^2 = d$ 是曲面方程. 请注意,这里``线性''的定义与直观与``线性代数''中的``线性''含义一致,因此在此后的学习中,我们遇见``线性''这一名词时,脑海中的定义——一次式的加减组合和直观——直线、平面都是将类似的.

进一步地,一组线性方程的集合称为线性方程组. 我们给出正式定义如下:

\begin{definition}{线性方程组}{线性方程组}
    一般地,一个由 $m$ 个线性方程组成的 $n$ 元(即变量数为 $n$)线性方程组可以表示为:
    \[ \begin{cases}
        \begin{aligned}
            a_{11}x_1+a_{12}x_2+\cdots+a_{1n}x_n & = b_1 \\
            a_{21}x_1+a_{22}x_2+\cdots+a_{2n}x_n & = b_2 \\
                                                & \vdotswithin{=} \\
            a_{m1}x_1+a_{m2}x_2+\cdots+a_{mn}x_n & = b_m
        \end{aligned}
    \end{cases} \]
\end{definition}

我们的目标是求解线性方程组,所谓线性方程组的解也就是找到一组满足所有方程的所有未知数的值. 考虑最简单的二元一次方程组:
\[ \begin{cases}
    \begin{aligned}
        a_{11}x_1+a_{12}x_2 & = b_1 \\
        a_{21}x_1+a_{22}x_2 & = b_2
    \end{aligned}
\end{cases} \]
从几何上这就相当于求解两条直线的交点,因此解的情况可能有如下三种:
\begin{itemize}
    \item 两直线相交于一点,此时方程组有唯一解;
    \item 两直线平行且不重合,此时方程组无解;
    \item 两直线重合,此时方程组有无穷多解.
\end{itemize}

本节介绍高斯-若当消元法的目标在于,提供一种系统的方法,通过一系列的变换将线性方程组化为更简单的形式,从而得到方程组属于有唯一解、无解还是无穷解的情况,并求出有解情况下的解. 在开始介绍高斯-若当消元法之前,我们先引入一些基本的概念.

\subsection{$n$ 元向量与矩阵的定义}

高中阶段我们已经学习了平面向量和空间向量,其中平面向量是两个实数元素构成的有序对,空间向量是三个实数元素构成的有序对,因此我们分别称其为二元实向量和三元实向量. 我们可以将向量的概念进一步推广,我们需要借助如下定义进行推广:

\begin{definition}{$n$ 元笛卡尔积}{}
    设 $A_i~(1\leqslant i\leqslant n)$ 是一族集合,我们把集合
    \[
    \prod_{i=1}^n A_i = \{(a_1, a_2, \cdots, a_n) \mid a_i\in A_i\}
    \]
    称为集族 $\{A_i\}$ 的 \term{$n$ 元笛卡尔积}. 特别地,当所有的 $A_i$ 都为集合 $A$ 时,记
    \[
    A^n = \prod_{i=1}^n A = \{(a_1, a_2, \cdots, a_n) \mid a_i \in A\}
    \]
    为集合 $A$ 的 $n$ 次\term{笛卡尔幂}.
\end{definition}

不难看出这一定义就是将\autoref{def:笛卡尔积}中定义的两个集合之间的笛卡尔积推广到 $n$ 个集合之间的笛卡尔积. 根据笛卡尔幂的定义,此前提到的二维平面点集 $\mathbf{R} \times \mathbf{R}$ 简记为 $\mathbf{R}^2$ 也是符合这一定义的. 类似地,$\mathbf{R}^3$ 表示的是三维空间中的点集,空间中的点可以用三个实数来表示,也就是形如 $(x, y, z)$ 的有序对,其中 $x, y, z \in \mathbf{R}$,在这种情况下我们有时会用``空间''来代指集合,而用``点''来代指集合的元素.

一般地,如果是 $n$ 个 $\mathbf{R}$ 进行笛卡尔积,我们便得到了集合 $\mathbf{R}^n$,其中的元素是由 $n$ 个实数组成的有序对,记为 $(x_1, x_2, \cdots, x_n)$,其中 $x_i \in \mathbf{R}$,这样的有序对我们称之为\term{ $n$ 元实向量}(如果类似定义 $\mathbf{C}^n$ 中的元素则称为 $n$ 元复向量). 其中的 $x_i$ 称为向量的第 $i$ 个分量. 如果 $n$ 个分量都为 $0$,我们称这个向量为\term{零向量}.

为了使后文的记号不会使人迷惑,我们必须在此强调一个记号问题. 严格来说 $\mathbf{R}^n$ 中的符号应当是沿竖向书写的(即应当是\term{列向量}),或者横向书写(即写成\term{行向量}的形式)时在右上角添加 $\mathrm{T}$ 作为上标,例如
\[
\begin{pmatrix}
    1 \\ 2 \\ 3
\end{pmatrix} = (1, 2, 3)^{\mathrm{T}} \in \mathbf{R}^3
\]
这个上标我们称之为\term{转置}符号,下面定义矩阵之后我们会给出转置的严格定义. 需要注意的是,本书中为了简便以及节省空间起见,书写时有时会省去右上角的转置符号,因此可能表达行向量或者列向量的含义,请读者在阅读时注意甄别. 但注意只要是 $\mathbf{R}^n$ 中的元素,它们严格来说都应当写成列向量,行向量表达的严格含义在\nameref{chap:对偶空间}一章中会给出解释.

下面我们可以给出 $n$ 元向量的加法和数乘的定义,这就是中学阶段平面向量、空间向量加法和数乘(数量积)运算的自然推广. 对于任意的 $n$ 元向量 $(x_1, x_2, \cdots, x_n)^\mathrm{T}$ 和 $(y_1, y_2, \cdots, y_n)^\mathrm{T}$,以及任意的实数 $k$,我们定义:
\begin{gather*}
    (x_1, x_2, \cdots, x_n)^\mathrm{T} + (y_1, y_2, \cdots, y_n)^\mathrm{T} = (x_1+y_1, x_2+y_2, \cdots, x_n+y_n)^\mathrm{T} \\
    k(x_1, x_2, \cdots, x_n)^\mathrm{T} = (kx_1, kx_2, \cdots, kx_n)^\mathrm{T}
\end{gather*}

接下来我们定义矩阵的相关概念,这能使得我们之后的讨论更加方便.

\begin{definition}{矩阵}{}
    域 $\mathbf{F}$ 中的 $m \times n$ 个元素 $a_{ij} \enspace (i = 1,\ldots,m, \enspace j=1,\ldots,n)$ 排成 $m$ 行 $n$ 列的矩形数表,称为域 $\mathbf{F}$ 上的一个 $m\times n$ 矩阵,记作
    \[A = \begin{pmatrix}
            a_{11} & a_{12} & \cdots & a_{1n} \\
            a_{21} & a_{22} & \cdots & a_{2n} \\
            \vdots & \vdots & \ddots & \vdots \\
            a_{m1} & a_{m2} & \cdots & a_{mn}
        \end{pmatrix}\]
    或简记为 $(a_{ij})_{m \times n}$,其中 $a_{ij}$ 表示矩阵 $A$ 的第 $i$ 行第 $j$ 列的元素.
\end{definition}

因此矩阵直观来看就是一个二维的数表. 此前我们提到的 $\mathbf{R}^n$ 中的元素($n$ 元列向量)就可以视为一个 $n \times 1$ 的矩阵,$n$ 元行向量则可以视为一个 $1 \times n$ 的矩阵. 接下来是此前提到过的转置的正式定义:

\begin{definition}{矩阵的转置}{矩阵的转置}
    设 $A = (a_{ij})_{m \times n}$,则 $A$ 的\term{转置矩阵}\index{zhuanzhi@转置 (transpose)}是一个 $n \times m$ 矩阵,记作 $A^\mathrm{T}$,它的第 $k$ 行正好是 $A$ 的第 $k$ 列($k = 1,2,\ldots,n$);它的第 $r$ 列正好是 $A$ 的第 $r$ 行($r = 1,2,\ldots,m$).
\end{definition}

根据定义,矩阵的转置显然等价于第 $i$ 行第 $j$ 列的元素变成转置后矩阵的第 $j$ 行第 $i$ 列的元素. 一个简单的例子是,令 $A = \begin{pmatrix}1 & 2 & 3 & 4 \\ 5 & 6 & 7 & 8 \\ 9 & 10 & 11 & 12\end{pmatrix}$,则 $A^\mathrm{T} = \begin{pmatrix}1 & 5 & 9 \\ 2 & 6 & 10 \\ 3 & 7 & 11 \\ 4 & 8 & 12\end{pmatrix}$. 此外,转置符号之前我们已经使用过,即列向量等于行向量的转置,不难验证这的确是符合更一般的矩阵转置定义的.

事实上,在求转置的时候,我们只需要把第一行保持顺序变成第一列,第二行保持顺序变成第二列,以此类推. 因为这样第一行第 $i$ 列的元素就变成了第 $i$ 行第一列的元素,符合转置的要求,其它行的元素也是类似的,从上面的例子也能看出这一点.

\subsection{高斯-若当消元法}

有了上述的基础之后,接下来便可以开始介绍高斯-若当消元法的具体操作. 我们先从一个简单的例子开始体会其基本思想:

\begin{example}{}{高斯-若当消元法引入}
    求解线性方程组
    \begin{numcases}{}
        3x_1+2x_2+x_3 = 39 \label{eq:高斯消元引入1} \\
        2x_1+3x_2+x_3 = 34 \label{eq:高斯消元引入2} \\
        x_1+2x_2+3x_3 = 26 \label{eq:高斯消元引入3}
    \end{numcases}

    将方程\eqref{eq:高斯消元引入3}分别乘以 $-3$ 和 $-2$ 加到方程\eqref{eq:高斯消元引入1}、\eqref{eq:高斯消元引入2}上消去两个方程中的 $x_1$,得

    \begin{numcases}{}
        -4x_2-8x_3=-39 \label{eq:高斯消元引入4} \\
        -x_2-5x_3=-18 \label{eq:高斯消元引入5}
    \end{numcases}

    再将方程\eqref{eq:高斯消元引入5}乘以 $-4$ 加到方程\eqref{eq:高斯消元引入4}上消去 $x_2$ 可知 $x_3 = \dfrac{11}{4}$,最后依次代回\eqref{eq:高斯消元引入4}和\eqref{eq:高斯消元引入1}可知依次解出 $x_2 = \dfrac{17}{4}$ 以及 $x_1 = \dfrac{37}{4}$.
\end{example}

可以看出,求解线性方程组无非是通过加、减、乘等尽可能地减少每个方程中未知数的个数,即消元. 这样的消元思想我们中学阶段就已经知道,只是我们没有系统地总结出来应用于更多个多元的线性方程组上. 事实上,这样的思想最早出现在中国古代数学著作《九章算术》中,相关内容在大约公元前 150 年就出现了. 在古代,方程这一词中的``方''为并列、对等,``程''表示以“算筹”列成竖式的计算过程. 实际上,上面的方程对应于九章算术上的原文:

\begin{quote}
    \kaishu
    今有上禾三秉,中禾二秉,下禾一秉,实三十九斗;上禾二秉,中禾三秉,下禾一秉,实三十四斗;上禾一秉,中禾二秉,下禾三秉,实二十六斗. 问上、中、下禾实一秉各几何?
\end{quote}

在欧洲,牛顿(Newton)最先发现了这种方法. 在 1670 年,牛顿写道他所知晓的所有代数教科书都缺少求解方程组的方法,而他随后补充了这一部分. 在牛顿离开学术生涯很久以后,剑桥大学才在 1707 年最终以 \textit{Arithmetica Univeralis} 的标题出版了他的笔记. 这些笔记被广泛复制,在 18 世纪末成为了代数课本的标准内容. 1810 年,高斯(Gauss)在预测谷神星轨道时引入了高斯消元法,这是种系统性的线性方程组解法,后来这种记法被手算员们广泛应用于解决最小二乘问题. 他在对天体轨道建模时处理了 12 个方程对应 6 个未知变量的情况. 他已经清楚地了解方程组有无穷解,唯一解还是无解的条. 后来这一算法由于对历史的混淆在 1950 年代被以高斯命名,称之为``高斯消元法''.

1888年,德国数学家若当(Jordan)发现了这种高斯消元法的变体(具体的变化在介绍完高斯-若当消元法后会给出). 然而,相同的方法也出现在克莱森(Clasen)在同年出版的文章中. 若当与克莱森有可能是各自独立地发现了高斯-若当消元法.

下面我们将详细介绍高斯-若当消元法这一系统的解线性方程组的方法. 我们将由 $m$ 个方程组成的 $n$ 元线性方程组的一般形式再次写下:
\[ \begin{cases} \begin{aligned}
            a_{11}x_1+a_{12}x_2+\cdots+a_{1n}x_n & = b_1           \\
            a_{21}x_1+a_{22}x_2+\cdots+a_{2n}x_n & = b_2           \\
                                                 & \vdotswithin{=} \\
            a_{m1}x_1+a_{m2}x_2+\cdots+a_{mn}x_n & = b_m
        \end{aligned} \end{cases} \]
将其系数按照如下的方式排列,可以得到一个矩阵,称其为该线性方程组的\term{系数矩阵},记为
\[A = \begin{pmatrix}
        a_{11} & a_{12} & \cdots & a_{1n} \\
        a_{21} & a_{22} & \cdots & a_{2n} \\
        \vdots & \vdots & \ddots & \vdots \\
        a_{m1} & a_{m2} & \cdots & a_{mn}
    \end{pmatrix}\]
且记 $\vec{b} = (b_1,b_2,\ldots,b_m)^\mathrm{T}$,若 $\vec{b} = \vec{0}$ 则称此方程为\term{齐次线性方程组},否则为\term{非齐次线性方程组}. 再将 $n$ 个未知量记为 $n$ 元列向量 $X = (x_1,x_2,\ldots,x_n)^\mathrm{T}$,我们便可以把方程组简记为
\begin{equation} \label{eq:线性方程组的矩阵表示}
    AX = \vec{b}
\end{equation}
展开写就是,$\forall (x_1,x_2,\ldots,x_n)^\mathrm{T} \in \mathbf{R}^n$,有
\[\begin{pmatrix}
        a_{11} & a_{12} & \cdots & a_{1n} \\
        a_{21} & a_{22} & \cdots & a_{2n} \\
        \vdots & \vdots & \ddots & \vdots \\
        a_{m1} & a_{m2} & \cdots & a_{mn}
    \end{pmatrix} \begin{pmatrix}
        x_1 \\ x_2 \\ \vdots \\ x_n
    \end{pmatrix} = \begin{pmatrix}
        b_1 \\ b_2 \\ \vdots \\ b_m
    \end{pmatrix} = \begin{pmatrix}
        a_{11} x_1 + a_{12} x_2 + \cdots + a_{1n} x_n \\
        a_{21} x_1 + a_{22} x_2 + \cdots + a_{2n} x_n \\
        \cdots\\
        a_{m1} x_1 + a_{m2} x_2 + \cdots + a_{mn} x_n
    \end{pmatrix}
\]

我们将在\nameref{sec:矩阵乘法}一节中发现,这就是矩阵乘法的一个特殊情况,但在这里我们只需要记住这一运算规则即可,也就是一个矩阵左乘一个列向量是如何进行的. 为了让这个规则看起来更加清晰,根据 $n$ 元向量的运算规则,我们可以做如下变换:
\[\begin{pmatrix}
    a_{11} x_1 + a_{12} x_2 + \cdots + a_{1n} x_n \\
    a_{21} x_1 + a_{22} x_2 + \cdots + a_{2n} x_n \\
    \cdots\\
    a_{m1} x_1 + a_{m2} x_2 + \cdots + a_{mn} x_n
\end{pmatrix} = x_1 \begin{pmatrix}
    a_{11} \\ a_{21} \\ \vdots \\ a_{m1}
\end{pmatrix} + x_2 \begin{pmatrix}
    a_{12} \\ a_{22} \\ \vdots \\ a_{m2}
\end{pmatrix} + \cdots + x_n \begin{pmatrix}
    a_{1n} \\ a_{2n} \\ \vdots \\ a_{mn}
\end{pmatrix} = \vec{b}\]
令 $\vec{\beta}_i = (a_{1i},a_{2i},\ldots,a_{mi})^\mathrm{T}$,即方程组系数矩阵 $A$ 的某一列,则方程组可以简记为
\begin{equation} \label{eq:线性方程的向量表示}
    x_1\vec{\beta}_1 + x_2\vec{\beta}_2 + \cdots + x_n\vec{\beta}_n = \vec{b}
\end{equation}
结合\eqref{eq:线性方程组的矩阵表示} 和 \eqref{eq:线性方程的向量表示},我们可以得到矩阵左乘列向量的清晰的运算规则,即
\begin{equation} \label{eq:矩阵左乘列向量}
    AX =  x_1\vec{\beta}_1 + x_2\vec{\beta}_2 + \cdots + x_n\vec{\beta}_n
\end{equation}
其中 $x_i$ 是列向量 $X$ 的第 $i$ 个分量,$\vec{\beta}_i$ 是系数矩阵 $A$ 的第 $i$ 列. 关于此式的更进一步的解释,我们放在\nameref{sec:分块矩阵}一节中给出. 我们使用带数值的例子来练习一下这一运算规则:
\[
    \begin{pmatrix}
        1 & 2 & 3 \\ 4 & 5 & 6
    \end{pmatrix} \begin{pmatrix}
        7 \\ 8 \\ 9
    \end{pmatrix} = 7 \begin{pmatrix}
        1 \\ 4
    \end{pmatrix} + 8 \begin{pmatrix}
        2 \\ 5
    \end{pmatrix} + 9 \begin{pmatrix}
        3 \\ 6
    \end{pmatrix} = \begin{pmatrix}
        50 \\ 122
    \end{pmatrix}
\]

在以上的记号下,我们定义几个概念:

增广矩阵:$(A,\vec{b})$,即左$n$列为系数矩阵,最右列为列向量$\vec{b}$的$b+1$列矩阵.

阶梯矩阵:系数全零行在最下方,并且非零行中,在下方的行的第一个非零元素一定在上方行的右侧(每行第一个非零元素称主元素).

简化阶梯矩阵:阶梯矩阵中每个主元素所在列的其余元素全为$0$.

有了这些定义,我们可以将解线性方程组的过程转化为矩阵的初等行变换. 高斯-若当消元法的一般步骤如下:
\begin{center}
    线性方程组$\overset{1}{\longrightarrow}$增广矩阵$\overset{2}{\longrightarrow}$阶梯矩阵$\overset{3}{\longrightarrow}$(行)简化阶梯矩阵$\overset{4}{\longrightarrow}$解
\end{center}

\begin{enumerate}[label=步骤\arabic*~]
    \item 只需要将线性方程组转化为$(A, \vec{b})$的形式,得到增广矩阵;

    \item 通过初等行变换后,得到阶梯矩阵.

    \item 将主元素化1后将主元素所在列的其他元素均通过初等行变换化为0即可,即化为简化阶梯矩阵.

    \item \label{item:1:解方程组}
          我们分三种情况讨论:
          \begin{enumerate}
              \item 有唯一解:没有全零行,最后一个主元素的行号与系数矩阵的列数相等,且简化阶梯矩阵对角线上全为1,其余元素均为0,此时可以直接写出解.前面的例子就属于这种情况.

              \item 无解:出现矛盾方程,即系数为0的行的行末元素不为0,此时直接写无解即可;

              \item 有无穷解:非上述情况. 此时设出自由未知量将其令为$k_1,k_2,\ldots$,然后代入增广矩阵对应的方程组即可. 注意选取自由未知量时,选取没有主元素出现的列对应的未知量是更常见的做法,当然选择其他作为自由未知量也可以.
          \end{enumerate}
\end{enumerate}

实际上,``高斯消元法''指代消元到阶梯形矩阵之前的过程,经过若当改进的``高斯-若当消元法''指代消元到简化阶梯矩阵的过程. 根据前面我们给出的例子,显然高斯-若当消元法中得到的简化阶梯矩阵对于判断解的情况以及写出解的目标更简单.

从高斯-若当消元法开始,我们正式进入线性代数的学习. 实际上,上述\ref*{item:1:解方程组}中关于方程组解的情况的讨论我们是浮于表面,是基于算法最后得到的矩阵的形式进行的讨论,但事实上,这背后蕴含着更深刻的意义. 我们将会在接下来的十余个章节中讲述线性代数中的核心概念,并在\nameref{chap:朝花夕拾}中回过头来重新审视线性方程组解的问题. 相信在那时,经历十余章各式抽象概念和运算技巧的洗礼后再来回味这一问题的你,定有``守得云开见月明''之感,对线性代数的理解也会更深一层.

\begin{summary}

    本讲为了后续章节讲述方便引入了一些基本概念和算法. 尽管这是一门面向理工科应用的数学课,但我们仍然希望以最自然的方式引入概念,而非填鸭式地轰炸,因此我们首先从大家最熟悉的实数集合开始,讨论在集合上定义运算的方法:我们逐步加强条件,引入了三种基本的代数结构——群、环和域,并且给出了一些例子,并简单讨论了定义代数系统的意义. 事实上,下一讲开始要介绍的线性空间也是一种特殊的代数结构,因此首先引入代数结构对于我们自然展开接下来的讨论有很大的帮助,不至于让读者觉得非常突兀.

    接下来我们也从域的定义入手,构造了$\mathbf{R}^2$上的乘法运算使其构成了一个域,并且我们发现这里的定义与高中学习的复数乘法是完全一致的. 之后我们引入了等价关系的概念,这一概念在后续的讲义中将会多次出现,其重要意义就是将一个集合划分成了几个等价的区域. 最后我们讨论了高斯-若当消元法的一般步骤,这是我们接下来解决线性空间中各类问题绕不开的算法.

\end{summary}

\begin{exercise}
    \exquote[浙江大学数学科学学院教授吴志祥]{我这门课很简单,只有简单的加减乘除四则运算,甚至除法都不太需要.}

    \begin{exgroup}
        \item 完善\autoref{thm:复数乘法构造} 中的证明,即证明$\mathbf{R}^2$在平面向量加法和如\autoref*{thm:复数乘法构造} 定义的乘法下构成一个域.

        \item 下列集合中的关系 $ R $ 是否是等价关系?如果是,说明等价类的意义,并写出集合关于 $ R $ 的商集.
        \begin{enumerate}
            \item 实数集上定义关系:$ xRy \iff x - y $ 为无理数或 0;
            \item 集合 $ S = \{ (x, y) | |x| \leq 1, |y| \leq 1 \} $ 上定义关系:\[ (x_1, y_1) R (x_2, y_2) \iff x_1 = x_2; \]
            \item 集合 $ S = \{ (x, y) | x| < +\infty, |y| < +\infty \} $ 上定义关系:\[ (x_1, y_1) R (x_2, y_2) \iff x_1^2 + y_1^2 = x_2^2 + y_2^2; \]
            \item 平面上所有直线组成的集合 $ L $ 上分别定义关系 $ R_1 $ 和 $ R_2 $ 为:\[ l_1 R_1 l_2 \iff l_1 \perp l_2; \quad l_1 R_2 l_2 \iff l_1 \text{与 $l_2$ 相交};\]
            \item 集合 $ S = \{ (x, y) | 0 < |x| < +\infty, 0 < |y| < +\infty \} $ 上定义关系:\[ (x_1, y_1) R (x_2, y_2) \iff x_1 x_2 > 0 \ \text{且} \ y_1 y_2 > 0; \]
            \item 集合 $ C^* = \{ a + bi | a, b \in \mathbb{R}, a \neq 0 \} $ 上定义关系:\[ (a_1 + b_1 i) R (a_2 + b_2 i) \iff a_1 a_2 > 0; \]
            \item 在集合 $ A = (a_1, a_2, a_3, a_4) $ 的幂集 $ P(A) $ 上定义关系:\[ A_1 R A_2 \iff |A_1| = |A_2| \]
            \item 在集合 $ \mathbb{N}^* \times \mathbb{N}^* $ 上定义关系:\[ (m_1, n_1) R (m_2, n_2) \iff m_1 n_2 = m_2 n_1.\]
        \end{enumerate}

        \begin{answer}
            \begin{enumerate}
                \item 不是;
                \item 是, 矩形内平行于 $y$ 轴的直线段, $S/R = \{ \text{线段} \ x = c (|y| \leq 1) : |c| \leq 1 \} $;
                \item 是, 以原点为心的同心圆, $S/R = \{ x^2 + y^2 = r^2 | r \geq 0 \} $;
                \item 均不是;
                \item 是, $S/R = \{ \text{不含坐标轴的四个象限} \} $;
                \item 是, $S/R = \{ \text{复平面的左半平面和右半平面} \} $;
                \item 是, $S/R = \{ S_i | i = 0, 1, 2, 3, 4 \}, S_i $ 是含 $i$ 个元素的子集组成的集合, $S_0 = \{ \emptyset \}$;
                \item 是, $S/R = \{ (m,n) \mid \frac{m}{n} = c | c \in \mathbb{Q} \}$.
            \end{enumerate}
        \end{answer}

        \item 求齐次线性方程组$\begin{cases}
                x_1+x_2+x_3+4x_4-3x_5=0   \\
                2x_1+x_2+3x_3+5x_4-5x_5=0 \\
                x_1-x_2+3x_3-2x_4-x_5=0   \\
                3x_1+x_2+5x_3+6x_4-7x_5=0
            \end{cases}$的通解.

        \begin{answer}
            \begin{align*}
                A ={} &
                \begin{pmatrix}
                    1 & 1  & 1 & 4  & -3 \\
                    2 & 1  & 3 & 5  & -5 \\
                    1 & -1 & 3 & -2 & -1 \\
                    3 & 1  & 5 & 6  & -7
                \end{pmatrix}
                \rightarrow
                \begin{pmatrix}
                    1 & 1  & 1 & 4  & -3 \\
                    0 & -1 & 1 & -3 & 1  \\
                    0 & -2 & 2 & -6 & 2  \\
                    0 & -2 & 2 & -6 & 2
                \end{pmatrix} \rightarrow \\
                      &
                \begin{pmatrix}
                    1 & 1 & 1  & 4 & -3 \\
                    0 & 1 & -1 & 3 & -1 \\
                    0 & 0 & 0  & 0 & 0  \\
                    0 & 0 & 0  & 0 & 0
                \end{pmatrix}
                \rightarrow
                \begin{pmatrix}
                    1 & 0 & 2  & 1 & -2 \\
                    0 & 1 & -1 & 3 & -1 \\
                    0 & 0 & 0  & 0 & 0  \\
                    0 & 0 & 0  & 0 & 0
                \end{pmatrix}.
            \end{align*}
            令 $ x_3 = k_1, x_4 = k_2, x_5 = k_3$,有 $x_1 = -2k_1 - k_2 + 2k_3,\enspace\allowbreak x_2 = k_1 - 3k_2 + k_3 $,则
            \[ X = (x_1, x_2, x_3, x_4, x_5)^\mathrm{T} = k_1 \begin{pmatrix} -2 \\ 1 \\ 1 \\ 0 \\ 0 \end{pmatrix} + k_2 \begin{pmatrix} -1 \\ -3 \\ 0 \\ 1 \\ 0 \end{pmatrix} + k_3 \begin{pmatrix} 2 \\ 1 \\ 0 \\ 0 \\ 1 \end{pmatrix} \qquad k_1, k_2, k_3 \in \mathbf{R} \]
        \end{answer}

        \item 求非齐次线性方程组$\begin{cases}
                x_1-x_2+2x_3-2x_4+3x_5=1     \\
                2x_1-x_2+5x_3-9x_4+8x_5=-1   \\
                3x_1-2x_2+7x_3-11x_4+11x_5=0 \\
                x_1-x_2+x_3-x_4+3x_5=3
            \end{cases}$的通解.

        \begin{answer}
            \begin{align*}
                \bar{A} ={} &
                \begin{pmatrix}
                    1 & -1 & 2 & -2  & 3  & 1  \\
                    2 & -1 & 5 & -9  & 8  & -1 \\
                    3 & -2 & 7 & -11 & 11 & 0  \\
                    1 & -1 & 1 & -1  & 3  & 3
                \end{pmatrix}
                \rightarrow
                \begin{pmatrix}
                    1 & -1 & 2  & -2 & 3 & 1  \\
                    0 & 1  & 1  & -5 & 2 & -3 \\
                    0 & 1  & 1  & -5 & 2 & -3 \\
                    0 & 0  & -1 & 1  & 0 & 2
                \end{pmatrix} \rightarrow \\
                            &
                \begin{pmatrix}
                    1 & -1 & 2  & -2 & 3 & 1  \\
                    0 & 1  & 1  & -5 & 2 & -3 \\
                    0 & 0  & -1 & 1  & 0 & 2  \\
                    0 & 0  & 0  & 0  & 0 & 0
                \end{pmatrix}
                \rightarrow
                \begin{pmatrix}
                    1 & 0 & 0 & -4 & 5 & 4  \\
                    0 & 1 & 0 & -4 & 2 & -1 \\
                    0 & 0 & 1 & -1 & 0 & -2 \\
                    0 & 0 & 0 & 0  & 0 & 0
                \end{pmatrix}.
            \end{align*}
            令 $ x_4 = k_1, x_5 = k_2 $,有 $x_1 = 4k_1 - 5k_2 + 4,\enspace\allowbreak x_2 = 4k_1 - 2k_2 - 1,\enspace\allowbreak x_3 = k_1 -2 $,则
            \[ X = (x_1, x_2, x_3, x_4, x_5)^\mathrm{T} = k_1 \begin{pmatrix} 4 \\ 4 \\ 1 \\ 1 \\ 0 \end{pmatrix} + k_2 \begin{pmatrix} -5 \\ -2 \\ 0 \\ 0 \\ 1 \end{pmatrix} + \begin{pmatrix} 4 \\ -1 \\ -2 \\ 0 \\ 0 \end{pmatrix} \qquad k_1, k_2 \in \mathbf{R} \]
        \end{answer}

        \item 求解线性方程组$\begin{cases}
                x_1+x_2+x_3=1   \\
                x_1+2x_2-5x_3=2 \\
                2x_1+3x_2-4x_3=5
            \end{cases}$.

        \begin{answer}
            见教材P33例3. 无解.
        \end{answer}
    \end{exgroup}

    \begin{exgroup}
        \item 设$A$是一个Abel群,$A$的运算是加法. 在$A$中定义乘法运算为$ab=0,\enspace\forall a,b\in A$. 证明:$A$为一个环,而且它的加法单位元与乘法单位元相同(我们称这种环为\term{零环}\index{huan!ling@零环 (zero ring)}).

        \item 证明:若集合$A$上的二元关系$R$满足
        \begin{enumerate}
            \item $a\,R\,a,\enspace\forall a\in A$;

            \item $\forall a,b,c\in A$,若$a\,R\,b$且$a\,R\,c$,则$b\,R\,c$.
        \end{enumerate}
        则$R$为$A$上的等价关系.
    \end{exgroup}

    \begin{exgroup}
        \item 证明:\autoref{ex:有限域} 中定义的$\langle \mathbf{Z}_n\colon\oplus,\circ\rangle$是域当且仅当$n$是素数.(提示:无论$n$是否为素数,$n\in\mathbf{Z}$且$n\geqslant 2$时$\langle \mathbf{Z}_n\colon\oplus,\circ\rangle$为交换环,因此是否为素数将决定这一结构中每个元素是否有逆元. 在初等数论中,我们熟知的裴蜀定理可以解决这一问题.)

        \item 本讲我们构造了$\mathbf{R}^2$上的乘法,从而定义了复数域的乘法运算. 本题希望探讨的是:$\mathbf{R}^3$无法构造出乘法使其成为一个域. 在高中的学习中我们知道,$\mathbf{R}^3$空间向量的一组基底为$\{\vec{e}_1=(1,0,0),\vec{e}_2=(0,1,0),\vec{e}_3=(0,0,1)\}$. 证明:$\mathbf{R}^3$没有乘法同时满足以下性质:
        \begin{enumerate}
            \item (单位元) $\forall \vec{u}\in\mathbf{R}^3,\enspace\vec{e}_1\cdot \vec{u}=\vec{u}\cdot \vec{e}_1$;

            \item (交换性) $\forall \vec{u},\vec{v}\in\mathbf{R}^3,\enspace\vec{u}\cdot \vec{v}=\vec{v}\cdot \vec{u}$;

            \item (长度可乘性) $\forall \vec{u},\vec{v}\in\mathbf{R}^3,\enspace|\vec{u}\cdot\vec{v}|=|\vec{u}||\vec{v}|$.
        \end{enumerate}
        按照如下思路给出详细证明过程:采用反证法. 假设乘法存在,则
        \begin{enumerate}
            \item 通过计算$(\vec{e}_1+\vec{e}_2)\cdot(\vec{e}_1-\vec{e}_2)$,$(\vec{e}_1+\vec{e}_3)\cdot(\vec{e}_1-\vec{e}_3)$,证明\[\vec{e}_2\cdot\vec{e}_2=\vec{e}_3\cdot\vec{e}_3=-\vec{e}_1.\]

            \item 证明$(\vec{e}_2+\vec{e}_3)\cdot(\vec{e}_2-\vec{e}_3)=0$得出矛盾.
        \end{enumerate}

        \item 尝试构造一个 4 元的域 $\mathbf{F}_4$.(提示:我们在上面的第一题中已经给出了 2 元的域 $\mathbf{Z}_2$,而且我们也知道 $\mathbf{Z}_4$ 不构成一个域,因此,我们考虑怎么把两个二元的域拼起来,可以尝试笛卡尔积的方式,并在其中定义二元运算来满足所需的性质. 进一步的两个事实是,对于所有的素数 $p$,都可以用类似的方法构造 $p^n$ 元的域;以及不是任意两个域的笛卡尔积都是域,这个例子已经被上面的第二题所说明.)
    \end{exgroup}
\end{exercise}
