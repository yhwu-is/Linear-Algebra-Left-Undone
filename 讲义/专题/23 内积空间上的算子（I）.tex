\chapter{内积空间上的算子(I)}

\section{正交矩阵和酉矩阵}

\section{正定矩阵}

可能很多同学对于行秩、列秩相等以及转置的几何意义很感兴趣.实际上我们有两种获得转置矩阵的
方式,第一种来源于我们之前讨论的对偶空间上的线性映射对应的矩阵,这种方式可能不够直观.
另一种获得的方法基于伴随算子.接下来我们将说明这些定义的统一性,深刻理解转置的内涵.

我们可以研究矩阵及其转置的关系,我们可以用一个图形来表示:

\begin{figure}[h]
    \centering
    \small
    \begin{tikzpicture}
        \tikzset{->-/.style={decoration={
            markings,
            mark=at position .6 with {\arrow{stealth}}},postaction={decorate}}}

        \draw[rotate=45] (0,6) rectangle (-3,3) rectangle (-5,0)
            (-3,3) rectangle(-3.35,3.35)
            coordinate (xr) at (-2,4)
            coordinate (xn) at (-4,2)
            coordinate (x) at (-2,2)
            coordinate (0n) at (-3,3)
            node at (-1,5) {行空间}
            node at (-4,1) {$A$的核空间}
            node at (-1.5,6.5) {$\dim r$}
            node at (-4,4) {$\mathbf{R}^n$}
            node at (-6,3) {$\dim n-r$};

        \draw[rotate=30] (6,2) rectangle (3.5,-2) rectangle (0,-4)
            (3.5,-2) rectangle (3.85,-2.35)
            coordinate (b) at (4.5,0.5)
            coordinate (0m) at (3.5,-2)
            node at (5,1.5) {列空间}
            node at (2,-3) {$A^{\mathrm{T}}$的核空间}
            node at (7,0) {$\dim r$}
            node at (5,-3) {$\mathbf{R}^m$}
            node at (4,-4.5) {$\dim m-r$};

        \foreach \point in {xr, x, xn, 0n, b, 0m}
            \fill[black] (\point) circle (1pt);

        \node [left] at (xr) {$x_r$};
        \node [below right] at (x) {$x=x_r+x_n$};
        \node [left] at (xn) {$x_n$};
        \node [right] at (0n) {0};
        \node [right] at (b) {$b$};

        \draw[->-,very thick] (xr) -- node[above,sloped] {$Ax_r = b$} (b);
        \draw[->-,very thick] (x) -- node[below,sloped] {$Ax = b$} (b);
        \draw[->-,very thick] (xn) -- node[below,sloped] {$Ax_n = 0$} (0m);

        \draw[dashed,thick] (xr) -- (x) -- (xn);

    \end{tikzpicture}
\end{figure}

我们观察到以下几点:
\begin{enumerate}
    \item 矩阵的行空间与解空间(零空间)互为正交补(直观理解两个空间就是互相垂直且互为补空间),这一点应当是在正交的内容中有所提及的;
    \item 矩阵的列空间与其转置矩阵的零空间互为正交补,这一点实际与上一条等价.
\end{enumerate}

接下来我们来看行秩(列秩比较显然,此处不再详细展开).我们首先得到解空间($N(A)$)的维数,这可以直接
根据维数公式得到:$\dim N(A) =n-r(A)$,根据正交补的性质,我们的可以得到行秩即为
$n-(n-r(A))=r(A)$.于是我们得到了一个基于正交补的行秩解释.

\vspace{2ex}
\centerline{\heiti \Large 内容总结}

\vspace{2ex}

\centerline{\heiti \Large 习题}
\vspace{2ex}
{\kaishu }
\begin{flushright}
    \kaishu

\end{flushright}
\centerline{\heiti A组}
\begin{enumerate}
    \item
\end{enumerate}
\centerline{\heiti B组}
\begin{enumerate}
    \item
\end{enumerate}
\centerline{\heiti C组}
\begin{enumerate}
    \item
\end{enumerate}
