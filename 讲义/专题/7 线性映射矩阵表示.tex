\chapter{线性映射矩阵表示}

\section{线性映射矩阵表示}
\begin{definition}\label{def:7:线性映射矩阵表示}
    设$B_1=\{\varepsilon_1,\varepsilon_2,\ldots,\varepsilon_n\}$是$V_1(F)$的基,$B_2=\{\alpha_1,\alpha_2,\cdots,\alpha_m\}$是$V_2(F)$的基.
    则线性映射$\sigma \in \mathcal{L}(V_1,V_2)$被它作用于基$B_1$的像
    \[\sigma(B_1)=\{\sigma(\varepsilon_1),\sigma(\varepsilon_2),\ldots,\sigma(\varepsilon_n)\}\]
    所唯一确定,而$\sigma(B_1)$是$V_2$的子空间,于是其中元素都可以被基$B_2$线性表示,即
    \[ \left\{
     \begin{array}{rcl}
        \sigma(\varepsilon_1)&=&a_{11}\alpha_1+a_{21}\alpha_2+\ldots+a_{m1}\alpha_m \\
        \sigma(\varepsilon_2)&=&a_{12}\alpha_1+a_{22}\alpha_2+\ldots+a_{m2}\alpha_m \\
        &\vdots& \\
        \sigma(\varepsilon_n)&=&a_{1n}\alpha_1+a_{2n}\alpha_2+\ldots+a_{mn}\alpha_m
     \end{array}
    \right. \]

    我们将$\sigma(B_1)=\{\sigma(\varepsilon_1),\sigma(\varepsilon_2),\ldots,\sigma(\varepsilon_n)\}$
    关于基$B_2$的坐标排列成矩阵$\mathbf{M}(\sigma)$,即
    \[\mathbf{M}(\sigma)=\begin{pmatrix}
        a_{11} & a_{12} & \cdots & a_{1n} \\
        a_{21} & a_{22} & \cdots & a_{2n} \\
        \vdots & \vdots & \ddots & \vdots \\
        a_{m1} & a_{m2} & \cdots & a_{mn}
    \end{pmatrix}\]
\end{definition}
更通俗来说,线性映射矩阵表示就是将线性映射在一组基上的像在另一组基下的坐标表示按列排列得到的结果.
这一整体过程我们也可以用如下记号表示:
\begin{equation}\label{eq:7:线性映射矩阵表示}
    (\sigma(\epsilon_1),\sigma(\epsilon_2),\ldots,\sigma(\epsilon_n))=(\alpha_1,\alpha_2,\ldots,\alpha_m)\mathbf{M}(\sigma).
\end{equation}

\begin{example}\label{ex:7:矩阵表示1}
    已知$\sigma \in \mathcal{L}(\mathbf{R}^3,\mathbf{R}^3)$且$\sigma(x_1,x_2,x_3)=(x_1+x_2,x_1-x_3, x_2)$
    \begin{enumerate}[label=(\arabic*)]
        \item 求$\sigma$的像空间和核空间;

        \item 求$\sigma$关于$\mathbf{R}^3$自然基的矩阵.
    \end{enumerate}
\end{example}
\begin{solution}
    \begin{enumerate}[label=(\arabic*)]
        \item 求像空间和核空间的方法我们在之前已经介绍过,我们为了计算方便取$\mathbf{R}^3$的自然基$e_1,e_2,e_3$计算有:
        \[\im\sigma=\spa(\sigma(e_1),\sigma(e_2),\sigma(e_3))=\spa((1,1,0),(1,0,1),(0,-1,0))=\mathbf{R}^3\]
        对于核空间,解方程$\sigma(\alpha)=0$即可,我们也可以用更简洁的方式书写:
        \[\ker\sigma=\{(x_1,x_2,x_3)\mid \sigma(x_1,x_2,x_3)=(0,0,0)\}=\{(0,0,0)\}\]
        即方程只有零解,核空间可以记为$\ker\sigma=\{0\}$(只含零元的空间的一般记法).
        \item 我们根据\autoref{def:7:线性映射矩阵表示},我们应先写出$\sigma$在出发空间一组基(按题目要求是$\mathbf{R}^3$
        自然基)下的像,并将像表示为到达空间基(按题目要求是$\mathbf{R}^3$自然基)的线性组合,即
        \begin{gather*}
            \sigma(e_1)=(1,1,0)=e_1+e_2=(e_1,e_2,e_3)\begin{pmatrix}
                1 \\ 1 \\ 0
            \end{pmatrix} \\
            \sigma(e_2)=(1,0,1)=e_1+e_3=(e_1,e_2,e_3)\begin{pmatrix}
                1 \\ 0 \\ 1
            \end{pmatrix} \\
            \sigma(e_3)=(0,-1,0)=-e_2=(e_1,e_2,e_3)\begin{pmatrix}
                0 \\ -1 \\ 0
            \end{pmatrix}
        \end{gather*}
        接下来我们把坐标依次按列称矩阵就得到了本题需要求解的矩阵:
        \[\mathbf{M}(\sigma)=\begin{pmatrix}
            1 & 1 & 0 \\
            1 & 0 & -1 \\
            0 & 1 & 0
        \end{pmatrix}\]
    \end{enumerate}
\end{solution}

有趣的是,在结合我个人的学习经历以及过往辅学的经验后,我总结出了第二问的一种常见的错误解法,这里我需要加粗强调,下面这种
解法是\textbf{完全错误的!!!},这里展示这一解法是为了让读者将前面所学的知识完全厘清:

\begin{solution}
    (\textbf{错误解法!!!})$\sigma(x_1,x_2,x_3)=(x_1+x_2,x_1-x_3, x_2)=(x_1,x_2,x_3)\begin{pmatrix}
        1 & 1 & 0 \\
        1 & 0 & 1 \\
        0 & -1 & 0
    \end{pmatrix}$
\end{solution}

我们惊奇地发现,这一结果和我们前面得到的标准答案在向量的排列方式上发生了变化,即标准答案的1、2、3行变为了这里的1、2、3列,
我们需要强调两点:
\begin{enumerate}
    \item 为什么这种解法是错误的:我们可以直接比较\autoref{eq:7:线性映射矩阵表示}和这一解法中,\autoref{eq:7:线性映射矩阵表示}
    的等号左边是$n$个向量在$\sigma$下的像,而上述解法$\sigma(x_1,x_2,x_3)$只是$\sigma$在一个向量下的像,这显然是不一样的!!!
    同样,等号右边括号内\autoref{eq:7:线性映射矩阵表示}是到达空间的一组基,而上述解法中仍然只是一个向量.我们从未定义过这样解题
    的结果是什么,所以千万不能做这种无意义的事!!!

    容易导致混淆的原因可能在于$(x,y,z)$向量是排列成一行的,可能看起来和$(e_1,e_2,e_3)$有点相似,但如果我们将后者拆分成
    $((1,0,0),(0,1,0),(0,0,1))$,你还会混淆吗?

    \item 为什么会出现行列互换这样的错误:事实上
    \[\sigma(x,y,z)=\sigma(xe_1+ye_2+ze_3)=x\sigma(e_1)+y\sigma(e_2)+z\sigma(e_3)=(x,y,z)\begin{pmatrix}
        \sigma(e_1) \\ \sigma(e_2) \\ \sigma(e_3)
    \end{pmatrix},\]
    这里将$\sigma(e_1),\sigma(e_2),\sigma(e_3)$的结果按行排列成矩阵,而标准答案是将$\sigma(e_1),\sigma(e_2),\sigma(e_3)$
    在$\mathbf{R}^3$自然基下的坐标按列排列成矩阵,回忆$\mathbf{R}^n$向量在自然基下坐标是其本身这一性质,标准答案就是将
    $\sigma(e_1),\sigma(e_2),\sigma(e_3)$按列排列成矩阵,由此我们解释了行列互换发生的原因.
\end{enumerate}

这也就是为什么我强调读者不要参考教材102页例3求解像空间的方法来求解像空间——很容易导致这里矩阵表示犯这样的错误,
并且容易导致初学时无法区分求解像空间和线性映射矩阵表示的方法.在这里我必须再次强调:在没有完全熟练掌握这些概念和方法前,
不要乱用方法!!!

还需需要特别强调的一点是,
之后我们会经常看见两种记号,即$(\sigma(\varepsilon_1),\sigma(\varepsilon_2),\ldots,\sigma(\varepsilon_n))$
和$\sigma(\varepsilon_1,\varepsilon_2,\ldots,\varepsilon_n)$.实际上是等价的,等价原因是
$(\sigma(\varepsilon_1),\sigma(\varepsilon_2),\ldots,\sigma(\varepsilon_n))A=(\sigma(\varepsilon_1,\varepsilon_2,\ldots,\varepsilon_n))A=\sigma((\varepsilon_1,\varepsilon_2,\ldots,\varepsilon_n)A)$成立,
这一性质在之后会有运用,证明并不复杂,可以自行尝试或参考我的矩阵辅学授课.

\section{线性映射矩阵表示的进一步讨论}
\subsection{一组简单的例子}
\begin{example}\label{example:5:矩阵表示1}
    已知$\sigma \in \mathcal{L}(\mathbf{R}^3,\mathbf{R}^3)$且$\sigma(x_1,x_2,x_3)=(x_1+x_2,x_1-x_3, x_2)$
    \begin{enumerate}[label=(\arabic*)]
        \item 求$\sigma$的像空间和核空间;

        \item 求$\sigma$关于$\mathbf{R}^3$自然基的矩阵.
    \end{enumerate}
\end{example}

\begin{example}\label{ex:6:矩阵表示2}
    设$A=\begin{pmatrix}1 & 0 & 2 \\ -1 & 2 & 1 \\ 1 & 2 & 5\end{pmatrix}$为两个三维线性空间之间的线性映射$\sigma$对应的矩阵,
    求$\sigma$的像空间和核空间.
\end{example}

\begin{example}\label{ex:6:矩阵表示3}
    已知3阶矩阵$A=\begin{pmatrix}
        1 & 0 & 1 \\ 0 & -1 & 0 \\ -1 & 1 & -1
    \end{pmatrix}$. 定义$\mathbf{F}^{3 \times 3}$上的线性变换$\sigma(X)=AX,\enspace X \in \mathbf{F}^{3 \times 3}$.
    求$\sigma$的像和核.
\end{example}
实际上,例题2.4.1和2.4.3都是属于已知映射求像和核的题目,具体方法在像和核一节已经讲述,并且求矩阵表示也是根据上面的定义
即可,都是程式化的.然而例7则有不同,但此题与例2.4.1、2.4.2也有关联.实际上
此类问题像空间就是以矩阵列空间为坐标的向量的线性扩张,核空间是以矩阵零空间的基(即$AX=0$的基础解系)为坐标的向量的线性扩张,
推导见例7解析或我的矩阵辅学,希望各位同学能掌握推导并理解这三个例题之间的关系与区别. % TODO 编号系统 autoref

\subsection{一些相似的定理}
\begin{theorem} \label{thm:6:线性映射对向量坐标的影响}
    \textbf{\heiti 线性映射对向量坐标的影响}

    设$\sigma \in \mathcal{L}(V_1,V_2)$关于$V_1$和$V_2$的基$B_1$和基$B_2$的矩阵为$A=(a_{ij})_{m \times n}$,
    且$\alpha$与$\sigma(\alpha)$在基$B_1$和基$B_2$下的坐标分别为$X$和$Y$,则$Y=AX$.
\end{theorem}
上述即教材定理4.1,这一定理给出一个向量经过线性映射之后,其坐标的变化. 我们可以用下图表示:

\begin{figure}[h]
    \centering
    \begin{tikzpicture}[>=Stealth]
        \node (V) at (0,0) {$V$};
        \node (W) at (3,0) {$W$};
        \node (Fn) at (0,-3) {$\mathbf{F}^n$};
        \node (Fm) at (3,-3) {$\mathbf{F}^m$};
        \draw[->,thick] (V) -- node[below]{表示矩阵:$A$} (W);
        \draw[<->,thick] (V) -- node[right]{同构} (Fn);
        \draw[<->,thick] (W) -- node[left]{同构} (Fm);
        \draw[->,thick] (Fn) -- node[above]{$\sigma(\alpha)=A\alpha$} (Fm);
    \end{tikzpicture}
\end{figure}

图中我们可以看出通过坐标映射后得到的新映射即为\autoref{thm:6:线性映射对向量坐标的影响} 描述的映射.

在描述下一定理之前,我们首先介绍过渡矩阵(变换矩阵)的概念.
\begin{definition}
    设$B_1=\{\alpha_1,\alpha_2,\ldots,\alpha_n\}$与$B_2=\{\beta_1,\beta_2,\ldots,\beta_n\}$是线性空间
    $V(\mathbf{F})$的任意两组基,$B_2$中每个基向量被基$B_1$表示为
    \[ \left\{
    \begin{array}{rcl}
        \beta_1&=&a_{11}\alpha_1+a_{21}\alpha_2+\cdots+a_{n1}\alpha_n \\
        \beta_2&=&a_{12}\alpha_1+a_{22}\alpha_2+\cdots+a_{n2}\alpha_n \\
        &\vdots& \\
        \beta_n&=&a_{1n}\alpha_1+a_{2n}\alpha_2+\cdots+a_{nn}\alpha_n
    \end{array}
    \right. \]
    将上式用矩阵表示为
    \[(\beta_1,\beta_2,\cdots,\beta_n)=(\alpha_1,\alpha_2,\cdots,\alpha_n)\begin{pmatrix}
        a_{11} & a_{12} & \cdots & a_{1n} \\
        a_{21} & a_{22} & \cdots & a_{2n} \\
        \vdots & \vdots & \ddots & \vdots \\
        a_{n1} & a_{n2} & \cdots & a_{nn}
    \end{pmatrix}\]
    我们将这一矩阵称为即$B_1$变为基$B_2$的变换矩阵(或过渡矩阵).
\end{definition}
简单而言就是将$B_2$中的向量在$B_1$下的坐标按列排列.需要注意表述中是$B_1$变为基$B_2$还是反过来,
这两个矩阵互逆.注意过渡矩阵一定是基与基之间的表示矩阵,并且过渡矩阵一定可逆.
\begin{theorem}
    \textbf{\heiti 基的选择对向量坐标的影响}

    设线性空间$V$的两组基为$B_1$和$B_2$,且基$B_1$到$B_2$的变换矩阵(过渡矩阵)为$A$,如果
    $\xi \in V(\mathbf{F})$,且在$B_1$和$B_2$下的坐标分别为$X$和$Y$,则$Y=A^{-1}X$.
\end{theorem}
上述即教材定理4.10,描述同一个向量在不同基下坐标之间的关系.事实上,这与本节同构关系紧密,因为
同构意味着两个线性空间结构一致,故同构映射可以保持向量组的线性关系不变.在同构关系下,
线性组合对应线性组合,线性无关对应线性无关,线性相关对应线性相关.我们有如下定理:
\begin{theorem}
    设$(\alpha_1,\alpha_2,\ldots,\alpha_n)$是线性无关的向量组,且
    \[(\beta_1,\beta_2,\ldots,\beta_s)=(\alpha_1,\alpha_2,\ldots,\alpha_n)A\]
    则向量组$(\beta_1,\beta_2,\ldots,\beta_s)$的秩等于矩阵$A$的秩.
\end{theorem}
定理的证明需要用到坐标映射是同构映射这一事实,我们不难发现等式左侧向量组与$A$的列向量组是等价的.
事实上我们也可以由此发现,过渡矩阵一定是可逆矩阵.
\begin{theorem}
    已知$\beta_i=a_{1i}\alpha_1+a_{2i}\alpha_2+\cdots+a_{ni}\alpha_n\enspace(i=1,2,\ldots,n)$,
    且$A=(a_{ij})$可逆,则$\alpha_1,\alpha_2,\ldots,\alpha_n$与$\beta_1,\beta_2,\ldots,\beta_n$
    等价.
\end{theorem}
实际上这一定理与上一定理的思想都是类似的,我们可以看一个例题练习一下:
\begin{example}
    已知$\beta_1=\alpha_2+\alpha_3,\enspace\beta_2=\alpha_1+\alpha_3,\enspace\beta_3=\alpha_1+\alpha_2$,
    证明$\alpha_1,\alpha_2,\alpha_3$与$\beta_1,\beta_2,\beta_3$等价.
\end{example}
\begin{theorem}
    \textbf{\heiti 基的选择对映射矩阵的影响}

    设线性变换$\sigma \in \mathcal{L}(V,V)$,$B_1=\{\alpha_1,\ldots,\alpha_n\}$和$B_2=\{\beta_1,\ldots,\beta_n\}$
    是线性空间的$V(\mathbf{F})$的两组基,基$B_1$变为基$B_2$的变换矩阵为$C$,如果$\sigma$在基$B_1$下的矩阵为$A$,
    则$\sigma$关于基$B_2$所对应的矩阵为$C^{-1}AC$.
\end{theorem}
上述即教材定理7.4,研究同一个映射在不同基下表示矩阵之间的关系.实际上我们将在下一专题初等矩阵一节进一步讨论.
这一定理的证明需要用到我们之前描述的两种线性映射矩阵表示的统一性.

\vspace{2ex}
\centerline{\heiti \Large 内容总结}

\vspace{2ex}

\centerline{\heiti \Large 习题}
\vspace{2ex}
{\kaishu }
\begin{flushright}
    \kaishu

\end{flushright}
\centerline{\heiti A组}
\begin{enumerate}
    \item 设$\sigma: V_1\to V_2$是线性映射,证明:$\sigma(W_1)$和$\sigma^{-1}(W_2)$分别是$V_2$和$V_1$的子空间
\end{enumerate}
\centerline{\heiti B组}
\begin{enumerate}
    \item
\end{enumerate}
\centerline{\heiti C组}
\begin{enumerate}
    \item
\end{enumerate}
