\chapter{有理标准形}

% 相抵、相似、相合的全系不变量
最后我们再来总结一个题型. 一些题目可能需要判断矩阵是否相似,实际上我们有如下基本方法:
\begin{enumerate}
    \item 定义法:找到$P$使得$P^{-1}AP=B$即可,这一般是$A,B$没给出具体矩阵的做法,例如上面的性质证明;

    \item 我们也可以先计算两者特征多项式是否相等(即特征值是否一致),若不一致则一定不相似,得到结论,若一致且均为实对称矩阵则相似,否则不一定相似(因为这是相似的必要条件). 对于这种特征值一致的情况,我们进行对角化,情况如下:
          \begin{enumerate}
              \item 若两矩阵均可对角化,则两矩阵相似:因为特征多项式相等则特征值相等,均可对角化那么对角矩阵也完全一致,因此二者与同一个对角矩阵相似,根据相似这一等价关系的传递性可知两矩阵相似;

              \item 若一个矩阵可对角化,另一个矩阵不可对角化,则一定不相似;

              \item 若两个矩阵都不可对角化,不一定相似. 需要两矩阵各个特征值的几何重数(即各个特征子空间维数)都一致才相似,否则不相似. 这是因为只有几何重数一致才有相同的若当标准形.
          \end{enumerate}
\end{enumerate}

\begin{example}
    设$A,B\in \mathbf{M}_n(\mathbf{F})$,证明:若$A$可逆,则$AB\sim BA$.
\end{example}

\begin{proof}

\end{proof}

\begin{example}
    设$A=\begin{pmatrix}
            0 & 0 & 1 \\ 0 & 1 & 0 \\ 1 & 0 & 0
        \end{pmatrix},B=\begin{pmatrix}
            -1 & 0 & 0 \\ 0 & 0 & 1 \\ 0 & -1 & 2
        \end{pmatrix}$,判断$A$与$B$是否相似.
\end{example}

\begin{solution}

\end{solution}

\vspace{2ex}
\centerline{\heiti \Large 内容总结}

\vspace{2ex}
\centerline{\heiti \Large 习题}

\vspace{2ex}
{\kaishu }
\begin{flushright}
    \kaishu

\end{flushright}

\centerline{\heiti A组}
\begin{enumerate}
    \item
\end{enumerate}

\centerline{\heiti B组}
\begin{enumerate}
    \item
\end{enumerate}

\centerline{\heiti C组}
\begin{enumerate}
    \item
\end{enumerate}
